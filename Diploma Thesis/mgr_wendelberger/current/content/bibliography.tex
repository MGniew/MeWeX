\begin{thebibliography}{9}

\bibitem{supermatrix}
Broda B., Piasecki M., \emph{Parallel, Massive Processing in SuperMatrix – a General Tool for Distributional Semantic Analysis of Corpora}, International Journal of Data Mining, Modelling and Management, 2011, p. 373-379.

\bibitem{buczynski}	
Buczyński A., \emph{Pozyskiwanie z Internetu tekstów do badań lingwistycznych}, praca magisterska napisana na Wydziale Matematyki, Informatyki i Mechaniki Uniwersytetu Warszawskiego, Warszawa, 2004.
	 
\bibitem{ccl_web}
CCL format - Corpus2 - nlp.pwr.wroc.pl: http://nlp.pwr.wroc.pl/redmine/projects/corpus2/wiki/CCL\_format , dostępna 3.02.2015 r.
	
\bibitem{fdpn}
Da Silva J. F., Lopes G. P., \emph{A Local Maxima method and a Fair Dispersion Normalization for extracting multi-word units from corpora}, [in:] \emph{Proceedings of the sixth meeting on mathematics of language (MOL6)}, Orlando, 1999, p. 369-381.

\bibitem{mutual_expectation}
Dias G., Lopes J. G. P., Guilloré S., \emph{Mutual Expectation: A Measure for Multiword Lexical Unit Extraction}, [in:] \emph{Proceedings of VEXTAL ’99}, Venezia, 1999, p. 133-138.

\bibitem{contingency_book}
Dillon W. R., \emph{Analyzing large multiway contingency tables: a simple method for selecting variables}, Journal of Marketing, Fall 1979, p. 99-102.

\bibitem{aggregation}	
Dinu A., Dinu L. P., Sorodoc I. T., \emph{Aggregation methods for efficient collocation detection}, [in:] \emph{Proceedings of the Ninth International Conference on Language Resources and Evaluation (LREC ’14)}, Reykjavik, 2014, p. 4041-4045.

\bibitem{evert}
Evert S., \emph{The Statistics of Word Cooccurences Word Pairs and Collocation Extraction}, PhD dissertation, University of Stuttgart, 2004.

\bibitem{firth}
Firth J. R., \emph{A synopsis of linguistic theory 1930-55}, [in:] \emph{Studies in linguistic analysis}, Oxford 1957, p. 1-32.

\bibitem{iob_chan_web}
Formaty wejściowe i wyjściowe - Corpus2 - nlp.pwr.wroc.pl: http://nlp.pwr.wroc.pl/redmine/projects/corpus2/wiki/Formaty\_wej\%C5\%9Bciowe\_i\_wyj\%C5\%9Bciowe\#IOB-CHAN , dostępna 3.02.2015 r.

\bibitem{dispersions}
Gries S. Th., \emph{Dispersions and adjusted frequencies in corpora}, International Journal of Corpus Linguistics 13:4, John Benjamins Publishing Company, 2008, p. 403-437.

\bibitem{wsec}
Hoang H. H., Kim S. N., Kan M.-Y.,
\emph{A Re-examination of Lexical Association Measures}, [in:]
\emph{Proceedings of the 2009 Workshop on Multiword Expressions, ACL-IJCNLP 2009}, Suntec, Singapore, 2009, p. 31–39. 

\bibitem{klyk}	
Kłyk Ł., \emph{Metody sztucznej inteligencji w zwiększaniu skuteczności klasyfikatora}, praca magisterska napisana na Wydziale Informatyki i Zarządzania Politechniki Wrocławskiej, Wrocław 2013.

\bibitem{korpus_rzeczpospolitej}	
Korpus Rzeczpospolitej: http://www.cs.put.poznan.pl/dweiss/research/rzeczpospolita/ , dostępna 3.02.2015 r.

\bibitem{hypermat}
Lim L.-H., \emph{Tensors and hypermatrices}, [in:] L. Hogben (Ed.), \emph{Handbook of Linear Algebra. Second Edition}, Boca Raton, 2013, p. 15.1–15.30.

\bibitem{mit}
Manning Ch. D., Schütze H., \emph{Foundations of Statistical Natural Language Processing}, The MIT Press, London, 2000.

\bibitem{contingency_lecture}	
Materiał wykładowy z zajęć prof. Jarretta Barbera, dostępny na dzień 24.01.2015 pod adresem: http://old.stat.duke.edu/courses/Spring02/sta102/chap16.pdf .
	
\bibitem{morfeusz}
Morphological analyser Morfeusz SGJP: http://sgjp.pl/morfeusz/ , dostępna 3.02.2015 r.	

\bibitem{paradowski_beta}
Paradowski M., \emph{Opracowanie formalnej analizy zależności pomiędzy współczynnikami służącymi do wykrywania wyrażeń wielowyrazowych oraz na tej podstawie opracowanie współczynnika uogólniającego}, Instytut Informatyki, Politechnika Wrocławska, 2014 (artykuł niepublikowany).

\bibitem{pecina_measures}
Pecina P., \emph{Lexical association measures and collocation extraction}, Language resources and evaluation, Vol. 44, 2010, p. 137 – 158.

\bibitem{pecina_resource}
Pecina P., \emph{Reference Data for Czech Collocation Extraction}, [in:] \emph{Proceedings of the LREC Workshop Towards a Shared Task for MWEs (MWE 2008)}, Marrakech, 2008, p. 11-14.

\bibitem{coling}
Pecina P., Schlesinger P., \emph{Combining Association Measures for Collocation Extraction}, [in:] \emph{Proceedings of the COLING/ACL 2006 Main Conference Poster Sessions}, Sydney, 2006, p. 651 – 658.

\bibitem{generalization_patterns}
Petrović S., Šnajder J., Bašić B. D., \emph{Extending lexical association measures for collocation extraction}, Computer Speech and Language, Vol. 24, Issue 2, April 2010, p. 383-394.
	 
\bibitem{corpus2_web}	 
Przegląd - Corpus2 - nlp.pwr.wroc.pl: http://nlp.pwr.wroc.pl/redmine/projects/corpus2/wiki , dostępna 3.02.2015 r.

\bibitem{korpus_ipi_pan_publikacja}
Przepiórkowski A., \emph{Korpus IPI PAN. Wersja wstępna/The IPI PAN Corpus: Preliminary version}, Warszawa, IPI/PAN, 2004.

\bibitem{wcrft}
Radziszewski A., \emph{A tiered CRF tagger for Polish}, [in:] \emph{Intelligent Tools for Building a Scientific Information Platform: Advanced Architectures and Solutions}, Springer, 2013, p. 215-230.

\bibitem{wccl}
Radziszewski A., Wardyński A., Śniatowski T., \emph{WCCL: A Morpho-syntactic Feature Toolkit}, [in:] \emph{Proceedings of the Balto-Slavonic Natural Language Processing Workshop}, Springer, 2011.

\bibitem{slowosiec}	
Słowosieć – witryna internetowa: http://plwordnet.pwr.wroc.pl/wordnet , dostępna 28.01.2015 r.

\bibitem{korpus_ipi_pan}
Strona internetowa autorów korpusu KIPI: www.korpus.pl , dostępna 29.11.2014 r.

\bibitem{smadja_xtract}
Smadja F., \emph{Retrieving Collocations from Text: Xtract}, Computational Linguistics – Special issue on using large corpora: I, Vol. 19, Issue 1, March 1993, p. 143-177.

\bibitem{fbmd}		
Thanopoulos A., Fakotakis N., Kokkinakis G., \emph{Comparative Evaluation of Collocation Extraction Metrics}, [in:] \emph{Proceedings of the Third International Conference on Language Resources and Evaluation (LREC-2002)}, Las Palmas, 2002, p. 620-625.

\bibitem{mmi_w11}		
Van de Cruys T., \emph{Two Multivariate Generalizations of Pointwise Mutual Information}, [in:] \emph{Proceedings of the Workshop on Distributional Semantics and Compositionality}, Portland, 2011, p. 16-20.
	
\bibitem{wccl_web}	
Wiki - WCCL - nlp.pwr.wroc.pl: http://nlp.pwr.wroc.pl/redmine/projects/joskipi/wiki , dostępna 3.02.2015 r.
	
\bibitem{wcrft_web}
Wiki - WCRFT - nlp.pwr.wroc.pl: http://nlp.pwr.wroc.pl/redmine/projects/wcrft/wiki , dostępna 3.02.2015 r.

\end{thebibliography}