\chapter{Definicja \protect\textit{kolokacji}}


\section{Próby definicji terminu \protect\textit{kolokacji} w literaturze}
Termin \emph{kolokacja} został pierwszy raz użyty w 1930 roku przez jego autora, angielskiego lingwistę Johna Ruperta Firtha. 
Początkowo zwrot ten miał opisywać charakterystyczne i typowe dla języka kombinacje słów, które łączą się ze sobą w konkretny sposób, dostarczając dzięki temu cenną wiedzę i informacje o tym języku \cite[str. 15]{evert}.
Firth sformułował zwrot \emph{You shall know a word by company it keeps!} \cite{firth}, którym zwrócił uwagę ludzi zajmujących się lingwistyką na fakt, że teksty nie są tylko zbitką losowo występujących słów ograniczonych jedynie przez zasady składniowe języka \cite[str. 15]{evert}. Zwrot ten w wolnym tłumaczeniu oznacza, że powinniśmy znać słowo na podstawie innych wyrazów w jego otoczeniu.
\par
Trudność jednoznacznego i ścisłego zdefiniowania kolokacji została opisana w obrazowy sposób przez Choueka i przytoczona przez Stefana Everta -- \emph{Even though any two lexicographers would agree that 'once upon a time', 'hit the road' and similar idioms are collocations, they would most certainly disagree on almost anything else.} \cite[str. 15]{evert}.
Cytat ten jest metaforą obrazującą bardzo niską korelację oceny wyrażeń wielowyrazowych jako kolokacji, nawet pomiędzy lingwistami.
\par
Na przestrzeni ponad osiemdziesięciu ostatnich lat definicja terminu \emph{kolokacja} była wielokrotnie modyfikowana, dostosowywana i poddawana uszczegółowieniu co sprawiło, że pojawiło się wiele jej wersji. 
Powstałe definicje można jednak w większości przyporządkować do jednej z dwóch grup reprezentujących różne podejścia do zagadnień kolokacji: \emph{distributional} oraz \emph{intensional} \cite[str. 15]{evert}.
Pierwsza z tych grup, zwana także \emph{szkołą Neo-Firthańską} opiera się głównie na wykorzystaniu informacji pozyskanych w sposób empiryczny z przebadanych zbiorów danych \cite[str. 15]{evert}. 
Druga grupa to natomiast koncept bardziej teoretyczny, skupia się na tym, że kolokacje są bytami umiejscowionymi pomiędzy wolną kombinacją słów a idiomami, parami składającymi się z jednego słowa wolnego, zwanego bazą i drugiego zdeterminowanego leksykalnie - podejście teoretyczne do tematu wyrażeń wielowyrazowych \cite[str. 16]{evert}. 
W dalszej części tej pracy na potrzeby realizacji tematu będę starał się określić definicję bliższą pierwszej z tych dwóch grup.
\par
Liczbę wielu różnych definicji jednostki wielowyrazowej odzwierciedla także liczba zwrotów o praktycznie tym samym znaczeniu, które są używane zamiennie jako terminy określające kolokację, przykładowo: \emph{jednostka wielowyrazowa (ang. MWU)}, \emph{wyrażenie wielowyrazowe (MWE)}, \emph{n-gramy} dla wieloelementowych kolokacji czy wręcz po prostu idiomy \cite[str. 16]{evert}. 


\section{Przykładowe definicje jednostki wielowyrazowej stosowane w literaturze}
Przykładowe definicje kolokacji z różnych źródeł wraz z krótkim komentarzem do nich zostały zamieszczone na poniższej liście:

\begin{enumerate}
\item 
Manning i Schütze -- \emph{collocations correspond to some conventional way of saying things} \cite[str. 151]{mit};
\newline
"kolokacje to sposób w jaki przyjęło się mówić pewne rzeczy."
\newline		
		
\item 
Manning i Schütze -- \emph{A collocation is any turn of phrase or accepted usage where somehow the whole is perceived to have an existence beyond the sum of the parts} \cite[str. 29]{mit}; 
\newline
"kolokacja to wyrażenie, którego znaczenie wykracza poza sumę znaczeń jej elementów składowych - ma pewną wartość dodaną do znaczenia lub ulega ono zmianie."
\newline

\item
Choueka -- \emph{a syntactic and semantic unit whose exact and unambiguous meaning or connotation cannot be derived directly from the meaning or connotation of its components} \cite[str. 1]{coling};
\newline
"wyrażenie wielowyrazowe to jednostka syntaktyczna i semantyczna, której dokładne znaczenie nie może być określone bezpośrednio na podstawie znaczeń jego składowych" -- definicja analogiczna w swym sensie do zamieszczonej przez Manninga i Schütze.		
\newline		
		
\item 
Stefan Evert -- \emph{A collocation is a word combination whose semantic and/or syntactic properties cannot be fully predicted from those of its components, and which therefore has to be listed in a lexicon} \cite[str. 17]{evert}; \newline
podobna definicja jak u poprzedników, jednak z dodaniem części mówiącej o tym, że kolokacja to związek, który powinien zostać zamieszczony w słowniku.
Rozszerzenie tej definicji było inspirowane pracami Choueka, który opracował sugestywną instrukcję do oceny czy dane wyrażenie jest kolokacją, a której esencja wyciągnięta przez Stefana Everta została sprowadzona do postaci pytania: \emph{Does it deserve a special entry in a dictionary or lexical database of the language?} - czy wyrażenie zasługuje na wpis w leksykonie? \cite[str. 17]{evert}.
Takie podsumowanie wydaje się być pewnym uproszczeniem.
\end{enumerate}

\par
Mimo ogromnej liczby definicji terminu \emph{kolokacja}, większość jest zgodna co do trzech cech, które wyrażenie powinno spełniać, aby zostać uznanym za wyrażenie wielowyrazowe, a są to: semantyczna niekompozycyjność, niezmienność syntaktyczna oraz niemożliwość podmiany słów składowych nawet na ich synonimy \cite[str. 16]{evert}. 
Wyjaśnienie tych oraz innych cech jednostek wielowyrazowych zostało zamieszczone w dalszej części pracy.	



\section{Cechy wyrażeń wielowyrazowych}

\subsection{Częściowa lub całkowita niekompozycyjność}
Specjalną cechą kolokacji jest ich ograniczona kompozycyjność lub całkowity jej brak.
Cecha ta jest uwzględniona w większości definicji kolokacji w literaturze.
Wyrażenie językowe w pełni kompozycyjne to takie, którego znaczenie może być przewidziane jedynie na podstawie sumy znaczeń jego elementów składowych \cite[str. 151, 184]{mit}.
\par
Jednostki wielowyrazowe mogą być częściowo lub całkowicie niekompozycyjne co oznacza, że ich znaczenie ulega pewnej zmianie, ma pewną wartość dodaną do niego w stosunku do sumy znaczeń składowych.
Zmiana ta może być niewielką modyfikacją sensu wyrażenia lub całkowicie zmieniać przekaz kolokacji. 
Warto wspomnieć, że większość anglojęzycznych kolokacji jest częściowo kompozycyjna \cite[str. 151]{mit}.
\par
Przykład niewielkiej zmiany sensu wyrażenia zobrazować można na podstawie dwóch wyrażeń: \emph{białe włosy} i \emph{białe wino}.
Obie frazy zawierają w sobie słowo \emph{białe} określający kolor pewnego obiektu - rzeczownika. 
Przymiotnik określa pewien odcień bieli, ale jednak nieco inny dla każdego z wyrażeń. 
W przypadku białych włosów kolor powinien być lekko szarawy, srebrzysty, a w przypadku wina: przezroczysty, o lekkim, żółtym zabarwieniu.
\par
Dodatkowo za przykład niewielkiej wartości dodanej do znaczenia kolokacji można podać wyrażenie \emph{czerwona kartka}. 
Faktycznie obiekt ten jest czerwonym kartonikiem, ale symbolizuje on coś jeszcze - wielokrotne lub poważne przewinienie zawodnika piłki nożnej podczas meczu.
\par
Przykładem całkowitego oderwania znaczenia wyrażenia od sumy znaczeń jego składowych są idiomy takie jak na przykład \emph{nawarzyć piwa} czy \emph{wyjść jak Zabłocki na mydle}.
Pierwsze z nich mówi o zaistnieniu pewnej sytuacji z powodu czynów danej osoby, której jednocześnie ta zazwyczaj nieprzyjemna sytuacja dotyczy.
Znaczenie tego zwrotu w zasadzie nie ma nic wspólnego z sumą znaczeń jego elementów składowych, zwłaszcza że wspomniany trunek jest zdecydowanie przez wiele osób ceniony i chętnie spożywany.
\par
Idiomy jasno obrazują, że są kolokacje, których znaczenie nie może być nawet przybliżone na podstawie sumy znaczeń składowych tego wyrażenia, ponieważ są ono zupełnie inne, niż znaczenie tego zwrotu rozpatrywanego jako całość.
W związku z powyższym w celu poprawnego określenia znaczenia jednostek wielowyrazowych należy rozpatrywać je jako całość, a nie skupiać się jedynie na każdej z ich składowych oddzielnie.


\subsection{Niezmienność szyku}
Szyk elementów składowych wyrażenia wielowyrazowego może być zmienny.
Tego typu kolokacje są trudniejsze do wykrycia niż te o stałym szyku, co wymaga innego podejścia przy ich wyszukiwaniu.
\par
Jeśli rozważymy pojęcia \emph{Unia Europejska} czy \emph{ptasie mleczko} to zmiana szyku poprzez zamianę miejscami składowych kolokacji w obrębie tych wyrażeń wielowyrazowych na odpowiednio \emph{Europejska Unia} oraz \emph{mleczko ptasie} sprawi, że ich znaczenie ulegnie zmianie.
Jeśli przyjęta definicja kolokacji uwzględnia tę cechę, wtedy według niej dwa powyższe zwroty powinny zostać uznane za kolokację.


\subsection{Nieciągłość}
Kolejną cechą niektórych jednostek wielowyrazowych jest ich nieciągłość wyrazowa. 
Wyrażenie nieciągłe składa się z określonych elementów tworzących kolokację, ale jednocześnie pomiędzy nimi znajdują się wyrazy, które do niej nie należą.
\par
Przykładem wyrażenia nieciągłego może być \emph{druga, straszna wojna światowa}. 
Kolokacją jest tutaj termin \emph{druga wojna światowa}, a przymiotnik \emph{straszna}, mimo że znajduje się pomiędzy jej elementami, nie powinien być uznany za składową tego wyrażenia wielowyrazowego.
\par
Trudniejszy można podać w analogii do problemu poruszonego w artykule \cite[str. 1]{fdpn}, przykład dotyczący byłej premier Wielkiej Brytanii -- \emph{Margaret Thatcher}.
Załóżmy, że rozważamy kolokację \emph{Karol Wojtyła} w tekście o byłym papieżu.
Istnieje duże prawdopodobieństwo, że słowo \emph{Wojtyła} wystąpi zaraz po wyrazie \emph{Karol} lub odwrotnie. 
Można sobie jednak wyobrazić sytuacje, w której pierwszy człon tej jednostki wielowyrazowej wystąpi w jednym zdaniu, a drugi w innej jego części lub nawet dopiero w zdaniu kolejnym. 
Oba słowa tej kolokacji zachowują się wtedy niczym synonimy określające tę samą osobę, ale nie występują w swoim bezpośrednim otoczeniu.
\par
Powyższe przykłady obrazują, że elementy składowe kolokacji nie muszą występować bezpośrednio po sobie, aby rozważany zwrot można było uznać za jednostkę wielowyrazową.


\subsection{Niezastępowalność składniowa}
Istotną cechą kolokacji jest niezastępowalność jej elementów składowych synonimami wyrazów, które ją stanowią, z zachowaniem znaczenia pierwotnego wyrażenia wielowyrazowego \cite[str. 184]{mit}.
Dla przykładu nie można zamienić zwrotu \emph{białe wino} na \emph{żółte wino}, mimo że drugie wyrażenie opisuje obiekt rodzaj trunku równie dobrze co pierwsze, a może i nawet trafniej.
Jednak drugi z tych zwrotów nie powinien zostać uznany za kolokację \cite[str. 184]{mit}. 
Podobnie ma się przykład z kolokacją \emph{mocna herbata} - zamiana przymiotnika \emph{mocna} na \emph{potężna} lub nawet na \emph{silna} zmieni znaczenie tego wyrażenia lub nawet je całkowicie zdeformuje.
Tym samym fraza ta może zostać uznana za wyrażenie wielowyrazowe w kontekście tej cechy.


\subsection{Niemodyfikowalność}
Kolokacja posiadająca tę cechę nie może być dowolnie rozszerzana lub modyfikowana poprzez zmianę szyku albo adaptację dodatkowych słów.
Ważne dla zachowania sensu wyrażenia wielowyrazowego jest także pozostawienie liczby, w której odmieniony jest rzeczownik \cite[str 184]{mit}.
Innymi słowy wyrażenia wielowyrazowe posiadające tę cechę muszą pozostać w jednej, określonej formie.
\par
Przykładowo idiom \emph{kopnąć w kalendarz} jest kolokacją, ale jego rozszerzenie do postaci \emph{kopnąć butem w kalendarz}, \emph{kopnąć mocno w kalendarz} lub \emph{kopnąć w wielki kalendarz} sprawi, że przekaz jaki niesie ze sobą to wyrażenie mocno się zmieni.


\subsection{Przynależność domenowa}
Wiele kolokacji jest mocno związanych z wiedzą i tematyką dziedzinową.
Żargon pewnej grupy osób, np. inżynierów jest często dla nich hermetyczny i zawiera wiele technicznych pojęć, przez co bywa niezrozumiały dla laików w danej dziedzinie wiedzy.
Ponadto słowa znane osobom niewtajemniczonym bywają używane w inny sposób, do określenia odmiennych obiektów czy zjawisk wewnątrz określonej grupy \cite[str. 4]{smadja_xtract}. 
\par
Przykładem zwrotu związanego z informatyczną wiedzą dziedzinową może być określenie \emph{twardy reset} (reset polegający na odłączeniu zasilania komputera w sposób mechaniczny) lub inne wyrażenie, które stało się już dość powszechne i przeniknęło do języka codziennego -- \emph{program się powiesił} (czyli przestał odpowiadać na akcje użytkownika lub obliczenia wykonywane przez program utknęły w martwym punkcie).

	

\section{Przyjęta definicja kolokacji}
Na potrzeby niniejszej pracy przyjęto następującą definicję kolokacji zamieszczoną poniżej:
\begin{center}
Za wyrażenie wielowyrazowe uznawane są wieloelementowe terminy specjalistyczne oraz niekompozycyjne terminy ogólne.
Mogą być one zarówno ciągłe, w szyku przemiennym, jak i ustalonym.
\end{center}

Motywacją do jej wyboru jest fakt, że definicja ta jest zbliżona do zestawu wytycznych stosowanych przez lingwistów pracujących w projekcie budowy słownika wyrażeń wielowyrazowych, z którego pozyskany został zbiór testowy poprawnych kolokacji, użyty przez autora niniejszej pracy.