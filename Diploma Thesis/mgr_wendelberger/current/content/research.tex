
\chapter{Badania}
Badania na potrzeby niniejszej pracy obejmują sprawdzenie jakości miar asocjacyjnych w procesie wydobywania wyrażeń wielowyrazowych z dużych korpusów języka polskiego.
Do badanych miar zaliczają się prawie wszystkie funkcje asocjacyjne zaimplementowane w oprogramowaniu \emph{MWeXtractor}, kombinacja liniowa rankingów wygenerowanych za pomocą tych funkcji, a także klasyfikator będący perceptronem wielowarstwowym.
Metodologia, proces badań, wyniki, ich omówienie oraz wnioski zostały zawarte w dalszej części tego rozdziału.

\section{Opis wykorzystanych zbiorów danych}
Podczas badań wykorzystane w różnych celach zostały dwa zbiory danych, które zostały dokładniej omówione w dalszej części tego rozdziału.
Pozyskanie wszystkich korpusów tekstowych przebiegło bardzo szybko ze względu na współpracę z Grupą Naukową G4.19, która udostępniła je na potrzeby badań prowadzonych przez autora niniejszej pracy.

\subsection{Korpus KIPI}
Właściwa nazwa korpusu to \emph{Korpus IPI PAN}\cite{korpus_ipi_pan_publikacja}, a został on stworzony przez Zespół Inżynierii Lingwistycznej w Instytucie Podstaw Informatyki Polskiej Akademii Nauk.
Według autorów korpus posiada 250 milionów segmentów anotowanych morfosyntaktycznie.
Korpus jest dostępny publicznie i składowany w formacie \emph{Poliqarp}.
Te i więcej informacji można pozyskać także ze strony internetowej autorów korpusu \cite{korpus_ipi_pan}.

\subsection{Korpus KGR7}
Korpus utworzony przez Grupę Naukową G4.19 składowany także w plikach o formacie \emph{Poliqarp}, otagowany tagsetem \emph{KIPI}.
Jedną z jego części jest korpus \emph{KIPI}.
Korpus \emph{KGR7} jest około siedmiokrotnie większy niż wspomniany wcześniej \emph{Korpus IPI PAN} i został oparty w dużej mierze na tekstach pobranych z Internetu\footnote{Informacja od Dr Macieja Piaseckiego.}.
Dodatkowo zawiera on także korpus Polskiej Wikipedii oraz Korpus Rzeczpospolitej\footnote{Informacja od Dr Macieja Piaseckiego.} \cite{korpus_rzeczpospolitej}.
\par
Tabela \ref{kgr7_stats} zawiera statystyki podkorpusów składowych korpus \emph{KGR7}.
\begin{table}[h!]
\centering
\begin{tabular}{l | r}
	\toprule
	\textbf{nazwa korpusu} & \textbf{liczba tokenów} \\
	\midrule
	1002 & 19 512 317 \\
	1003 & 10 006 539 \\
	blogi & 9 613 618 \\
	interia & 611 402 \\
	kipi & 255 516 328 \\
	knigi\_joined & 1 010 676 150 \\
	naukawe & 2 594 225 \\
	ornitologia & 544 937 \\
	plwiki20120428 & 275 578 635 \\
	pogoda & 593 538 \\
	poig\_biznes\_data\_sub\_0 & 35 439 099 \\
	poig\_biznes\_data\_sub\_1 & 30 676 362 \\
	polityka & 82 480 654 \\
	prace & 12 665 419 \\
	pryzmat & 2 183 403 \\
	rzepa & 116 317 357 \\
	sjp & 2 177 299 \\
	wordpress & 439 304 \\
	zwiazki & 820 991 \\
	\hline
	suma & 1 868 447 577 \\
	\bottomrule
\end{tabular}
\caption[Podkorpusy i statystyki korpusu \emph{KGR7}]{Podkorpusy i statystyki dotyczące korpusu \emph{KGR7}}
\label{kgr7_stats}
\end{table}

Ze względu na ograniczony czas, autorowi pracy nie udało się jednak wykorzystać niniejszego zbioru danych -- jednak dał on pewne pojęcia o mniejszych badaniach prowadzonych na inne potrzeby niż niniejsza praca oraz w trakcie tworzenia oprogramowania \emph{MWeXtractor}.

\subsection{Słowosieć i praca lingwistów}
Wspomniany wcześniej \emph{polski Wordnet} -- \emph{Słowosieć}, został w procesie badań wykorzystany jako baza wiedzy, z której pozyskano wyrażenia wielowyrazowe uznane za poprawne.
Dodatkowo w trakcie prac opisanych w niniejszym dokumencie grupa lingwistów na bieżąco oceniała kolejne zestawy kandydatów na jednostki wielowyrazowe.
Oba zbiory kolokacji zostały ze sobą połączone w jeden, który następnie wykorzystywany był jako dane niezbędne do oceny wyników generowanych przez rankery, a także do generacji cech dla sprawdzonych klasyfikatorów.

\section{Przygotowanie danych}
Tak jak wspomniano wcześniej, wszystkie dane zostały udostępnione przez Grupę Naukową G4.19.
Otrzymane korpusy były składowane w formacie \emph{Poliqarp} i opisane za pomocą tagsetu \emph{KIPI}.
Korpusy \emph{KGR7} oraz \emph{KIPI} zostały przygotowane do badań w taki sam sposób, jak opisany w dalszej części tej sekcji.
Dodać jednak trzeba, że z korpusu \emph{KGR7} wyłączony został podkorpus \emph{sjp}.
Motywacją do tego były problemy związane z błędami w pliku z danymi oraz tym, że jest to zestaw definicji słownikowych, a nie spójny tekst traktujący na konkretne tematy.

\subsection{Tagowanie}
Autor niniejszej pracy postanowił dokonać rozłożenia danych otagowanych do nieotagowanego tekstu ciągłego, a następnie ponownego otagowania tych danych.
Do wykonania tego zadania użyte zostało narzędzie, tager \emph{WCRFT2}\footnote{Omówione w rozdziale 4.2.}.
Tekst został otagowany tagsetem \emph{NKJP}, a tekst już przetworzony jest składowany w formacie \emph{IOB-chan}.
Motywacje do tego działa były dwie.
Pierwsza to chęć otagowania danych za pomocą znanego tagera i modelu wyuczonego przez grupę G4.19, a nie korzystanie z już gotowych danych otagowanych. 
Drugą motywacją były pewne problemy związane z czytaniem korpusów w formacie \emph{Poliqarp}.
Jako format zapisu na nowo otagowanego tekstu wybrano wspomniany \emph{IOB-chan} ze względu na małą objętość pamięciową i szybkość jego wczytywania, a także ze względu na jego przejrzystość.

\subsection{Ekstrakcja kandydatów na kolokacje}
Gotowe, otagowane korpusy należało wczytywać i wydobyć z nich potrzebne informacje o kandydatach na kolokacje, takie jak składowe krotek i ich częstości czy liczność konkretnych słów.
Częstości kandydatów na kolokacje składowane są z podziałem na konkretne podkorpusy, co umożliwia zastosowanie funkcji dyspersji w przypadku korpusów składających się z podkorpusów.
Wszystkie zebrane informacje zapisane są w omówionym uprzednio \emph{składzie krotek}.
Do wydobywania kandydatów na kolokacje użyto omawianego wcześniej języka ograniczeń \emph{WCCL}.
Skorzystano z 80 relacji w celu ekstrakcji 2-elementowych kandydatów na kolokacje oraz 12 w przypadku ekstrakcji relacji 3-elementowych.
Większość wykorzystanych operatorów sprawdzała części mowy wyrazów składowych kandydata na kolokacje, ale część także pewne ich cechy i powiązania, takie jak na przykład uzgodnienie rzeczownika z przymiotnikiem.
Część użytych wyrażeń \emph{WCCL} miała na celu zebranie wszystkich kandydatów na kolokacje, których można było utworzyć w oknie o danej długości.
Przy opisach konkretnych badań zostały zamieszczone spisz użytych relacji.

\section{Opis planowanego przebiegu badań}
Zaplanowany proces badania metod ekstrakcji został podzielony na kilka kroków.
Wszystkie zostały opisane w dalszej części tego rozdziału, a wykonane zostały z wykorzystaniem narzędzi z pakietu \emph{MWeXtractor}.
Skrótowy opis przebiegu badania został zawarty w tabeli \ref{research_plan}.

\begin{table}[h!]
\centering
\begin{tabular}{l | p{0.4\linewidth} | p{0.5\linewidth}}
	\toprule
	\textbf{nr.}	& \textbf{zadanie}	& \textbf{cel i motywacje}	\\
	\midrule
	1	& zbadanie dwuelementowych miar asocjacyjnych na korpusie \emph{KIPI} & zdobycie informacji pozwalających na ocenę jakości miar w zadaniu ekstrakcji kolokacji dwuelementowych; zdobycie wiedzy pozwalającej na ocenę, które funkcje asocjacyjne powinny zostać wybrane jako składowe generatora cech dla klasyfikatorów; punkt wyjścia do dalszych badań \\
	\hline
	2	& zbadanie dwuelementowych miar asocjacyjnych na korpusie \emph{KIPI} poddanym dyspersji za pomocą miary TF-IDF & sprawdzenie wpływu zastosowania miary dyspersji na zmianę jakości wyników \\
	\hline
	3	& zbadanie dwuelementowych miar asocjacyjnych na korpusie \emph{KIPI} po wykonaniu podpróbkowania klasy negatywnej & korpus zawiera w znacznej przewadze krotki niebędące kolokacjami; stężenie wyrażeń wielowyrazowych wśród kandydatów na kolokacje jest mniejszy niż pół procenta -- dokładne statystyki zostały zawarte w dalszej części tej pracy; ze względu na to autor pracy postanowił sprawdzić jakość miar po wykonaniu podpróbkowania klasy negatywnej; zostało ono wykonane dla miar, aby umożliwić ewentualne porównanie rezultatów z wynikami w artykule Pavla Peciny \cite{coling} \\
	\hline
	4	& zbadanie wariantów perceptronu wielowarstwowego dla kolokacji dwuelementowych & sprawdzenie jakości rozwiązań generowanych przez klasyfikatory; przetestowanie wpływu liczby neuronów i warstw ukrytych na jakość rozwiązań dostarczanych przez ten klasyfikator \\
	\hline
	5	& zbadanie wariantów perceptronu wielowarstwowego dla kolokacji dwuelementowych po wykonaniu podpróbkowania klasy negatywnej & sprawdzenie jakości sieci neuronowej po zrównoważeniu liczby instancji z obu klas \\
	\hline
	6	& zbadanie wariantów perceptronu wielowarstwowego dla kolokacji dwuelementowych po wykonaniu podpróbkowania klasy negatywnej i zmianie zestawu cech & sprawdzenie jakości sieci neuronowej po zrównoważeniu liczby instancji z obu klas oraz po dodaniu jednej cechy -- częstości -- do zestawu cech stosowanego w poprzednich badaniach perceptronu wielowarstwowego; motywacją do tego było osiąganie przez tę funkcję dobrych wyników w niektórych przypadkach, ale jednoczesne jej pominięcie w poprzednim zestawie cech \\
	\hline
	7	& zbadanie miar trójelementowych & sprawdzenie jakości generalizacji miar zaproponowanych w literaturze oraz przez autora niniejszej pracy; \\
	\hline
	8	& zbadanie sieci neuronowej dla kolokacji trójelementowych & sprawdzenie jakości rozwiązań generowanych przez sieci neuronowe w zadaniu ekstrakcji wyrażeń trójelementowych; \\
	\bottomrule
\end{tabular}
\caption[Planowany przebieg badań]{Planowany przebieg badań}
\label{research_plan}
\end{table}


\section{Miary dwuelementowe, korpus KIPI}
Celem tego badania było zdobycie informacji o jakości miar dwuelementowych wykorzystanych do ekstrakcji kolokacji z korpusu \emph{KIPI}.
Na podstawie wyników ocenić można było, które funkcje dwuelementowe generują najlepsze wyniki, co jest jednocześnie wstępem do dalszych badań opisanych w tym rozdziale -- informacja ta jest pomocna w doborze miary dla specjalnych funkcji N-elementowych, bazujących na rozbijaniu krotek N-elementowych na 2-elementowe.
Dodatkowo wiedza ta posłuży do doboru miar będących generatorami cech dla badanych klasyfikatorów -- sieci neuronowych.

\subsection{Przygotowanie i zbadanie podkorpusów KIPI}
Pierwszym etapem tego badania było sprawdzenie podzbiorów danych korpusu \emph{KIPI}, czyli zebranie i zestawienie statystyk o nich w celu ich lepszego poznania oraz oceny zastosowanych relacji.
Dane zostały pozyskane na sześć różnych sposobów -- sposoby różniły się zestawem wykorzystanych operatorów \emph{WCCL}.
Możliwym było wydobycie ich tylko w jeden sposób ze względu na możliwość wykorzystania funkcjonalności pakietu \emph{MWeXtractor}, takich jak między innymi filtrowanie.
Jednak podejście z wieloma osobnymi zbiorami danych może przyspieszyć kolejne etapy badań ze względu na szybkość wczytywania plików, a także zmniejszyć ilość pamięci RAM potrzebnej do obliczeń.
Wydobyte informacje zostały zapisane w strukturach współpracujących ze wspomnianym oprogramowaniem -- składach krotek.
Gotowe składy krotek poddano analizie z wykorzystaniem programu \emph{Cover}, który generuje dwa zestawy informacji opisane we wcześniejszej części tej pracy\footnote{Opisany w rozdziale 4.6.4.}.
Dla przypomnienia, pierwsza z nich to macierz liczb z etykietami na wierszach i kolumnach zawierająca dane o tym, w jakim stopniu relacje zachodzą na siebie -- ile krotek zostało przyporządkowanych do konkretnych relacji.
Drugi wynik to liczba wyrażeń wielowyrazowych odnalezionych w każdej z relacji z osobna.
Dane te posłużyły do oceny wykorzystanych operatorów i wyboru wzorców \emph{częstych}, czyli takich wyrażeń \emph{WCCL}, które miałyby odnajdować w tekście zbiory kandydatów, wśród których byłoby stosunkowo dużo poprawnych wyrażeń wielowyrazowych.
Do oceny wyrażeń pod kątem tego, czy są \emph{częste}, wykorzystano informacje o liczbie jednostek wielowyrazowych należących do danego wzorca, jak i o stosunku tej liczby do liczby wszystkich krotek należących do rozpatrywanej relacji.
Wzorzec został uznany za \emph{częsty} jeśli stosunek liczby jednostek wielowyrazowych w nim zawartych do wszystkich krotek należących do tego wzorca był większy od jednego procenta.
W dalszej części tej pracy podane zostaną szczegóły dotyczące przygotowanych i sprawdzonych zbiorów danych.


\subsubsection{Podzbiór \protect\textit{2W}}
Statystki dotyczące podzbioru \emph{2W} zostały zamieszczone w tabeli \ref{KIPI_2W_stats}.

\begin{table}[h!]
\centering
\begin{tabular}{ l | r | r | r | l }
	\toprule
	\textbf{relacje} 	& \textbf{liczba krotek} & \textbf{liczba JW} & \textbf{procent JW} & \textbf{częsta?} 	\\
	\midrule
	Window2P0	&	19752289	&	41221	&	0,20869	&	nie	\\
	Window2P1	&	19752289	&	25560	&	0,12940	&	nie	\\
	\midrule									
	Suma:	&	39504578	&	66781	&	0,16904	&		\\
	\bottomrule
\end{tabular}
\caption[Statystyki podzbioru danych \emph{KIPI} 2W]{Statystyki dotyczące podzbioru danych 2W pozyskanego z korpusu \emph{KIPI}.}
\label{KIPI_2W_stats}
\end{table}

Ważną obserwacją jest udział procentowy jednostek wielowyrazowych wśród wszystkich kandydatów na kolokacje -- tylko niecałe 0,17\%\footnote{Zaznaczyć trzeba jednak, że zbiór wzorcowy był bardzo niepełny.}.
Tak niska ich zawartość może obrazować poziom trudności zadania ekstrakcji kolokacji z korpusów tekstowych.


\subsubsection{Podzbiór \protect\textit{2R}}
Statystki dotyczące podzbioru \emph{2R} zostały zamieszczone w tabeli \ref{KIPI_2R_stats}.

\begin{table}[h!]
\centering
\footnotesize\setlength{\tabcolsep}{2.5pt}
\begin{tabular}{ l | r | r | r | l }
	\toprule
	\textbf{relacje} 	& \textbf{liczba krotek} & \textbf{liczba JW} & \textbf{procent JW} & \textbf{częsta?} 	\\
	\midrule
	AdjGenCosP0	&	128	&	0	&	0,00000	&	nie	\\
	AdjGenCosP1	&	840	&	0	&	0,00000	&	nie	\\
	AgrAdjSubstP0	&	1336463	&	1230	&	0,09203	&	nie	\\
	AgrAdjSubstP1	&	706331	&	645	&	0,09132	&	nie	\\
	AgrSubstAdjP0	&	706331	&	18162	&	2,57132	&	$ TAK $	\\
	AgrSubstAdjP1	&	1336463	&	11774	&	0,88098	&	$ TAK $	\\
	AllAdvPartP0	&	122560	&	6	&	0,00490	&	nie	\\
	AllAdvPartP1	&	74187	&	1	&	0,00135	&	nie	\\
	AllBurkSubstP0	&	2544	&	30	&	1,17925	&	$ TAK $	\\
	AllBurkSubstP1	&	7126	&	2	&	0,02807	&	nie	\\
	AllGerQubP0	&	11518	&	2012	&	17,46831	&	$ TAK $	\\
	AllGerQubP1	&	20772	&	1016	&	4,89120	&	$ TAK $	\\
	AllNumSubstP0	&	69889	&	8	&	0,01145	&	nie	\\
	AllNumSubstP1	&	50929	&	4	&	0,00785	&	nie	\\
	AllPartAdvP0	&	74187	&	1171	&	1,57844	&	$ TAK $	\\
	AllPartAdvP1	&	122560	&	1064	&	0,86815	&	$ TAK $	\\
	AllQubGerP0	&	20772	&	1	&	0,00481	&	nie	\\
	AllQubGerP1	&	11518	&	1	&	0,00868	&	nie	\\
	AllSiebieSubstP0	&	8149	&	0	&	0,00000	&	nie	\\
	AllSiebieSubstP1	&	2017	&	0	&	0,00000	&	nie	\\
	AllSubstBurkP0	&	7126	&	9	&	0,12630	&	nie	\\
	AllSubstBurkP1	&	2544	&	0	&	0,00000	&	nie	\\
	AllSubstNumP0	&	50929	&	3	&	0,00589	&	nie	\\
	AllSubstNumP1	&	69889	&	1	&	0,00143	&	nie	\\
	AllSubstSiebieP0	&	2017	&	12	&	0,59494	&	nie	\\
	AllSubstSiebieP1	&	8149	&	3	&	0,03681	&	nie	\\
	AllSubstSubstP0	&	4124657	&	3847	&	0,09327	&	nie	\\
	AllSubstSubstP1	&	4124657	&	730	&	0,01770	&	nie	\\
	CosAdjGenP0	&	840	&	47	&	5,59524	&	$ TAK $	\\
	CosAdjGenP1	&	128	&	5	&	3,90625	&	$ TAK $	\\
	GndAdjSubstP0	&	42964	&	67	&	0,15594	&	nie	\\
	GndAdjSubstP1	&	82688	&	63	&	0,07619	&	nie	\\
	GndSubstAdjP0	&	82676	&	3234	&	3,91166	&	$ TAK $	\\
	GndSubstAdjP1	&	42920	&	779	&	1,81500	&	$ TAK $	\\
	Ppron3GenSubstP0	&	18323	&	6	&	0,03275	&	nie	\\
	Ppron3GenSubstP1	&	10350	&	5	&	0,04831	&	nie	\\
	SubstPpron3GenP0	&	10350	&	5	&	0,04831	&	nie	\\
	SubstPpron3GenP1	&	18323	&	10	&	0,05458	&	nie	\\
	\midrule									
	Suma	&	13384814	&	45953	&	0,34332	&		\\
	Suma częstych	&	2400939	&	39294	&	1,63661	&		\\
	\bottomrule
\end{tabular}
\caption[Statystyki podzbioru danych \emph{KIPI} 2R]{Statystyki dotyczące podzbioru danych 2R pozyskanego z korpusu \emph{KIPI}.}
\label{KIPI_2R_stats}
\end{table}


Autor pracy postanowił wykluczyć relacje \emph{AllSubstSubstP0} i \emph{AllSubstSubstP1} z grona \emph{częstych} ze względu na niski procent jednostek wielowyrazowych w gronie kandydatów przez nie wybranych.
Wprawdzie relacje odnalazły w sumie $ 4577 $ poprawnych wyrażeń wielowyrazowych, co stanowi znaczącą liczbę (niemal $ 10 \% $ wszystkich poprawnych kolokacji), ale zauważyć trzeba, że relacje te, będąc dwiema z 38, wysunęły prawie $ 62\% $ wszystkich kandydatów na kolokacje.
Podsumowując, dwa wspomniane operatory wygenerowały znaczącą liczbę jednostek wielowyrazowych, ale jednocześnie także nieporównywalnie większą ilość szumu w postaci błędnych kandydatów.
Możliwe jest, że metodom wydobywającym udałoby się odnaleźć część z poprawnych wyrażeń wielowyrazowych należących do omawianych relacji, ale jednak procent jednostek wielowyrazowych w niej wydaje się na tyle mały, że ich wydobywanie mogłoby znacząco pogorszyć ogólny wynik pod kątem precyzji.

\par
Wśród wszystkich krotek pozyskanych za pomocą przedstawionych w \emph{2R} operatorów \emph{WCCL} znajduje się około $ 68,8\% $ wszystkich jednostek wielowyrazowych wykrytych w tekście za pomocą operatora okna.
Wynik ten otrzymano przy zmniejszeniu grona kandydatów na kolokacje do $ 33,9\% $.
Natomiast procent wyrażający stosunek jednostek wielowyrazowych do wszystkich kandydatów wzrósł z niecałych $ 0,17\% $ do ponad $ 0,34\% $.
Ponadto w przypadku wzięcia pod uwagę tylko relacji częstych, wartości te będą wynosić odpowiednio ponad $ 58,8\% $, prawie $ 6,1\% $ oraz niecałe $ 1,64 $.

\par
Podsumowując powyższe obserwacje zauważyć można, że poprzez zastosowanie filtrów opartych o części mowy dla korpusu \emph{KIPI} i zestawu jednostek wielowyrazowych pozyskanych ze Słowosieci, da się zachować około $ 58,8\% $ jednostek wielowyrazowych przy zmniejszeniu liczby kandydatów do jedynie niecałych $ 6,1\% $.
Skutkuje to także około dziesięciokrotnym zwiększeniem procentu jednostek wielowyrazowych wśród kandydatów -- efektem tego może być znaczny wzrost dokładności systemu wykrywającego kolokacje.

\par
Zauważyć można także kilkukrotne różnice w procencie jednostek wielowyrazowych w całym zestawie kandydatów z danej relacji w zależności od wybranego szyku; przykładowo w przypadku relacji \emph{AllSubstSubstP0} i \emph{AllSubstSubstP1} różnica ta jest około czterokrotna, a w przypadku \emph{CosAdjGenP0} i \emph{CosAdjGenP1} ponad dziewięciokrotna.


\subsubsection{Podzbiór \protect\textit{2RW}}
Omawiany tutaj zbiór powstał poprzez połączenie dwóch zestawów relacji, jednego wykorzystanego do utworzenia podzbioru \emph{2W} oraz drugiego przygotowanego na potrzeby generacji podzbioru \emph{2R}.
Statystyki niniejszego podkorpusu są takie same jak korpusów \emph{2R} i \emph{2W}.
Zmianie uległa jedynie statystyka dotycząca sumy.


\subsubsection{Podzbiór \protect\textit{2W1H}}
Zbiór został rozszerzony o operatory akceptujące wszystkie pary wyrazów oddzielone dowolnym słowem znajdującym się pomiędzy nimi.
Operatory te generują wszystkie możliwe do utworzenia pary wyrazów, tworząc zestaw kandydatów na kolokacje nieciągłe.
Nazwy nowych operatorów zostały zmodyfikowane poprzez dodanie do ich nazw fragmentu \emph{H1} reprezentującego nieciągłość wielkości pojedynczego wyrazu.
Tabela \ref{KIPI_2R1H_stats} prezentuje statystyki wygenerowane przy wykorzystaniu operatorów oknowych -- zarówno ciągłych jak i nieciągłych.

\begin{table}[h!]
\centering
\begin{tabular}{ l | r | r | r | l }
	\toprule
	\textbf{relacje} 	& \textbf{liczba krotek} & \textbf{liczba JW} & \textbf{procent JW} & \textbf{częsta?} 	\\
	\midrule
	Window2P0	&	19752289	&	41221	&	0,20869	&	nie	\\
	Window2P1	&	19752289	&	25560	&	0,12940	&	nie	\\
	Window2H1P0	&	29740688	&	20176	&	0,06784	&	nie	\\
	Window2H1P1	&	29740688	&	24998	&	0,08405	&	nie	\\
	\midrule									
	Suma nieciągłych&	59481376	&	45174	&	0,0759465	&	\\
	Suma wszystkich	&	98985954	&	111955	&	0,11310	&		\\
	\bottomrule
\end{tabular}
\caption[Statystyki podzbioru danych \emph{KIPI} 2W1H]{Statystyki dotyczące podzbioru danych 2W1H, pozyskanego z korpusu \emph{KIPI}.}
\label{KIPI_2W1H_stats}
\end{table}

\par
Zastanawiające wydaje się to, że kolokacji nieciągłych jest więcej niż ciągłych, mimo że dla zdania N-elementowego zawsze można wygenerować $ N - 1 $ bi-gramów oraz $ N - 2 $ tri-gramów.
Zadać sobie można pytanie dlaczego zatem tri-gramów jest więcej?
Odpowiedź jest następująca: jest ich w rzeczywistości mniej.
Statystyki pokazują, ilu różnych kandydatów udało się utworzyć za pomocą danych relacji, ale trzeba pamiętać, że każdy z kandydatów mógł wystąpić wielokrotnie.
Omawiany tutaj zbiór zawiera w sumie $ 98 985 954 $ różnych krotek, których suma częstości jest równa $ 942 472 698 $, gdzie tylko $ 457 972 144 $ par zostało wygenerowanych przez relacje wyszukujące kolokacje nieciągłe, a tym samym $ 484 500 554 $ przez relacje wyszukujące kandydatów ciągłych.
Kandydaci wyszukiwani przez relacje ciągłe byli mniej zróżnicowani, ale było ich więcej niż w zestawie kandydatów pozyskanych dzięki operatorom ciągłym.
Sytuacja może pojawiać się także przy innych relacjach niż oknowe, ale powód takiego stanu rzeczy może być analogiczny, a dodatkowo pojawia się także inne wyjaśnienie -- po prostu wyrazy tak się ułożyły, że konkretne pary częściej występowały oddzielone jakimś słowem niż bezpośrednio obok siebie.

\par
Wyniki z tabeli \ref{KIPI_2R1H_stats} obrazują duży potencjał operatorów nieciągłych w zwiększeniu liczby możliwych do wykrycia wyrażeń wielowyrazowych, a ponadto wpływają na dane statystyczne wykorzystywane przez miary powiązania i klasyfikatory podczas ich pracy.


\subsubsection{Podzbiór \protect\textit{2R1H}}
Niniejszy podzbiór jest nadzbiorem \emph{2R}.
Zawarte w nim są te same relacje co w zbiorze \emph{2R}, ale rozszerzone o akceptowanie także nieciągłych kandydatów na kolokacje, w których odległość pomiędzy wyrazami składowymi krotki była równa dwa (jeden dowolny wyraz pomiędzy składowymi).
Liczba relacji zwiększyła się dwukrotnie (do 76), a dodatkowo, dzięki zastosowaniu omawianego mechanizmu, informacja statystyczna uległa zmianie.
Szczegółowe informacje na temat tego podzbioru zawarte zostały w tabeli \ref{KIPI_2R1H_stats}.
Zauważyć jednak należy, że podano w niej tylko informacje, które pozyskano poprzez zastosowanie nowych relacji -- nieciągłych, ponieważ dane wydobyte za pomocą relacji ciągłych są takie same jak w przypadku podzbioru \emph{2R}.

\begin{table}[h!]
\centering
\footnotesize\setlength{\tabcolsep}{2.5pt}
\begin{tabular}{ l | r | r | r | l }
	\toprule
	\textbf{relacje} 	& \textbf{liczba krotek} & \textbf{liczba JW} & \textbf{procent JW} & \textbf{częsta?} 	\\
	\midrule
	AdjGenCosH1P0	&	405	&	0	&	0	&	nie	\\
	AdjGenCosH1P1	&	564	&	0	&	0	&	nie	\\
	AgrAdjSubstH1P0	&	441381	&	470	&	0,1064839674	&	nie	\\
	AgrAdjSubstH1P1	&	435442	&	410	&	0,0941572012	&	nie	\\
	AgrSubstAdjH1P0	&	435442	&	5097	&	1,170534767	&	$ TAK $	\\
	AgrSubstAdjH1P1	&	441381	&	6691	&	1,5159238844	&	$ TAK $	\\
	AllAdvPartH1P0	&	92077	&	3	&	0,0032581426	&	nie	\\
	AllAdvPartH1P1	&	70914	&	3	&	0,0042304764	&	nie	\\
	AllBurkSubstH1P0	&	7411	&	6	&	0,080960734	&	nie	\\
	AllBurkSubstH1P1	&	8195	&	6	&	0,0732153752	&	nie	\\
	AllGerQubH1P0	&	14164	&	614	&	4,3349336346	&	$ TAK $	\\
	AllGerQubH1P1	&	30860	&	1491	&	4,8314970836	&	$ TAK $	\\
	AllNumSubstH1P0	&	79359	&	6	&	0,0075605791	&	nie	\\
	AllNumSubstH1P1	&	105045	&	5	&	0,0047598648	&	nie	\\
	AllPartAdvH1P0	&	70914	&	604	&	0,8517359055	&	$ TAK $	\\
	AllPartAdvH1P1	&	92077	&	1555	&	1,6888039358	&	$ TAK $	\\
	AllQubGerH1P0	&	30860	&	1	&	0,0032404407	&	nie	\\
	AllQubGerH1P1	&	14164	&	1	&	0,0070601525	&	nie	\\
	AllSiebieSubstH1P0	&	10628	&	0	&	0	&	nie	\\
	AllSiebieSubstH1P1	&	7743	&	0	&	0	&	nie	\\
	AllSubstBurkH1P0	&	8195	&	2	&	0,0244051251	&	nie	\\
	AllSubstBurkH1P1	&	7411	&	2	&	0,0269869113	&	nie	\\
	AllSubstNumH1P0	&	105045	&	2	&	0,0019039459	&	nie	\\
	AllSubstNumH1P1	&	79359	&	3	&	0,0037802896	&	nie	\\
	AllSubstSiebieH1P0	&	7743	&	6	&	0,0774893452	&	nie	\\
	AllSubstSiebieH1P1	&	10628	&	7	&	0,0658637561	&	nie	\\
	AllSubstSubstH1P0	&	7888228	&	1790	&	0,0226920419	&	nie	\\
	AllSubstSubstH1P1	&	7888228	&	1415	&	0,0179381225	&	nie	\\
	CosAdjGenH1P0	&	564	&	46	&	8,1560283688	&	$ TAK $	\\
	CosAdjGenH1P1	&	405	&	25	&	6,1728395062	&	$ TAK $	\\
	GndAdjSubstH1P0	&	149805	&	95	&	0,0634157738	&	nie	\\
	GndAdjSubstH1P1	&	156749	&	99	&	0,0631582977	&	nie	\\
	GndSubstAdjH1P0	&	156749	&	1780	&	1,1355734327	&	$ TAK $	\\
	GndSubstAdjH1P1	&	149805	&	1931	&	1,2890090451	&	$ TAK $	\\
	Ppron3GenSubstH1P0	&	17518	&	9	&	0,0513757278	&	nie	\\
	Ppron3GenSubstH1P1	&	19023	&	6	&	0,0315407664	&	nie	\\
	SubstPpron3GenH1P0	&	19023	&	4	&	0,0210271776	&	nie	\\
	SubstPpron3GenH1P1	&	17518	&	12	&	0,0685009704	&	nie	\\
	\midrule									
	Suma ciągłych	&	13384814	&	45953	&	0,34332	&		\\
	Suma nieciągłych	&	19071022	&	24197	&	0,1268783603	&		\\
	Suma wszystkich	&	32455836	&	70150	&	0,2161398646	&		\\
	Suma ciągłych częstych	&	2400939	&	39294	&	1,63661	&		\\
	Suma nieciągłych częstych	&	1392361	&	19834	&	1,4244868967	&		\\
	Suma wszystkich częstych	&	3793300	&	59128	&	1,5587483194	&		\\
	\bottomrule
\end{tabular}
\caption[Statystyki podzbioru danych \emph{KIPI} 2R1H]{Statystyki dotyczące podzbioru danych 2R1H, pozyskanego z korpusu \emph{KIPI}.}
\label{KIPI_2R1H_stats}
\end{table}

\par
Ciekawą obserwacją jest, że relacje \emph{AgrAdjSubstH1P0} i \emph{AgrAdjSubstH1P1} generują prawie $ 48,7\% $ wszystkich nieciągłych wyrażeń wielowyrazowych na podstawie badanego tekstu, spośród wszystkich przedstawionych relacji.
Warto dodać, że zbiór relacji oznaczonych jako częste prawie nie uległ zmianie, z wyjątkiem \emph{AllBurkSubstP0}.
Nie uległ zmianie w takim rozumieniu, że jeśli relacje ciągłe A i B weszły w skład relacji częstych, to ich wersje wykrywające relacje nieciągłe (z dodatkiem \emph{H1} w nazwie) także znalazły się w tym zbiorze.

\par
Przedstawione statystyki obrazują, że ogólna jakość rozwiązań na podstawie tak zebranych danych teoretycznie powinna spaść w przypadku zastosowania wyboru wyrażeń wielowyrazowych w procesie losowania
Wniosek taki wysunąć można na podstawie kolumny \emph{procent JW} tabeli \ref{KIPI_2R1H_stats}, która obrazuje spadek udziału procentowego jednostek wielowyrazowych w stosunku do podzbioru \emph{2R}.
Zaznaczyć jednak trzeba, że także tutaj usunięcie relacji \emph{AllSubstSubstH1P0} i \emph{AllSubstSubstH1P1} powinno zaowocować znaczną poprawą wyniku, ponieważ według obliczeń zaledwie niespełna $ 0,041\% $ jednostek w niej zawartych stanowią jednostki wielowyrazowe, a do tych relacji należy ponad $ 82,7\% $ wszystkich kandydatów na kolokacje.
Nie poprawia to jednak jakości rozwiązań losowych generowanych na podstawie danych zebranych przez inne relacje w swoim obrębie.
Trzeba jednak mieć na uwadze to o czym autor niniejszej pracy napisał przy okazji omawiania zbioru \emph{2W1H} -- zebranie informacji statystycznych w ten sposób może zmienić w znacznym stopniu wyniki generowane przez funkcje asocjacyjne i klasyfikatory dzięki pozyskaniu nowych danych statystycznych z korpusu tekstowego.
Wyniki mogą zostać poprawione dlatego, że sieci neuronowe i miary powiązania są bardziej skomplikowane niż wybór losowy jednostek wielowyrazowych z grona kandydatów.

\par
Statystyki z tabeli pozwalają na wywnioskowanie, że zastosowanie filtrów opartych o dane lingwistyczne -- części mowy i inne, może jeszcze bardziej zmniejszyć grono kandydatów na wyrażenia wielowyrazowe w przypadku wyszukiwania zarówno kandydatów ciągłych jak i nieciągłych.
W przypadku wszystkich relacji zbiór kandydatów na kolokacje został zmniejszony do niecałych $ 32,8\% $, a maksymalna możliwa do osiągnięcia kompletność to niecałe $ 62,7\% $.
Jeśli natomiast rozważymy tylko relacje uznane za częste, zbiór kandydatów zostanie zmniejszony do zaledwie nieco ponad $ 3,83\% $, a maksymalna kompletność wyniesie ponad $ 52,8\% $.
Tak duże ograniczenie kandydatów na kolokacje skutkować może także kilkudziesięciokrotnym przyspieszeniem wydobywania wyrażeń wielowyrazowych.

\par
Tak duże zmiany w tych wartościach mogą być godne uwagi i dowodzą, że filtry części mowy są w stanie znacznie zmniejszyć grono kandydatów na kolokacje, a w konsekwencji tego zmienić wyniki pracy metod wykrywających wyrażenia wielowyrazowe poprzez zwiększenie ich szans na osiągnięcie większych poziomów precyzji.
Trzeba pamiętać jednak, że nie jest to idealne rozwiązanie, ponieważ kosztem jego zastosowania może być spadek kompletności wyniku końcowego.


\subsubsection{Podzbiór \protect\textit{2RW1H}}
Omawiany tutaj zbiór powstał w sposób analogiczny do \emph{2W} i \emph{2R}, ale poprzez połączenie dwóch innych zestawów relacji: jednego wykorzystanego do utworzenia podzbioru \emph{2W1H} oraz drugiego przygotowanego na potrzeby generacji podzbioru \emph{2R1H}.
Statystyki niniejszego podkorpusu są takie same jak korpusów \emph{2R1H} i \emph{2W1H}.
Zmianie uległy jedynie statystyki dotyczące sum.


\subsubsection{Wersje podkorpusów poddane dyspersji}
Oprócz sześciu opisanych w poprzedniej części tej sekcji podkorpusów korpusu \emph{KIPI}, utworzone zostały też ich odpowiedniki powstałe po podzieleniu korpusu na pewną liczbę mniej więcej równych części.
Sposób podziału korpusu \emph{KIPI} polegał na rozbiciu go na 10 ciągłych i podobnych rozmiarem części.
Rozmiar był determinowany przez liczbę tokenów, a sam korpus był dzielony z dokładnością do zdania, żeby uniknąć cięcia w jego środku.

\par
Po wykonaniu podziału zostały przygotowane odpowiedniki podkorpusów \emph{2R}, \emph{2W}, \emph{2RW}, \emph{2R1H}, \emph{2W1H}, \emph{2RW1H} poprzez wykorzystanie tych samych relacji, których użyto podczas przygotowywania zbioru \emph{KIPI} niepoddanego dyspersji.
Z racji, że żaden token czy zdanie nie zostało pominięte, a cięcia następowały z dokładnością do zdania, statystyki podkorpusów nie uległy zmianie.

\par
Przygotowane podkorpusy następnie poddano dyspersji z wykorzystaniem miary TF-IDF, ponieważ w literaturze uchodzi ona za dobrą miarę w zadaniach ekstrakcji informacji.
Specyfiką tej miary dyspersji jest to, że części kandydatów przyporządkowuje ona częstość równą zero -- tym krotkom, które wystąpiły we wszystkich podkorpusach.
Kandydaci, których częstość po wykonaniu dyspersji była równa zero, zostali usunięci z grona kandydatów.


\subsubsection{Wersje poddane podpróbkowaniu klasy negatywnej}
Ze względu na fakt, że klasa pozytywna reprezentująca wyrażenia wielowyrazowe jest słabo reprezentowana (poniżej 0,5\% wszystkich kandydatów) autor niniejszej pracy dokonał podpróbkowania klasy negatywnej w celu utworzenia zbioru zawierającego znacznie bardziej reprezentatywną liczbę instancji klasy pozytywnej\footnote{Podpróbkowanie klasy negatywnej w przypadku miar asocjacyjnych może wydawać się złym pomysłem, ponieważ nie zachodzi tutaj proces uczenia. Autor pracy postanowił jednak zbadać także jakość funkcji po wykonaniu tego zabiegu, ponieważ umożliwia to ewentualne odniesienie się i porównanie osiągniętych wyników z rezultatami prac Pavla Peciny i Pavla Schlesingera \cite{coling}.}.

\par
Podpróbkowanie polegało na wybraniu z zestawu kandydatów wszystkich wyrażeń wielowyrazowych, a następnie, z wykorzystaniem rozkładu jednorodnego, usuwaniu kolokacji z klasy negatywnej aż do momentu uzyskania zadanego udziału procentowego jednostek wielowyrazowych.
Pożądany stosunek jednostek pozytywnych do negatywnych został ustalony na poziomie przynajmniej $ 20\% $.
Motywacją do wyboru takiego progu był artykuł Pavla Peciny i Pavla Schlesingera \cite{coling}, gdzie badano zbiór kolokacji, w którym liczba instancji pozytywnych została ustalona na poziomie prawie $ 21\% $\footnote{Zaznaczyć także warto, że zbiór testowy we wspomnianej pracy został przygotowany przez grupę lingwistów oraz był zbiorem pełnym. Każdy kandydat został sklasyfikowany jako wyrażenie wielowyrazowe lub niebędący wyrażeniem wielowyrazowym. Nie występował zatem problem związany z kandydatami, którym nie została przypisana żadna z etykiet, tak jak miało to miejsce podczas prowadzenia badań opisanych w niniejszej pracy.}.


\subsection{Zbiór testowy}
Wykorzystany w tych badaniach zbiór testowy został pozyskany głównie ze \emph{Słowosieci}, która jest obecnie na bieżąco rozwijana i rozszerzana przez lingwistów.
Pozyskany zbiór testowy zawiera także jednostki oznaczone jako pozytywne przez lingwistów i zamieszczone w zewnętrznym źródle w stosunku do \emph{Słowosieci}\footnote{Plik ten został już zintegrowany ze \emph{Słowosiecią}, jednak w chwili tworzenia zbioru testowego był jeszcze źródłem zewnętrznym.}.
Wykorzystany zbiór jednostek pochodzi 20.10.2014 roku.

\par
Wykorzystana miara oceny to uśredniona średnia precyzja, a jej sposób obliczania został opisany w sekcji zawierającej wyniki.

\par
Sposób porównywania kandydatów z zbiorem testowym polegał na sprawdzeniu, czy formy słownikowe wyrazów składowych kandydata są takie same, jak wyrazy składowe którejś z krotek ze zbioru testowego.
Nie są sprawdzane części mowy wyrazów, jedynie same słowa.


\subsection{Szczegółowy opis przebiegu tej części badań}
Tabela \ref{KIPI_2_research_types} przedstawia zestaw 30 różnych wariantów badań przeprowadzonych dla funkcji dwuelementowych na korpusie \emph{KIPI}.
Filtr nazwany \emph{morfeusz} jest związany z analizatorem morfologicznym \emph{Morfeusz SGJP} \cite{morfeusz} i polega na sprawdzeniu czy każde ze słów kandydata na kolokacje jest zawarte w słowniku tego narzędzia (Morfeusza).

\begin{table}[h!]
\centering
\footnotesize\setlength{\tabcolsep}{2.5pt}
\begin{tabular}{ l || l | l | l }
	\toprule
	\textbf{nr} 	& \textbf{źródło danych statystycznych}			& \textbf{źródło kandydatów}		& \textbf{filtry}					\\
	\midrule
	1	& okno ciągłe 							& okno ciągłe			&							\\
	2	& okno ciągłe 							& okno ciągłe			& morfeusz					\\
	3	& okno ciągłe 							& okno ciągłe			& morfeusz, częstość $>$ 5	\\
	4	& okna ciągłe i nieciągłe 				& okno ciągłe i nieciągłe			&							\\
	5	& okna ciągłe i nieciągłe 				& okno ciągłe i nieciągłe			& morfeusz					\\
	6	& okna ciągłe i nieciągłe 				& okno ciągłe i nieciągłe			& morfeusz, częstość $>$ 5	\\
	7	& relacje ciągłe						& relacje ciągłe		&							\\
	8	& relacje ciągłe						& relacje ciągłe		& morfeusz					\\
	9	& relacje ciągłe						& relacje ciągłe		& morfeusz, częstość $>$ 5	\\
	10	& relacje ciągłe						& częste relacje ciągłe &							\\
	11	& relacje ciągłe						& częste relacje ciągłe & morfeusz					\\
	12	& relacje ciągłe						& częste relacje ciągłe & morfeusz, częstość $>$ 5	\\
	13	& relacje i okno, ciągłe				& relacje ciągłe		& 							\\
	14	& relacje i okno, ciągłe				& relacje ciągłe 		& morfeusz 					\\
	15	& relacje i okno, ciągłe				& relacje ciągłe		& morfeusz, częstość $>$ 5	\\
	16	& relacje i okno, ciągłe				& częste relacje ciągłe	&							\\
	17	& relacje i okno, ciągłe	 			& częste relacje ciągłe	& morfeusz					\\
	18	& relacje i okno, ciągłe				& częste relacje ciągłe	& morfeusz, częstość $>$ 5	\\
	19	& relacje ciągłe i nieciągłe			& relacje ciągłe i nieciągłe 		& 							\\
	20	& relacje ciągłe i nieciągłe			& relacje ciągłe i nieciągłe		& morfeusz					\\
	21	& relacje ciągłe i nieciągłe			& relacje ciągłe i nieciągłe		& morfeusz, częstość $>$ 5	\\
	22	& relacje ciągłe i nieciągłe			& częste relacje ciągłe i nieciągłe	& 							\\
	23	& relacje ciągłe i nieciągłe			& częste relacje ciągłe i nieciągłe	& morfeusz					\\
	24	& relacje ciągłe i nieciągłe			& częste relacje ciągłe i nieciągłe	& morfeusz, częstość $>$ 5	\\
	25	& relacje i okno, ciągłe i nieciągłe	& relacje ciągłe i nieciągłe		&							\\
	26	& relacje i okno, ciągłe i nieciągłe	& relacje ciągłe i nieciągłe		& morfeusz					\\
	27	& relacje i okno, ciągłe i nieciągłe	& relacje ciągłe i nieciągłe		& morfeusz, częstość $>$ 5	\\
	28	& relacje i okno, ciągłe i nieciągłe	& częste relacje ciągłe i nieciągłe	&							\\
	29	& relacje i okno, ciągłe i nieciągłe	& częste relacje ciągłe	i nieciągłe	& morfeusz					\\
	30	& relacje i okno, ciągłe i nieciągłe	& częste relacje ciągłe	i nieciągłe	& morfeusz, częstość $>$ 5	\\
	\bottomrule
\end{tabular}
\caption[Zestaw przeprowadzonych badań dla funkcji dwuelementowych na korpusie \emph{KIPI}]{Zestaw przeprowadzonych badań dla funkcji dwuelementowych na korpusie \emph{KIPI}.}
\label{KIPI_2_research_types}
\end{table}

Tabela \ref{KIPI_2_function_set} przestawia zestaw zbadanych funkcji na korpusie \emph{KIPI}.

\begin{table}[h!]
\centering
\footnotesize\setlength{\tabcolsep}{2.5pt}
\begin{tabular}{ l | l || l | l }
	\toprule
	\textbf{nr} 	& \textbf{nazwa}	& \textbf{nr}	& \textbf{nazwa}	\\
	\midrule
	1	&	Frequency$()$									& 37	&	Specific Exponential Correlation$($e=4.7$)$		\\
	2	&	Expected Frequency$()$							& 38	&	Specific Exponential Correlation$($e=4.8$)$		\\
	3	&	Inversed Expected Frequency$()$					& 39	&	Specific Exponential Correlation$($e=4.9$)$		\\
	4	&	Jaccard$()$										& 40	&	Specific Exponential Correlation$($e=5$)$		\\
	5	&	Dice$()$										& 41	&	Specific Exponential Correlation$($e=5.1$)$		\\
	6	&	Sorgenfrei$()$									& 42	&	Specific Exponential Correlation$($e=5.2$)$		\\
	7	&	Odds Ratio$()$									& 43	&	Specific Exponential Correlation$($e=5.3$)$		\\
	8	&	Unigram Subtuples$()$							& 44	&	Specific Exponential Correlation$($e=5.4$)$		\\
	9	&	Consonni T1$()$									& 45	&	Specific Exponential Correlation$($e=5.5$)$		\\
	10	&	Consonni T2$()$									& 46	&	Specific Exponential Correlation$($e=5.6$)$		\\
	11	&	Mutual Expectation$()$							& 47	&	Specific Exponential Correlation$($e=5.7$)$		\\
	12	&	Specific Correlation$()$						& 48	&	Specific Exponential Correlation$($e=5.8$)$		\\
	13	&	W Specific Correlation$()$						& 49	&	Specific Exponential Correlation$($e=5.9$)$		\\
	14	&	Specific Mutual Dependency$()$					& 50	&	Specific Exponential Correlation$($e=6$)$		\\
	15	&	Specific Frequency Biased Mutual Dependency$()$	& 51	&	W Specific Exponential Correlation$($e=1.05$)$	\\
	16	&	Tscore$()$										& 52 	& 	W Specific Exponential Correlation$($e=1.1$)$	\\
	17	&	Zscore$()$										& 53 	& 	W Specific Exponential Correlation$($e=1.15$)$	\\
	18	&	Pearsons Chi Square$()$							& 54	&	W Specific Exponential Correlation$($e=1.2$)$	\\
	19	&	W Chi Square$()$								& 55	&	W Specific Exponential Correlation$($e=1.25$)$	\\
	20	&	Loglikelihood$()$								& 56	&	W Specific Exponential Correlation$($e=1.3$)$	\\
	21	&	Specific Exponential Correlation$($e=3.1$)$		& 57	&	W Specific Exponential Correlation$($e=1.35$)$	\\
	22	&	Specific Exponential Correlation$($e=3.2$)$		& 58	&	W Specific Exponential Correlation$($e=1.4$)$	\\
	23	&	Specific Exponential Correlation$($e=3.3$)$		& 59	&	W Specific Exponential Correlation$($e=1.45$)$	\\
	24	&	Specific Exponential Correlation$($e=3.4$)$		& 60	&	W Specific Exponential Correlation$($e=1.5$)$	\\
	25	&	Specific Exponential Correlation$($e=3.5$)$		& 61	&	W Specific Exponential Correlation$($e=1.55$)$	\\
	26	&	Specific Exponential Correlation$($e=3.6$)$		& 62	&	W Specific Exponential Correlation$($e=1.6$)$	\\
	27	&	Specific Exponential Correlation$($e=3.7$)$		& 63	&	W Specific Exponential Correlation$($e=1.65$)$	\\
	28	&	Specific Exponential Correlation$($e=3.8$)$		& 64	&	W Specific Exponential Correlation$($e=1.7$)$	\\
	29	&	Specific Exponential Correlation$($e=3.9$)$		& 65	&	W Specific Exponential Correlation$($e=1.75$)$	\\
	30	&	Specific Exponential Correlation$($e=4$)$		& 66	&	W Specific Exponential Correlation$($e=1.8$)$	\\
	31	&	Specific Exponential Correlation$($e=4.1$)$		& 67	&	W Specific Exponential Correlation$($e=1.85$)$	\\
	32	&	Specific Exponential Correlation$($e=4.2$)$		& 68	&	W Specific Exponential Correlation$($e=1.9$)$	\\
	33	&	Specific Exponential Correlation$($e=4.3$)$		& 69	&	W Specific Exponential Correlation$($e=1.95$)$	\\
	34	&	Specific Exponential Correlation$($e=4.4$)$		& 70	&	W Specific Exponential Correlation$($e=2$)$		\\
	35	&	Specific Exponential Correlation$($e=4.5$)$		& 71	&	W Order$()$										\\
	36	&	Specific Exponential Correlation$($e=4.6$)$		& 72	&	W Term Frequency Order$()$						\\
	\bottomrule
\end{tabular}
\caption[Zestaw zbadanych funkcji dwuelementowych na korpusie \emph{KIPI}]{Zestaw zbadanych funkcji dwuelementowych na korpusie \emph{KIPI}.}
\label{KIPI_2_function_set}
\end{table}


Tabela \ref{KIPI_2_classifiers_set} przestawia zestaw zbadanych klasyfikatorów etykietujących krotki 3-elementowe na korpusie \emph{KIPI}.
Wykorzystany zestaw cech został ustalony na podstawie wyników zamieszczonych w dalszych częściach pracy oraz wiedzy pozyskanej z artykułu autorstwa Mariusza Paradowskiego \cite{paradowski_beta}; przyjęto 12 następujących cech:
\begin{itemize}
	\item W Specific Correlation;
	\item Mutual Expectation;
	\item Specific Frequency Biased Mutual Dependency;
	\item Tscore;
	\item Loglikelihood;
	\item Jaccard;
	\item Sorgenfrei;
	\item Unigram Subtuples;
	\item Specific Exponential Correlation z parametrem o wartości 3,8;
	\item W Specific Exponential Correlation z parametrem o wartości 1,15;
	\item W Order;
	\item W Term Frequency Order.
\end{itemize}

Dla każdej sieci momentum było stałe i na poziomie 0.5.
Skrótowiec \emph{LNWU} oznacza liczbę neuronów w każdej z warstw ukrytych sieci, a \emph{WU} to współczynnik uczenia perceptronu wielowarstwowego.

\begin{table}[h!]
\centering
\footnotesize\setlength{\tabcolsep}{2.5pt}
\begin{tabular}{ l | c | l || l | c | l }
	\toprule
	\textbf{nr} 	& \textbf{LNWU}	& \textbf{WU}	& \textbf{nr}	& \textbf{LNWU}	& \textbf{WU}	\\
	\midrule
	1	&	5, 2	& 0.2	& 25	& 5, 3	& 0.2 \\
	2	&	6, 2	& 0.2	& 26	& 6, 3	& 0.2 \\
	3	&	7, 2	& 0.2	& 27	& 7, 3 	& 0.2 \\
	4	&	8, 2	& 0.2	& 28	& 8, 3	& 0.2 \\
	5	&	9, 2	& 0.2	& 29	& 9, 3	& 0.2 \\
	6	&	10, 2	& 0.2	& 30	& 10, 3	& 0.2 \\
	7	&	11, 2	& 0.2	& 31	& 11, 3	& 0.2 \\
	8	&	12, 2	& 0.2	& 32	& 12, 3	& 0.2 \\
	9	&	13, 2	& 0.2	& 33	& 13, 3	& 0.2 \\
	10	&	14, 2	& 0.2	& 34	& 14, 3	& 0.2 \\
	11	&	15, 2	& 0.2	& 35	& 15, 3	& 0.2 \\
	12	&	16, 2	& 0.2	& 36	& 16, 3	& 0.2 \\
	13	&	5, 2	& 0.1	& 37	& 5, 3	& 0.1 \\
	14	&	6, 2	& 0.1	& 38	& 6, 3	& 0.1 \\
	15	&	7, 2	& 0.1	& 39	& 7, 3	& 0.1 \\
	16	&	8, 2	& 0.1	& 40	& 8, 3	& 0.1 \\
	17	&	9, 2	& 0.1	& 41	& 9, 3	& 0.1 \\
	18	&	10, 2	& 0.1	& 42	& 10, 3	& 0.1 \\
	19	&	11, 2	& 0.1	& 43	& 11, 3	& 0.1 \\
	20	&	12, 2	& 0.1	& 44	& 12, 3	& 0.1 \\
	21	&	13, 2	& 0.1	& 45	& 13, 3	& 0.1 \\
	22	&	14, 2	& 0.1	& 46	& 14, 3	& 0.1 \\
	23	&	15, 2	& 0.1	& 47	& 15, 3	& 0.1 \\
	24	&	16, 2	& 0.1	& 48	& 16, 3	& 0.1 \\
	\bottomrule
\end{tabular}
\caption[Zestaw zbadanych binarnych perceptronów wielowarstwowych dla problemu ekstrakcji kolokacji dwuelementowych na korpusie \emph{KIPI}]{Zestaw zbadanych binarnych perceptronów wielowarstwowych dla problemu ekstrakcji kolokacji dwuelementowych na korpusie \emph{KIPI}.}
\label{KIPI_2_classifiers_set}
\end{table}


\subsection{Wyniki}
Podczas przedstawiania jakości osiąganych przez miary dwuelementowe dla korpusu \emph{KIPI} wyników, w celu identyfikacji rodzaju badania używane będą odpowiednie numery badań zamiast ich pełnych nazw.
Analogicznie stosowane będą identyfikatory miar zamiast ich pełnych nazw.
Cały zestaw funkcji został zbadany dla każdego z wariantów badań w przypadku korpusu \emph{KIPI}.

\par
Wyniki badań zostały w dużym stopniu skompresowane do tabel zawierających wartości średniej uśrednionej precyzji ze względu na ilość otrzymanych wyników.
Średnia uśredniona precyzja jest rozumiana jako średnia z wyników miary \emph{Average Precision}.
Miara \emph{średniej precyzji} została opisana we wcześniejszej części tej pracy\footnote{por. rozdz. 4.6.13.}.
Przebieg procesu polegał na ocenie pewnego zakresu\footnote{Autor niniejszej pracy postanowił ograniczyć zakres obliczania średniej uśrednionej precyzji z wartości z rankingów od ich 10\% do 90\%. Decyzja taka została podjęta ze względu na inspiracje artykułem \cite{coling}.} rankingu wygenerowanego przez miarę, dla każdego foldu z osobna.
Efektem takiego działania był zestaw wartości (dokładności) dla każdego rankingu ze wszystkich foldów z osobna.
Następnie obliczona została średnia wartość z tak otrzymanych wyników.
Rezultatem końcowym była pojedyncza liczba, będąca jednocześnie miarą jakości danego rozwiązania.

\par
Najlepsza funkcja oraz zestaw funkcji o porównywalnie dobrym wyniku został oznaczony w tabelach za pomocą czcionki pogrubionej.
Podobnie dobre funkcje i klasyfikatory to te, które osiągnęły wynik na poziomie zbliżonym jakościowo\footnote{Zbliżony jakościowo wynik to taki, który osiągnął jakość na poziomie przynajmniej równym pewnemu procentowi. Procent ten może się różnić dla poszczególnych badań, ale zawsze jest zapisany w pierwszej komórce tabeli przedstawiającej wyniki badania.} do najlepszej miary/klasyfikatora w danym badaniu (kolumnie).


\subsubsection{Wyniki badań miar dwuelementowych}
Cztery tabele \ref{KIPI_part_1}, \ref{KIPI_part_2}, \ref{KIPI_part_3} oraz \ref{KIPI_part_4} prezentują jakość wyników osiągniętych przez 72 funkcje w 30 różnych badaniach -- 30 zestawów danych pozyskanych z korpusu \emph{KIPI}.
Ze względu na ilość wyników autor pracy postanowił zmniejszyć rozmiary tabel przedstawiających jakość wyników poprzez używanie liczb zamiast pełnych nazw badań i funkcji.
Etykiety kolumn odpowiadają numerowi badania, a wiersze numerom funkcji.
Zestaw funkcji i typów badań (sposobu przygotowania korpusu \emph{KIPI}) został opisany we wcześniejszej części tego rozdziału w tabelach odpowiednio \ref{KIPI_2_function_set} oraz \ref{KIPI_2_research_types}.

\begin{table}[htp!]
\centering
\footnotesize\setlength{\tabcolsep}{2.5pt}
 \begin{adjustwidth}{-2cm}{}
\begin{tabular}{ l | *{15}{| r}}
	\toprule
	\textbf{95\%} &	\textbf{1}	&	\textbf{2}	&	\textbf{3}	&	\textbf{4}	&	\textbf{5}	&	\textbf{6}	&	\textbf{7}	&	\textbf{8}	&	\textbf{9}	&	\textbf{10}	&	\textbf{11}	&	\textbf{12}	&	\textbf{13}	&	\textbf{14}	&	\textbf{15}	\\
	\midrule
1	&	0,0184	&	0,0261	&	0,0370	&	0,0100	&	0,0136	&	0,0219	&	0,0522	&	0,0612	&	0,0906	&	0,2095	&	0,2117	&	0,3000	&	0,0523	&	0,0614	&	0,0910	\\
2	&	0,0027	&	0,0038	&	0,0067	&	0,0024	&	0,0033	&	0,0053	&	0,0071	&	0,0085	&	0,0189	&	0,0372	&	0,0376	&	0,0792	&	0,0078	&	0,0094	&	0,0198	\\
3	&	0,0014	&	0,0022	&	0,0244	&	0,0007	&	0,0011	&	0,0141	&	0,0024	&	0,0042	&	0,0324	&	0,0109	&	0,0110	&	0,0930	&	0,0024	&	0,0045	&	0,0362	\\
4	&	0,0078	&	0,0171	&	0,0584	&	0,0032	&	0,0058	&	0,0300	&	0,0117	&	0,0313	&	0,0810	&	0,0836	&	0,0849	&	0,2490	&	0,0122	&	0,0359	&	0,0892	\\
5	&	0,0078	&	0,0171	&	0,0584	&	0,0032	&	0,0058	&	0,0300	&	0,0117	&	0,0313	&	0,0810	&	0,0836	&	0,0849	&	0,2490	&	0,0122	&	0,0359	&	0,0892	\\
6	&	0,0081	&	0,0231	&	0,0714	&	0,0034	&	0,0079	&	0,0392	&	0,0107	&	0,0368	&	0,0868	&	0,0979	&	0,0993	&	0,2698	&	0,0106	&	0,0413	&	0,0958	\\
7	&	0,0030	&	0,0057	&	0,0471	&	0,0014	&	0,0023	&	0,0252	&	0,0047	&	0,0109	&	0,0576	&	0,0274	&	0,0277	&	0,1717	&	0,0045	&	0,0111	&	0,0598	\\
8	&	0,0048	&	0,0103	&	0,0522	&	0,0020	&	0,0038	&	0,0278	&	0,0077	&	0,0191	&	0,0651	&	0,0486	&	0,0492	&	0,1966	&	0,0067	&	0,0195	&	0,0684	\\
9	&	0,0015	&	0,0020	&	0,0215	&	0,0008	&	0,0010	&	0,0115	&	0,0029	&	0,0041	&	0,0331	&	0,0107	&	0,0108	&	0,0949	&	0,0030	&	0,0045	&	0,0359	\\
10	&	0,0015	&	0,0020	&	0,0215	&	0,0008	&	0,0010	&	0,0115	&	0,0029	&	0,0041	&	0,0332	&	0,0107	&	0,0108	&	0,0949	&	0,0030	&	0,0045	&	0,0359	\\
11	&	0,0264	&	0,0368	&	0,0646	&	0,0115	&	0,0164	&	0,0374	&	0,0500	&	0,0705	&	0,1116	&	0,2242	&	0,2278	&	0,3584	&	0,0486	&	0,0711	&	0,1140	\\
12	&	0,0028	&	0,0053	&	0,0418	&	0,0013	&	0,0023	&	0,0231	&	0,0043	&	0,0094	&	0,0507	&	0,0238	&	0,0239	&	0,1483	&	0,0044	&	0,0107	&	0,0578	\\
13	&	0,0344	&	0,0440	&	0,0698	&	0,0169	&	0,0211	&	0,0466	&	\textbf{0,0621}	&	0,0744	&	\textbf{0,1155}	&	\textbf{0,2430}	&	\textbf{0,2463}	&	\textbf{0,3753}	&	\textbf{0,0610}	&	0,0740	&	\textbf{0,1186}	\\
14	&	0,0081	&	0,0231	&	0,0714	&	0,0034	&	0,0079	&	0,0392	&	0,0107	&	0,0368	&	0,0868	&	0,0979	&	0,0993	&	0,2698	&	0,0106	&	0,0413	&	0,0958	\\
15	&	0,0223	&	\textbf{0,0466}	&	\textbf{0,0815}	&	0,0090	&	0,0193	&	\textbf{0,0483}	&	0,0336	&	0,0745	&	\textbf{0,1172}	&	0,2345	&	0,2377	&	\textbf{0,3794}	&	0,0310	&	\textbf{0,0759}	&	\textbf{0,1219}	\\
16	&	0,0292	&	0,0376	&	0,0566	&	0,0160	&	0,0195	&	0,0400	&	0,0557	&	0,0656	&	0,1003	&	0,2172	&	0,2205	&	0,3286	&	0,0563	&	0,0665	&	0,1049	\\
17	&	0,0079	&	0,0224	&	0,0711	&	0,0031	&	0,0072	&	0,0385	&	0,0106	&	0,0362	&	0,0864	&	0,0965	&	0,0978	&	0,2685	&	0,0103	&	0,0401	&	0,0950	\\
18	&	0,0080	&	0,0227	&	0,0711	&	0,0033	&	0,0078	&	0,0382	&	0,0107	&	0,0367	&	0,0867	&	0,0977	&	0,0991	&	0,2692	&	0,0107	&	0,0417	&	0,0958	\\
19	&	0,0079	&	0,0224	&	0,0711	&	0,0031	&	0,0071	&	0,0385	&	0,0106	&	0,0362	&	0,0864	&	0,0964	&	0,0977	&	0,2684	&	0,0103	&	0,0401	&	0,0950	\\
20	&	0,0287	&	0,0384	&	0,0649	&	0,0097	&	0,0136	&	0,0300	&	\textbf{0,0628}	&	\textbf{0,0751}	&	\textbf{0,1181}	&	\textbf{0,2509}	&	\textbf{0,2530}	&	\textbf{0,3869}	&	0,0569	&	0,0678	&	0,1143	\\
21	&	0,0238	&	\textbf{0,0469}	&	\textbf{0,0808}	&	0,0097	&	0,0198	&	\textbf{0,0481}	&	0,0367	&	\textbf{0,0758}	&	\textbf{0,1184}	&	0,2397	&	0,2430	&	\textbf{0,3835}	&	0,0338	&	\textbf{0,0769}	&	\textbf{0,1228}	\\
22	&	0,0251	&	\textbf{0,0470}	&	\textbf{0,0800}	&	0,0104	&	0,0202	&	\textbf{0,0479}	&	0,0398	&	\textbf{0,0767}	&	\textbf{0,1193}	&	\textbf{0,2438}	&	\textbf{0,2471}	&	\textbf{0,3866}	&	0,0367	&	\textbf{0,0777}	&	\textbf{0,1234}	\\
23	&	0,0263	&	\textbf{0,0469}	&	\textbf{0,0790}	&	0,0110	&	0,0205	&	\textbf{0,0475}	&	0,0428	&	\textbf{0,0775}	&	\textbf{0,1199}	&	\textbf{0,2470}	&	\textbf{0,2503}	&	\textbf{0,3889}	&	0,0395	&	\textbf{0,0783}	&	\textbf{0,1238}	\\
24	&	0,0273	&	\textbf{0,0467}	&	\textbf{0,0780}	&	0,0115	&	0,0207	&	\textbf{0,0470}	&	0,0455	&	\textbf{0,0780}	&	\textbf{0,1204}	&	\textbf{0,2494}	&	\textbf{0,2526}	&	\textbf{0,3906}	&	0,0423	&	\textbf{0,0787}	&	\textbf{0,1241}	\\
25	&	0,0281	&	\textbf{0,0464}	&	0,0770	&	0,0120	&	0,0208	&	0,0465	&	0,0479	&	\textbf{0,0783}	&	\textbf{0,1207}	&	\textbf{0,2511}	&	\textbf{0,2544}	&	\textbf{0,3917}	&	0,0448	&	\textbf{0,0789}	&	\textbf{0,1242}	\\
26	&	0,0287	&	\textbf{0,0460}	&	0,0759	&	0,0124	&	0,0209	&	0,0459	&	0,0501	&	\textbf{0,0786}	&	\textbf{0,1209}	&	\textbf{0,2524}	&	\textbf{0,2557}	&	\textbf{0,3922}	&	0,0471	&	\textbf{0,0791}	&	\textbf{0,1242}	\\
27	&	0,0292	&	\textbf{0,0456}	&	0,0748	&	0,0128	&	0,0209	&	0,0454	&	0,0519	&	\textbf{0,0787}	&	\textbf{0,1209}	&	\textbf{0,2533}	&	\textbf{0,2566}	&	\textbf{0,3924}	&	0,0491	&	\textbf{0,0792}	&	\textbf{0,1241}	\\
28	&	0,0296	&	\textbf{0,0452}	&	0,0737	&	0,0131	&	0,0208	&	0,0448	&	0,0535	&	\textbf{0,0788}	&	\textbf{0,1209}	&	\textbf{0,2540}	&	\textbf{0,2572}	&	\textbf{0,3923}	&	0,0508	&	\textbf{0,0792}	&	\textbf{0,1239}	\\
29	&	0,0298	&	\textbf{0,0448}	&	0,0727	&	0,0134	&	0,0208	&	0,0442	&	0,0549	&	\textbf{0,0788}	&	\textbf{0,1208}	&	\textbf{0,2543}	&	\textbf{0,2576}	&	\textbf{0,3919}	&	0,0524	&	\textbf{0,0791}	&	\textbf{0,1237}	\\
30	&	0,0299	&	0,0444	&	0,0717	&	0,0136	&	0,0207	&	0,0436	&	0,0560	&	\textbf{0,0787}	&	\textbf{0,1206}	&	\textbf{0,2545}	&	\textbf{0,2577}	&	\textbf{0,3913}	&	0,0537	&	\textbf{0,0790}	&	\textbf{0,1234}	\\
31	&	0,0300	&	0,0439	&	0,0707	&	0,0137	&	0,0206	&	0,0430	&	0,0569	&	\textbf{0,0786}	&	\textbf{0,1204}	&	\textbf{0,2545}	&	\textbf{0,2578}	&	\textbf{0,3906}	&	0,0548	&	\textbf{0,0789}	&	\textbf{0,1231}	\\
32	&	0,0299	&	0,0435	&	0,0697	&	0,0138	&	0,0205	&	0,0424	&	0,0577	&	\textbf{0,0785}	&	\textbf{0,1201}	&	\textbf{0,2544}	&	\textbf{0,2577}	&	\textbf{0,3897}	&	0,0557	&	\textbf{0,0788}	&	\textbf{0,1227}	\\
33	&	0,0299	&	0,0431	&	0,0688	&	0,0139	&	0,0204	&	0,0418	&	0,0584	&	\textbf{0,0783}	&	\textbf{0,1198}	&	\textbf{0,2542}	&	\textbf{0,2575}	&	\textbf{0,3887}	&	0,0565	&	\textbf{0,0786}	&	\textbf{0,1223}	\\
34	&	0,0298	&	0,0427	&	0,0679	&	0,0140	&	0,0203	&	0,0413	&	0,0590	&	\textbf{0,0782}	&	\textbf{0,1195}	&	\textbf{0,2540}	&	\textbf{0,2572}	&	\textbf{0,3877}	&	0,0572	&	\textbf{0,0784}	&	\textbf{0,1219}	\\
35	&	0,0296	&	0,0423	&	0,0670	&	0,0140	&	0,0202	&	0,0407	&	0,0594	&	\textbf{0,0780}	&	\textbf{0,1191}	&	\textbf{0,2536}	&	\textbf{0,2568}	&	\textbf{0,3866}	&	0,0578	&	\textbf{0,0782}	&	\textbf{0,1215}	\\
36	&	0,0295	&	0,0419	&	0,0662	&	0,0140	&	0,0201	&	0,0402	&	\textbf{0,0598}	&	\textbf{0,0778}	&	\textbf{0,1187}	&	\textbf{0,2533}	&	\textbf{0,2564}	&	\textbf{0,3854}	&	\textbf{0,0582}	&	\textbf{0,0780}	&	\textbf{0,1211}	\\
	\bottomrule
\end{tabular}
 \end{adjustwidth}
\caption[Wyniki badań miar dwuelementowych dla korpusu \emph{KIPI}, część 1]{Wyniki badań miar dwuelementowych dla korpusu \emph{KIPI}, część 1.}
\label{KIPI_part_1}
\end{table}

\begin{table}[htp!]
\centering
\footnotesize\setlength{\tabcolsep}{2.5pt}
 \begin{adjustwidth}{-2cm}{}
\begin{tabular}{ l | *{15}{| r}}
	\toprule												
	\textbf{95\%} &	\textbf{16}	&	\textbf{17}	&	\textbf{18}	&	\textbf{19}	&	\textbf{20}	&	\textbf{21}	&	\textbf{22}	&	\textbf{23}	&	\textbf{24}	&	\textbf{25}	&	\textbf{26}	&	\textbf{27}	&	\textbf{28}	&	\textbf{29}	&	\textbf{30}	\\
	\midrule
1	&	0,2097	&	0,2097	&	0,3011	&	0,0273	&	0,0340	&	0,0625	&	0,1577	&	0,1585	&	0,2688	&	0,0273	&	0,0337	&	0,0628	&	0,1589	&	0,1588	&	0,2691	\\
2	&	0,0396	&	0,0400	&	0,0804	&	0,0047	&	0,0062	&	0,0127	&	0,0383	&	0,0388	&	0,0850	&	0,0067	&	0,0085	&	0,0155	&	0,0423	&	0,0432	&	0,0870	\\
3	&	0,0121	&	0,0123	&	0,1105	&	0,0014	&	0,0025	&	0,0223	&	0,0095	&	0,0095	&	0,0863	&	0,0013	&	0,0024	&	0,0232	&	0,0103	&	0,0104	&	0,1013	\\
4	&	0,1057	&	0,1028	&	0,2917	&	0,0054	&	0,0139	&	0,0517	&	0,0524	&	0,0531	&	0,2080	&	0,0052	&	0,0141	&	0,0538	&	0,0660	&	0,0645	&	0,2417	\\
5	&	0,1057	&	0,1028	&	0,2917	&	0,0054	&	0,0139	&	0,0517	&	0,0524	&	0,0531	&	0,2080	&	0,0052	&	0,0141	&	0,0538	&	0,0660	&	0,0645	&	0,2417	\\
6	&	0,1202	&	0,1178	&	0,3175	&	0,0053	&	0,0182	&	0,0610	&	0,0653	&	0,0659	&	0,2356	&	0,0047	&	0,0171	&	0,0624	&	0,0762	&	0,0748	&	0,2670	\\
7	&	0,0294	&	0,0297	&	0,1879	&	0,0025	&	0,0058	&	0,0403	&	0,0208	&	0,0209	&	0,1530	&	0,0022	&	0,0052	&	0,0379	&	0,0222	&	0,0222	&	0,1633	\\
8	&	0,0520	&	0,0521	&	0,2186	&	0,0037	&	0,0095	&	0,0445	&	0,0341	&	0,0343	&	0,1725	&	0,0030	&	0,0084	&	0,0432	&	0,0362	&	0,0360	&	0,1876	\\
9	&	0,0121	&	0,0124	&	0,1092	&	0,0015	&	0,0023	&	0,0203	&	0,0087	&	0,0088	&	0,0826	&	0,0015	&	0,0023	&	0,0210	&	0,0103	&	0,0104	&	0,0969	\\
10	&	0,0121	&	0,0124	&	0,1092	&	0,0015	&	0,0023	&	0,0203	&	0,0087	&	0,0088	&	0,0826	&	0,0015	&	0,0023	&	0,0210	&	0,0103	&	0,0104	&	0,0969	\\
11	&	0,2315	&	0,2331	&	0,3763	&	0,0232	&	0,0371	&	0,0791	&	0,1449	&	0,1464	&	0,3033	&	0,0211	&	0,0354	&	0,0784	&	0,1537	&	0,1536	&	0,3210	\\
12	&	0,0285	&	0,0286	&	0,1813	&	0,0023	&	0,0053	&	0,0349	&	0,0192	&	0,0193	&	0,1352	&	0,0021	&	0,0051	&	0,0367	&	0,0219	&	0,0219	&	0,1598	\\
13	&	0,2440	&	0,2461	&	0,3932	&	0,0331	&	0,0422	&	0,0875	&	0,1629	&	0,1633	&	\textbf{0,3285}	&	0,0286	&	0,0373	&	0,0858	&	0,1571	&	0,1569	&	\textbf{0,3422}	\\
14	&	0,1202	&	0,1178	&	0,3175	&	0,0053	&	0,0182	&	0,0610	&	0,0653	&	0,0659	&	0,2356	&	0,0047	&	0,0171	&	0,0624	&	0,0762	&	0,0748	&	0,2670	\\
15	&	\textbf{0,2482}	&	0,2489	&	\textbf{0,4117}	&	0,0154	&	0,0408	&	0,0871	&	0,1512	&	0,1523	&	\textbf{0,3277}	&	0,0129	&	0,0370	&	0,0855	&	0,1596	&	0,1587	&	\textbf{0,3504}	\\
16	&	0,2193	&	0,2216	&	0,3458	&	0,0302	&	0,0374	&	0,0748	&	0,1502	&	0,1508	&	0,2908	&	0,0273	&	0,0342	&	0,0761	&	0,1457	&	0,1460	&	0,3071	\\
17	&	0,1168	&	0,1142	&	0,3150	&	0,0052	&	0,0177	&	0,0605	&	0,0637	&	0,0640	&	0,2338	&	0,0044	&	0,0160	&	0,0613	&	0,0715	&	0,0699	&	0,2624	\\
18	&	0,1214	&	0,1194	&	0,3179	&	0,0054	&	0,0186	&	0,0613	&	0,0666	&	0,0674	&	0,2371	&	0,0051	&	0,0190	&	0,0635	&	0,0840	&	0,0835	&	0,2731	\\
19	&	0,1167	&	0,1142	&	0,3149	&	0,0052	&	0,0177	&	0,0605	&	0,0636	&	0,0640	&	0,2338	&	0,0044	&	0,0160	&	0,0613	&	0,0714	&	0,0698	&	0,2624	\\
20	&	0,2371	&	0,2345	&	0,3895	&	0,0324	&	0,0413	&	0,0869	&	0,1711	&	0,1688	&	\textbf{0,3354}	&	0,0248	&	0,0307	&	0,0696	&	0,1304	&	0,1270	&	0,2848	\\
21	&	\textbf{0,2521}	&	\textbf{0,2532}	&	\textbf{0,4141}	&	0,0169	&	0,0418	&	0,0883	&	0,1556	&	0,1566	&	\textbf{0,3316}	&	0,0141	&	0,0380	&	0,0866	&	0,1632	&	0,1625	&	\textbf{0,3534}	\\
22	&	\textbf{0,2550}	&	\textbf{0,2563}	&	\textbf{0,4157}	&	0,0184	&	0,0426	&	0,0893	&	0,1591	&	0,1601	&	\textbf{0,3347}	&	0,0154	&	0,0388	&	0,0874	&	0,1661	&	0,1656	&	\textbf{0,3557}	\\
23	&	\textbf{0,2571}	&	\textbf{0,2586}	&	\textbf{0,4165}	&	0,0199	&	0,0433	&	0,0900	&	0,1620	&	0,1631	&	\textbf{0,3372}	&	0,0167	&	0,0395	&	0,0881	&	0,1685	&	0,1681	&	\textbf{0,3574}	\\
24	&	\textbf{0,2586}	&	\textbf{0,2603}	&	\textbf{0,4168}	&	0,0214	&	0,0438	&	0,0905	&	0,1644	&	0,1655	&	\textbf{0,3390}	&	0,0179	&	0,0400	&	0,0886	&	0,1704	&	0,1701	&	\textbf{0,3585}	\\
25	&	\textbf{0,2596}	&	\textbf{0,2614}	&	\textbf{0,4167}	&	0,0228	&	0,0442	&	0,0909	&	0,1664	&	0,1674	&	\textbf{0,3404}	&	0,0192	&	0,0404	&	0,0889	&	0,1720	&	0,1717	&	\textbf{0,3591}	\\
26	&	\textbf{0,2603}	&	\textbf{0,2621}	&	\textbf{0,4162}	&	0,0242	&	0,0444	&	0,0911	&	0,1680	&	0,1690	&	\textbf{0,3413}	&	0,0204	&	0,0407	&	0,0891	&	0,1732	&	0,1730	&	\textbf{0,3593}	\\
27	&	\textbf{0,2606}	&	\textbf{0,2625}	&	\textbf{0,4153}	&	0,0255	&	0,0446	&	0,0913	&	0,1693	&	0,1703	&	\textbf{0,3419}	&	0,0216	&	0,0410	&	0,0892	&	0,1743	&	0,1741	&	\textbf{0,3592}	\\
28	&	\textbf{0,2607}	&	\textbf{0,2627}	&	\textbf{0,4142}	&	0,0266	&	0,0448	&	0,0913	&	0,1703	&	0,1713	&	\textbf{0,3421}	&	0,0227	&	0,0412	&	0,0892	&	0,1751	&	0,1750	&	\textbf{0,3588}	\\
29	&	\textbf{0,2606}	&	\textbf{0,2626}	&	\textbf{0,4129}	&	0,0276	&	0,0448	&	0,0912	&	0,1712	&	0,1722	&	\textbf{0,3422}	&	0,0238	&	0,0413	&	0,0892	&	0,1757	&	0,1756	&	\textbf{0,3583}	\\
30	&	\textbf{0,2604}	&	\textbf{0,2624}	&	\textbf{0,4115}	&	0,0284	&	0,0449	&	0,0911	&	0,1719	&	0,1729	&	\textbf{0,3420}	&	0,0246	&	0,0414	&	0,0891	&	0,1762	&	0,1762	&	\textbf{0,3575}	\\
31	&	\textbf{0,2601}	&	\textbf{0,2621}	&	\textbf{0,4099}	&	0,0291	&	0,0449	&	0,0909	&	0,1725	&	0,1735	&	\textbf{0,3416}	&	0,0254	&	0,0415	&	0,0889	&	0,1766	&	0,1766	&	\textbf{0,3567}	\\
32	&	\textbf{0,2597}	&	\textbf{0,2617}	&	\textbf{0,4083}	&	0,0297	&	0,0448	&	0,0907	&	0,1729	&	0,1739	&	\textbf{0,3411}	&	0,0261	&	0,0415	&	0,0887	&	0,1769	&	0,1769	&	\textbf{0,3557}	\\
33	&	\textbf{0,2592}	&	\textbf{0,2612}	&	\textbf{0,4067}	&	0,0302	&	0,0448	&	0,0904	&	0,1733	&	0,1743	&	\textbf{0,3405}	&	0,0267	&	0,0415	&	0,0885	&	0,1771	&	0,1772	&	\textbf{0,3546}	\\
34	&	\textbf{0,2587}	&	\textbf{0,2607}	&	\textbf{0,4050}	&	0,0307	&	0,0447	&	0,0901	&	0,1736	&	0,1746	&	\textbf{0,3399}	&	0,0272	&	0,0415	&	0,0883	&	0,1773	&	0,1774	&	\textbf{0,3535}	\\
35	&	\textbf{0,2581}	&	\textbf{0,2601}	&	\textbf{0,4033}	&	0,0310	&	0,0446	&	0,0898	&	0,1738	&	0,1748	&	\textbf{0,3391}	&	0,0277	&	0,0415	&	0,0880	&	0,1774	&	0,1775	&	\textbf{0,3524}	\\
36	&	\textbf{0,2575}	&	\textbf{0,2595}	&	\textbf{0,4016}	&	0,0313	&	0,0445	&	0,0895	&	0,1740	&	0,1750	&	\textbf{0,3383}	&	0,0280	&	0,0415	&	0,0877	&	0,1775	&	0,1776	&	\textbf{0,3512}	\\
	\bottomrule
\end{tabular}
 \end{adjustwidth}
\caption[Wyniki badań miar dwuelementowych dla korpusu \emph{KIPI}, część 2]{Wyniki badań miar dwuelementowych dla korpusu \emph{KIPI}, część 2.}
\label{KIPI_part_2}
\end{table}

\begin{table}[htp!]
\centering
\footnotesize\setlength{\tabcolsep}{2.5pt}
 \begin{adjustwidth}{-2cm}{}
\begin{tabular}{ l | *{15}{| r}}
	\toprule																		
	\textbf{95\%} &	\textbf{1}	&	\textbf{2}	&	\textbf{3}	&	\textbf{4}	&	\textbf{5}	&	\textbf{6}	&	\textbf{7}	&	\textbf{8}	&	\textbf{9}	&	\textbf{10}	&	\textbf{11}	&	\textbf{12}	&	\textbf{13}	&	\textbf{14}	&	\textbf{15}	\\
	\midrule
37	&	0,0293	&	0,0415	&	0,0654	&	0,0140	&	0,0199	&	0,0397	&	\textbf{0,0601}	&	\textbf{0,0776}	&	\textbf{0,1184}	&	\textbf{0,2529}	&	\textbf{0,2560}	&	\textbf{0,3842}	&	\textbf{0,0586}	&	\textbf{0,0778}	&	\textbf{0,1206}	\\
38	&	0,0292	&	0,0412	&	0,0646	&	0,0140	&	0,0198	&	0,0393	&	\textbf{0,0604}	&	\textbf{0,0774}	&	\textbf{0,1180}	&	\textbf{0,2524}	&	\textbf{0,2556}	&	\textbf{0,3830}	&	\textbf{0,0590}	&	\textbf{0,0776}	&	\textbf{0,1202}	\\
39	&	0,0290	&	0,0408	&	0,0639	&	0,0140	&	0,0197	&	0,0388	&	\textbf{0,0606}	&	\textbf{0,0772}	&	\textbf{0,1176}	&	\textbf{0,2520}	&	\textbf{0,2551}	&	\textbf{0,3818}	&	\textbf{0,0593}	&	\textbf{0,0774}	&	\textbf{0,1198}	\\
40	&	0,0288	&	0,0405	&	0,0632	&	0,0139	&	0,0196	&	0,0384	&	\textbf{0,0607}	&	\textbf{0,0770}	&	\textbf{0,1173}	&	\textbf{0,2515}	&	\textbf{0,2546}	&	\textbf{0,3806}	&	\textbf{0,0595}	&	\textbf{0,0772}	&	\textbf{0,1193}	\\
41	&	0,0286	&	0,0402	&	0,0625	&	0,0139	&	0,0195	&	0,0379	&	\textbf{0,0609}	&	\textbf{0,0768}	&	\textbf{0,1169}	&	\textbf{0,2510}	&	\textbf{0,2541}	&	\textbf{0,3794}	&	\textbf{0,0597}	&	\textbf{0,0769}	&	\textbf{0,1189}	\\
42	&	0,0284	&	0,0398	&	0,0619	&	0,0138	&	0,0194	&	0,0375	&	\textbf{0,0610}	&	\textbf{0,0766}	&	\textbf{0,1165}	&	\textbf{0,2505}	&	\textbf{0,2536}	&	\textbf{0,3782}	&	\textbf{0,0598}	&	\textbf{0,0767}	&	\textbf{0,1185}	\\
43	&	0,0282	&	0,0395	&	0,0613	&	0,0138	&	0,0192	&	0,0371	&	\textbf{0,0611}	&	\textbf{0,0764}	&	\textbf{0,1161}	&	\textbf{0,2500}	&	\textbf{0,2531}	&	\textbf{0,3770}	&	\textbf{0,0600}	&	\textbf{0,0765}	&	\textbf{0,1180}	\\
44	&	0,0281	&	0,0392	&	0,0607	&	0,0138	&	0,0191	&	0,0368	&	\textbf{0,0611}	&	\textbf{0,0762}	&	\textbf{0,1157}	&	\textbf{0,2495}	&	\textbf{0,2526}	&	\textbf{0,3758}	&	\textbf{0,0601}	&	\textbf{0,0763}	&	0,1176	\\
45	&	0,0279	&	0,0390	&	0,0602	&	0,0137	&	0,0190	&	0,0364	&	\textbf{0,0611}	&	\textbf{0,0760}	&	\textbf{0,1154}	&	\textbf{0,2490}	&	\textbf{0,2520}	&	\textbf{0,3747}	&	\textbf{0,0601}	&	\textbf{0,0761}	&	0,1172	\\
46	&	0,0277	&	0,0387	&	0,0596	&	0,0137	&	0,0189	&	0,0361	&	\textbf{0,0612}	&	\textbf{0,0758}	&	\textbf{0,1150}	&	\textbf{0,2485}	&	\textbf{0,2515}	&	\textbf{0,3735}	&	\textbf{0,0602}	&	\textbf{0,0759}	&	0,1168	\\
47	&	0,0275	&	0,0384	&	0,0591	&	0,0136	&	0,0188	&	0,0357	&	\textbf{0,0612}	&	\textbf{0,0756}	&	0,1147	&	\textbf{0,2480}	&	\textbf{0,2510}	&	0,3724	&	\textbf{0,0603}	&	\textbf{0,0757}	&	0,1164	\\
48	&	0,0274	&	0,0382	&	0,0586	&	0,0136	&	0,0187	&	0,0354	&	\textbf{0,0612}	&	\textbf{0,0754}	&	0,1143	&	\textbf{0,2475}	&	\textbf{0,2505}	&	0,3713	&	\textbf{0,0603}	&	\textbf{0,0755}	&	0,1161	\\
49	&	0,0272	&	0,0379	&	0,0581	&	0,0135	&	0,0186	&	0,0351	&	\textbf{0,0612}	&	\textbf{0,0752}	&	0,1140	&	\textbf{0,2471}	&	\textbf{0,2500}	&	0,3702	&	\textbf{0,0603}	&	\textbf{0,0753}	&	0,1157	\\
50	&	0,0270	&	0,0377	&	0,0577	&	0,0135	&	0,0186	&	0,0348	&	\textbf{0,0611}	&	\textbf{0,0750}	&	0,1136	&	\textbf{0,2466}	&	\textbf{0,2495}	&	0,3692	&	\textbf{0,0603}	&	0,0751	&	0,1153	\\
51	&	\textbf{0,0364}	&	\textbf{0,0455}	&	0,0744	&	0,0161	&	0,0201	&	\textbf{0,0484}	&	\textbf{0,0621}	&	0,0747	&	\textbf{0,1174}	&	\textbf{0,2432}	&	\textbf{0,2464}	&	\textbf{0,3814}	&	\textbf{0,0603}	&	0,0738	&	\textbf{0,1206}	\\
52	&	\textbf{0,0369}	&	\textbf{0,0455}	&	\textbf{0,0783}	&	0,0143	&	0,0181	&	\textbf{0,0491}	&	\textbf{0,0617}	&	0,0746	&	\textbf{0,1191}	&	\textbf{0,2420}	&	\textbf{0,2451}	&	\textbf{0,3869}	&	\textbf{0,0584}	&	0,0723	&	\textbf{0,1216}	\\
53	&	\textbf{0,0352}	&	0,0436	&	\textbf{0,0806}	&	0,0116	&	0,0150	&	\textbf{0,0480}	&	\textbf{0,0605}	&	0,0737	&	\textbf{0,1202}	&	0,2384	&	0,2410	&	\textbf{0,3904}	&	0,0546	&	0,0688	&	\textbf{0,1212}	\\
54	&	0,0312	&	0,0391	&	\textbf{0,0804}	&	0,0081	&	0,0110	&	0,0446	&	0,0581	&	0,0714	&	\textbf{0,1204}	&	0,2308	&	0,2326	&	\textbf{0,3903}	&	0,0483	&	0,0624	&	\textbf{0,1182}	\\
55	&	0,0253	&	0,0330	&	0,0761	&	0,0053	&	0,0078	&	0,0394	&	0,0538	&	0,0671	&	\textbf{0,1189}	&	0,2162	&	0,2172	&	\textbf{0,3849}	&	0,0401	&	0,0544	&	0,1122	\\
56	&	0,0199	&	0,0279	&	0,0702	&	0,0029	&	0,0047	&	0,0348	&	0,0475	&	0,0604	&	\textbf{0,1152}	&	0,1936	&	0,1940	&	0,3722	&	0,0308	&	0,0448	&	0,1037	\\
57	&	0,0141	&	0,0219	&	0,0645	&	0,0017	&	0,0025	&	0,0294	&	0,0394	&	0,0516	&	0,1088	&	0,1639	&	0,1636	&	0,3510	&	0,0207	&	0,0335	&	0,0927	\\
58	&	0,0081	&	0,0148	&	0,0570	&	0,0012	&	0,0018	&	0,0235	&	0,0297	&	0,0405	&	0,0992	&	0,1263	&	0,1253	&	0,3185	&	0,0119	&	0,0221	&	0,0791	\\
59	&	0,0039	&	0,0079	&	0,0476	&	0,0010	&	0,0014	&	0,0179	&	0,0200	&	0,0293	&	0,0871	&	0,0874	&	0,0872	&	0,2761	&	0,0060	&	0,0118	&	0,0644	\\
60	&	0,0023	&	0,0037	&	0,0377	&	0,0008	&	0,0011	&	0,0136	&	0,0119	&	0,0188	&	0,0735	&	0,0527	&	0,0524	&	0,2303	&	0,0038	&	0,0063	&	0,0507	\\
61	&	0,0016	&	0,0025	&	0,0282	&	0,0007	&	0,0009	&	0,0107	&	0,0063	&	0,0102	&	0,0599	&	0,0256	&	0,0257	&	0,1857	&	0,0027	&	0,0044	&	0,0390	\\
62	&	0,0013	&	0,0019	&	0,0210	&	0,0006	&	0,0008	&	0,0089	&	0,0041	&	0,0061	&	0,0477	&	0,0168	&	0,0169	&	0,1453	&	0,0022	&	0,0035	&	0,0301	\\
63	&	0,0011	&	0,0016	&	0,0162	&	0,0005	&	0,0007	&	0,0076	&	0,0030	&	0,0045	&	0,0376	&	0,0127	&	0,0128	&	0,1125	&	0,0018	&	0,0029	&	0,0241	\\
64	&	0,0009	&	0,0013	&	0,0132	&	0,0005	&	0,0007	&	0,0067	&	0,0024	&	0,0035	&	0,0298	&	0,0101	&	0,0102	&	0,0890	&	0,0016	&	0,0025	&	0,0201	\\
65	&	0,0008	&	0,0012	&	0,0111	&	0,0005	&	0,0006	&	0,0060	&	0,0020	&	0,0029	&	0,0244	&	0,0084	&	0,0085	&	0,0731	&	0,0014	&	0,0022	&	0,0173	\\
66	&	0,0008	&	0,0011	&	0,0096	&	0,0004	&	0,0006	&	0,0055	&	0,0017	&	0,0025	&	0,0205	&	0,0073	&	0,0074	&	0,0617	&	0,0013	&	0,0020	&	0,0154	\\
67	&	0,0007	&	0,0010	&	0,0087	&	0,0004	&	0,0006	&	0,0051	&	0,0016	&	0,0023	&	0,0178	&	0,0065	&	0,0065	&	0,0532	&	0,0013	&	0,0019	&	0,0141	\\
68	&	0,0007	&	0,0009	&	0,0079	&	0,0004	&	0,0006	&	0,0048	&	0,0014	&	0,0021	&	0,0157	&	0,0059	&	0,0060	&	0,0470	&	0,0012	&	0,0018	&	0,0131	\\
69	&	0,0007	&	0,0009	&	0,0074	&	0,0004	&	0,0006	&	0,0046	&	0,0013	&	0,0019	&	0,0142	&	0,0056	&	0,0057	&	0,0425	&	0,0012	&	0,0018	&	0,0124	\\
70	&	0,0006	&	0,0009	&	0,0070	&	0,0004	&	0,0006	&	0,0045	&	0,0013	&	0,0018	&	0,0131	&	0,0055	&	0,0055	&	0,0393	&	0,0012	&	0,0018	&	0,0119	\\
71	&	0,0150	&	0,0191	&	0,0226	&	\textbf{0,0280}	&	\textbf{0,0364}	&	0,0430	&	0,0356	&	0,0422	&	0,0498	&	0,1569	&	0,1587	&	0,1873	&	0,0357	&	0,0423	&	0,0500	\\
72	&	0,0182	&	0,0239	&	0,0310	&	0,0250	&	0,0327	&	0,0425	&	0,0418	&	0,0493	&	0,0647	&	0,1826	&	0,1843	&	0,2364	&	0,0418	&	0,0494	&	0,0649	\\
	\bottomrule
\end{tabular}
 \end{adjustwidth}
\caption[Wyniki badań miar dwuelementowych dla korpusu \emph{KIPI}, część 3]{Wyniki badań miar dwuelementowych dla korpusu \emph{KIPI}, część 3.}
\label{KIPI_part_3}
\end{table}

\begin{table}[htp!]
\centering
\footnotesize\setlength{\tabcolsep}{2.5pt}
 \begin{adjustwidth}{-2cm}{}
\begin{tabular}{ l | *{15}{| r}}
	\toprule
	\textbf{95\%} &	\textbf{16}	&	\textbf{17}	&	\textbf{18}	&	\textbf{19}	&	\textbf{20}	&	\textbf{21}	&	\textbf{22}	&	\textbf{23}	&	\textbf{24}	&	\textbf{25}	&	\textbf{26}	&	\textbf{27}	&	\textbf{28}	&	\textbf{29}	&	\textbf{30}	\\
	\midrule
37	&	\textbf{0,2569}	&	\textbf{0,2589}	&	\textbf{0,3999}	&	0,0316	&	0,0444	&	0,0891	&	0,1741	&	0,1751	&	\textbf{0,3375}	&	0,0284	&	0,0414	&	0,0874	&	0,1775	&	0,1776	&	\textbf{0,3500}	\\
38	&	\textbf{0,2563}	&	\textbf{0,2582}	&	\textbf{0,3982}	&	0,0318	&	0,0443	&	0,0888	&	0,1742	&	0,1752	&	\textbf{0,3366}	&	0,0287	&	0,0414	&	0,0871	&	0,1775	&	0,1776	&	\textbf{0,3488}	\\
39	&	\textbf{0,2557}	&	\textbf{0,2576}	&	\textbf{0,3965}	&	0,0320	&	0,0442	&	0,0884	&	0,1743	&	0,1752	&	\textbf{0,3357}	&	0,0289	&	0,0413	&	0,0867	&	0,1775	&	0,1776	&	\textbf{0,3476}	\\
40	&	\textbf{0,2551}	&	\textbf{0,2569}	&	0,3949	&	0,0321	&	0,0441	&	0,0880	&	0,1743	&	0,1753	&	\textbf{0,3348}	&	0,0291	&	0,0412	&	0,0864	&	0,1775	&	0,1775	&	\textbf{0,3464}	\\
41	&	\textbf{0,2544}	&	\textbf{0,2563}	&	0,3933	&	0,0322	&	0,0440	&	0,0877	&	0,1743	&	0,1753	&	\textbf{0,3339}	&	0,0293	&	0,0412	&	0,0861	&	0,1774	&	0,1775	&	\textbf{0,3451}	\\
42	&	\textbf{0,2538}	&	\textbf{0,2556}	&	0,3917	&	0,0323	&	0,0439	&	0,0873	&	0,1743	&	0,1753	&	\textbf{0,3330}	&	0,0295	&	0,0411	&	0,0857	&	0,1773	&	0,1774	&	\textbf{0,3440}	\\
43	&	\textbf{0,2532}	&	\textbf{0,2550}	&	0,3902	&	0,0324	&	0,0437	&	0,0869	&	0,1743	&	0,1752	&	\textbf{0,3321}	&	0,0296	&	0,0410	&	0,0854	&	0,1772	&	0,1773	&	\textbf{0,3428}	\\
44	&	\textbf{0,2526}	&	\textbf{0,2543}	&	0,3886	&	0,0324	&	0,0436	&	0,0866	&	0,1742	&	0,1752	&	\textbf{0,3312}	&	0,0297	&	0,0410	&	0,0851	&	0,1771	&	0,1772	&	\textbf{0,3416}	\\
45	&	\textbf{0,2519}	&	\textbf{0,2537}	&	0,3871	&	0,0325	&	0,0435	&	0,0862	&	0,1742	&	0,1751	&	\textbf{0,3303}	&	0,0298	&	0,0409	&	0,0848	&	0,1770	&	0,1771	&	0,3405	\\
46	&	\textbf{0,2513}	&	\textbf{0,2531}	&	0,3857	&	0,0325	&	0,0434	&	0,0859	&	0,1741	&	0,1750	&	\textbf{0,3294}	&	0,0299	&	0,0408	&	0,0845	&	0,1769	&	0,1770	&	0,3394	\\
47	&	\textbf{0,2508}	&	\textbf{0,2525}	&	0,3842	&	0,0326	&	0,0433	&	0,0855	&	0,1740	&	0,1750	&	\textbf{0,3286}	&	0,0300	&	0,0407	&	0,0841	&	0,1768	&	0,1768	&	0,3383	\\
48	&	\textbf{0,2502}	&	\textbf{0,2518}	&	0,3829	&	0,0326	&	0,0431	&	0,0852	&	0,1739	&	0,1749	&	\textbf{0,3277}	&	0,0301	&	0,0407	&	0,0838	&	0,1766	&	0,1767	&	0,3372	\\
49	&	\textbf{0,2496}	&	\textbf{0,2513}	&	0,3815	&	0,0326	&	0,0430	&	0,0849	&	0,1738	&	0,1748	&	\textbf{0,3269}	&	0,0301	&	0,0406	&	0,0835	&	0,1765	&	0,1766	&	0,3362	\\
50	&	\textbf{0,2491}	&	\textbf{0,2507}	&	0,3802	&	0,0326	&	0,0429	&	0,0845	&	0,1737	&	0,1747	&	\textbf{0,3261}	&	0,0302	&	0,0405	&	0,0832	&	0,1763	&	0,1764	&	0,3351	\\
51	&	0,2437	&	0,2455	&	\textbf{0,4011}	&	0,0328	&	0,0419	&	0,0893	&	0,1600	&	0,1600	&	\textbf{0,3325}	&	0,0274	&	0,0363	&	0,0869	&	0,1526	&	0,1520	&	\textbf{0,3464}	\\
52	&	0,2394	&	0,2405	&	\textbf{0,4061}	&	0,0319	&	0,0411	&	0,0906	&	0,1552	&	0,1549	&	\textbf{0,3353}	&	0,0252	&	0,0341	&	0,0869	&	0,1436	&	0,1427	&	\textbf{0,3466}	\\
53	&	0,2279	&	0,2283	&	\textbf{0,4060}	&	0,0303	&	0,0396	&	0,0912	&	0,1484	&	0,1478	&	\textbf{0,3358}	&	0,0216	&	0,0302	&	0,0848	&	0,1272	&	0,1260	&	0,3396	\\
54	&	0,2054	&	0,2060	&	\textbf{0,3966}	&	0,0280	&	0,0372	&	0,0907	&	0,1388	&	0,1378	&	\textbf{0,3330}	&	0,0169	&	0,0250	&	0,0804	&	0,1045	&	0,1043	&	0,3236	\\
55	&	0,1752	&	0,1781	&	0,3775	&	0,0245	&	0,0333	&	0,0884	&	0,1246	&	0,1231	&	0,3245	&	0,0117	&	0,0191	&	0,0733	&	0,0791	&	0,0799	&	0,2982	\\
56	&	0,1402	&	0,1448	&	0,3488	&	0,0198	&	0,0281	&	0,0840	&	0,1056	&	0,1034	&	0,3099	&	0,0067	&	0,0124	&	0,0644	&	0,0499	&	0,0514	&	0,2654	\\
57	&	0,0990	&	0,1049	&	0,3104	&	0,0141	&	0,0212	&	0,0771	&	0,0807	&	0,0779	&	0,2872	&	0,0035	&	0,0066	&	0,0537	&	0,0254	&	0,0270	&	0,2254	\\
58	&	0,0586	&	0,0652	&	0,2642	&	0,0085	&	0,0138	&	0,0677	&	0,0531	&	0,0506	&	0,2545	&	0,0023	&	0,0039	&	0,0427	&	0,0163	&	0,0170	&	0,1827	\\
59	&	0,0262	&	0,0306	&	0,2129	&	0,0044	&	0,0076	&	0,0562	&	0,0287	&	0,0278	&	0,2140	&	0,0017	&	0,0029	&	0,0326	&	0,0122	&	0,0128	&	0,1408	\\
60	&	0,0162	&	0,0177	&	0,1637	&	0,0028	&	0,0045	&	0,0451	&	0,0180	&	0,0178	&	0,1742	&	0,0014	&	0,0023	&	0,0246	&	0,0097	&	0,0101	&	0,1075	\\
61	&	0,0120	&	0,0129	&	0,1220	&	0,0021	&	0,0033	&	0,0349	&	0,0137	&	0,0136	&	0,1367	&	0,0011	&	0,0019	&	0,0192	&	0,0080	&	0,0084	&	0,0852	\\
62	&	0,0095	&	0,0102	&	0,0935	&	0,0017	&	0,0026	&	0,0268	&	0,0109	&	0,0109	&	0,1074	&	0,0010	&	0,0017	&	0,0156	&	0,0070	&	0,0073	&	0,0699	\\
63	&	0,0079	&	0,0084	&	0,0748	&	0,0014	&	0,0022	&	0,0211	&	0,0089	&	0,0090	&	0,0860	&	0,0009	&	0,0015	&	0,0132	&	0,0063	&	0,0065	&	0,0596	\\
64	&	0,0068	&	0,0072	&	0,0623	&	0,0012	&	0,0019	&	0,0172	&	0,0077	&	0,0077	&	0,0713	&	0,0008	&	0,0014	&	0,0115	&	0,0058	&	0,0060	&	0,0519	\\
65	&	0,0061	&	0,0065	&	0,0535	&	0,0010	&	0,0017	&	0,0146	&	0,0068	&	0,0068	&	0,0610	&	0,0008	&	0,0013	&	0,0103	&	0,0056	&	0,0057	&	0,0465	\\
66	&	0,0058	&	0,0060	&	0,0472	&	0,0010	&	0,0015	&	0,0126	&	0,0062	&	0,0062	&	0,0532	&	0,0008	&	0,0012	&	0,0094	&	0,0055	&	0,0056	&	0,0426	\\
67	&	0,0056	&	0,0057	&	0,0431	&	0,0009	&	0,0014	&	0,0111	&	0,0058	&	0,0058	&	0,0472	&	0,0007	&	0,0012	&	0,0088	&	0,0054	&	0,0055	&	0,0401	\\
68	&	0,0055	&	0,0056	&	0,0402	&	0,0008	&	0,0013	&	0,0100	&	0,0056	&	0,0056	&	0,0429	&	0,0007	&	0,0012	&	0,0083	&	0,0054	&	0,0054	&	0,0383	\\
69	&	0,0054	&	0,0055	&	0,0383	&	0,0008	&	0,0013	&	0,0092	&	0,0055	&	0,0055	&	0,0398	&	0,0007	&	0,0012	&	0,0080	&	0,0053	&	0,0054	&	0,0371	\\
70	&	0,0053	&	0,0054	&	0,0370	&	0,0008	&	0,0013	&	0,0087	&	0,0054	&	0,0054	&	0,0376	&	0,0007	&	0,0012	&	0,0077	&	0,0053	&	0,0054	&	0,0362	\\
71	&	0,1571	&	0,1572	&	0,1879	&	\textbf{0,0643}	&	\textbf{0,0792}	&	\textbf{0,0987}	&	\textbf{0,2668}	&	\textbf{0,2680}	&	\textbf{0,3298}	&	\textbf{0,0643}	&	\textbf{0,0786}	&	\textbf{0,0993}	&	\textbf{0,2688}	&	\textbf{0,2686}	&	0,3302	\\
72	&	0,1827	&	0,1828	&	0,2371	&	0,0572	&	0,0706	&	\textbf{0,0962}	&	0,2519	&	0,2531	&	\textbf{0,3278}	&	0,0572	&	0,0701	&	\textbf{0,0967}	&	0,2537	&	0,2536	&	0,3281	\\
	\bottomrule
\end{tabular}
 \end{adjustwidth}
\caption[Wyniki badań miar dwuelementowych dla korpusu \emph{KIPI}, część 4]{Wyniki badań miar dwuelementowych dla korpusu \emph{KIPI}, część 4.}
\label{KIPI_part_4}
\end{table}
	
Na podstawie analizy danych ze wspomnianych czterech tabel zauważyć można, że najlepsze jakościowo wyniki zostały osiągnięte dla następujących miar asocjacyjnych: \emph{W Specific Correlation}, \emph{Specific Frequency Biased Mutual Dependency}, \emph{Loglikelihood}, \emph{W Order}, \emph{W Term Frequency Order} oraz zestaw miar \emph{Specific Exponential Correlation} i \emph{W Specific Exponential Correlation} dla pewnych wartości ich jedynego parametru -- wykładnika.

\par
Analizując jakość wyników dla miar \emph{Specific Exponential Correlation} i \emph{W Specific Exponential Correlation} należy skupić się także na typie badania.
Wartości w tabeli obrazują, że sposób przygotowania danych ma wpływ na optymalny dobór wartości parametru dla obu tych funkcji z osobna.

\par
Zauważyć można, że im dane są bardziej przefiltrowane, tym lepsze wyniki są osiągane dla mniejszych wartości wykładnika.
Przykładowo dla miary \emph{Specific Exponential Correlation} w zadaniu ekstrakcji kolokacji z grona kandydatów pozyskanego operatorami oknowymi, wartości parametru w okolicach liczby 3,0 są preferowane dla bardziej przefiltrowanych danych, a do ekstrakcji kolokacji z mniej przefiltrowanych danych lepsze wydają się być wartości bliższe 4,0.
Dodatkowo dla zadań związanych z ekstrakcją wyrażeń wielowyrazowych preferowane dla obu funkcji są w większości raczej niskie wartości ich parametru -- niskie jak na przyjęty przedział optymalizacji.

\par
Na podstawie wyników można spróbować wysnuć wniosek, że w okolicach wartości optymalnej parametru funkcji \emph{Specific Exponential Correlation} i \emph{W Specific Exponential Correlation} dopuszczalny margines błędu podczas dostrajania wartości wykładnika jest dość duży.
Innymi słowy, jeśli zostanie znaleziona wartość bliska optymalnej, osiągane przez funkcje wyniki też będą bliskie optimum.
Nie występują duże skoki w jakości rozwiązania, co może uprościć dostrajanie tego parametru tej funkcji.
Tym samym funkcje wydają się być odporne na niewielkie zmiany wartości parametrów w okolicach ich optimów.
Przykładowo dla wartości parametru 1,05 i 1,1 dla badania ze składem oznaczonym numerem 2, różnica w wyniku jest minimalna lub wręcz niezauważalna, ale z drugiej strony zmiana z 1,5 na 1,45 zaowocowała poprawieniem wyniku ponad dwukrotnie.
Na uwadze trzeba jednak mieć to, że w pierwszym przypadku dla obu wartości otrzymywane są stosunkowo dobre wyniki w porównaniu z innymi funkcjami, a w przypadku drugiej zmiany obie wartości parametru były znacząco bardziej oddalone od wartości optymalnej.

\par
Dodatkowo w przypadku operatorów oknowych dodanie kandydatów nieciągłych pogarsza wyniki dla każdej z miar z wyjątkiem dwóch -- \emph{W Order} oraz \emph{W Term Frequency Order}, dla których jakość wyników została poprawiona o około 37\% dla wszystkich trzech przypadków filtrowania -- braku filtrowania, filtru opartego o Morfeusza oraz filtru Morfeusza i częstości.
Powodem takiego wzrostu może być to, że obie miary badają szyk kandydata kolokacji, a tym samym przy dodaniu kolejnych relacji i kandydatów nieciągłych funkcje te miały około dwa razy więcej informacji o kandydatach, niż przy braku kandydatów nieciągłych.
Dodać także należy, że dzięki poszerzeniu grona kandydatów miary te wysunęły się na prowadzenie jeśli chodzi o jakość wyników dla zbioru bez filtracji i filtracji z użyciem Morfeusza.
Natomiast w przypadku zbioru poddanego filtrowaniu zbiorem słów Morfeusza i częstości funkcje te przestały być tak skuteczne, jednak ich wyniki w dalszym ciągu były godne uwagi.
Z kolei w przypadku operatorów relacyjnych sytuacja jest prawie taka sama, czyli zachodzi ogólne pogorszenie wyników z wyjątkiem uzyskanych miarami opartych o szyk -- \emph{W Order} oraz \emph{W Term Frequency Order}.
Istnieją jednak pewne różnice.
Tym razem miary oparte o szyk wysunęły się na prowadzenie we wszystkich trzech przypadkach związanych z filtrowaniem danych, a nie tylko dwóch pierwszych.
Dodatkowo w tym przypadku wzrost jakości rozwiązań generowanych przez miary oparte o szyk wyniósł około 81\%, 88\% oraz 98\% dla odpowiednio trzech kolejnych sposobów filtracji -- braku filtracji, Morfeusz, Morfeusz i częstość większa od pięciu.

\par
Istotną obserwacją może być też to, że funkcje z rodziny \emph{Specific Exponential Correlation} i \emph{W Specific Exponential Correlation} zdają się mieć jedynie pojedyncze optima dla swojego parametru, a jeśli tak jest (wyniki wydają się to potwierdzać), to proces optymalizacji tego parametru -- w dodatku jednego, powinien być zadaniem dość prostym.
Niestety optymalna wartość parametru dla obu funkcji jest różna, a dodatkowo zmienna dla odmiennych zestawów danych.
Mimo to wydaje się, że można wydzielić pewne zakresy wartości dla tych parametrów, w obrębie których można próbować optymalizować wartość wykładnika.
Dla \emph{W Specific Exponential Correlation} zakres taki można spróbować ustalić w przedziale od 0,0+ do 1,3.
Sytuacja jest trudniejsza dla \emph{Specific Exponential Correlation}, ponieważ w większości przypadków zakres taki można byłoby ograniczyć do wartości od mniej niż 3,0 do 4,1, ale w kilku sytuacjach zakres ten należałoby rozszerzyć do nawet 6,0.
Należy jednak nadmienić, że w sytuacji, w której \emph{Specific Exponential Correlation} osiąga dobre wyniki dla parametru o wartości 6,0, nie znajduje się ona w czołówce najlepszych funkcji (ich 5\%).
Dodatkowo istotnym jest, że są to tylko propozycje zakresów wartości parametrów dla tych miar ustalone przez autora tej pracy na podstawie konkretnego zbioru danych, które jednak mogą być pomocne do wyznaczenia wartości początkowej optymalizowanego parametru.

\par
Ciekawym jest także fakt, że przyjęta przez autora niniejszej pracy ziarnistość optymalizacji tych parametrów pozwala na dokładność optymalizacji umożliwiającą dostrojenie parametrów do wartości bliskiej optymalnej i dla której ich ewentualna zmiana o tę właśnie ziarnistość powoduje jedynie niewielkie zmiany jakości -- na poziomie czwartej cyfry licząc od pierwszej cyfry niezerowej wyniku\footnote{Przy zmianach wartości parametru zgodnie z przyjętą ziarnistością wyniki generowane przez miary praktycznie nie ulegały zmianom jeśli optymalizacja przebiegała w okolicy wartości parametru, dla którego osiągane przez funkcje rezultaty były najlepsze.}.
Jednak powodem tego może być też to, o czym autor niniejszej pracy napisał wcześniej -- funkcje wydają się być odporne na niewielkie zmiany wartości parametrów w okolicach ich optimów dla danego zbioru danych.

\par
Należy jednak mieć na uwadze, że wszystkie zamieszczone tutaj wnioski pochodzą z pewnego określonego zbioru danych, a tym samym co do części z nich nie ma pewności, czy powtórzą się w przypadku pracy z innym zbiorem tekstów.


\subsubsection{Wyniki badań miar dwuelementowych na korpusie podzielonym na 10 fragmentów i poddanym dyspersji}
Cztery tabele \ref{KIPI_TFIDF_10_part_1}, \ref{KIPI_TFIDF_10_part_2}, \ref{KIPI_TFIDF_10_part_3} oraz \ref{KIPI_TFIDF_10_part_4} prezentują jakość wyników osiągniętych przez 72 funkcje w 30 różnych badaniach (30 zestawów danych pozyskanych z korpusu \emph{KIPI}).
Różnica pomiędzy tym a poprzednim badaniem jest taka, że w tym przypadku korpus \emph{KIPI} został podzielony na 10 części i poddany dyspersji funkcją \emph{TF-IDF}.
Indeksy miar i typów badań pozostały takie same, jak w poprzednim badaniu.

\begin{table}[htp!]
\centering
\footnotesize\setlength{\tabcolsep}{2.5pt}
 \begin{adjustwidth}{-2cm}{}
\begin{tabular}{ l | *{15}{| r}}
	\toprule 
	\textbf{95\%} &	\textbf{1}	&	\textbf{2}	&	\textbf{3}	&	\textbf{4}	&	\textbf{5}	&	\textbf{6}	&	\textbf{7}	&	\textbf{8}	&	\textbf{9}	&	\textbf{10}	&	\textbf{11}	&	\textbf{12}	&	\textbf{13}	&	\textbf{14}	&	\textbf{15}	\\
	\midrule
1	&	0,0072	&	0,0116	&	0,0196	&	0,0042	&	0,0062	&	0,0111	&	0,0176	&	0,0246	&	0,0397	&	0,1098	&	0,1093	&	0,1869	&	0,0176	&	0,0248	&	0,0393	\\
2	&	0,0011	&	0,0015	&	0,0028	&	0,0012	&	0,0016	&	0,0027	&	0,0031	&	0,0042	&	0,0095	&	0,0206	&	0,0206	&	0,0479	&	0,0037	&	0,0050	&	0,0102	\\
3	&	0,0015	&	0,0026	&	0,0279	&	0,0008	&	0,0013	&	0,0158	&	0,0027	&	0,0052	&	0,0378	&	0,0130	&	0,0132	&	0,0963	&	0,0025	&	0,0051	&	0,0400	\\
4	&	0,0034	&	0,0054	&	0,0321	&	0,0015	&	0,0021	&	0,0157	&	0,0058	&	0,0118	&	0,0487	&	0,0301	&	0,0311	&	0,1496	&	0,0060	&	0,0130	&	0,0514	\\
5	&	0,0034	&	0,0054	&	0,0321	&	0,0015	&	0,0021	&	0,0157	&	0,0058	&	0,0118	&	0,0487	&	0,0301	&	0,0311	&	0,1496	&	0,0060	&	0,0130	&	0,0514	\\
6	&	0,0033	&	0,0074	&	\textbf{0,0400}	&	0,0014	&	0,0028	&	0,0206	&	0,0052	&	0,0142	&	0,0531	&	0,0366	&	0,0373	&	0,1641	&	0,0049	&	0,0147	&	0,0557	\\
7	&	0,0022	&	0,0042	&	0,0363	&	0,0010	&	0,0018	&	0,0192	&	0,0037	&	0,0084	&	0,0467	&	0,0209	&	0,0212	&	0,1327	&	0,0034	&	0,0082	&	0,0484	\\
8	&	0,0029	&	0,0056	&	0,0363	&	0,0012	&	0,0023	&	0,0189	&	0,0050	&	0,0115	&	0,0495	&	0,0290	&	0,0295	&	0,1492	&	0,0043	&	0,0114	&	0,0519	\\
9	&	0,0015	&	0,0019	&	0,0207	&	0,0008	&	0,0010	&	0,0110	&	0,0030	&	0,0044	&	0,0337	&	0,0108	&	0,0110	&	0,0881	&	0,0030	&	0,0046	&	0,0359	\\
10	&	0,0015	&	0,0019	&	0,0207	&	0,0008	&	0,0010	&	0,0110	&	0,0030	&	0,0044	&	0,0337	&	0,0108	&	0,0110	&	0,0881	&	0,0030	&	0,0046	&	0,0359	\\
11	&	0,0067	&	0,0118	&	0,0335	&	0,0027	&	0,0043	&	0,0170	&	0,0118	&	0,0247	&	0,0545	&	0,0815	&	0,0830	&	0,2053	&	0,0116	&	0,0256	&	0,0549	\\
12	&	0,0022	&	0,0040	&	0,0339	&	0,0010	&	0,0018	&	0,0182	&	0,0035	&	0,0079	&	0,0445	&	0,0196	&	0,0199	&	0,1210	&	0,0033	&	0,0080	&	0,0476	\\
13	&	\textbf{0,0108}	&	\textbf{0,0175}	&	0,0337	&	0,0040	&	0,0067	&	0,0192	&	\textbf{0,0196}	&	\textbf{0,0303}	&	0,0529	&	\textbf{0,1204}	&	\textbf{0,1197}	&	\textbf{0,2260}	&	0,0178	&	\textbf{0,0292}	&	0,0535	\\
14	&	0,0033	&	0,0074	&	\textbf{0,0400}	&	0,0014	&	0,0028	&	0,0206	&	0,0052	&	0,0142	&	0,0531	&	0,0366	&	0,0373	&	0,1642	&	0,0049	&	0,0147	&	0,0558	\\
15	&	0,0052	&	0,0125	&	\textbf{0,0407}	&	0,0021	&	0,0044	&	\textbf{0,0212}	&	0,0080	&	0,0232	&	\textbf{0,0578}	&	0,0700	&	0,0711	&	0,2081	&	0,0075	&	0,0235	&	\textbf{0,0586}	\\
16	&	0,0100	&	0,0153	&	0,0249	&	0,0050	&	0,0073	&	0,0154	&	\textbf{0,0195}	&	0,0268	&	0,0429	&	0,1127	&	0,1126	&	0,1955	&	\textbf{0,0190}	&	0,0268	&	0,0442	\\
17	&	0,0033	&	0,0073	&	\textbf{0,0400}	&	0,0014	&	0,0027	&	\textbf{0,0207}	&	0,0052	&	0,0141	&	0,0531	&	0,0364	&	0,0370	&	0,1640	&	0,0049	&	0,0145	&	0,0557	\\
18	&	0,0033	&	0,0073	&	\textbf{0,0400}	&	0,0014	&	0,0028	&	0,0206	&	0,0052	&	0,0141	&	0,0531	&	0,0364	&	0,0371	&	0,1640	&	0,0049	&	0,0146	&	0,0557	\\
19	&	0,0033	&	0,0073	&	\textbf{0,0400}	&	0,0014	&	0,0027	&	\textbf{0,0207}	&	0,0052	&	0,0141	&	0,0531	&	0,0364	&	0,0370	&	0,1639	&	0,0049	&	0,0145	&	0,0557	\\
20	&	0,0100	&	\textbf{0,0171}	&	0,0357	&	0,0033	&	0,0058	&	0,0191	&	\textbf{0,0188}	&	\textbf{0,0307}	&	0,0549	&	\textbf{0,1202}	&	\textbf{0,1195}	&	\textbf{0,2311}	&	0,0171	&	\textbf{0,0297}	&	0,0553	\\
21	&	0,0054	&	0,0129	&	\textbf{0,0405}	&	0,0021	&	0,0045	&	\textbf{0,0211}	&	0,0084	&	0,0240	&	\textbf{0,0579}	&	0,0736	&	0,0747	&	0,2110	&	0,0078	&	0,0242	&	\textbf{0,0586}	\\
22	&	0,0056	&	0,0134	&	\textbf{0,0403}	&	0,0022	&	0,0047	&	\textbf{0,0210}	&	0,0088	&	0,0247	&	\textbf{0,0580}	&	0,0771	&	0,0781	&	0,2135	&	0,0081	&	0,0249	&	\textbf{0,0586}	\\
23	&	0,0058	&	0,0138	&	\textbf{0,0400}	&	0,0023	&	0,0049	&	\textbf{0,0209}	&	0,0091	&	0,0254	&	\textbf{0,0580}	&	0,0805	&	0,0815	&	0,2157	&	0,0085	&	0,0255	&	\textbf{0,0585}	\\
24	&	0,0060	&	0,0141	&	\textbf{0,0397}	&	0,0024	&	0,0050	&	\textbf{0,0208}	&	0,0095	&	0,0260	&	\textbf{0,0580}	&	0,0838	&	0,0847	&	0,2177	&	0,0088	&	0,0260	&	\textbf{0,0584}	\\
25	&	0,0062	&	0,0145	&	0,0395	&	0,0024	&	0,0052	&	\textbf{0,0207}	&	0,0099	&	0,0266	&	\textbf{0,0579}	&	0,0869	&	0,0878	&	0,2195	&	0,0092	&	0,0265	&	\textbf{0,0582}	\\
26	&	0,0064	&	0,0148	&	0,0392	&	0,0025	&	0,0053	&	0,0206	&	0,0103	&	0,0271	&	\textbf{0,0579}	&	0,0898	&	0,0907	&	\textbf{0,2210}	&	0,0095	&	0,0270	&	\textbf{0,0581}	\\
27	&	0,0066	&	0,0151	&	0,0389	&	0,0026	&	0,0054	&	0,0205	&	0,0107	&	0,0275	&	\textbf{0,0578}	&	0,0926	&	0,0934	&	\textbf{0,2223}	&	0,0099	&	0,0274	&	\textbf{0,0579}	\\
28	&	0,0068	&	0,0153	&	0,0385	&	0,0027	&	0,0056	&	0,0204	&	0,0110	&	0,0279	&	\textbf{0,0576}	&	0,0952	&	0,0959	&	\textbf{0,2234}	&	0,0102	&	0,0277	&	\textbf{0,0577}	\\
29	&	0,0070	&	0,0155	&	0,0382	&	0,0028	&	0,0057	&	0,0202	&	0,0114	&	0,0283	&	\textbf{0,0575}	&	0,0976	&	0,0982	&	\textbf{0,2244}	&	0,0106	&	0,0280	&	\textbf{0,0575}	\\
30	&	0,0072	&	0,0157	&	0,0379	&	0,0028	&	0,0058	&	0,0201	&	0,0118	&	0,0286	&	\textbf{0,0573}	&	0,0998	&	0,1004	&	\textbf{0,2252}	&	0,0109	&	0,0283	&	\textbf{0,0573}	\\
31	&	0,0074	&	0,0159	&	0,0376	&	0,0029	&	0,0059	&	0,0200	&	0,0121	&	0,0289	&	\textbf{0,0571}	&	0,1018	&	0,1024	&	\textbf{0,2259}	&	0,0113	&	\textbf{0,0286}	&	\textbf{0,0570}	\\
32	&	0,0076	&	0,0161	&	0,0373	&	0,0030	&	0,0060	&	0,0198	&	0,0125	&	0,0291	&	\textbf{0,0570}	&	0,1037	&	0,1042	&	\textbf{0,2265}	&	0,0116	&	\textbf{0,0288}	&	\textbf{0,0568}	\\
33	&	0,0077	&	0,0162	&	0,0370	&	0,0030	&	0,0061	&	0,0197	&	0,0128	&	\textbf{0,0294}	&	\textbf{0,0568}	&	0,1055	&	0,1059	&	\textbf{0,2269}	&	0,0119	&	\textbf{0,0290}	&	\textbf{0,0566}	\\
34	&	0,0079	&	0,0163	&	0,0367	&	0,0031	&	0,0062	&	0,0195	&	0,0132	&	\textbf{0,0295}	&	\textbf{0,0566}	&	0,1071	&	0,1074	&	\textbf{0,2273}	&	0,0122	&	\textbf{0,0291}	&	\textbf{0,0563}	\\
35	&	0,0080	&	0,0164	&	0,0363	&	0,0032	&	0,0063	&	0,0194	&	0,0135	&	\textbf{0,0297}	&	0,0564	&	0,1085	&	0,1089	&	\textbf{0,2275}	&	0,0125	&	\textbf{0,0293}	&	\textbf{0,0561}	\\
36	&	0,0081	&	0,0165	&	0,0360	&	0,0032	&	0,0064	&	0,0193	&	0,0138	&	\textbf{0,0299}	&	0,0562	&	0,1098	&	0,1101	&	\textbf{0,2277}	&	0,0128	&	\textbf{0,0294}	&	0,0559	\\
	\bottomrule
\end{tabular}
 \end{adjustwidth}
\caption[Wyniki badań miar dwuelementowych dla korpusu \emph{KIPI} podzielonego na 10 części, i poddanego dyspersji miarą TF-IDF, część 1]{Wyniki badań miar dwuelementowych dla korpusu \emph{KIPI} podzielonego na 10 części, i poddanego dyspersji miarą TF-IDF, część 1.}
\label{KIPI_TFIDF_10_part_1}
\end{table}

\begin{table}[htp!]
\centering
\footnotesize\setlength{\tabcolsep}{2.5pt}
 \begin{adjustwidth}{-2cm}{}
\begin{tabular}{ l | *{15}{| r}}
	\toprule 	
	\textbf{95\%} &	\textbf{16}	&	\textbf{17}	&	\textbf{18}	&	\textbf{19}	&	\textbf{20}	&	\textbf{21}	&	\textbf{22}	&	\textbf{23}	&	\textbf{24}	&	\textbf{25}	&	\textbf{26}	&	\textbf{27}	&	\textbf{28}	&	\textbf{29}	&	\textbf{30}	\\
	\midrule			
1	&	0,1097	&	0,1114	&	0,1873	&	0,0091	&	0,0129	&	0,0239	&	0,0811	&	0,0810	&	0,1593	&	0,0091	&	0,0129	&	0,0238	&	0,0809	&	0,0812	&	0,1599	\\
2	&	0,0223	&	0,0225	&	0,0488	&	0,0022	&	0,0031	&	0,0061	&	0,0244	&	0,0246	&	0,0563	&	0,0037	&	0,0049	&	0,0082	&	0,0280	&	0,0280	&	0,0587	\\
3	&	0,0131	&	0,0135	&	0,1084	&	0,0015	&	0,0031	&	0,0264	&	0,0109	&	0,0110	&	0,0875	&	0,0013	&	0,0026	&	0,0255	&	0,0107	&	0,0109	&	0,0974	\\
4	&	0,0360	&	0,0367	&	0,1702	&	0,0027	&	0,0052	&	0,0294	&	0,0192	&	0,0198	&	0,1192	&	0,0026	&	0,0052	&	0,0294	&	0,0235	&	0,0237	&	0,1384	\\
5	&	0,0360	&	0,0367	&	0,1702	&	0,0027	&	0,0052	&	0,0294	&	0,0192	&	0,0198	&	0,1192	&	0,0026	&	0,0052	&	0,0294	&	0,0235	&	0,0237	&	0,1384	\\
6	&	0,0403	&	0,0411	&	0,1853	&	0,0027	&	0,0070	&	0,0357	&	0,0250	&	0,0254	&	0,1386	&	0,0022	&	0,0060	&	0,0342	&	0,0259	&	0,0262	&	0,1526	\\
7	&	0,0210	&	0,0216	&	0,1428	&	0,0020	&	0,0046	&	0,0325	&	0,0158	&	0,0160	&	0,1154	&	0,0016	&	0,0038	&	0,0300	&	0,0155	&	0,0158	&	0,1221	\\
8	&	0,0299	&	0,0307	&	0,1623	&	0,0025	&	0,0058	&	0,0331	&	0,0204	&	0,0207	&	0,1256	&	0,0019	&	0,0049	&	0,0317	&	0,0204	&	0,0207	&	0,1358	\\
9	&	0,0119	&	0,0122	&	0,0993	&	0,0015	&	0,0024	&	0,0211	&	0,0085	&	0,0087	&	0,0749	&	0,0015	&	0,0023	&	0,0212	&	0,0098	&	0,0099	&	0,0872	\\
10	&	0,0119	&	0,0122	&	0,0993	&	0,0015	&	0,0024	&	0,0211	&	0,0085	&	0,0087	&	0,0749	&	0,0015	&	0,0023	&	0,0212	&	0,0098	&	0,0099	&	0,0872	\\
11	&	0,0908	&	0,0927	&	0,2179	&	0,0050	&	0,0104	&	0,0336	&	0,0444	&	0,0456	&	0,1611	&	0,0047	&	0,0101	&	0,0327	&	0,0520	&	0,0524	&	0,1753	\\
12	&	0,0207	&	0,0212	&	0,1389	&	0,0019	&	0,0044	&	0,0304	&	0,0153	&	0,0155	&	0,1073	&	0,0016	&	0,0037	&	0,0295	&	0,0154	&	0,0157	&	0,1203	\\
13	&	\textbf{0,1175}	&	\textbf{0,1196}	&	\textbf{0,2341}	&	0,0090	&	0,0154	&	0,0361	&	0,0743	&	0,0734	&	\textbf{0,1882}	&	0,0066	&	0,0122	&	0,0347	&	0,0651	&	0,0650	&	\textbf{0,1943}	\\
14	&	0,0403	&	0,0411	&	0,1854	&	0,0027	&	0,0070	&	0,0357	&	0,0250	&	0,0254	&	0,1385	&	0,0022	&	0,0060	&	0,0342	&	0,0259	&	0,0262	&	0,1526	\\
15	&	0,0763	&	0,0775	&	0,2244	&	0,0038	&	0,0108	&	0,0390	&	0,0418	&	0,0423	&	0,1703	&	0,0032	&	0,0092	&	\textbf{0,0365}	&	0,0433	&	0,0435	&	0,1812	\\
16	&	0,1126	&	0,1150	&	0,2029	&	0,0103	&	0,0146	&	0,0286	&	0,0769	&	0,0761	&	0,1665	&	0,0083	&	0,0124	&	0,0294	&	0,0690	&	0,0691	&	0,1750	\\
17	&	0,0398	&	0,0405	&	0,1850	&	0,0027	&	0,0070	&	0,0357	&	0,0248	&	0,0251	&	0,1383	&	0,0022	&	0,0058	&	0,0342	&	0,0250	&	0,0253	&	0,1520	\\
18	&	0,0401	&	0,0409	&	0,1851	&	0,0027	&	0,0070	&	0,0357	&	0,0249	&	0,0253	&	0,1385	&	0,0023	&	0,0063	&	0,0345	&	0,0269	&	0,0275	&	0,1536	\\
19	&	0,0398	&	0,0405	&	0,1850	&	0,0027	&	0,0070	&	0,0357	&	0,0248	&	0,0251	&	0,1383	&	0,0022	&	0,0058	&	0,0342	&	0,0250	&	0,0253	&	0,1520	\\
20	&	\textbf{0,1184}	&	\textbf{0,1207}	&	\textbf{0,2395}	&	0,0084	&	0,0156	&	0,0376	&	0,0753	&	0,0754	&	\textbf{0,1923}	&	0,0071	&	0,0128	&	0,0349	&	0,0657	&	0,0651	&	\textbf{0,1962}	\\
21	&	0,0799	&	0,0812	&	0,2266	&	0,0040	&	0,0112	&	0,0391	&	0,0436	&	0,0442	&	0,1725	&	0,0033	&	0,0095	&	\textbf{0,0365}	&	0,0451	&	0,0453	&	0,1831	\\
22	&	0,0833	&	0,0847	&	0,2285	&	0,0041	&	0,0115	&	0,0392	&	0,0454	&	0,0460	&	0,1746	&	0,0034	&	0,0098	&	\textbf{0,0365}	&	0,0469	&	0,0471	&	0,1848	\\
23	&	0,0866	&	0,0880	&	\textbf{0,2302}	&	0,0043	&	0,0119	&	0,0392	&	0,0472	&	0,0478	&	0,1764	&	0,0035	&	0,0101	&	\textbf{0,0365}	&	0,0486	&	0,0489	&	0,1863	\\
24	&	0,0896	&	0,0912	&	\textbf{0,2316}	&	0,0044	&	0,0122	&	0,0392	&	0,0489	&	0,0495	&	0,1781	&	0,0036	&	0,0103	&	\textbf{0,0365}	&	0,0503	&	0,0506	&	\textbf{0,1876}	\\
25	&	0,0925	&	0,0941	&	\textbf{0,2328}	&	0,0046	&	0,0125	&	0,0391	&	0,0507	&	0,0512	&	0,1796	&	0,0038	&	0,0106	&	\textbf{0,0364}	&	0,0519	&	0,0523	&	\textbf{0,1888}	\\
26	&	0,0951	&	0,0969	&	\textbf{0,2338}	&	0,0047	&	0,0128	&	0,0391	&	0,0523	&	0,0529	&	0,1809	&	0,0039	&	0,0108	&	\textbf{0,0364}	&	0,0535	&	0,0539	&	\textbf{0,1898}	\\
27	&	0,0976	&	0,0995	&	\textbf{0,2346}	&	0,0049	&	0,0130	&	0,0390	&	0,0539	&	0,0545	&	0,1821	&	0,0040	&	0,0111	&	\textbf{0,0363}	&	0,0551	&	0,0554	&	\textbf{0,1907}	\\
28	&	0,0999	&	0,1018	&	\textbf{0,2353}	&	0,0051	&	0,0133	&	0,0389	&	0,0555	&	0,0561	&	\textbf{0,1831}	&	0,0042	&	0,0113	&	\textbf{0,0362}	&	0,0565	&	0,0569	&	\textbf{0,1914}	\\
29	&	0,1020	&	0,1040	&	\textbf{0,2358}	&	0,0052	&	0,0135	&	0,0387	&	0,0570	&	0,0575	&	\textbf{0,1840}	&	0,0043	&	0,0115	&	\textbf{0,0360}	&	0,0579	&	0,0583	&	\textbf{0,1921}	\\
30	&	0,1039	&	0,1060	&	\textbf{0,2362}	&	0,0054	&	0,0137	&	0,0386	&	0,0585	&	0,0590	&	\textbf{0,1848}	&	0,0044	&	0,0117	&	\textbf{0,0359}	&	0,0593	&	0,0597	&	\textbf{0,1926}	\\
31	&	0,1057	&	0,1078	&	\textbf{0,2365}	&	0,0055	&	0,0139	&	0,0385	&	0,0598	&	0,0603	&	\textbf{0,1855}	&	0,0045	&	0,0119	&	\textbf{0,0358}	&	0,0605	&	0,0609	&	\textbf{0,1930}	\\
32	&	0,1073	&	0,1095	&	\textbf{0,2366}	&	0,0057	&	0,0141	&	0,0383	&	0,0612	&	0,0617	&	\textbf{0,1861}	&	0,0047	&	0,0120	&	\textbf{0,0357}	&	0,0617	&	0,0622	&	\textbf{0,1934}	\\
33	&	0,1087	&	0,1110	&	\textbf{0,2367}	&	0,0058	&	0,0142	&	0,0381	&	0,0624	&	0,0629	&	\textbf{0,1866}	&	0,0048	&	0,0122	&	\textbf{0,0355}	&	0,0629	&	0,0633	&	\textbf{0,1936}	\\
34	&	0,1100	&	0,1124	&	\textbf{0,2367}	&	0,0060	&	0,0144	&	0,0380	&	0,0636	&	0,0641	&	\textbf{0,1870}	&	0,0049	&	0,0123	&	0,0354	&	0,0640	&	0,0644	&	\textbf{0,1939}	\\
35	&	0,1112	&	0,1136	&	\textbf{0,2367}	&	0,0061	&	0,0145	&	0,0378	&	0,0647	&	0,0652	&	\textbf{0,1874}	&	0,0051	&	0,0125	&	0,0352	&	0,0650	&	0,0655	&	\textbf{0,1940}	\\
36	&	0,1123	&	0,1147	&	\textbf{0,2366}	&	0,0063	&	0,0146	&	0,0376	&	0,0658	&	0,0662	&	\textbf{0,1877}	&	0,0052	&	0,0126	&	0,0351	&	0,0659	&	0,0664	&	\textbf{0,1941}	\\
	\bottomrule
\end{tabular}
 \end{adjustwidth}
\caption[Wyniki badań miar dwuelementowych dla korpusu \emph{KIPI} podzielonego na 10 części, i poddanego dyspersji miarą TF-IDF, część 2]{Wyniki badań miar dwuelementowych dla korpusu \emph{KIPI} podzielonego na 10 części, i poddanego dyspersji miarą TF-IDF, część 2.}
\label{KIPI_TFIDF_10_part_2}
\end{table}

\begin{table}[htp!]
\centering
\footnotesize\setlength{\tabcolsep}{2.5pt}
 \begin{adjustwidth}{-2cm}{}
\begin{tabular}{ l | *{15}{| r}}
	\toprule 																		
	\textbf{95\%} &	\textbf{1}	&	\textbf{2}	&	\textbf{3}	&	\textbf{4}	&	\textbf{5}	&	\textbf{6}	&	\textbf{7}	&	\textbf{8}	&	\textbf{9}	&	\textbf{10}	&	\textbf{11}	&	\textbf{12}	&	\textbf{13}	&	\textbf{14}	&	\textbf{15}	\\
	\midrule 																							
37	&	0,0083	&	0,0166	&	0,0357	&	0,0033	&	0,0065	&	0,0191	&	0,0141	&	\textbf{0,0300}	&	0,0559	&	0,1110	&	0,1113	&	\textbf{0,2278}	&	0,0131	&	\textbf{0,0295}	&	0,0556	\\
38	&	0,0084	&	\textbf{0,0166}	&	0,0355	&	0,0034	&	0,0066	&	0,0190	&	0,0144	&	\textbf{0,0301}	&	0,0557	&	0,1121	&	0,1123	&	\textbf{0,2279}	&	0,0134	&	\textbf{0,0296}	&	0,0554	\\
39	&	0,0085	&	\textbf{0,0167}	&	0,0352	&	0,0034	&	0,0066	&	0,0189	&	0,0147	&	\textbf{0,0302}	&	0,0555	&	0,1131	&	0,1133	&	\textbf{0,2279}	&	0,0137	&	\textbf{0,0297}	&	0,0552	\\
40	&	0,0086	&	\textbf{0,0167}	&	0,0349	&	0,0035	&	0,0067	&	0,0187	&	0,0150	&	\textbf{0,0303}	&	0,0553	&	0,1140	&	\textbf{0,1142}	&	\textbf{0,2279}	&	0,0139	&	\textbf{0,0297}	&	0,0549	\\
41	&	0,0087	&	\textbf{0,0167}	&	0,0346	&	0,0035	&	0,0068	&	0,0186	&	0,0152	&	\textbf{0,0303}	&	0,0551	&	\textbf{0,1149}	&	\textbf{0,1150}	&	\textbf{0,2278}	&	0,0142	&	\textbf{0,0298}	&	0,0547	\\
42	&	0,0087	&	\textbf{0,0167}	&	0,0344	&	0,0036	&	0,0068	&	0,0185	&	0,0155	&	\textbf{0,0304}	&	0,0549	&	\textbf{0,1156}	&	\textbf{0,1157}	&	\textbf{0,2277}	&	0,0144	&	\textbf{0,0298}	&	0,0545	\\
43	&	0,0088	&	\textbf{0,0167}	&	0,0341	&	0,0036	&	0,0069	&	0,0184	&	0,0157	&	\textbf{0,0304}	&	0,0547	&	\textbf{0,1163}	&	\textbf{0,1163}	&	\textbf{0,2276}	&	0,0146	&	\textbf{0,0299}	&	0,0543	\\
44	&	0,0089	&	\textbf{0,0167}	&	0,0338	&	0,0037	&	0,0069	&	0,0182	&	0,0159	&	\textbf{0,0304}	&	0,0545	&	\textbf{0,1169}	&	\textbf{0,1169}	&	\textbf{0,2274}	&	0,0149	&	\textbf{0,0299}	&	0,0540	\\
45	&	0,0089	&	\textbf{0,0167}	&	0,0336	&	0,0037	&	0,0070	&	0,0181	&	0,0161	&	\textbf{0,0305}	&	0,0543	&	\textbf{0,1175}	&	\textbf{0,1174}	&	\textbf{0,2272}	&	0,0151	&	\textbf{0,0299}	&	0,0538	\\
46	&	0,0090	&	\textbf{0,0167}	&	0,0334	&	0,0038	&	0,0070	&	0,0180	&	0,0163	&	\textbf{0,0305}	&	0,0541	&	\textbf{0,1180}	&	\textbf{0,1179}	&	\textbf{0,2270}	&	0,0153	&	\textbf{0,0299}	&	0,0536	\\
47	&	0,0091	&	\textbf{0,0167}	&	0,0331	&	0,0038	&	0,0070	&	0,0179	&	0,0165	&	\textbf{0,0305}	&	0,0539	&	\textbf{0,1184}	&	\textbf{0,1183}	&	\textbf{0,2267}	&	0,0155	&	\textbf{0,0299}	&	0,0534	\\
48	&	0,0091	&	\textbf{0,0167}	&	0,0329	&	0,0039	&	0,0071	&	0,0178	&	0,0167	&	\textbf{0,0305}	&	0,0537	&	\textbf{0,1188}	&	\textbf{0,1187}	&	\textbf{0,2265}	&	0,0156	&	\textbf{0,0299}	&	0,0532	\\
49	&	0,0091	&	\textbf{0,0167}	&	0,0327	&	0,0039	&	0,0071	&	0,0177	&	0,0169	&	\textbf{0,0305}	&	0,0535	&	\textbf{0,1192}	&	\textbf{0,1191}	&	\textbf{0,2262}	&	0,0158	&	\textbf{0,0299}	&	0,0530	\\
50	&	0,0092	&	\textbf{0,0167}	&	0,0325	&	0,0039	&	0,0071	&	0,0176	&	0,0171	&	\textbf{0,0305}	&	0,0533	&	\textbf{0,1195}	&	\textbf{0,1194}	&	\textbf{0,2259}	&	0,0160	&	\textbf{0,0299}	&	0,0528	\\
51	&	\textbf{0,0107}	&	\textbf{0,0174}	&	0,0352	&	0,0036	&	0,0061	&	0,0200	&	\textbf{0,0192}	&	\textbf{0,0302}	&	0,0540	&	\textbf{0,1187}	&	\textbf{0,1178}	&	\textbf{0,2282}	&	0,0170	&	\textbf{0,0287}	&	0,0548	\\
52	&	\textbf{0,0103}	&	\textbf{0,0169}	&	0,0369	&	0,0030	&	0,0052	&	\textbf{0,0208}	&	0,0186	&	\textbf{0,0298}	&	0,0552	&	\textbf{0,1157}	&	\textbf{0,1146}	&	\textbf{0,2303}	&	0,0158	&	0,0276	&	\textbf{0,0562}	\\
53	&	0,0095	&	0,0158	&	0,0385	&	0,0024	&	0,0041	&	\textbf{0,0214}	&	0,0176	&	0,0291	&	\textbf{0,0565}	&	0,1110	&	0,1095	&	\textbf{0,2320}	&	0,0140	&	0,0258	&	\textbf{0,0576}	\\
54	&	0,0081	&	0,0139	&	\textbf{0,0400}	&	0,0018	&	0,0031	&	\textbf{0,0217}	&	0,0161	&	0,0277	&	\textbf{0,0577}	&	0,1035	&	0,1016	&	\textbf{0,2326}	&	0,0118	&	0,0230	&	\textbf{0,0586}	\\
55	&	0,0064	&	0,0115	&	\textbf{0,0411}	&	0,0014	&	0,0024	&	\textbf{0,0216}	&	0,0142	&	0,0255	&	\textbf{0,0588}	&	0,0927	&	0,0906	&	\textbf{0,2317}	&	0,0093	&	0,0193	&	\textbf{0,0590}	\\
56	&	0,0048	&	0,0087	&	\textbf{0,0415}	&	0,0012	&	0,0019	&	\textbf{0,0211}	&	0,0120	&	0,0226	&	\textbf{0,0595}	&	0,0787	&	0,0764	&	\textbf{0,2285}	&	0,0069	&	0,0149	&	\textbf{0,0585}	\\
57	&	0,0035	&	0,0062	&	\textbf{0,0411}	&	0,0009	&	0,0015	&	0,0200	&	0,0096	&	0,0188	&	\textbf{0,0594}	&	0,0619	&	0,0597	&	\textbf{0,2219}	&	0,0051	&	0,0106	&	\textbf{0,0568}	\\
58	&	0,0026	&	0,0044	&	\textbf{0,0396}	&	0,0008	&	0,0013	&	0,0186	&	0,0073	&	0,0145	&	\textbf{0,0586}	&	0,0442	&	0,0425	&	0,2116	&	0,0038	&	0,0075	&	0,0538	\\
59	&	0,0020	&	0,0033	&	0,0372	&	0,0007	&	0,0011	&	0,0170	&	0,0055	&	0,0106	&	\textbf{0,0567}	&	0,0300	&	0,0291	&	0,1974	&	0,0030	&	0,0056	&	0,0500	\\
60	&	0,0016	&	0,0026	&	0,0341	&	0,0006	&	0,0010	&	0,0153	&	0,0042	&	0,0077	&	0,0538	&	0,0214	&	0,0211	&	0,1796	&	0,0024	&	0,0045	&	0,0454	\\
61	&	0,0013	&	0,0021	&	0,0306	&	0,0006	&	0,0009	&	0,0138	&	0,0033	&	0,0059	&	0,0498	&	0,0163	&	0,0162	&	0,1591	&	0,0021	&	0,0037	&	0,0407	\\
62	&	0,0012	&	0,0018	&	0,0270	&	0,0006	&	0,0008	&	0,0123	&	0,0028	&	0,0048	&	0,0454	&	0,0130	&	0,0130	&	0,1386	&	0,0018	&	0,0032	&	0,0365	\\
63	&	0,0010	&	0,0016	&	0,0236	&	0,0005	&	0,0008	&	0,0111	&	0,0024	&	0,0040	&	0,0410	&	0,0108	&	0,0109	&	0,1195	&	0,0017	&	0,0029	&	0,0326	\\
64	&	0,0010	&	0,0015	&	0,0205	&	0,0005	&	0,0008	&	0,0101	&	0,0021	&	0,0035	&	0,0368	&	0,0094	&	0,0095	&	0,1036	&	0,0016	&	0,0027	&	0,0294	\\
65	&	0,0009	&	0,0014	&	0,0180	&	0,0005	&	0,0007	&	0,0093	&	0,0019	&	0,0031	&	0,0333	&	0,0084	&	0,0085	&	0,0909	&	0,0015	&	0,0025	&	0,0268	\\
66	&	0,0009	&	0,0013	&	0,0160	&	0,0005	&	0,0007	&	0,0086	&	0,0018	&	0,0029	&	0,0301	&	0,0078	&	0,0079	&	0,0805	&	0,0015	&	0,0024	&	0,0247	\\
67	&	0,0008	&	0,0013	&	0,0144	&	0,0005	&	0,0007	&	0,0080	&	0,0017	&	0,0027	&	0,0275	&	0,0074	&	0,0074	&	0,0725	&	0,0014	&	0,0023	&	0,0232	\\
68	&	0,0008	&	0,0012	&	0,0132	&	0,0005	&	0,0007	&	0,0076	&	0,0016	&	0,0026	&	0,0255	&	0,0071	&	0,0071	&	0,0667	&	0,0014	&	0,0023	&	0,0218	\\
69	&	0,0008	&	0,0012	&	0,0122	&	0,0005	&	0,0007	&	0,0072	&	0,0016	&	0,0025	&	0,0238	&	0,0068	&	0,0069	&	0,0622	&	0,0014	&	0,0022	&	0,0207	\\
70	&	0,0008	&	0,0012	&	0,0114	&	0,0005	&	0,0007	&	0,0069	&	0,0015	&	0,0024	&	0,0226	&	0,0067	&	0,0067	&	0,0589	&	0,0013	&	0,0022	&	0,0197	\\
71	&	0,0057	&	0,0083	&	0,0151	&	\textbf{0,0058}	&	\textbf{0,0086}	&	0,0139	&	0,0121	&	0,0165	&	0,0265	&	0,0773	&	0,0770	&	0,1315	&	0,0120	&	0,0167	&	0,0263	\\
72	&	0,0069	&	0,0104	&	0,0177	&	\textbf{0,0058}	&	\textbf{0,0087}	&	0,0138	&	0,0148	&	0,0205	&	0,0316	&	0,0960	&	0,0956	&	0,1577	&	0,0147	&	0,0207	&	0,0314	\\
	\bottomrule
\end{tabular}
 \end{adjustwidth}
\caption[Wyniki badań miar dwuelementowych dla korpusu \emph{KIPI} podzielonego na 10 części, i poddanego dyspersji miarą TF-IDF, część 3]{Wyniki badań miar dwuelementowych dla korpusu \emph{KIPI} podzielonego na 10 części, i poddanego dyspersji miarą TF-IDF, część 3.}
\label{KIPI_TFIDF_10_part_3}
\end{table}

\begin{table}[htp!]
\centering
\footnotesize\setlength{\tabcolsep}{2.5pt}
 \begin{adjustwidth}{-2cm}{}
\begin{tabular}{ l | *{15}{| r}}
	\toprule 
	\textbf{95\%} &	\textbf{16}	&	\textbf{17}	&	\textbf{18}	&	\textbf{19}	&	\textbf{20}	&	\textbf{21}	&	\textbf{22}	&	\textbf{23}	&	\textbf{24}	&	\textbf{25}	&	\textbf{26}	&	\textbf{27}	&	\textbf{28}	&	\textbf{29}	&	\textbf{30}	\\
	\midrule
37	&	0,1133	&	0,1158	&	\textbf{0,2364}	&	0,0064	&	0,0147	&	0,0374	&	0,0668	&	0,0672	&	\textbf{0,1879}	&	0,0053	&	0,0127	&	0,0349	&	0,0668	&	0,0673	&	\textbf{0,1942}	\\
38	&	\textbf{0,1142}	&	\textbf{0,1167}	&	\textbf{0,2362}	&	0,0065	&	0,0148	&	0,0373	&	0,0678	&	0,0681	&	\textbf{0,1881}	&	0,0054	&	0,0128	&	0,0348	&	0,0677	&	0,0682	&	\textbf{0,1942}	\\
39	&	\textbf{0,1150}	&	\textbf{0,1175}	&	\textbf{0,2359}	&	0,0067	&	0,0149	&	0,0371	&	0,0687	&	0,0690	&	\textbf{0,1882}	&	0,0055	&	0,0129	&	0,0346	&	0,0685	&	0,0690	&	\textbf{0,1941}	\\
40	&	\textbf{0,1157}	&	\textbf{0,1182}	&	\textbf{0,2357}	&	0,0068	&	0,0150	&	0,0369	&	0,0695	&	0,0699	&	\textbf{0,1883}	&	0,0056	&	0,0130	&	0,0345	&	0,0692	&	0,0698	&	\textbf{0,1941}	\\
41	&	\textbf{0,1164}	&	\textbf{0,1189}	&	\textbf{0,2354}	&	0,0069	&	0,0151	&	0,0367	&	0,0703	&	0,0706	&	\textbf{0,1884}	&	0,0058	&	0,0131	&	0,0344	&	0,0699	&	0,0705	&	\textbf{0,1940}	\\
42	&	\textbf{0,1169}	&	\textbf{0,1195}	&	\textbf{0,2350}	&	0,0070	&	0,0152	&	0,0365	&	0,0711	&	0,0714	&	\textbf{0,1884}	&	0,0059	&	0,0131	&	0,0342	&	0,0706	&	0,0712	&	\textbf{0,1939}	\\
43	&	\textbf{0,1175}	&	\textbf{0,1201}	&	\textbf{0,2347}	&	0,0071	&	0,0152	&	0,0364	&	0,0718	&	0,0721	&	\textbf{0,1884}	&	0,0060	&	0,0132	&	0,0341	&	0,0712	&	0,0718	&	\textbf{0,1937}	\\
44	&	\textbf{0,1179}	&	\textbf{0,1205}	&	\textbf{0,2343}	&	0,0072	&	0,0153	&	0,0362	&	0,0724	&	0,0727	&	\textbf{0,1884}	&	0,0061	&	0,0133	&	0,0339	&	0,0718	&	0,0724	&	\textbf{0,1936}	\\
45	&	\textbf{0,1184}	&	\textbf{0,1210}	&	\textbf{0,2339}	&	0,0074	&	0,0153	&	0,0360	&	0,0731	&	0,0733	&	\textbf{0,1883}	&	0,0062	&	0,0133	&	0,0338	&	0,0723	&	0,0729	&	\textbf{0,1934}	\\
46	&	\textbf{0,1187}	&	\textbf{0,1214}	&	\textbf{0,2335}	&	0,0075	&	0,0154	&	0,0358	&	0,0737	&	0,0739	&	\textbf{0,1883}	&	0,0063	&	0,0134	&	0,0336	&	0,0728	&	0,0735	&	\textbf{0,1932}	\\
47	&	\textbf{0,1191}	&	\textbf{0,1217}	&	\textbf{0,2331}	&	0,0076	&	0,0154	&	0,0357	&	0,0742	&	0,0744	&	\textbf{0,1882}	&	0,0064	&	0,0135	&	0,0335	&	0,0733	&	0,0739	&	\textbf{0,1930}	\\
48	&	\textbf{0,1194}	&	\textbf{0,1220}	&	\textbf{0,2327}	&	0,0077	&	0,0154	&	0,0355	&	0,0747	&	0,0749	&	\textbf{0,1881}	&	0,0065	&	0,0135	&	0,0334	&	0,0738	&	0,0744	&	\textbf{0,1928}	\\
49	&	\textbf{0,1197}	&	\textbf{0,1223}	&	\textbf{0,2323}	&	0,0078	&	0,0154	&	0,0354	&	0,0752	&	0,0754	&	\textbf{0,1880}	&	0,0065	&	0,0135	&	0,0332	&	0,0742	&	0,0748	&	\textbf{0,1926}	\\
50	&	\textbf{0,1199}	&	\textbf{0,1226}	&	\textbf{0,2319}	&	0,0078	&	0,0155	&	0,0352	&	0,0757	&	0,0758	&	\textbf{0,1878}	&	0,0066	&	0,0136	&	0,0331	&	0,0746	&	0,0752	&	\textbf{0,1923}	\\
51	&	\textbf{0,1141}	&	0,1158	&	\textbf{0,2371}	&	0,0086	&	0,0150	&	0,0373	&	0,0710	&	0,0699	&	\textbf{0,1895}	&	0,0058	&	0,0113	&	\textbf{0,0357}	&	0,0591	&	0,0589	&	\textbf{0,1953}	\\
52	&	0,1082	&	0,1094	&	\textbf{0,2395}	&	0,0079	&	0,0144	&	0,0385	&	0,0666	&	0,0653	&	\textbf{0,1905}	&	0,0049	&	0,0100	&	\textbf{0,0366}	&	0,0508	&	0,0505	&	\textbf{0,1952}	\\
53	&	0,0984	&	0,0991	&	\textbf{0,2407}	&	0,0071	&	0,0134	&	\textbf{0,0397}	&	0,0606	&	0,0592	&	\textbf{0,1908}	&	0,0040	&	0,0083	&	\textbf{0,0372}	&	0,0406	&	0,0402	&	\textbf{0,1930}	\\
54	&	0,0843	&	0,0847	&	\textbf{0,2395}	&	0,0061	&	0,0120	&	\textbf{0,0407}	&	0,0529	&	0,0513	&	\textbf{0,1899}	&	0,0031	&	0,0065	&	\textbf{0,0372}	&	0,0300	&	0,0299	&	\textbf{0,1874}	\\
55	&	0,0665	&	0,0669	&	\textbf{0,2348}	&	0,0050	&	0,0102	&	\textbf{0,0413}	&	0,0436	&	0,0420	&	\textbf{0,1870}	&	0,0024	&	0,0049	&	\textbf{0,0366}	&	0,0216	&	0,0217	&	0,1785	\\
56	&	0,0465	&	0,0472	&	0,2251	&	0,0040	&	0,0082	&	\textbf{0,0414}	&	0,0337	&	0,0324	&	0,1820	&	0,0019	&	0,0037	&	0,0352	&	0,0161	&	0,0163	&	0,1663	\\
57	&	0,0300	&	0,0308	&	0,2106	&	0,0032	&	0,0063	&	\textbf{0,0408}	&	0,0252	&	0,0243	&	0,1741	&	0,0016	&	0,0030	&	0,0330	&	0,0126	&	0,0129	&	0,1509	\\
58	&	0,0204	&	0,0211	&	0,1914	&	0,0025	&	0,0049	&	0,0393	&	0,0192	&	0,0187	&	0,1630	&	0,0013	&	0,0025	&	0,0305	&	0,0104	&	0,0106	&	0,1346	\\
59	&	0,0152	&	0,0158	&	0,1691	&	0,0021	&	0,0039	&	0,0373	&	0,0151	&	0,0149	&	0,1501	&	0,0012	&	0,0021	&	0,0279	&	0,0088	&	0,0091	&	0,1185	\\
60	&	0,0120	&	0,0125	&	0,1462	&	0,0017	&	0,0032	&	0,0345	&	0,0123	&	0,0122	&	0,1347	&	0,0010	&	0,0019	&	0,0253	&	0,0078	&	0,0081	&	0,1040	\\
61	&	0,0099	&	0,0104	&	0,1247	&	0,0015	&	0,0027	&	0,0316	&	0,0104	&	0,0103	&	0,1198	&	0,0010	&	0,0017	&	0,0231	&	0,0072	&	0,0074	&	0,0918	\\
62	&	0,0086	&	0,0090	&	0,1068	&	0,0013	&	0,0024	&	0,0287	&	0,0090	&	0,0090	&	0,1057	&	0,0009	&	0,0016	&	0,0209	&	0,0068	&	0,0070	&	0,0816	\\
63	&	0,0078	&	0,0082	&	0,0922	&	0,0012	&	0,0021	&	0,0260	&	0,0081	&	0,0081	&	0,0934	&	0,0009	&	0,0016	&	0,0192	&	0,0065	&	0,0067	&	0,0740	\\
64	&	0,0073	&	0,0076	&	0,0813	&	0,0011	&	0,0020	&	0,0237	&	0,0075	&	0,0075	&	0,0837	&	0,0009	&	0,0015	&	0,0178	&	0,0063	&	0,0065	&	0,0679	\\
65	&	0,0069	&	0,0072	&	0,0731	&	0,0011	&	0,0019	&	0,0216	&	0,0071	&	0,0071	&	0,0754	&	0,0008	&	0,0015	&	0,0166	&	0,0062	&	0,0063	&	0,0633	\\
66	&	0,0067	&	0,0069	&	0,0670	&	0,0010	&	0,0018	&	0,0198	&	0,0068	&	0,0069	&	0,0689	&	0,0008	&	0,0014	&	0,0157	&	0,0061	&	0,0062	&	0,0598	\\
67	&	0,0065	&	0,0067	&	0,0626	&	0,0010	&	0,0017	&	0,0184	&	0,0066	&	0,0067	&	0,0639	&	0,0008	&	0,0014	&	0,0149	&	0,0060	&	0,0062	&	0,0573	\\
68	&	0,0064	&	0,0065	&	0,0592	&	0,0010	&	0,0017	&	0,0173	&	0,0065	&	0,0065	&	0,0602	&	0,0008	&	0,0014	&	0,0143	&	0,0060	&	0,0061	&	0,0552	\\
69	&	0,0063	&	0,0064	&	0,0565	&	0,0010	&	0,0016	&	0,0164	&	0,0064	&	0,0064	&	0,0574	&	0,0008	&	0,0014	&	0,0137	&	0,0059	&	0,0060	&	0,0535	\\
70	&	0,0062	&	0,0063	&	0,0545	&	0,0009	&	0,0016	&	0,0156	&	0,0063	&	0,0063	&	0,0553	&	0,0008	&	0,0014	&	0,0132	&	0,0059	&	0,0060	&	0,0522	\\
71	&	0,0773	&	0,0785	&	0,1317	&	\textbf{0,0122}	&	\textbf{0,0175}	&	0,0280	&	\textbf{0,0995}	&	\textbf{0,0994}	&	0,1602	&	\textbf{0,0122}	&	\textbf{0,0175}	&	0,0280	&	\textbf{0,0992}	&	\textbf{0,0996}	&	0,1608	\\
72	&	0,0959	&	0,0974	&	0,1580	&	\textbf{0,0120}	&	\textbf{0,0173}	&	0,0279	&	\textbf{0,1008}	&	\textbf{0,1007}	&	0,1644	&	\textbf{0,0120}	&	\textbf{0,0174}	&	0,0278	&	\textbf{0,1005}	&	\textbf{0,1009}	&	0,1650	\\
	\bottomrule
\end{tabular}
 \end{adjustwidth}
\caption[Wyniki badań miar dwuelementowych dla korpusu \emph{KIPI} podzielonego na 10 części, i poddanego dyspersji miarą TF-IDF, część 4]{Wyniki badań miar dwuelementowych dla korpusu \emph{KIPI} podzielonego na 10 części, i poddanego dyspersji miarą TF-IDF, część 4.}
\label{KIPI_TFIDF_10_part_4}
\end{table}

Najlepsze jakościowo wyniki zostały osiągnięte dla następujących miar asocjacyjnych: \emph{Sorgenfrei}, \emph{Specific Mutual Dependency}, \emph{Specific Frequency Biased Mutual Dependency}, \emph{W Specific Correlation}, \emph{T-Score}, \emph{Z-Score}, $ Pearson's $ $ Chi^{2} $, \emph{Loglikelihood}, \emph{W Order}, \emph{W Term Frequency Order} oraz zestaw miar \emph{Specific Exponential Correlation} i \emph{W Specific Exponential Correlation} dla pewnych wartości ich jedynego parametru -- wykładnika.
Zestaw najlepszych funkcji wyznaczony na podstawie wyników niniejszego badania zawiera ich znaczną liczbę.
Jednak część z nich okazała się pomocna jedynie przy ekstrakcji kolokacji z kandydatów wyznaczonych za pomocą operatorów oknowych.

\par
Wykonanie podziału korpusu na 10 części i wykonanie dyspersji kandydatów na kolokacje za pomocą miary \emph{TF-IDF} pogorszyło jakość wyników generowanych przez wszystkie funkcje.
Efekt taki nie był spodziewany, a powodem zaistniałej sytuacji może być przykładowo to, że teksty źródłowe w korpusie \emph{KIPI} nie zostały pogrupowane tematycznie.
Brak grupowania może zaowocować rozrzuceniem zwrotów i sformułowań z danej dziedziny tematycznej na przestrzeni całego korpusu, a to mogłoby tłumaczyć problem pogorszenia się wyników.
Strona internetowa \emph{Korpusu IPI PAN} \cite{korpus_ipi_pan} nie zawierała informacji o zawartości tematycznej tekstów składowych korpusu (jedynie jego próbki) ani informacji o ich zgrupowaniu lub jego braku.
Autor niniejszej pracy nie znalazł także takich informacji w publikacji traktującej o korpusie \emph{KIPI}\cite{korpus_ipi_pan_publikacja}.

\par
Istnieje też możliwość wystąpienia innego problemu -- zbyt małej, wybranej arbitralnie ziarnistości podziału korpusu na części.
Na potrzeby tego badania korpus \emph{KIPI} został podzielony na 10 ciągłych części (każda z nich zawierała około 10\% tekstu z korpusu).
Źródła nie podają informacji o zawartości tematycznej artykułów ani o procentowej zawartości każdego działu tematycznego w pełnym korpusie \emph{KIPI} \cite{korpus_ipi_pan}\cite{korpus_ipi_pan_publikacja}.
Takie informacje dostępne są jedynie dla próbki korpusu i znajdują się one na stronie internetowej \emph{Korpusu IPI PAN} \cite{korpus_ipi_pan}.


\subsubsection{Wyniki badań miar dwuelementowych na korpusie KIPI podzielonym na 20 fragmentów i poddanym dyspersji}
Autor niniejszej pracy postanowił sprawdzić, jak zmienią się wyniki po zwiększeniu ziarnistości podziału do 20, czyli po podziale na 20 części, z których każda będzie zawierać 5\% tekstów całego korpusu.
Procent ten został wybrany po zasugerowaniu się danymi dotyczącymi próbki z korpusu \emph{KIPI} \cite{korpus_ipi_pan}.
Zauważyć można, że najmniejszy dział tematyczny stanowi właśnie około 5\% całego korpusu \emph{KIPI}.
Jednak zaznaczyć trzeba, że autor niniejszej pracy zdaje sobie sprawę, że próbka nie odzwierciedla statystyk całego korpusu -- po prostu taka informacja wydaje się lepsza niż brak jakiejkolwiek informacji.
Innymi słowy ze względu na brak istotnych informacji o składzie korpusu \emph{KIPI} podział na 10 czy 20 części jest tak samo dobry jak na każdą inną ich liczbę.

\par
Cztery tabele \ref{KIPI_TFIDF_20_part_1}, \ref{KIPI_TFIDF_20_part_2}, \ref{KIPI_TFIDF_20_part_3} oraz \ref{KIPI_TFIDF_20_part_4} prezentują jakość wyników osiągniętych przez 72 funkcje w 30 różnych badaniach (30 zestawów danych pozyskanych z korpusu \emph{KIPI}).
Różnica pomiędzy tym a poprzednim badaniem jest taka, że w tym przypadku korpus \emph{KIPI} został podzielony na 20 części.
Wykorzystana została natomiast ta sama miara dyspersji -- TF-IDF.
Indeksy miar i typów badań pozostały takie same, jak w poprzednich badaniach.

\begin{table}[htp!]
\centering
\footnotesize\setlength{\tabcolsep}{2.5pt}
 \begin{adjustwidth}{-2cm}{}
\begin{tabular}{ l | *{15}{| r}}
	\toprule 
	\textbf{95\%} &	\textbf{1}	&	\textbf{2}	&	\textbf{3}	&	\textbf{4}	&	\textbf{5}	&	\textbf{6}	&	\textbf{7}	&	\textbf{8}	&	\textbf{9}	&	\textbf{10}	&	\textbf{11}	&	\textbf{12}	&	\textbf{13}	&	\textbf{14}	&	\textbf{15}	\\
	\midrule
1	&	0,0098	&	0,0158	&	0,0235	&	0,0056	&	0,0083	&	0,0133	&	0,0236	&	0,0323	&	0,0469	&	0,1409	&	0,1405	&	0,2139	&	0,0236	&	0,0328	&	0,0468	\\
2	&	0,0014	&	0,0020	&	0,0035	&	0,0015	&	0,0020	&	0,0032	&	0,0037	&	0,0050	&	0,0114	&	0,0243	&	0,0244	&	0,0546	&	0,0046	&	0,0060	&	0,0124	\\
3	&	0,0016	&	0,0026	&	0,0288	&	0,0008	&	0,0013	&	0,0161	&	0,0027	&	0,0052	&	0,0392	&	0,0131	&	0,0132	&	0,1036	&	0,0025	&	0,0051	&	0,0416	\\
4	&	0,0039	&	0,0065	&	0,0341	&	0,0017	&	0,0025	&	0,0167	&	0,0067	&	0,0137	&	0,0515	&	0,0351	&	0,0361	&	0,1562	&	0,0068	&	0,0153	&	0,0549	\\
5	&	0,0039	&	0,0065	&	0,0341	&	0,0017	&	0,0025	&	0,0167	&	0,0067	&	0,0137	&	0,0516	&	0,0351	&	0,0361	&	0,1562	&	0,0068	&	0,0153	&	0,0549	\\
6	&	0,0038	&	0,0087	&	0,0425	&	0,0016	&	0,0033	&	0,0220	&	0,0058	&	0,0162	&	0,0563	&	0,0418	&	0,0426	&	0,1719	&	0,0056	&	0,0171	&	0,0593	\\
7	&	0,0023	&	0,0045	&	0,0387	&	0,0011	&	0,0019	&	0,0203	&	0,0039	&	0,0090	&	0,0493	&	0,0225	&	0,0228	&	0,1415	&	0,0036	&	0,0088	&	0,0507	\\
8	&	0,0032	&	0,0064	&	0,0385	&	0,0014	&	0,0025	&	0,0201	&	0,0054	&	0,0127	&	0,0520	&	0,0319	&	0,0324	&	0,1537	&	0,0047	&	0,0126	&	0,0544	\\
9	&	0,0016	&	0,0020	&	0,0222	&	0,0008	&	0,0010	&	0,0116	&	0,0031	&	0,0045	&	0,0357	&	0,0112	&	0,0114	&	0,0957	&	0,0031	&	0,0047	&	0,0381	\\
10	&	0,0016	&	0,0020	&	0,0222	&	0,0008	&	0,0010	&	0,0116	&	0,0031	&	0,0045	&	0,0357	&	0,0112	&	0,0114	&	0,0957	&	0,0031	&	0,0047	&	0,0381	\\
11	&	0,0087	&	0,0162	&	0,0380	&	0,0035	&	0,0059	&	0,0192	&	0,0146	&	0,0320	&	0,0609	&	0,1049	&	0,1067	&	0,2209	&	0,0145	&	0,0334	&	0,0620	\\
12	&	0,0023	&	0,0043	&	0,0352	&	0,0011	&	0,0019	&	0,0189	&	0,0038	&	0,0084	&	0,0466	&	0,0210	&	0,0213	&	0,1292	&	0,0036	&	0,0086	&	0,0498	\\
13	&	\textbf{0,0152}	&	\textbf{0,0243}	&	0,0410	&	0,0056	&	0,0093	&	0,0230	&	\textbf{0,0263}	&	\textbf{0,0397}	&	0,0622	&	\textbf{0,1564}	&	\textbf{0,1554}	&	\textbf{0,2575}	&	0,0238	&	\textbf{0,0382}	&	0,0628	\\
14	&	0,0038	&	0,0087	&	0,0425	&	0,0016	&	0,0033	&	0,0220	&	0,0058	&	0,0162	&	0,0563	&	0,0418	&	0,0426	&	0,1718	&	0,0056	&	0,0171	&	0,0593	\\
15	&	0,0064	&	0,0162	&	\textbf{0,0453}	&	0,0025	&	0,0056	&	\textbf{0,0236}	&	0,0095	&	0,0287	&	0,0631	&	0,0858	&	0,0870	&	0,2193	&	0,0089	&	0,0294	&	\textbf{0,0645}	\\
16	&	0,0142	&	0,0215	&	0,0311	&	\textbf{0,0069}	&	0,0100	&	0,0192	&	\textbf{0,0262}	&	0,0355	&	0,0513	&	0,1464	&	0,1463	&	0,2263	&	\textbf{0,0255}	&	0,0355	&	0,0531	\\
17	&	0,0038	&	0,0086	&	0,0425	&	0,0016	&	0,0031	&	0,0220	&	0,0058	&	0,0161	&	0,0562	&	0,0415	&	0,0422	&	0,1715	&	0,0055	&	0,0167	&	0,0592	\\
18	&	0,0038	&	0,0086	&	0,0425	&	0,0016	&	0,0032	&	0,0220	&	0,0058	&	0,0161	&	0,0562	&	0,0416	&	0,0423	&	0,1716	&	0,0055	&	0,0169	&	0,0593	\\
19	&	0,0038	&	0,0086	&	0,0425	&	0,0016	&	0,0031	&	0,0220	&	0,0058	&	0,0161	&	0,0562	&	0,0415	&	0,0422	&	0,1715	&	0,0055	&	0,0167	&	0,0592	\\
20	&	0,0133	&	0,0228	&	0,0425	&	0,0042	&	0,0072	&	0,0217	&	\textbf{0,0250}	&	\textbf{0,0400}	&	\textbf{0,0641}	&	\textbf{0,1553}	&	\textbf{0,1543}	&	\textbf{0,2612}	&	0,0224	&	\textbf{0,0381}	&	\textbf{0,0644}	\\
21	&	0,0067	&	0,0169	&	\textbf{0,0453}	&	0,0026	&	0,0059	&	\textbf{0,0236}	&	0,0100	&	0,0298	&	\textbf{0,0634}	&	0,0908	&	0,0920	&	0,2229	&	0,0094	&	0,0305	&	\textbf{0,0647}	\\
22	&	0,0070	&	0,0176	&	\textbf{0,0453}	&	0,0028	&	0,0061	&	\textbf{0,0237}	&	0,0105	&	0,0309	&	\textbf{0,0637}	&	0,0957	&	0,0969	&	0,2263	&	0,0098	&	0,0314	&	\textbf{0,0649}	\\
23	&	0,0073	&	0,0182	&	\textbf{0,0452}	&	0,0029	&	0,0064	&	\textbf{0,0237}	&	0,0110	&	0,0319	&	\textbf{0,0640}	&	0,1005	&	0,1016	&	0,2294	&	0,0103	&	0,0324	&	\textbf{0,0650}	\\
24	&	0,0076	&	0,0189	&	\textbf{0,0451}	&	0,0030	&	0,0066	&	\textbf{0,0237}	&	0,0115	&	0,0328	&	\textbf{0,0642}	&	0,1051	&	0,1062	&	0,2323	&	0,0107	&	0,0332	&	\textbf{0,0651}	\\
25	&	0,0079	&	0,0194	&	\textbf{0,0450}	&	0,0031	&	0,0069	&	\textbf{0,0236}	&	0,0120	&	0,0337	&	\textbf{0,0644}	&	0,1095	&	0,1105	&	0,2350	&	0,0112	&	0,0340	&	\textbf{0,0652}	\\
26	&	0,0083	&	0,0199	&	\textbf{0,0448}	&	0,0032	&	0,0071	&	\textbf{0,0236}	&	0,0126	&	0,0344	&	\textbf{0,0645}	&	0,1137	&	0,1147	&	0,2374	&	0,0117	&	0,0347	&	\textbf{0,0652}	\\
27	&	0,0086	&	0,0204	&	\textbf{0,0447}	&	0,0034	&	0,0074	&	\textbf{0,0236}	&	0,0131	&	0,0351	&	\textbf{0,0646}	&	0,1176	&	0,1186	&	0,2396	&	0,0122	&	0,0353	&	\textbf{0,0652}	\\
28	&	0,0089	&	0,0209	&	\textbf{0,0445}	&	0,0035	&	0,0076	&	\textbf{0,0235}	&	0,0137	&	0,0358	&	\textbf{0,0646}	&	0,1213	&	0,1222	&	0,2415	&	0,0127	&	0,0359	&	\textbf{0,0652}	\\
29	&	0,0092	&	0,0212	&	\textbf{0,0443}	&	0,0036	&	0,0078	&	\textbf{0,0234}	&	0,0142	&	0,0364	&	\textbf{0,0646}	&	0,1248	&	0,1256	&	0,2433	&	0,0132	&	0,0364	&	\textbf{0,0651}	\\
30	&	0,0095	&	0,0216	&	0,0441	&	0,0037	&	0,0080	&	\textbf{0,0234}	&	0,0147	&	0,0369	&	\textbf{0,0646}	&	0,1280	&	0,1288	&	0,2449	&	0,0137	&	0,0368	&	\textbf{0,0650}	\\
31	&	0,0097	&	0,0219	&	0,0438	&	0,0038	&	0,0082	&	\textbf{0,0233}	&	0,0153	&	0,0374	&	\textbf{0,0646}	&	0,1310	&	0,1317	&	0,2464	&	0,0142	&	0,0372	&	\textbf{0,0649}	\\
32	&	0,0100	&	0,0222	&	0,0436	&	0,0040	&	0,0084	&	\textbf{0,0232}	&	0,0158	&	0,0378	&	\textbf{0,0646}	&	0,1338	&	0,1344	&	0,2476	&	0,0147	&	0,0376	&	\textbf{0,0648}	\\
33	&	0,0103	&	0,0224	&	0,0433	&	0,0041	&	0,0086	&	0,0231	&	0,0163	&	0,0381	&	\textbf{0,0645}	&	0,1363	&	0,1368	&	\textbf{0,2488}	&	0,0152	&	\textbf{0,0379}	&	\textbf{0,0647}	\\
34	&	0,0105	&	0,0226	&	0,0431	&	0,0042	&	0,0087	&	0,0230	&	0,0168	&	\textbf{0,0385}	&	\textbf{0,0644}	&	0,1386	&	0,1391	&	\textbf{0,2498}	&	0,0157	&	\textbf{0,0382}	&	\textbf{0,0646}	\\
35	&	0,0108	&	0,0228	&	0,0429	&	0,0043	&	0,0089	&	0,0229	&	0,0173	&	\textbf{0,0387}	&	\textbf{0,0643}	&	0,1407	&	0,1411	&	\textbf{0,2506}	&	0,0161	&	\textbf{0,0384}	&	\textbf{0,0645}	\\
36	&	0,0110	&	0,0229	&	0,0426	&	0,0044	&	0,0090	&	0,0228	&	0,0178	&	\textbf{0,0390}	&	\textbf{0,0642}	&	0,1426	&	0,1430	&	\textbf{0,2514}	&	0,0166	&	\textbf{0,0387}	&	\textbf{0,0643}	\\
	\bottomrule
\end{tabular}
 \end{adjustwidth}
\caption[Wyniki badań miar dwuelementowych dla korpusu \emph{KIPI} podzielonego na 20 części, i poddanego dyspersji miarą TF-IDF, część 1]{Wyniki badań miar dwuelementowych dla korpusu \emph{KIPI} podzielonego na 20 części, i poddanego dyspersji miarą TF-IDF, część 1.}
\label{KIPI_TFIDF_20_part_1}
\end{table}

\begin{table}[htp!]
\centering
\footnotesize\setlength{\tabcolsep}{2.5pt}
 \begin{adjustwidth}{-2cm}{}
\begin{tabular}{ l | *{15}{| r}}
	\toprule 
	\textbf{95\%} &	\textbf{16}	&	\textbf{17}	&	\textbf{18}	&	\textbf{19}	&	\textbf{20}	&	\textbf{21}	&	\textbf{22}	&	\textbf{23}	&	\textbf{24}	&	\textbf{25}	&	\textbf{26}	&	\textbf{27}	&	\textbf{28}	&	\textbf{29}	&	\textbf{30}	\\
	\midrule
1	&	0,1429	&	0,1421	&	0,2136	&	0,0123	&	0,0173	&	0,0286	&	0,1048	&	0,1051	&	0,1846	&	0,0123	&	0,0172	&	0,0286	&	0,1055	&	0,1044	&	0,1839	\\
2	&	0,0266	&	0,0266	&	0,0562	&	0,0025	&	0,0036	&	0,0073	&	0,0276	&	0,0280	&	0,0634	&	0,0043	&	0,0056	&	0,0099	&	0,0318	&	0,0317	&	0,0665	\\
3	&	0,0134	&	0,0138	&	0,1163	&	0,0015	&	0,0031	&	0,0271	&	0,0111	&	0,0111	&	0,0943	&	0,0013	&	0,0027	&	0,0261	&	0,0110	&	0,0112	&	0,1044	\\
4	&	0,0428	&	0,0426	&	0,1775	&	0,0031	&	0,0061	&	0,0311	&	0,0223	&	0,0231	&	0,1261	&	0,0029	&	0,0060	&	0,0311	&	0,0278	&	0,0277	&	0,1458	\\
5	&	0,0428	&	0,0426	&	0,1775	&	0,0031	&	0,0061	&	0,0311	&	0,0223	&	0,0231	&	0,1261	&	0,0029	&	0,0060	&	0,0311	&	0,0278	&	0,0277	&	0,1458	\\
6	&	0,0474	&	0,0472	&	0,1935	&	0,0030	&	0,0082	&	0,0378	&	0,0288	&	0,0292	&	0,1471	&	0,0025	&	0,0070	&	0,0362	&	0,0305	&	0,0304	&	0,1609	\\
7	&	0,0229	&	0,0234	&	0,1511	&	0,0021	&	0,0049	&	0,0342	&	0,0170	&	0,0172	&	0,1240	&	0,0017	&	0,0040	&	0,0312	&	0,0169	&	0,0171	&	0,1296	\\
8	&	0,0336	&	0,0339	&	0,1681	&	0,0027	&	0,0065	&	0,0348	&	0,0226	&	0,0229	&	0,1318	&	0,0021	&	0,0054	&	0,0331	&	0,0230	&	0,0231	&	0,1420	\\
9	&	0,0125	&	0,0127	&	0,1074	&	0,0016	&	0,0025	&	0,0222	&	0,0089	&	0,0091	&	0,0817	&	0,0015	&	0,0024	&	0,0221	&	0,0102	&	0,0103	&	0,0942	\\
10	&	0,0125	&	0,0127	&	0,1074	&	0,0016	&	0,0025	&	0,0222	&	0,0089	&	0,0091	&	0,0817	&	0,0015	&	0,0024	&	0,0221	&	0,0102	&	0,0103	&	0,0942	\\
11	&	0,1194	&	0,1182	&	0,2359	&	0,0063	&	0,0140	&	0,0376	&	0,0590	&	0,0607	&	0,1755	&	0,0059	&	0,0134	&	0,0368	&	0,0702	&	0,0692	&	0,1915	\\
12	&	0,0226	&	0,0230	&	0,1471	&	0,0021	&	0,0047	&	0,0316	&	0,0164	&	0,0166	&	0,1149	&	0,0017	&	0,0040	&	0,0306	&	0,0168	&	0,0170	&	0,1277	\\
13	&	\textbf{0,1545}	&	\textbf{0,1529}	&	\textbf{0,2646}	&	0,0127	&	0,0211	&	0,0430	&	0,0993	&	0,0984	&	\textbf{0,2176}	&	0,0090	&	0,0162	&	\textbf{0,0407}	&	0,0873	&	0,0852	&	\textbf{0,2208}	\\
14	&	0,0474	&	0,0472	&	0,1935	&	0,0030	&	0,0082	&	0,0378	&	0,0288	&	0,0292	&	0,1471	&	0,0025	&	0,0070	&	0,0362	&	0,0305	&	0,0304	&	0,1609	\\
15	&	0,0964	&	0,0949	&	0,2376	&	0,0046	&	0,0138	&	0,0427	&	0,0522	&	0,0531	&	0,1828	&	0,0038	&	0,0116	&	\textbf{0,0400}	&	0,0555	&	0,0546	&	0,1938	\\
16	&	0,1487	&	0,1479	&	0,2345	&	0,0143	&	0,0200	&	0,0349	&	0,1011	&	0,1007	&	0,1953	&	0,0113	&	0,0165	&	0,0355	&	0,0919	&	0,0902	&	0,2037	\\
17	&	0,0466	&	0,0463	&	0,1930	&	0,0030	&	0,0080	&	0,0377	&	0,0284	&	0,0288	&	0,1468	&	0,0024	&	0,0066	&	0,0361	&	0,0291	&	0,0289	&	0,1601	\\
18	&	0,0470	&	0,0469	&	0,1933	&	0,0030	&	0,0081	&	0,0378	&	0,0287	&	0,0292	&	0,1471	&	0,0026	&	0,0073	&	0,0365	&	0,0318	&	0,0320	&	0,1624	\\
19	&	0,0465	&	0,0463	&	0,1930	&	0,0030	&	0,0080	&	0,0377	&	0,0284	&	0,0288	&	0,1468	&	0,0024	&	0,0066	&	0,0361	&	0,0291	&	0,0289	&	0,1601	\\
20	&	\textbf{0,1534}	&	\textbf{0,1513}	&	\textbf{0,2688}	&	0,0116	&	0,0208	&	0,0441	&	0,1004	&	0,1005	&	\textbf{0,2212}	&	0,0091	&	0,0152	&	\textbf{0,0402}	&	0,0781	&	0,0749	&	\textbf{0,2177}	\\
21	&	0,1015	&	0,1000	&	0,2406	&	0,0048	&	0,0143	&	0,0429	&	0,0550	&	0,0559	&	0,1857	&	0,0039	&	0,0121	&	\textbf{0,0402}	&	0,0583	&	0,0573	&	0,1964	\\
22	&	0,1064	&	0,1048	&	0,2434	&	0,0050	&	0,0149	&	0,0432	&	0,0577	&	0,0586	&	0,1885	&	0,0041	&	0,0126	&	\textbf{0,0404}	&	0,0611	&	0,0600	&	0,1987	\\
23	&	0,1111	&	0,1095	&	0,2459	&	0,0052	&	0,0154	&	0,0433	&	0,0604	&	0,0614	&	0,1910	&	0,0043	&	0,0130	&	\textbf{0,0405}	&	0,0638	&	0,0627	&	0,2009	\\
24	&	0,1156	&	0,1140	&	0,2482	&	0,0054	&	0,0159	&	0,0435	&	0,0631	&	0,0641	&	0,1934	&	0,0045	&	0,0134	&	\textbf{0,0406}	&	0,0665	&	0,0653	&	0,2029	\\
25	&	0,1198	&	0,1182	&	0,2503	&	0,0057	&	0,0164	&	0,0436	&	0,0657	&	0,0667	&	0,1956	&	0,0046	&	0,0138	&	\textbf{0,0407}	&	0,0690	&	0,0678	&	0,2048	\\
26	&	0,1237	&	0,1221	&	0,2521	&	0,0059	&	0,0169	&	0,0437	&	0,0683	&	0,0693	&	0,1977	&	0,0048	&	0,0142	&	\textbf{0,0408}	&	0,0715	&	0,0703	&	0,2065	\\
27	&	0,1273	&	0,1258	&	0,2538	&	0,0061	&	0,0173	&	0,0438	&	0,0708	&	0,0718	&	0,1995	&	0,0050	&	0,0146	&	\textbf{0,0408}	&	0,0739	&	0,0727	&	0,2080	\\
28	&	0,1307	&	0,1292	&	0,2552	&	0,0064	&	0,0177	&	0,0438	&	0,0732	&	0,0742	&	0,2013	&	0,0052	&	0,0150	&	\textbf{0,0408}	&	0,0762	&	0,0749	&	0,2094	\\
29	&	0,1338	&	0,1323	&	\textbf{0,2565}	&	0,0066	&	0,0181	&	0,0438	&	0,0755	&	0,0765	&	0,2028	&	0,0054	&	0,0153	&	\textbf{0,0408}	&	0,0784	&	0,0771	&	\textbf{0,2106}	\\
30	&	0,1366	&	0,1352	&	\textbf{0,2576}	&	0,0069	&	0,0185	&	0,0438	&	0,0778	&	0,0787	&	0,2043	&	0,0056	&	0,0156	&	\textbf{0,0408}	&	0,0805	&	0,0792	&	\textbf{0,2118}	\\
31	&	0,1392	&	0,1379	&	\textbf{0,2586}	&	0,0071	&	0,0188	&	0,0437	&	0,0799	&	0,0808	&	0,2056	&	0,0058	&	0,0159	&	\textbf{0,0408}	&	0,0824	&	0,0811	&	\textbf{0,2128}	\\
32	&	0,1415	&	0,1403	&	\textbf{0,2594}	&	0,0073	&	0,0191	&	0,0437	&	0,0819	&	0,0828	&	0,2067	&	0,0060	&	0,0162	&	\textbf{0,0407}	&	0,0843	&	0,0830	&	\textbf{0,2137}	\\
33	&	0,1437	&	0,1425	&	\textbf{0,2601}	&	0,0076	&	0,0194	&	0,0436	&	0,0838	&	0,0847	&	0,2078	&	0,0062	&	0,0164	&	\textbf{0,0407}	&	0,0860	&	0,0847	&	\textbf{0,2145}	\\
34	&	0,1456	&	0,1444	&	\textbf{0,2607}	&	0,0078	&	0,0196	&	0,0436	&	0,0856	&	0,0865	&	0,2088	&	0,0064	&	0,0167	&	\textbf{0,0406}	&	0,0876	&	0,0863	&	\textbf{0,2152}	\\
35	&	0,1473	&	0,1463	&	\textbf{0,2612}	&	0,0081	&	0,0198	&	0,0435	&	0,0873	&	0,0882	&	0,2096	&	0,0066	&	0,0169	&	\textbf{0,0406}	&	0,0892	&	0,0879	&	\textbf{0,2158}	\\
36	&	0,1489	&	0,1479	&	\textbf{0,2616}	&	0,0083	&	0,0201	&	0,0434	&	0,0889	&	0,0897	&	\textbf{0,2104}	&	0,0068	&	0,0171	&	\textbf{0,0405}	&	0,0906	&	0,0893	&	\textbf{0,2163}	\\
	\bottomrule
\end{tabular}
 \end{adjustwidth}
\caption[Wyniki badań miar dwuelementowych dla korpusu \emph{KIPI} podzielonego na 20 części, i poddanego dyspersji miarą TF-IDF, część 2]{Wyniki badań miar dwuelementowych dla korpusu \emph{KIPI} podzielonego na 20 części, i poddanego dyspersji miarą TF-IDF, część 2.}
\label{KIPI_TFIDF_20_part_2}
\end{table}

\begin{table}[htp!]
\centering
\footnotesize\setlength{\tabcolsep}{2.5pt}
 \begin{adjustwidth}{-2cm}{}
\begin{tabular}{ l | *{15}{| r}}
	\toprule 
	\textbf{95\%} &	\textbf{1}	&	\textbf{2}	&	\textbf{3}	&	\textbf{4}	&	\textbf{5}	&	\textbf{6}	&	\textbf{7}	&	\textbf{8}	&	\textbf{9}	&	\textbf{10}	&	\textbf{11}	&	\textbf{12}	&	\textbf{13}	&	\textbf{14}	&	\textbf{15}	\\
	\midrule
37	&	0,0112	&	0,0230	&	0,0423	&	0,0045	&	0,0092	&	0,0227	&	0,0183	&	\textbf{0,0392}	&	\textbf{0,0641}	&	0,1444	&	0,1447	&	\textbf{0,2521}	&	0,0170	&	\textbf{0,0388}	&	\textbf{0,0642}	\\
38	&	0,0114	&	\textbf{0,0231}	&	0,0421	&	0,0046	&	0,0093	&	0,0226	&	0,0187	&	\textbf{0,0394}	&	\textbf{0,0640}	&	0,1460	&	0,1462	&	\textbf{0,2527}	&	0,0175	&	\textbf{0,0390}	&	\textbf{0,0640}	\\
39	&	0,0116	&	\textbf{0,0232}	&	0,0418	&	0,0047	&	0,0094	&	0,0225	&	0,0192	&	\textbf{0,0396}	&	\textbf{0,0639}	&	0,1474	&	0,1476	&	\textbf{0,2532}	&	0,0179	&	\textbf{0,0391}	&	\textbf{0,0638}	\\
40	&	0,0117	&	\textbf{0,0233}	&	0,0416	&	0,0048	&	0,0095	&	0,0224	&	0,0196	&	\textbf{0,0397}	&	\textbf{0,0637}	&	\textbf{0,1487}	&	\textbf{0,1488}	&	\textbf{0,2536}	&	0,0183	&	\textbf{0,0393}	&	\textbf{0,0637}	\\
41	&	0,0119	&	\textbf{0,0233}	&	0,0413	&	0,0049	&	0,0096	&	0,0223	&	0,0200	&	\textbf{0,0398}	&	\textbf{0,0636}	&	\textbf{0,1499}	&	\textbf{0,1500}	&	\textbf{0,2540}	&	0,0187	&	\textbf{0,0394}	&	\textbf{0,0635}	\\
42	&	0,0120	&	\textbf{0,0234}	&	0,0411	&	0,0050	&	0,0097	&	0,0222	&	0,0204	&	\textbf{0,0399}	&	\textbf{0,0634}	&	\textbf{0,1509}	&	\textbf{0,1510}	&	\textbf{0,2543}	&	0,0190	&	\textbf{0,0394}	&	\textbf{0,0633}	\\
43	&	0,0122	&	\textbf{0,0234}	&	0,0408	&	0,0051	&	0,0098	&	0,0220	&	0,0207	&	\textbf{0,0400}	&	\textbf{0,0633}	&	\textbf{0,1519}	&	\textbf{0,1519}	&	\textbf{0,2545}	&	0,0194	&	\textbf{0,0395}	&	\textbf{0,0632}	\\
44	&	0,0123	&	\textbf{0,0234}	&	0,0406	&	0,0052	&	0,0099	&	0,0219	&	0,0211	&	\textbf{0,0401}	&	\textbf{0,0631}	&	\textbf{0,1528}	&	\textbf{0,1528}	&	\textbf{0,2547}	&	0,0198	&	\textbf{0,0396}	&	\textbf{0,0630}	\\
45	&	0,0124	&	\textbf{0,0234}	&	0,0404	&	0,0053	&	0,0099	&	0,0218	&	0,0214	&	\textbf{0,0401}	&	0,0630	&	\textbf{0,1536}	&	\textbf{0,1535}	&	\textbf{0,2548}	&	0,0201	&	\textbf{0,0396}	&	\textbf{0,0628}	\\
46	&	0,0125	&	\textbf{0,0234}	&	0,0401	&	0,0053	&	0,0100	&	0,0217	&	0,0218	&	\textbf{0,0401}	&	0,0628	&	\textbf{0,1543}	&	\textbf{0,1542}	&	\textbf{0,2549}	&	0,0204	&	\textbf{0,0397}	&	0,0627	\\
47	&	0,0126	&	\textbf{0,0234}	&	0,0399	&	0,0054	&	0,0100	&	0,0216	&	0,0221	&	\textbf{0,0402}	&	0,0627	&	\textbf{0,1549}	&	\textbf{0,1548}	&	\textbf{0,2550}	&	0,0207	&	\textbf{0,0397}	&	0,0625	\\
48	&	0,0127	&	\textbf{0,0234}	&	0,0397	&	0,0055	&	0,0101	&	0,0215	&	0,0223	&	\textbf{0,0402}	&	0,0625	&	\textbf{0,1555}	&	\textbf{0,1553}	&	\textbf{0,2550}	&	0,0210	&	\textbf{0,0397}	&	0,0623	\\
49	&	0,0128	&	\textbf{0,0233}	&	0,0394	&	0,0055	&	\textbf{0,0101}	&	0,0214	&	0,0226	&	\textbf{0,0402}	&	0,0624	&	\textbf{0,1560}	&	\textbf{0,1558}	&	\textbf{0,2550}	&	0,0213	&	\textbf{0,0397}	&	0,0621	\\
50	&	0,0128	&	\textbf{0,0233}	&	0,0392	&	0,0056	&	\textbf{0,0102}	&	0,0213	&	0,0229	&	\textbf{0,0402}	&	0,0622	&	\textbf{0,1565}	&	\textbf{0,1563}	&	\textbf{0,2549}	&	0,0215	&	\textbf{0,0397}	&	0,0620	\\
51	&	\textbf{0,0149}	&	\textbf{0,0239}	&	0,0424	&	0,0048	&	0,0081	&	\textbf{0,0237}	&	\textbf{0,0256}	&	\textbf{0,0394}	&	\textbf{0,0632}	&	\textbf{0,1533}	&	\textbf{0,1520}	&	\textbf{0,2588}	&	0,0224	&	0,0371	&	\textbf{0,0639}	\\
52	&	0,0140	&	0,0228	&	0,0439	&	0,0038	&	0,0067	&	\textbf{0,0242}	&	0,0244	&	\textbf{0,0385}	&	\textbf{0,0643}	&	0,1482	&	0,1465	&	\textbf{0,2596}	&	0,0203	&	0,0351	&	\textbf{0,0650}	\\
53	&	0,0123	&	0,0207	&	\textbf{0,0452}	&	0,0028	&	0,0050	&	\textbf{0,0244}	&	0,0227	&	0,0370	&	\textbf{0,0652}	&	0,1403	&	0,1381	&	\textbf{0,2594}	&	0,0176	&	0,0321	&	\textbf{0,0658}	\\
54	&	0,0101	&	0,0176	&	\textbf{0,0461}	&	0,0021	&	0,0035	&	\textbf{0,0242}	&	0,0204	&	0,0346	&	\textbf{0,0660}	&	0,1285	&	0,1260	&	\textbf{0,2575}	&	0,0143	&	0,0278	&	\textbf{0,0661}	\\
55	&	0,0076	&	0,0138	&	\textbf{0,0464}	&	0,0015	&	0,0025	&	\textbf{0,0235}	&	0,0176	&	0,0312	&	\textbf{0,0665}	&	0,1128	&	0,1100	&	\textbf{0,2536}	&	0,0108	&	0,0225	&	\textbf{0,0655}	\\
56	&	0,0054	&	0,0100	&	\textbf{0,0458}	&	0,0012	&	0,0019	&	0,0224	&	0,0143	&	0,0268	&	\textbf{0,0663}	&	0,0931	&	0,0903	&	0,2464	&	0,0076	&	0,0165	&	\textbf{0,0636}	\\
57	&	0,0036	&	0,0065	&	\textbf{0,0442}	&	0,0009	&	0,0015	&	0,0210	&	0,0109	&	0,0215	&	\textbf{0,0651}	&	0,0705	&	0,0677	&	0,2353	&	0,0053	&	0,0111	&	0,0606	\\
58	&	0,0025	&	0,0043	&	0,0418	&	0,0008	&	0,0012	&	0,0193	&	0,0079	&	0,0158	&	0,0630	&	0,0477	&	0,0456	&	0,2202	&	0,0038	&	0,0074	&	0,0566	\\
59	&	0,0019	&	0,0031	&	0,0388	&	0,0007	&	0,0010	&	0,0176	&	0,0056	&	0,0108	&	0,0600	&	0,0304	&	0,0294	&	0,2024	&	0,0029	&	0,0053	&	0,0517	\\
60	&	0,0015	&	0,0024	&	0,0352	&	0,0006	&	0,0009	&	0,0161	&	0,0042	&	0,0075	&	0,0560	&	0,0208	&	0,0204	&	0,1814	&	0,0023	&	0,0042	&	0,0469	\\
61	&	0,0012	&	0,0020	&	0,0316	&	0,0006	&	0,0008	&	0,0148	&	0,0032	&	0,0056	&	0,0514	&	0,0155	&	0,0154	&	0,1597	&	0,0020	&	0,0035	&	0,0424	\\
62	&	0,0011	&	0,0017	&	0,0283	&	0,0005	&	0,0008	&	0,0136	&	0,0026	&	0,0044	&	0,0466	&	0,0122	&	0,0122	&	0,1397	&	0,0018	&	0,0030	&	0,0386	\\
63	&	0,0010	&	0,0015	&	0,0255	&	0,0005	&	0,0007	&	0,0126	&	0,0022	&	0,0037	&	0,0424	&	0,0102	&	0,0102	&	0,1229	&	0,0016	&	0,0027	&	0,0352	\\
64	&	0,0009	&	0,0013	&	0,0230	&	0,0005	&	0,0007	&	0,0118	&	0,0020	&	0,0032	&	0,0387	&	0,0088	&	0,0089	&	0,1092	&	0,0015	&	0,0025	&	0,0325	\\
65	&	0,0009	&	0,0013	&	0,0210	&	0,0005	&	0,0007	&	0,0111	&	0,0018	&	0,0028	&	0,0356	&	0,0079	&	0,0079	&	0,0982	&	0,0014	&	0,0023	&	0,0304	\\
66	&	0,0008	&	0,0012	&	0,0193	&	0,0005	&	0,0007	&	0,0105	&	0,0017	&	0,0026	&	0,0329	&	0,0073	&	0,0073	&	0,0893	&	0,0014	&	0,0022	&	0,0288	\\
67	&	0,0008	&	0,0011	&	0,0180	&	0,0005	&	0,0007	&	0,0100	&	0,0016	&	0,0024	&	0,0307	&	0,0069	&	0,0069	&	0,0823	&	0,0013	&	0,0021	&	0,0275	\\
68	&	0,0008	&	0,0011	&	0,0169	&	0,0005	&	0,0007	&	0,0096	&	0,0015	&	0,0023	&	0,0291	&	0,0066	&	0,0066	&	0,0774	&	0,0013	&	0,0021	&	0,0264	\\
69	&	0,0008	&	0,0011	&	0,0160	&	0,0005	&	0,0007	&	0,0092	&	0,0015	&	0,0022	&	0,0278	&	0,0064	&	0,0064	&	0,0735	&	0,0013	&	0,0020	&	0,0255	\\
70	&	0,0007	&	0,0011	&	0,0152	&	0,0005	&	0,0007	&	0,0089	&	0,0014	&	0,0022	&	0,0268	&	0,0063	&	0,0063	&	0,0707	&	0,0013	&	0,0020	&	0,0246	\\
71	&	0,0061	&	0,0087	&	0,0150	&	0,0068	&	0,0100	&	0,0147	&	0,0129	&	0,0174	&	0,0267	&	0,0786	&	0,0784	&	0,1281	&	0,0129	&	0,0176	&	0,0267	\\
72	&	0,0083	&	0,0123	&	0,0188	&	\textbf{0,0071}	&	\textbf{0,0106}	&	0,0154	&	0,0173	&	0,0234	&	0,0333	&	0,1075	&	0,1073	&	0,1624	&	0,0173	&	0,0237	&	0,0333	\\
	\bottomrule
\end{tabular}
 \end{adjustwidth}
\caption[Wyniki badań miar dwuelementowych dla korpusu \emph{KIPI} podzielonego na 20 części, i poddanego dyspersji miarą TF-IDF, część 3]{Wyniki badań miar dwuelementowych dla korpusu \emph{KIPI} podzielonego na 20 części, i poddanego dyspersji miarą TF-IDF, część 3.}
\label{KIPI_TFIDF_20_part_3}
\end{table}

\begin{table}[htp!]
\centering
\footnotesize\setlength{\tabcolsep}{2.5pt}
 \begin{adjustwidth}{-2cm}{}
\begin{tabular}{ l | *{15}{| r}}
	\toprule 
	\textbf{95\%} &	\textbf{16}	&	\textbf{17}	&	\textbf{18}	&	\textbf{19}	&	\textbf{20}	&	\textbf{21}	&	\textbf{22}	&	\textbf{23}	&	\textbf{24}	&	\textbf{25}	&	\textbf{26}	&	\textbf{27}	&	\textbf{28}	&	\textbf{29}	&	\textbf{30}	\\
	\midrule																							
37	&	0,1504	&	0,1494	&	\textbf{0,2619}	&	0,0085	&	0,0202	&	0,0433	&	0,0904	&	0,0912	&	\textbf{0,2111}	&	0,0070	&	0,0173	&	\textbf{0,0404}	&	0,0919	&	0,0907	&	\textbf{0,2168}	\\
38	&	0,1516	&	0,1507	&	\textbf{0,2621}	&	0,0087	&	0,0204	&	0,0432	&	0,0918	&	0,0926	&	\textbf{0,2118}	&	0,0072	&	0,0174	&	\textbf{0,0403}	&	0,0932	&	0,0919	&	\textbf{0,2172}	\\
39	&	\textbf{0,1528}	&	\textbf{0,1519}	&	\textbf{0,2623}	&	0,0090	&	0,0206	&	0,0430	&	0,0931	&	0,0939	&	\textbf{0,2123}	&	0,0074	&	0,0176	&	\textbf{0,0402}	&	0,0943	&	0,0931	&	\textbf{0,2176}	\\
40	&	\textbf{0,1538}	&	\textbf{0,1530}	&	\textbf{0,2624}	&	0,0092	&	0,0207	&	0,0429	&	0,0944	&	0,0951	&	\textbf{0,2128}	&	0,0076	&	0,0178	&	\textbf{0,0401}	&	0,0954	&	0,0942	&	\textbf{0,2179}	\\
41	&	\textbf{0,1548}	&	\textbf{0,1540}	&	\textbf{0,2625}	&	0,0094	&	0,0208	&	0,0428	&	0,0955	&	0,0962	&	\textbf{0,2132}	&	0,0078	&	0,0179	&	\textbf{0,0400}	&	0,0964	&	0,0952	&	\textbf{0,2181}	\\
42	&	\textbf{0,1556}	&	\textbf{0,1548}	&	\textbf{0,2625}	&	0,0096	&	0,0209	&	0,0427	&	0,0966	&	0,0973	&	\textbf{0,2136}	&	0,0080	&	0,0180	&	\textbf{0,0399}	&	0,0973	&	0,0962	&	\textbf{0,2183}	\\
43	&	\textbf{0,1564}	&	\textbf{0,1556}	&	\textbf{0,2624}	&	0,0098	&	0,0210	&	0,0425	&	0,0976	&	0,0982	&	\textbf{0,2139}	&	0,0081	&	0,0181	&	0,0398	&	0,0982	&	0,0970	&	\textbf{0,2184}	\\
44	&	\textbf{0,1570}	&	\textbf{0,1563}	&	\textbf{0,2624}	&	0,0100	&	0,0211	&	0,0424	&	0,0985	&	0,0992	&	\textbf{0,2142}	&	0,0083	&	0,0182	&	0,0397	&	0,0990	&	0,0979	&	\textbf{0,2186}	\\
45	&	\textbf{0,1576}	&	\textbf{0,1569}	&	\textbf{0,2622}	&	0,0102	&	0,0212	&	0,0423	&	0,0994	&	0,1000	&	\textbf{0,2144}	&	0,0085	&	0,0183	&	0,0396	&	0,0998	&	0,0986	&	\textbf{0,2186}	\\
46	&	\textbf{0,1582}	&	\textbf{0,1575}	&	\textbf{0,2621}	&	0,0103	&	0,0213	&	0,0421	&	0,1002	&	0,1008	&	\textbf{0,2146}	&	0,0086	&	0,0184	&	0,0395	&	0,1005	&	0,0993	&	\textbf{0,2187}	\\
47	&	\textbf{0,1587}	&	\textbf{0,1580}	&	\textbf{0,2620}	&	0,0105	&	0,0213	&	0,0420	&	0,1009	&	0,1015	&	\textbf{0,2148}	&	0,0088	&	0,0185	&	0,0394	&	0,1011	&	0,1000	&	\textbf{0,2187}	\\
48	&	\textbf{0,1591}	&	\textbf{0,1584}	&	\textbf{0,2618}	&	0,0107	&	0,0214	&	0,0418	&	0,1016	&	0,1022	&	\textbf{0,2149}	&	0,0089	&	0,0185	&	0,0393	&	0,1017	&	0,1006	&	\textbf{0,2187}	\\
49	&	\textbf{0,1595}	&	\textbf{0,1588}	&	\textbf{0,2615}	&	0,0108	&	0,0214	&	0,0417	&	0,1023	&	0,1028	&	\textbf{0,2150}	&	0,0091	&	0,0186	&	0,0392	&	0,1023	&	0,1012	&	\textbf{0,2187}	\\
50	&	\textbf{0,1598}	&	\textbf{0,1592}	&	\textbf{0,2613}	&	0,0110	&	0,0214	&	0,0416	&	0,1029	&	0,1034	&	\textbf{0,2150}	&	0,0092	&	0,0187	&	0,0391	&	0,1028	&	0,1018	&	\textbf{0,2186}	\\
51	&	0,1485	&	0,1464	&	\textbf{0,2661}	&	0,0119	&	0,0204	&	0,0440	&	0,0943	&	0,0931	&	\textbf{0,2181}	&	0,0078	&	0,0147	&	\textbf{0,0415}	&	0,0781	&	0,0759	&	\textbf{0,2203}	\\
52	&	0,1386	&	0,1360	&	\textbf{0,2665}	&	0,0108	&	0,0192	&	\textbf{0,0451}	&	0,0875	&	0,0860	&	\textbf{0,2178}	&	0,0063	&	0,0126	&	\textbf{0,0420}	&	0,0656	&	0,0635	&	\textbf{0,2179}	\\
53	&	0,1236	&	0,1206	&	\textbf{0,2648}	&	0,0095	&	0,0176	&	\textbf{0,0460}	&	0,0785	&	0,0768	&	\textbf{0,2163}	&	0,0048	&	0,0101	&	\textbf{0,0420}	&	0,0505	&	0,0487	&	\textbf{0,2124}	\\
54	&	0,1029	&	0,1001	&	\textbf{0,2598}	&	0,0079	&	0,0154	&	\textbf{0,0465}	&	0,0671	&	0,0653	&	\textbf{0,2131}	&	0,0035	&	0,0074	&	\textbf{0,0413}	&	0,0350	&	0,0339	&	0,2028	\\
55	&	0,0783	&	0,0762	&	0,2502	&	0,0062	&	0,0127	&	\textbf{0,0465}	&	0,0537	&	0,0518	&	0,2073	&	0,0026	&	0,0052	&	0,0398	&	0,0233	&	0,0230	&	0,1899	\\
56	&	0,0521	&	0,0511	&	0,2353	&	0,0047	&	0,0096	&	\textbf{0,0458}	&	0,0395	&	0,0378	&	0,1985	&	0,0020	&	0,0038	&	0,0375	&	0,0166	&	0,0166	&	0,1742	\\
57	&	0,0314	&	0,0315	&	0,2163	&	0,0035	&	0,0069	&	0,0442	&	0,0274	&	0,0264	&	0,1867	&	0,0016	&	0,0029	&	0,0346	&	0,0126	&	0,0128	&	0,1566	\\
58	&	0,0202	&	0,0206	&	0,1940	&	0,0026	&	0,0050	&	0,0419	&	0,0198	&	0,0193	&	0,1726	&	0,0013	&	0,0024	&	0,0316	&	0,0102	&	0,0105	&	0,1390	\\
59	&	0,0147	&	0,0151	&	0,1701	&	0,0021	&	0,0038	&	0,0389	&	0,0151	&	0,0149	&	0,1565	&	0,0011	&	0,0021	&	0,0287	&	0,0086	&	0,0088	&	0,1227	\\
60	&	0,0114	&	0,0119	&	0,1478	&	0,0017	&	0,0031	&	0,0357	&	0,0120	&	0,0119	&	0,1396	&	0,0010	&	0,0018	&	0,0262	&	0,0076	&	0,0078	&	0,1093	\\
61	&	0,0094	&	0,0098	&	0,1287	&	0,0014	&	0,0026	&	0,0324	&	0,0100	&	0,0100	&	0,1240	&	0,0009	&	0,0017	&	0,0240	&	0,0069	&	0,0072	&	0,0983	\\
62	&	0,0081	&	0,0085	&	0,1133	&	0,0013	&	0,0022	&	0,0294	&	0,0087	&	0,0087	&	0,1102	&	0,0009	&	0,0016	&	0,0222	&	0,0065	&	0,0067	&	0,0894	\\
63	&	0,0074	&	0,0077	&	0,1008	&	0,0012	&	0,0020	&	0,0270	&	0,0077	&	0,0078	&	0,0994	&	0,0009	&	0,0015	&	0,0208	&	0,0063	&	0,0065	&	0,0826	\\
64	&	0,0069	&	0,0071	&	0,0913	&	0,0011	&	0,0018	&	0,0249	&	0,0071	&	0,0072	&	0,0906	&	0,0008	&	0,0014	&	0,0196	&	0,0061	&	0,0063	&	0,0773	\\
65	&	0,0065	&	0,0068	&	0,0843	&	0,0010	&	0,0017	&	0,0231	&	0,0068	&	0,0068	&	0,0832	&	0,0008	&	0,0014	&	0,0187	&	0,0060	&	0,0061	&	0,0731	\\
66	&	0,0063	&	0,0065	&	0,0790	&	0,0010	&	0,0017	&	0,0217	&	0,0065	&	0,0065	&	0,0773	&	0,0008	&	0,0014	&	0,0180	&	0,0059	&	0,0060	&	0,0701	\\
67	&	0,0062	&	0,0063	&	0,0750	&	0,0009	&	0,0016	&	0,0206	&	0,0063	&	0,0064	&	0,0730	&	0,0008	&	0,0013	&	0,0174	&	0,0058	&	0,0060	&	0,0679	\\
68	&	0,0060	&	0,0062	&	0,0721	&	0,0009	&	0,0016	&	0,0196	&	0,0062	&	0,0062	&	0,0696	&	0,0008	&	0,0013	&	0,0168	&	0,0058	&	0,0059	&	0,0662	\\
69	&	0,0060	&	0,0061	&	0,0699	&	0,0009	&	0,0015	&	0,0189	&	0,0061	&	0,0061	&	0,0671	&	0,0008	&	0,0013	&	0,0164	&	0,0058	&	0,0059	&	0,0648	\\
70	&	0,0059	&	0,0060	&	0,0681	&	0,0009	&	0,0015	&	0,0183	&	0,0061	&	0,0061	&	0,0653	&	0,0008	&	0,0013	&	0,0160	&	0,0057	&	0,0058	&	0,0636	\\
71	&	0,0799	&	0,0794	&	0,1279	&	\textbf{0,0157}	&	\textbf{0,0226}	&	0,0319	&	\textbf{0,1190}	&	\textbf{0,1194}	&	0,1707	&	\textbf{0,0157}	&	\textbf{0,0225}	&	0,0320	&	\textbf{0,1199}	&	\textbf{0,1186}	&	0,1700	\\
72	&	0,1091	&	0,1084	&	0,1622	&	\textbf{0,0157}	&	\textbf{0,0226}	&	0,0325	&	\textbf{0,1218}	&	\textbf{0,1222}	&	0,1799	&	\textbf{0,0157}	&	\textbf{0,0225}	&	0,0326	&	\textbf{0,1227}	&	\textbf{0,1214}	&	0,1792	\\
	\bottomrule
\end{tabular}
 \end{adjustwidth}
\caption[Wyniki badań miar dwuelementowych dla korpusu \emph{KIPI} podzielonego na 20 części, i poddanego dyspersji miarą TF-IDF, część 4]{Wyniki badań miar dwuelementowych dla korpusu \emph{KIPI} podzielonego na 20 części, i poddanego dyspersji miarą TF-IDF, część 4.}
\label{KIPI_TFIDF_20_part_4}
\end{table}

Dwukrotne zwiększenie ziarnistości spowodowało poprawę jakości generowanych rozwiązań w stosunku do wyników z poprzedniego badania w każdym przypadku.
Wzrost jakości sięgnął nawet około 50\% w stosunku do poprzedniego badania, ale taki przeskok nie spowodował nawet wyrównania wyników osiągniętych przy badaniu korpusu bez podziału i dyspersji.
Wykonanie dyspersji tą metodą spowodowało osiągnięcie wyników na poziomie od około 30\% do 60\% jakości wyników badań przeprowadzonych na korpusie \emph{KIPI} bez dyspersji.

\par
Najlepsze jakościowo wyniki zostały osiągnięte dla następujących miar asocjacyjnych: \emph{W Specific Correlation}, \emph{Specific Frequency Biased Mutual Dependency}, \emph{T-Score}, \emph{Loglikelihood}, \emph{W Order}, \emph{W Term Frequency Order} oraz zestaw miar \emph{Specific Exponential Correlation} i \emph{W Specific Exponential Correlation} dla pewnych wartości ich parametru.
Opisany zestaw funkcji uległ zmianie w stosunku do poprzedniego badania wpływu dyspersji na jakość wyników, ale w konkretny sposób, ponieważ nie pojawiła się żadna nowa funkcja w tym zestawie, a został on jedynie okrojony.
Pierwszą obserwacją jest to, że być może dobre wyniki pewnych funkcji w poprzednim badaniu były spowodowane jedynie ogólnie wynikami niskiej jakości.
Potwierdzeniem poprawności takiej obserwacji może być przykład miary \emph{Sorgenfrei}, która okazała się być jedną z najlepszych w poprzednim badaniu (jej wynik w poprzednim badaniu jest zbliżony do wyniku w tym badaniu), natomiast wynik funkcji \emph{Loglikelihood} jest zauważalnie lepszy.
Innymi słowy pojawienie się miary \emph{Sorgenfrei} w gronie najlepszych mogło być spowodowane ogólnym obniżeniem jakości wszystkich funkcji z wyjątkiem właśnie tej -- było jej łatwiej osiągnąć wynik pozwalający na uplasowanie się w czołówce.

\par
Niniejsze i poprzednie badanie pokazują, że taka metoda dyspersji nie spełnia oczekiwań w przypadku badanego korpusu \emph{KIPI}.
Nie można jednak jednoznacznie stwierdzić, że winna jest sama metoda.
Powodem takiej niemożności są informacje zamieszczone przez autora niniejszej pracy przy omawianiu poprzedniego badania, dyspersji korpusu podzielonego na 10 części, a mianowicie brak informacji o grupowaniu tematycznym tekstów składowych korpusu \emph{KIPI}.


\subsubsection{Wyniki badań jakości wyników dla miar asocjacyjnych po podpróbkowaniu klasy negatywnej do 80\%}
Różnica pomiędzy tym, a poprzednimi badaniami jest taka, że w tym przypadku krotki z korpusu zostały poddane podpróbkowaniu klasy negatywnej.
Proces ten został wykonany dla każdego z 6 zestawów danych opisanych w poprzedniej części pracy -- \emph{2R}, \emph{2W}, \emph{2R1H}, \emph{2W1H}, \emph{2RW} oraz \emph{2RW1H}.
Składy krotek zostały najpierw poddane ewentualnemu filtrowaniu, a dopiero później podpróbkowaniu w celu zapewnienia pożądanego stosunku liczby wyrażeń wielowyrazowych do wszystkich kandydatów.
Pożądany stosunek wyrażeń wielowyrazowych do wszystkich kandydatów został ustalony na poziomie 20\%, a idea dobrania takiej ich liczby pochodzi z artykułu Pavla Peciny i Pavla Schlesingera \cite{coling}.
Tak przygotowane zestawy krotek zostały wykorzystane w tej części badań.

\par
Cztery tabele \ref{KIPI_subsampled_20_part_1}, \ref{KIPI_subsampled_20_part_2}, \ref{KIPI_subsampled_20_part_3} oraz \ref{KIPI_subsampled_20_part_4} prezentują jakość wyników osiągniętych przez 72 funkcje w 30 różnych badaniach (30 zestawów danych pozyskanych z korpusu \emph{KIPI}).
Indeksy miar i typów badań pozostały takie same, jak w poprzednich badaniach.


\begin{table}[htp!]
\centering
\footnotesize\setlength{\tabcolsep}{2.5pt}
 \begin{adjustwidth}{-2cm}{}
\begin{tabular}{ l | *{15}{| r}}
	\toprule 
	\textbf{95\%} &	\textbf{1}	&	\textbf{2}	&	\textbf{3}	&	\textbf{4}	&	\textbf{5}	&	\textbf{6}	&	\textbf{7}	&	\textbf{8}	&	\textbf{9}	&	\textbf{10}	&	\textbf{11}	&	\textbf{12}	&	\textbf{13}	&	\textbf{14}	&	\textbf{15}	\\
	\midrule
1	&	0,7085	&	0,7200	&	\textbf{0,4872}	&	0,6583	&	0,6700	&	0,4657	&	\textbf{0,7588}	&	\textbf{0,7432}	&	\textbf{0,5387}	&	\textbf{0,9410}	&	\textbf{0,9218}	&	\textbf{0,8379}	&	\textbf{0,8312}	&	\textbf{0,8243}	&	\textbf{0,6595}	\\
2	&	\textbf{0,8174}	&	\textbf{0,8035}	&	0,4562	&	\textbf{0,8428}	&	\textbf{0,8389}	&	\textbf{0,5147}	&	\textbf{0,7262}	&	0,6764	&	0,3780	&	0,8473	&	0,8043	&	0,6263	&	\textbf{0,8218}	&	\textbf{0,7971}	&	0,5405	\\
3	&	0,0625	&	0,0601	&	0,1541	&	0,0541	&	0,0547	&	0,1465	&	0,0613	&	0,0655	&	0,1383	&	0,2473	&	0,2040	&	0,3796	&	0,0907	&	0,0992	&	0,1980	\\
4	&	0,1993	&	0,2128	&	0,3357	&	0,1624	&	0,1764	&	0,3092	&	0,1874	&	0,2258	&	0,2952	&	0,5560	&	0,5080	&	0,6511	&	0,2604	&	0,3145	&	0,3925	\\
5	&	0,1995	&	0,2128	&	0,3358	&	0,1624	&	0,1764	&	0,3090	&	0,1873	&	0,2259	&	0,2957	&	0,5556	&	0,5080	&	0,6513	&	0,2606	&	0,3145	&	0,3923	\\
6	&	0,1870	&	0,2079	&	0,3305	&	0,1468	&	0,1681	&	0,3018	&	0,1759	&	0,2201	&	0,2887	&	0,5393	&	0,4948	&	0,6437	&	0,2287	&	0,2893	&	0,3615	\\
7	&	0,1207	&	0,1250	&	0,2409	&	0,0910	&	0,0975	&	0,2201	&	0,1218	&	0,1390	&	0,2186	&	0,3992	&	0,3494	&	0,5334	&	0,1326	&	0,1561	&	0,2392	\\
8	&	0,1769	&	0,1808	&	0,2765	&	0,1313	&	0,1385	&	0,2491	&	0,1826	&	0,2005	&	0,2450	&	0,5175	&	0,4659	&	0,5793	&	0,1984	&	0,2273	&	0,2876	\\
9	&	0,0887	&	0,0817	&	0,1919	&	0,0751	&	0,0722	&	0,1786	&	0,0836	&	0,0862	&	0,1679	&	0,3037	&	0,2521	&	0,4380	&	0,1188	&	0,1245	&	0,2331	\\
10	&	0,0888	&	0,0816	&	0,1917	&	0,0751	&	0,0722	&	0,1785	&	0,0836	&	0,0863	&	0,1682	&	0,3037	&	0,2522	&	0,4380	&	0,1188	&	0,1245	&	0,2331	\\
11	&	0,4833	&	0,4908	&	0,4496	&	0,3933	&	0,4058	&	0,4190	&	0,5416	&	0,5548	&	0,4587	&	0,8656	&	0,8364	&	\textbf{0,8116}	&	0,6242	&	0,6491	&	0,5675	\\
12	&	0,0896	&	0,0902	&	0,2157	&	0,0750	&	0,0792	&	0,2022	&	0,0840	&	0,0950	&	0,1848	&	0,3174	&	0,2719	&	0,4749	&	0,1214	&	0,1412	&	0,2534	\\
13	&	0,5738	&	0,5841	&	0,4586	&	0,4659	&	0,4732	&	0,4170	&	0,6570	&	0,6518	&	0,5091	&	\textbf{0,9073}	&	\textbf{0,8825}	&	\textbf{0,8399}	&	0,7150	&	0,7165	&	0,6123	\\
14	&	0,1869	&	0,2080	&	0,3304	&	0,1468	&	0,1681	&	0,3018	&	0,1760	&	0,2203	&	0,2888	&	0,5394	&	0,4948	&	0,6438	&	0,2287	&	0,2894	&	0,3614	\\
15	&	0,4143	&	0,4389	&	0,4297	&	0,3113	&	0,3409	&	0,3957	&	0,4449	&	0,4806	&	0,4195	&	0,8168	&	0,7858	&	0,7865	&	0,4942	&	0,5542	&	0,5005	\\
16	&	0,6358	&	0,6454	&	\textbf{0,4630}	&	0,5293	&	0,5335	&	0,4154	&	0,7109	&	0,6988	&	\textbf{0,5283}	&	\textbf{0,9226}	&	\textbf{0,8995}	&	\textbf{0,8382}	&	0,7668	&	0,7613	&	\textbf{0,6403}	\\
17	&	0,1843	&	0,2054	&	0,3268	&	0,1448	&	0,1654	&	0,2975	&	0,1739	&	0,2183	&	0,2862	&	0,5364	&	0,4927	&	0,6402	&	0,2274	&	0,2874	&	0,3601	\\
18	&	0,1865	&	0,2073	&	0,3290	&	0,1462	&	0,1672	&	0,2999	&	0,1765	&	0,2201	&	0,2881	&	0,5390	&	0,4945	&	0,6428	&	0,2283	&	0,2887	&	0,3606	\\
19	&	0,1838	&	0,2049	&	0,3263	&	0,1445	&	0,1650	&	0,2968	&	0,1738	&	0,2182	&	0,2860	&	0,5362	&	0,4922	&	0,6403	&	0,2274	&	0,2873	&	0,3601	\\
20	&	0,5695	&	0,5833	&	0,4583	&	0,4768	&	0,4918	&	0,4247	&	0,6562	&	0,6539	&	0,5044	&	\textbf{0,9107}	&	\textbf{0,8875}	&	\textbf{0,8405}	&	0,7233	&	0,7293	&	0,6052	\\
21	&	0,4307	&	0,4541	&	0,4348	&	0,3262	&	0,3550	&	0,4011	&	0,4645	&	0,4966	&	0,4276	&	0,8279	&	0,7976	&	0,7931	&	0,5163	&	0,5711	&	0,5102	\\
22	&	0,4456	&	0,4677	&	0,4393	&	0,3402	&	0,3681	&	0,4060	&	0,4819	&	0,5108	&	0,4350	&	0,8373	&	0,8077	&	\textbf{0,7987}	&	0,5361	&	0,5861	&	0,5192	\\
23	&	0,4590	&	0,4800	&	0,4433	&	0,3534	&	0,3803	&	0,4103	&	0,4975	&	0,5235	&	0,4417	&	0,8453	&	0,8163	&	\textbf{0,8036}	&	0,5541	&	0,5995	&	0,5274	\\
24	&	0,4709	&	0,4911	&	0,4468	&	0,3655	&	0,3915	&	0,4142	&	0,5115	&	0,5348	&	0,4477	&	0,8521	&	0,8237	&	\textbf{0,8079}	&	0,5701	&	0,6115	&	0,5349	\\
25	&	0,4818	&	0,5012	&	0,4500	&	0,3768	&	0,4019	&	0,4177	&	0,5241	&	0,5450	&	0,4532	&	0,8581	&	0,8300	&	\textbf{0,8116}	&	0,5844	&	0,6223	&	0,5418	\\
26	&	0,4917	&	0,5104	&	0,4528	&	0,3872	&	0,4115	&	0,4208	&	0,5353	&	0,5542	&	0,4582	&	0,8633	&	0,8356	&	\textbf{0,8149}	&	0,5972	&	0,6320	&	0,5481	\\
27	&	0,5007	&	0,5188	&	0,4552	&	0,3968	&	0,4205	&	0,4236	&	0,5454	&	0,5626	&	0,4628	&	0,8678	&	0,8405	&	\textbf{0,8177}	&	0,6087	&	0,6407	&	0,5539	\\
28	&	0,5089	&	0,5266	&	0,4574	&	0,4058	&	0,4290	&	0,4261	&	0,5547	&	0,5702	&	0,4669	&	0,8719	&	0,8449	&	\textbf{0,8202}	&	0,6192	&	0,6487	&	0,5592	\\
29	&	0,5165	&	0,5337	&	0,4594	&	0,4142	&	0,4369	&	0,4284	&	0,5631	&	0,5772	&	0,4708	&	0,8755	&	0,8488	&	\textbf{0,8225}	&	0,6285	&	0,6559	&	0,5640	\\
30	&	0,5235	&	0,5404	&	0,4612	&	0,4221	&	0,4443	&	0,4305	&	0,5708	&	0,5836	&	0,4743	&	0,8787	&	0,8524	&	\textbf{0,8244}	&	0,6371	&	0,6626	&	0,5685	\\
31	&	0,5300	&	0,5465	&	0,4628	&	0,4295	&	0,4513	&	0,4323	&	0,5779	&	0,5895	&	0,4775	&	0,8816	&	0,8555	&	\textbf{0,8262}	&	0,6449	&	0,6687	&	0,5727	\\
32	&	0,5361	&	0,5523	&	\textbf{0,4643}	&	0,4365	&	0,4579	&	0,4340	&	0,5845	&	0,5949	&	0,4805	&	0,8843	&	0,8584	&	\textbf{0,8278}	&	0,6520	&	0,6744	&	0,5766	\\
33	&	0,5417	&	0,5577	&	\textbf{0,4656}	&	0,4430	&	0,4641	&	0,4355	&	0,5905	&	0,6000	&	0,4832	&	0,8867	&	0,8611	&	\textbf{0,8292}	&	0,6586	&	0,6797	&	0,5801	\\
34	&	0,5470	&	0,5628	&	\textbf{0,4668}	&	0,4492	&	0,4700	&	0,4369	&	0,5961	&	0,6048	&	0,4857	&	0,8890	&	0,8635	&	\textbf{0,8304}	&	0,6647	&	0,6846	&	0,5834	\\
35	&	0,5519	&	0,5676	&	\textbf{0,4679}	&	0,4551	&	0,4756	&	0,4382	&	0,6013	&	0,6092	&	0,4881	&	0,8910	&	0,8658	&	\textbf{0,8315}	&	0,6704	&	0,6892	&	0,5865	\\
36	&	0,5566	&	0,5721	&	\textbf{0,4690}	&	0,4606	&	0,4809	&	0,4394	&	0,6062	&	0,6134	&	0,4902	&	0,8929	&	0,8679	&	\textbf{0,8325}	&	0,6756	&	0,6935	&	0,5894	\\
	\bottomrule
\end{tabular}
 \end{adjustwidth}
\caption[Wyniki badań miar dwuelementowych dla korpusu \emph{KIPI} poddanego podpróbkowaniu klasy negatywnej do 80\%, część 1]{Wyniki badań miar dwuelementowych dla korpusu \emph{KIPI} poddanego podpróbkowaniu klasy negatywnej do 80\%, część 1.}
\label{KIPI_subsampled_20_part_1}
\end{table}

\begin{table}[htp!]
\centering
\footnotesize\setlength{\tabcolsep}{2.5pt}
 \begin{adjustwidth}{-2cm}{}
\begin{tabular}{ l | *{15}{| r}}
	\toprule 
	\textbf{95\%} &	\textbf{16}	&	\textbf{17}	&	\textbf{18}	&	\textbf{19}	&	\textbf{20}	&	\textbf{21}	&	\textbf{22}	&	\textbf{23}	&	\textbf{24}	&	\textbf{25}	&	\textbf{26}	&	\textbf{27}	&	\textbf{28}	&	\textbf{29}	&	\textbf{30}	\\
	\midrule
1	&	\textbf{0,9612}	&	\textbf{0,9510}	&	\textbf{0,8879}	&	0,6980	&	0,6840	&	\textbf{0,5425}	&	\textbf{0,9444}	&	\textbf{0,9221}	&	\textbf{0,8631}	&	0,7779	&	0,7737	&	\textbf{0,6615}	&	\textbf{0,9643}	&	\textbf{0,9516}	&	\textbf{0,9146}	\\
2	&	0,9005	&	0,8839	&	0,7540	&	\textbf{0,7896}	&	\textbf{0,7388}	&	0,4258	&	0,9055	&	0,8671	&	0,7376	&	\textbf{0,8699}	&	\textbf{0,8451}	&	0,5981	&	\textbf{0,9434}	&	\textbf{0,9280}	&	0,8446	\\
3	&	0,3500	&	0,3001	&	0,5118	&	0,0528	&	0,0588	&	0,1320	&	0,3110	&	0,2559	&	0,4457	&	0,0781	&	0,0929	&	0,2049	&	0,4336	&	0,3771	&	0,6141	\\
4	&	0,6917	&	0,6500	&	0,7743	&	0,1522	&	0,1887	&	0,2753	&	0,6039	&	0,5569	&	0,6947	&	0,2154	&	0,2747	&	0,3907	&	0,7466	&	0,7012	&	0,8279	\\
5	&	0,6917	&	0,6499	&	0,7744	&	0,1522	&	0,1888	&	0,2751	&	0,6041	&	0,5568	&	0,6952	&	0,2153	&	0,2746	&	0,3907	&	0,7465	&	0,7012	&	0,8280	\\
6	&	0,6566	&	0,6152	&	0,7488	&	0,1421	&	0,1854	&	0,2729	&	0,5867	&	0,5404	&	0,6894	&	0,1897	&	0,2578	&	0,3679	&	0,7157	&	0,6703	&	0,8085	\\
7	&	0,4624	&	0,4123	&	0,5959	&	0,0951	&	0,1129	&	0,2085	&	0,4477	&	0,3901	&	0,5866	&	0,1119	&	0,1424	&	0,2555	&	0,5435	&	0,4884	&	0,6893	\\
8	&	0,5829	&	0,5330	&	0,6645	&	0,1399	&	0,1602	&	0,2312	&	0,5525	&	0,4949	&	0,6289	&	0,1606	&	0,1989	&	0,2970	&	0,6455	&	0,5925	&	0,7441	\\
9	&	0,4173	&	0,3622	&	0,5748	&	0,0702	&	0,0743	&	0,1549	&	0,3686	&	0,3083	&	0,5021	&	0,1013	&	0,1140	&	0,2330	&	0,5017	&	0,4424	&	0,6717	\\
10	&	0,4174	&	0,3623	&	0,5748	&	0,0701	&	0,0743	&	0,1548	&	0,3688	&	0,3083	&	0,5022	&	0,1013	&	0,1140	&	0,2331	&	0,5017	&	0,4423	&	0,6717	\\
11	&	0,9126	&	0,8934	&	\textbf{0,8766}	&	0,4295	&	0,4531	&	0,4476	&	0,8477	&	0,8110	&	\textbf{0,8263}	&	0,5112	&	0,5516	&	0,5653	&	0,9029	&	0,8785	&	\textbf{0,8998}	\\
12	&	0,4381	&	0,3893	&	0,6097	&	0,0719	&	0,0849	&	0,1760	&	0,3830	&	0,3282	&	0,5392	&	0,1046	&	0,1320	&	0,2625	&	0,5247	&	0,4696	&	0,7017	\\
13	&	\textbf{0,9306}	&	\textbf{0,9117}	&	\textbf{0,8859}	&	0,5376	&	0,5388	&	0,5036	&	0,8836	&	0,8486	&	\textbf{0,8510}	&	0,5884	&	0,5989	&	0,6046	&	0,9074	&	0,8817	&	\textbf{0,9045}	\\
14	&	0,6565	&	0,6153	&	0,7489	&	0,1421	&	0,1853	&	0,2730	&	0,5868	&	0,5404	&	0,6893	&	0,1896	&	0,2578	&	0,3678	&	0,7157	&	0,6703	&	0,8085	\\
15	&	0,8688	&	0,8442	&	\textbf{0,8505}	&	0,3369	&	0,3835	&	0,4073	&	0,7989	&	0,7592	&	0,8039	&	0,3882	&	0,4648	&	0,5032	&	0,8624	&	0,8331	&	\textbf{0,8795}	\\
16	&	\textbf{0,9423}	&	\textbf{0,9255}	&	\textbf{0,8859}	&	0,6000	&	0,5914	&	\textbf{0,5211}	&	0,9068	&	0,8747	&	\textbf{0,8551}	&	0,6430	&	0,6423	&	0,6239	&	0,9219	&	0,8980	&	\textbf{0,9050}	\\
17	&	0,6541	&	0,6124	&	0,7474	&	0,1409	&	0,1839	&	0,2705	&	0,5839	&	0,5376	&	0,6870	&	0,1882	&	0,2550	&	0,3657	&	0,7119	&	0,6657	&	0,8066	\\
18	&	0,6555	&	0,6144	&	0,7481	&	0,1423	&	0,1856	&	0,2730	&	0,5868	&	0,5407	&	0,6889	&	0,1900	&	0,2594	&	0,3683	&	0,7179	&	0,6734	&	0,8088	\\
19	&	0,6540	&	0,6124	&	0,7472	&	0,1408	&	0,1838	&	0,2704	&	0,5838	&	0,5374	&	0,6869	&	0,1882	&	0,2549	&	0,3656	&	0,7117	&	0,6655	&	0,8065	\\
20	&	\textbf{0,9367}	&	\textbf{0,9208}	&	\textbf{0,8863}	&	0,5626	&	0,5678	&	0,5078	&	0,9006	&	0,8694	&	\textbf{0,8537}	&	0,6500	&	0,6694	&	0,6197	&	\textbf{0,9362}	&	0,9190	&	\textbf{0,9110}	\\
21	&	0,8774	&	0,8537	&	\textbf{0,8550}	&	0,3542	&	0,3976	&	0,4162	&	0,8084	&	0,7693	&	0,8094	&	0,4066	&	0,4794	&	0,5129	&	0,8689	&	0,8406	&	\textbf{0,8829}	\\
22	&	0,8846	&	0,8618	&	\textbf{0,8590}	&	0,3700	&	0,4103	&	0,4244	&	0,8167	&	0,7780	&	0,8142	&	0,4239	&	0,4927	&	0,5218	&	0,8747	&	0,8471	&	\textbf{0,8859}	\\
23	&	0,8908	&	0,8687	&	\textbf{0,8624}	&	0,3843	&	0,4219	&	0,4320	&	0,8240	&	0,7858	&	0,8185	&	0,4402	&	0,5048	&	0,5300	&	0,8797	&	0,8529	&	\textbf{0,8885}	\\
24	&	0,8961	&	0,8747	&	\textbf{0,8655}	&	0,3975	&	0,4324	&	0,4388	&	0,8304	&	0,7926	&	\textbf{0,8223}	&	0,4553	&	0,5158	&	0,5375	&	0,8842	&	0,8580	&	\textbf{0,8908}	\\
25	&	0,9007	&	0,8799	&	\textbf{0,8681}	&	0,4094	&	0,4420	&	0,4450	&	0,8360	&	0,7987	&	\textbf{0,8256}	&	0,4690	&	0,5259	&	0,5444	&	0,8883	&	0,8626	&	\textbf{0,8929}	\\
26	&	0,9047	&	0,8844	&	\textbf{0,8705}	&	0,4202	&	0,4508	&	0,4507	&	0,8411	&	0,8042	&	\textbf{0,8286}	&	0,4816	&	0,5351	&	0,5507	&	0,8919	&	0,8667	&	\textbf{0,8947}	\\
27	&	0,9082	&	0,8883	&	\textbf{0,8725}	&	0,4302	&	0,4589	&	0,4558	&	0,8457	&	0,8092	&	\textbf{0,8312}	&	0,4930	&	0,5435	&	0,5565	&	0,8951	&	0,8705	&	\textbf{0,8964}	\\
28	&	0,9113	&	0,8918	&	\textbf{0,8743}	&	0,4393	&	0,4664	&	0,4605	&	0,8498	&	0,8137	&	\textbf{0,8336}	&	0,5034	&	0,5513	&	0,5619	&	0,8981	&	0,8739	&	\textbf{0,8978}	\\
29	&	\textbf{0,9140}	&	0,8950	&	\textbf{0,8759}	&	0,4477	&	0,4733	&	0,4648	&	0,8536	&	0,8179	&	\textbf{0,8357}	&	0,5130	&	0,5586	&	0,5668	&	0,9008	&	0,8770	&	\textbf{0,8991}	\\
30	&	\textbf{0,9165}	&	0,8978	&	\textbf{0,8772}	&	0,4556	&	0,4798	&	0,4688	&	0,8571	&	0,8217	&	\textbf{0,8376}	&	0,5218	&	0,5653	&	0,5713	&	0,9033	&	0,8798	&	\textbf{0,9003}	\\
31	&	\textbf{0,9187}	&	0,9003	&	\textbf{0,8785}	&	0,4628	&	0,4859	&	0,4725	&	0,8603	&	0,8253	&	\textbf{0,8393}	&	0,5300	&	0,5716	&	0,5754	&	0,9057	&	0,8825	&	\textbf{0,9014}	\\
32	&	\textbf{0,9207}	&	0,9026	&	\textbf{0,8796}	&	0,4696	&	0,4915	&	0,4758	&	0,8633	&	0,8286	&	\textbf{0,8409}	&	0,5375	&	0,5775	&	0,5793	&	0,9078	&	0,8850	&	\textbf{0,9023}	\\
33	&	\textbf{0,9225}	&	\textbf{0,9047}	&	\textbf{0,8805}	&	0,4760	&	0,4969	&	0,4789	&	0,8661	&	0,8317	&	\textbf{0,8423}	&	0,5446	&	0,5831	&	0,5828	&	0,9098	&	0,8873	&	\textbf{0,9032}	\\
34	&	\textbf{0,9242}	&	\textbf{0,9066}	&	\textbf{0,8814}	&	0,4819	&	0,5020	&	0,4818	&	0,8687	&	0,8346	&	\textbf{0,8436}	&	0,5512	&	0,5883	&	0,5861	&	0,9117	&	0,8894	&	\textbf{0,9040}	\\
35	&	\textbf{0,9257}	&	\textbf{0,9084}	&	\textbf{0,8823}	&	0,4875	&	0,5067	&	0,4844	&	0,8711	&	0,8373	&	\textbf{0,8447}	&	0,5575	&	0,5932	&	0,5892	&	0,9134	&	0,8914	&	\textbf{0,9047}	\\
36	&	\textbf{0,9271}	&	\textbf{0,9100}	&	\textbf{0,8830}	&	0,4928	&	0,5113	&	0,4869	&	0,8734	&	0,8399	&	\textbf{0,8458}	&	0,5633	&	0,5979	&	0,5921	&	0,9151	&	0,8933	&	\textbf{0,9053}	\\
	\bottomrule
\end{tabular}
 \end{adjustwidth}
\caption[Wyniki badań miar dwuelementowych dla korpusu \emph{KIPI} poddanego podpróbkowaniu klasy negatywnej do 80\%, część 2]{Wyniki badań miar dwuelementowych dla korpusu \emph{KIPI} poddanego podpróbkowaniu klasy negatywnej do 80\%, część 2.}
\label{KIPI_subsampled_20_part_2}
\end{table}

\begin{table}[htp!]
\centering
\footnotesize\setlength{\tabcolsep}{2.5pt}
 \begin{adjustwidth}{-2cm}{}
\begin{tabular}{ l | *{15}{| r}}
	\toprule 
	\textbf{95\%} &	\textbf{1}	&	\textbf{2}	&	\textbf{3}	&	\textbf{4}	&	\textbf{5}	&	\textbf{6}	&	\textbf{7}	&	\textbf{8}	&	\textbf{9}	&	\textbf{10}	&	\textbf{11}	&	\textbf{12}	&	\textbf{13}	&	\textbf{14}	&	\textbf{15}	\\
	\midrule
37	&	0,5610	&	0,5764	&	\textbf{0,4699}	&	0,4659	&	0,4860	&	0,4404	&	0,6108	&	0,6173	&	0,4923	&	\textbf{0,8947}	&	0,8699	&	\textbf{0,8334}	&	0,6805	&	0,6975	&	0,5921	\\
38	&	0,5652	&	0,5804	&	\textbf{0,4707}	&	0,4709	&	0,4908	&	0,4414	&	0,6151	&	0,6210	&	0,4941	&	\textbf{0,8964}	&	0,8717	&	\textbf{0,8342}	&	0,6851	&	0,7013	&	0,5946	\\
39	&	0,5691	&	0,5843	&	\textbf{0,4715}	&	0,4757	&	0,4954	&	0,4424	&	0,6192	&	0,6245	&	0,4959	&	\textbf{0,8979}	&	0,8734	&	\textbf{0,8350}	&	0,6894	&	0,7048	&	0,5969	\\
40	&	0,5729	&	0,5880	&	\textbf{0,4722}	&	0,4803	&	0,4998	&	0,4432	&	0,6231	&	0,6278	&	0,4975	&	\textbf{0,8994}	&	0,8750	&	\textbf{0,8356}	&	0,6934	&	0,7082	&	0,5991	\\
41	&	0,5765	&	0,5914	&	\textbf{0,4729}	&	0,4847	&	0,5040	&	0,4440	&	0,6267	&	0,6310	&	0,4991	&	\textbf{0,9008}	&	\textbf{0,8765}	&	\textbf{0,8362}	&	0,6973	&	0,7114	&	0,6011	\\
42	&	0,5799	&	0,5948	&	\textbf{0,4735}	&	0,4889	&	0,5080	&	0,4448	&	0,6302	&	0,6340	&	0,5005	&	\textbf{0,9021}	&	\textbf{0,8780}	&	\textbf{0,8368}	&	0,7009	&	0,7145	&	0,6031	\\
43	&	0,5831	&	0,5979	&	\textbf{0,4741}	&	0,4929	&	0,5118	&	0,4455	&	0,6335	&	0,6368	&	0,5019	&	\textbf{0,9033}	&	\textbf{0,8793}	&	\textbf{0,8373}	&	0,7043	&	0,7174	&	0,6049	\\
44	&	0,5862	&	0,6010	&	\textbf{0,4746}	&	0,4967	&	0,5155	&	0,4461	&	0,6366	&	0,6395	&	0,5031	&	\textbf{0,9045}	&	\textbf{0,8806}	&	\textbf{0,8377}	&	0,7076	&	0,7201	&	0,6066	\\
45	&	0,5892	&	0,6038	&	\textbf{0,4751}	&	0,5004	&	0,5190	&	0,4467	&	0,6396	&	0,6421	&	0,5043	&	\textbf{0,9056}	&	\textbf{0,8818}	&	\textbf{0,8381}	&	0,7107	&	0,7227	&	0,6083	\\
46	&	0,5920	&	0,6066	&	\textbf{0,4756}	&	0,5039	&	0,5224	&	0,4473	&	0,6425	&	0,6446	&	0,5055	&	\textbf{0,9066}	&	\textbf{0,8830}	&	\textbf{0,8385}	&	0,7137	&	0,7252	&	0,6098	\\
47	&	0,5948	&	0,6093	&	\textbf{0,4760}	&	0,5073	&	0,5257	&	0,4479	&	0,6452	&	0,6469	&	0,5065	&	\textbf{0,9076}	&	\textbf{0,8841}	&	\textbf{0,8388}	&	0,7165	&	0,7276	&	0,6112	\\
48	&	0,5974	&	0,6118	&	\textbf{0,4765}	&	0,5106	&	0,5288	&	0,4484	&	0,6479	&	0,6492	&	0,5075	&	\textbf{0,9085}	&	\textbf{0,8851}	&	\textbf{0,8391}	&	0,7192	&	0,7299	&	0,6126	\\
49	&	0,5999	&	0,6143	&	\textbf{0,4768}	&	0,5137	&	0,5318	&	0,4489	&	0,6504	&	0,6514	&	0,5085	&	\textbf{0,9094}	&	\textbf{0,8861}	&	\textbf{0,8394}	&	0,7218	&	0,7321	&	0,6139	\\
50	&	0,6022	&	0,6166	&	\textbf{0,4772}	&	0,5167	&	0,5346	&	0,4493	&	0,6528	&	0,6534	&	0,5094	&	\textbf{0,9103}	&	\textbf{0,8871}	&	\textbf{0,8396}	&	0,7242	&	0,7342	&	0,6152	\\
51	&	0,5482	&	0,5574	&	0,4514	&	0,4344	&	0,4404	&	0,4071	&	0,6440	&	0,6394	&	0,5045	&	\textbf{0,9012}	&	0,8753	&	\textbf{0,8383}	&	0,6961	&	0,6980	&	0,6040	\\
52	&	0,5140	&	0,5218	&	0,4415	&	0,3981	&	0,4023	&	0,3951	&	0,6285	&	0,6243	&	0,4985	&	0,8932	&	0,8659	&	\textbf{0,8359}	&	0,6720	&	0,6742	&	0,5934	\\
53	&	0,4741	&	0,4803	&	0,4280	&	0,3608	&	0,3647	&	0,3805	&	0,6099	&	0,6062	&	0,4910	&	0,8831	&	0,8540	&	\textbf{0,8324}	&	0,6466	&	0,6493	&	0,5798	\\
54	&	0,4509	&	0,4570	&	0,4134	&	0,3360	&	0,3402	&	0,3674	&	0,5878	&	0,5845	&	0,4815	&	0,8700	&	0,8391	&	\textbf{0,8273}	&	0,6210	&	0,6243	&	0,5644	\\
55	&	0,4342	&	0,4396	&	0,4065	&	0,3163	&	0,3201	&	0,3621	&	0,5601	&	0,5575	&	0,4696	&	0,8520	&	0,8183	&	\textbf{0,8200}	&	0,5913	&	0,5951	&	0,5467	\\
56	&	0,4165	&	0,4209	&	0,4029	&	0,2970	&	0,2999	&	0,3589	&	0,5272	&	0,5252	&	0,4543	&	0,8277	&	0,7909	&	\textbf{0,8091}	&	0,5557	&	0,5601	&	0,5255	\\
57	&	0,3958	&	0,3989	&	0,3981	&	0,2762	&	0,2784	&	0,3534	&	0,4913	&	0,4897	&	0,4354	&	0,7996	&	0,7593	&	0,7929	&	0,5126	&	0,5174	&	0,4990	\\
58	&	0,3714	&	0,3729	&	0,3898	&	0,2537	&	0,2546	&	0,3434	&	0,4531	&	0,4523	&	0,4132	&	0,7686	&	0,7250	&	0,7716	&	0,4637	&	0,4689	&	0,4673	\\
59	&	0,3417	&	0,3420	&	0,3753	&	0,2280	&	0,2279	&	0,3271	&	0,4117	&	0,4112	&	0,3885	&	0,7307	&	0,6838	&	0,7453	&	0,4095	&	0,4157	&	0,4309	\\
60	&	0,3076	&	0,3069	&	0,3548	&	0,1998	&	0,1990	&	0,3036	&	0,3670	&	0,3667	&	0,3597	&	0,6870	&	0,6361	&	0,7122	&	0,3534	&	0,3592	&	0,3905	\\
61	&	0,2698	&	0,2679	&	0,3269	&	0,1706	&	0,1695	&	0,2765	&	0,3198	&	0,3201	&	0,3289	&	0,6349	&	0,5814	&	0,6734	&	0,2959	&	0,3008	&	0,3478	\\
62	&	0,2294	&	0,2261	&	0,2954	&	0,1441	&	0,1426	&	0,2473	&	0,2719	&	0,2715	&	0,2964	&	0,5774	&	0,5198	&	0,6276	&	0,2431	&	0,2467	&	0,3059	\\
63	&	0,1889	&	0,1841	&	0,2613	&	0,1238	&	0,1223	&	0,2175	&	0,2239	&	0,2233	&	0,2624	&	0,5114	&	0,4518	&	0,5763	&	0,1995	&	0,2023	&	0,2689	\\
64	&	0,1527	&	0,1476	&	0,2262	&	0,1077	&	0,1064	&	0,1899	&	0,1819	&	0,1807	&	0,2302	&	0,4462	&	0,3865	&	0,5233	&	0,1677	&	0,1702	&	0,2362	\\
65	&	0,1273	&	0,1225	&	0,1954	&	0,0948	&	0,0935	&	0,1681	&	0,1478	&	0,1456	&	0,2003	&	0,3918	&	0,3343	&	0,4723	&	0,1430	&	0,1452	&	0,2084	\\
66	&	0,1093	&	0,1048	&	0,1702	&	0,0844	&	0,0829	&	0,1508	&	0,1226	&	0,1217	&	0,1741	&	0,3469	&	0,2933	&	0,4255	&	0,1229	&	0,1243	&	0,1841	\\
67	&	0,0953	&	0,0905	&	0,1509	&	0,0753	&	0,0741	&	0,1368	&	0,1054	&	0,1040	&	0,1530	&	0,3102	&	0,2587	&	0,3861	&	0,1053	&	0,1073	&	0,1651	\\
68	&	0,0830	&	0,0785	&	0,1346	&	0,0674	&	0,0661	&	0,1250	&	0,0907	&	0,0892	&	0,1348	&	0,2775	&	0,2299	&	0,3512	&	0,0906	&	0,0937	&	0,1521	\\
69	&	0,0726	&	0,0681	&	0,1210	&	0,0609	&	0,0597	&	0,1152	&	0,0775	&	0,0767	&	0,1200	&	0,2527	&	0,2067	&	0,3198	&	0,0850	&	0,0883	&	0,1447	\\
70	&	0,0598	&	0,0563	&	0,1097	&	0,0530	&	0,0526	&	0,1067	&	0,0602	&	0,0615	&	0,1075	&	0,2261	&	0,1831	&	0,2940	&	0,0808	&	0,0843	&	0,1402	\\
71	&	0,4967	&	0,5063	&	0,2366	&	0,6084	&	0,6225	&	0,2683	&	0,6219	&	0,5999	&	0,3211	&	0,8824	&	0,8484	&	0,6506	&	0,7218	&	0,7094	&	0,4368	\\
72	&	0,5830	&	0,5942	&	0,3001	&	0,5863	&	0,6005	&	0,2787	&	0,6808	&	0,6574	&	0,3799	&	\textbf{0,9115}	&	\textbf{0,8843}	&	0,7175	&	0,7740	&	0,7597	&	0,5025	\\
	\bottomrule
\end{tabular}
 \end{adjustwidth}
\caption[Wyniki badań miar dwuelementowych dla korpusu \emph{KIPI} poddanego podpróbkowaniu klasy negatywnej do 80\%, część 3]{Wyniki badań miar dwuelementowych dla korpusu \emph{KIPI} poddanego podpróbkowaniu klasy negatywnej do 80\%, część 3.}
\label{KIPI_subsampled_20_part_3}
\end{table}

\begin{table}[htp!]
\centering
\footnotesize\setlength{\tabcolsep}{2.5pt}
 \begin{adjustwidth}{-2cm}{}
\begin{tabular}{ l | *{15}{| r}}
	\toprule 
	\textbf{95\%} &	\textbf{16}	&	\textbf{17}	&	\textbf{18}	&	\textbf{19}	&	\textbf{20}	&	\textbf{21}	&	\textbf{22}	&	\textbf{23}	&	\textbf{24}	&	\textbf{25}	&	\textbf{26}	&	\textbf{27}	&	\textbf{28}	&	\textbf{29}	&	\textbf{30}	\\
	\midrule
37	&	\textbf{0,9284}	&	\textbf{0,9116}	&	\textbf{0,8836}	&	0,4979	&	0,5156	&	0,4892	&	0,8756	&	0,8423	&	\textbf{0,8468}	&	0,5688	&	0,6024	&	0,5947	&	0,9166	&	0,8951	&	\textbf{0,9060}	\\
38	&	\textbf{0,9297}	&	\textbf{0,9130}	&	\textbf{0,8841}	&	0,5027	&	0,5197	&	0,4913	&	0,8777	&	0,8447	&	\textbf{0,8477}	&	0,5740	&	0,6066	&	0,5972	&	0,9181	&	0,8968	&	\textbf{0,9065}	\\
39	&	\textbf{0,9308}	&	\textbf{0,9143}	&	\textbf{0,8847}	&	0,5072	&	0,5236	&	0,4933	&	0,8796	&	0,8469	&	\textbf{0,8485}	&	0,5789	&	0,6107	&	0,5995	&	0,9195	&	0,8984	&	\textbf{0,9070}	\\
40	&	\textbf{0,9319}	&	\textbf{0,9156}	&	\textbf{0,8852}	&	0,5116	&	0,5274	&	0,4952	&	0,8815	&	0,8490	&	\textbf{0,8493}	&	0,5836	&	0,6145	&	0,6017	&	0,9208	&	0,9000	&	\textbf{0,9075}	\\
41	&	\textbf{0,9329}	&	\textbf{0,9167}	&	\textbf{0,8856}	&	0,5157	&	0,5309	&	0,4969	&	0,8832	&	0,8510	&	\textbf{0,8500}	&	0,5881	&	0,6182	&	0,6038	&	0,9221	&	0,9014	&	\textbf{0,9079}	\\
42	&	\textbf{0,9338}	&	\textbf{0,9178}	&	\textbf{0,8860}	&	0,5197	&	0,5344	&	0,4985	&	0,8849	&	0,8529	&	\textbf{0,8506}	&	0,5924	&	0,6217	&	0,6057	&	0,9233	&	0,9028	&	\textbf{0,9083}	\\
43	&	\textbf{0,9347}	&	\textbf{0,9189}	&	\textbf{0,8863}	&	0,5235	&	0,5377	&	0,5000	&	0,8865	&	0,8547	&	\textbf{0,8512}	&	0,5965	&	0,6251	&	0,6075	&	0,9244	&	0,9041	&	\textbf{0,9086}	\\
44	&	\textbf{0,9356}	&	\textbf{0,9199}	&	\textbf{0,8867}	&	0,5272	&	0,5408	&	0,5015	&	0,8881	&	0,8564	&	\textbf{0,8518}	&	0,6004	&	0,6283	&	0,6092	&	0,9255	&	0,9054	&	\textbf{0,9090}	\\
45	&	\textbf{0,9364}	&	\textbf{0,9208}	&	\textbf{0,8870}	&	0,5307	&	0,5438	&	0,5029	&	0,8895	&	0,8581	&	\textbf{0,8523}	&	0,6041	&	0,6314	&	0,6108	&	0,9265	&	0,9066	&	\textbf{0,9093}	\\
46	&	\textbf{0,9371}	&	\textbf{0,9217}	&	\textbf{0,8873}	&	0,5341	&	0,5467	&	0,5041	&	0,8909	&	0,8597	&	\textbf{0,8528}	&	0,6078	&	0,6344	&	0,6123	&	0,9275	&	0,9077	&	\textbf{0,9096}	\\
47	&	\textbf{0,9378}	&	\textbf{0,9226}	&	\textbf{0,8875}	&	0,5374	&	0,5495	&	0,5053	&	0,8923	&	0,8612	&	\textbf{0,8533}	&	0,6112	&	0,6373	&	0,6137	&	0,9284	&	0,9088	&	\textbf{0,9099}	\\
48	&	\textbf{0,9385}	&	\textbf{0,9234}	&	\textbf{0,8877}	&	0,5405	&	0,5522	&	0,5065	&	0,8935	&	0,8626	&	\textbf{0,8537}	&	0,6146	&	0,6401	&	0,6151	&	0,9293	&	0,9099	&	\textbf{0,9101}	\\
49	&	\textbf{0,9392}	&	\textbf{0,9241}	&	\textbf{0,8880}	&	0,5436	&	0,5548	&	0,5075	&	0,8948	&	0,8640	&	\textbf{0,8541}	&	0,6178	&	0,6427	&	0,6164	&	\textbf{0,9302}	&	0,9109	&	\textbf{0,9104}	\\
50	&	\textbf{0,9398}	&	\textbf{0,9249}	&	\textbf{0,8881}	&	0,5465	&	0,5573	&	0,5086	&	0,8959	&	0,8654	&	\textbf{0,8545}	&	0,6209	&	0,6453	&	0,6176	&	\textbf{0,9310}	&	0,9119	&	\textbf{0,9106}	\\
51	&	\textbf{0,9227}	&	0,9018	&	\textbf{0,8836}	&	0,5207	&	0,5225	&	0,4977	&	0,8742	&	0,8375	&	\textbf{0,8483}	&	0,5647	&	0,5757	&	0,5953	&	0,8958	&	0,8681	&	\textbf{0,9014}	\\
52	&	0,9112	&	0,8877	&	\textbf{0,8800}	&	0,5013	&	0,5036	&	0,4903	&	0,8626	&	0,8238	&	\textbf{0,8445}	&	0,5380	&	0,5495	&	0,5839	&	0,8814	&	0,8514	&	\textbf{0,8976}	\\
53	&	0,8979	&	0,8724	&	\textbf{0,8742}	&	0,4790	&	0,4821	&	0,4812	&	0,8484	&	0,8071	&	\textbf{0,8397}	&	0,5117	&	0,5243	&	0,5696	&	0,8670	&	0,8349	&	\textbf{0,8920}	\\
54	&	0,8846	&	0,8569	&	\textbf{0,8673}	&	0,4540	&	0,4576	&	0,4700	&	0,8310	&	0,7872	&	\textbf{0,8333}	&	0,4861	&	0,4997	&	0,5543	&	0,8531	&	0,8190	&	\textbf{0,8861}	\\
55	&	0,8684	&	0,8379	&	\textbf{0,8585}	&	0,4260	&	0,4297	&	0,4562	&	0,8096	&	0,7631	&	\textbf{0,8245}	&	0,4577	&	0,4715	&	0,5363	&	0,8371	&	0,8002	&	\textbf{0,8786}	\\
56	&	0,8470	&	0,8136	&	\textbf{0,8466}	&	0,3957	&	0,3994	&	0,4396	&	0,7850	&	0,7351	&	0,8127	&	0,4242	&	0,4387	&	0,5139	&	0,8165	&	0,7767	&	0,8680	\\
57	&	0,8182	&	0,7805	&	0,8299	&	0,3655	&	0,3693	&	0,4191	&	0,7598	&	0,7069	&	0,7968	&	0,3855	&	0,4010	&	0,4867	&	0,7899	&	0,7471	&	0,8539	\\
58	&	0,7818	&	0,7397	&	0,8073	&	0,3330	&	0,3373	&	0,3955	&	0,7312	&	0,6752	&	0,7773	&	0,3425	&	0,3585	&	0,4541	&	0,7580	&	0,7114	&	0,8350	\\
59	&	0,7373	&	0,6916	&	0,7773	&	0,2982	&	0,3027	&	0,3689	&	0,6972	&	0,6375	&	0,7545	&	0,2972	&	0,3138	&	0,4176	&	0,7197	&	0,6702	&	0,8106	\\
60	&	0,6843	&	0,6340	&	0,7394	&	0,2615	&	0,2660	&	0,3396	&	0,6580	&	0,5953	&	0,7269	&	0,2521	&	0,2677	&	0,3777	&	0,6759	&	0,6231	&	0,7803	\\
61	&	0,6239	&	0,5695	&	0,6944	&	0,2235	&	0,2274	&	0,3079	&	0,6129	&	0,5465	&	0,6937	&	0,2102	&	0,2235	&	0,3364	&	0,6280	&	0,5715	&	0,7443	\\
62	&	0,5584	&	0,5024	&	0,6430	&	0,1863	&	0,1901	&	0,2745	&	0,5639	&	0,4949	&	0,6559	&	0,1752	&	0,1882	&	0,2957	&	0,5804	&	0,5217	&	0,7036	\\
63	&	0,5002	&	0,4419	&	0,5924	&	0,1522	&	0,1549	&	0,2402	&	0,5122	&	0,4421	&	0,6113	&	0,1505	&	0,1622	&	0,2608	&	0,5402	&	0,4799	&	0,6645	\\
64	&	0,4508	&	0,3937	&	0,5438	&	0,1276	&	0,1298	&	0,2086	&	0,4669	&	0,3979	&	0,5650	&	0,1312	&	0,1425	&	0,2315	&	0,5052	&	0,4444	&	0,6277	\\
65	&	0,4090	&	0,3523	&	0,4989	&	0,1098	&	0,1120	&	0,1818	&	0,4276	&	0,3600	&	0,5221	&	0,1146	&	0,1252	&	0,2056	&	0,4734	&	0,4136	&	0,5916	\\
66	&	0,3730	&	0,3171	&	0,4588	&	0,0957	&	0,0976	&	0,1601	&	0,3938	&	0,3287	&	0,4832	&	0,1005	&	0,1103	&	0,1844	&	0,4455	&	0,3874	&	0,5588	\\
67	&	0,3460	&	0,2906	&	0,4268	&	0,0837	&	0,0858	&	0,1424	&	0,3648	&	0,3020	&	0,4496	&	0,0888	&	0,0986	&	0,1678	&	0,4252	&	0,3682	&	0,5313	\\
68	&	0,3242	&	0,2707	&	0,4017	&	0,0733	&	0,0751	&	0,1268	&	0,3388	&	0,2785	&	0,4182	&	0,0793	&	0,0895	&	0,1558	&	0,4094	&	0,3531	&	0,5100	\\
69	&	0,3156	&	0,2632	&	0,3864	&	0,0638	&	0,0659	&	0,1138	&	0,3182	&	0,2604	&	0,3901	&	0,0757	&	0,0860	&	0,1487	&	0,4035	&	0,3476	&	0,4959	\\
70	&	0,3091	&	0,2583	&	0,3759	&	0,0524	&	0,0561	&	0,1019	&	0,2972	&	0,2412	&	0,3659	&	0,0729	&	0,0835	&	0,1443	&	0,3995	&	0,3428	&	0,4866	\\
71	&	\textbf{0,9215}	&	0,9005	&	0,7425	&	\textbf{0,7796}	&	\textbf{0,7644}	&	0,4549	&	\textbf{0,9649}	&	\textbf{0,9501}	&	0,8153	&	\textbf{0,8456}	&	\textbf{0,8411}	&	0,5793	&	\textbf{0,9783}	&	\textbf{0,9695}	&	\textbf{0,8813}	\\
72	&	\textbf{0,9411}	&	\textbf{0,9242}	&	0,7961	&	0,7471	&	\textbf{0,7288}	&	0,4528	&	\textbf{0,9572}	&	\textbf{0,9394}	&	0,8139	&	0,8207	&	\textbf{0,8146}	&	0,5778	&	\textbf{0,9733}	&	\textbf{0,9627}	&	\textbf{0,8803}	\\
	\bottomrule
\end{tabular}
 \end{adjustwidth}
\caption[Wyniki badań miar dwuelementowych dla korpusu \emph{KIPI} poddanego podpróbkowaniu klasy negatywnej do 80\%, część 4]{Wyniki badań miar dwuelementowych dla korpusu \emph{KIPI} poddanego podpróbkowaniu klasy negatywnej do 80\%, część 4.}
\label{KIPI_subsampled_20_part_4}
\end{table}

Zgodnie z oczekiwaniami jakość wyników wzrosła znacząco (kilkukrotnie), w niektórych przypadkach wzrost sięgał całego rzędu, a w przypadku funkcji \emph{Expected Frequency} okazał się niespodziewanie duży -- kilkudziesięciokrotny.

\par
Zmianie uległ zestaw funkcji osiągających najlepsze wyniki i był następujący: \emph{Frequency}, \emph{Expected Frequency}, \emph{Mutual Expectation}, \emph{W Specific Correlation}, \emph{Specific Frequency Biased Mutual Dependency}, \emph{T-Score}, \emph{Loglikelihood}, \emph{W Order}, \emph{W Term Frequency Order} oraz funkcje parametryczne -- \emph{Specific Exponential Correlation} i \emph{W Specific Exponential Correlation}.
Zauważyć można, że zestaw funkcji jest bardzo podobny do tego, z poprzednich badań, ale w stosunku do badania wcześniejszego grono najlepszych zostało rozszerzone o miary \emph{Mutual Expectation} i \emph{Specific Frequency Biased Mutual Dependency}.


\subsubsection{Rezultat badań jakości wyników dla miar asocjacyjnych po podpróbkowaniu klasy negatywnej do 95\%}
Cztery tabele \ref{KIPI_subsampled_5_part_1}, \ref{KIPI_subsampled_5_part_2}, \ref{KIPI_subsampled_5_part_3} oraz \ref{KIPI_subsampled_5_part_4} prezentują jakość wyników osiągniętych przez 72 funkcje w 30 różnych badaniach (30 zestawów danych pozyskanych z korpusu \emph{KIPI}).
Indeksy miar i typów badań pozostały takie same, jak w poprzednich badaniach.

\par
Różnica pomiędzy tym, a poprzednim badaniem jest taka, że w tym przypadku krotki z klasy negatywnej stanowią około 95\% wszystkich kandydatów, a nie tylko 80\%.
Motywacją do tego badania były stosunkowo dobre wyniki przy podpróbkowaniu na poziomie 80\% i chęć sprawdzenia jak zmienią się wyniki przy kilkukrotnym ograniczeniu liczby jednostek wielowyrazowych wśród wszystkich kandydatów.

\begin{table}[htp!]
\centering
\footnotesize\setlength{\tabcolsep}{2.5pt}
 \begin{adjustwidth}{-2cm}{}
\begin{tabular}{ l | *{15}{| r}}
	\toprule 
	\textbf{95\%} &	\textbf{1}	&	\textbf{2}	&	\textbf{3}	&	\textbf{4}	&	\textbf{5}	&	\textbf{6}	&	\textbf{7}	&	\textbf{8}	&	\textbf{9}	&	\textbf{10}	&	\textbf{11}	&	\textbf{12}	&	\textbf{13}	&	\textbf{14}	&	\textbf{15}	\\
	\midrule
1	&	\textbf{0,3540}	&	\textbf{0,3708}	&	0,1683	&	0,3038	&	0,3154	&	0,1581	&	\textbf{0,4210}	&	\textbf{0,4046}	&	0,1995	&	\textbf{0,7649}	&	\textbf{0,7165}	&	0,5163	&	\textbf{0,5299}	&	\textbf{0,5152}	&	0,2929	\\
2	&	0,3148	&	0,3084	&	0,0645	&	\textbf{0,3698}	&	\textbf{0,3672}	&	0,0824	&	0,2539	&	0,2134	&	0,0614	&	0,4959	&	0,4206	&	0,2062	&	0,3796	&	0,3399	&	0,1151	\\
3	&	0,0214	&	0,0219	&	0,0700	&	0,0180	&	0,0192	&	0,0637	&	0,0202	&	0,0237	&	0,0585	&	0,0829	&	0,0686	&	0,1694	&	0,0310	&	0,0375	&	0,0898	\\
4	&	0,0949	&	0,1151	&	0,1732	&	0,0687	&	0,0809	&	0,1473	&	0,0823	&	0,1228	&	0,1444	&	0,3365	&	0,3044	&	0,3999	&	0,1250	&	0,1831	&	0,2123	\\
5	&	0,0949	&	0,1151	&	0,1733	&	0,0686	&	0,0810	&	0,1474	&	0,0823	&	0,1229	&	0,1443	&	0,3365	&	0,3042	&	0,3998	&	0,1251	&	0,1831	&	0,2123	\\
6	&	0,0919	&	0,1225	&	0,1851	&	0,0641	&	0,0843	&	0,1587	&	0,0775	&	0,1262	&	0,1470	&	0,3372	&	0,3076	&	0,4099	&	0,1086	&	0,1769	&	0,2047	\\
7	&	0,0458	&	0,0526	&	0,1260	&	0,0327	&	0,0379	&	0,1089	&	0,0428	&	0,0578	&	0,1008	&	0,1771	&	0,1517	&	0,2880	&	0,0526	&	0,0724	&	0,1273	\\
8	&	0,0750	&	0,0854	&	0,1435	&	0,0500	&	0,0577	&	0,1216	&	0,0701	&	0,0921	&	0,1151	&	0,2668	&	0,2342	&	0,3265	&	0,0837	&	0,1145	&	0,1524	\\
9	&	0,0292	&	0,0273	&	0,0771	&	0,0240	&	0,0232	&	0,0678	&	0,0271	&	0,0292	&	0,0657	&	0,1017	&	0,0833	&	0,1875	&	0,0415	&	0,0448	&	0,1004	\\
10	&	0,0292	&	0,0273	&	0,0770	&	0,0240	&	0,0232	&	0,0678	&	0,0271	&	0,0292	&	0,0657	&	0,1017	&	0,0833	&	0,1876	&	0,0415	&	0,0448	&	0,1004	\\
11	&	0,2723	&	0,2800	&	0,2139	&	0,2002	&	0,2110	&	0,1937	&	0,2988	&	0,3171	&	0,2105	&	0,6614	&	0,6191	&	0,5451	&	0,3769	&	0,4053	&	0,2940	\\
12	&	0,0362	&	0,0401	&	0,1110	&	0,0285	&	0,0324	&	0,0975	&	0,0320	&	0,0417	&	0,0875	&	0,1339	&	0,1148	&	0,2499	&	0,0489	&	0,0660	&	0,1311	\\
13	&	0,3070	&	0,3176	&	0,2074	&	0,2276	&	0,2342	&	0,1918	&	0,3660	&	0,3621	&	\textbf{0,2202}	&	0,7149	&	0,6700	&	\textbf{0,5649}	&	0,4439	&	0,4422	&	\textbf{0,3059}	\\
14	&	0,0919	&	0,1225	&	0,1852	&	0,0641	&	0,0843	&	0,1588	&	0,0774	&	0,1262	&	0,1471	&	0,3371	&	0,3076	&	0,4099	&	0,1086	&	0,1768	&	0,2047	\\
15	&	0,2349	&	0,2697	&	\textbf{0,2286}	&	0,1566	&	0,1897	&	\textbf{0,2047}	&	0,2323	&	0,2838	&	0,2060	&	0,6158	&	0,5776	&	0,5453	&	0,2729	&	0,3525	&	0,2758	\\
16	&	0,3179	&	0,3298	&	0,1810	&	0,2382	&	0,2426	&	0,1667	&	0,3911	&	0,3780	&	0,2050	&	\textbf{0,7315}	&	\textbf{0,6828}	&	0,5299	&	0,4765	&	0,4645	&	\textbf{0,2957}	\\
17	&	0,0906	&	0,1204	&	0,1838	&	0,0627	&	0,0819	&	0,1566	&	0,0766	&	0,1251	&	0,1462	&	0,3350	&	0,3054	&	0,4082	&	0,1075	&	0,1746	&	0,2037	\\
18	&	0,0912	&	0,1213	&	0,1844	&	0,0634	&	0,0832	&	0,1580	&	0,0774	&	0,1259	&	0,1468	&	0,3362	&	0,3069	&	0,4094	&	0,1084	&	0,1766	&	0,2043	\\
19	&	0,0903	&	0,1201	&	0,1836	&	0,0625	&	0,0816	&	0,1563	&	0,0767	&	0,1250	&	0,1462	&	0,3349	&	0,3053	&	0,4082	&	0,1075	&	0,1745	&	0,2036	\\
20	&	0,3073	&	0,3215	&	0,2171	&	0,2356	&	0,2497	&	\textbf{0,2053}	&	0,3677	&	0,3680	&	\textbf{0,2243}	&	0,7243	&	\textbf{0,6810}	&	\textbf{0,5749}	&	0,4533	&	0,4589	&	\textbf{0,3101}	\\
21	&	0,2457	&	0,2777	&	\textbf{0,2291}	&	0,1656	&	0,1976	&	\textbf{0,2061}	&	0,2460	&	0,2927	&	0,2090	&	0,6282	&	0,5897	&	\textbf{0,5509}	&	0,2898	&	0,3630	&	0,2800	\\
22	&	0,2552	&	0,2846	&	\textbf{0,2293}	&	0,1741	&	0,2047	&	\textbf{0,2071}	&	0,2585	&	0,3005	&	0,2115	&	0,6388	&	0,6001	&	\textbf{0,5555}	&	0,3057	&	0,3725	&	0,2837	\\
23	&	0,2634	&	0,2907	&	\textbf{0,2292}	&	0,1820	&	0,2112	&	\textbf{0,2078}	&	0,2696	&	0,3074	&	\textbf{0,2136}	&	0,6479	&	0,6090	&	\textbf{0,5594}	&	0,3202	&	0,3809	&	0,2870	\\
24	&	0,2706	&	0,2960	&	\textbf{0,2288}	&	0,1894	&	0,2170	&	\textbf{0,2081}	&	0,2794	&	0,3136	&	\textbf{0,2154}	&	0,6558	&	0,6167	&	\textbf{0,5626}	&	0,3334	&	0,3885	&	0,2899	\\
25	&	0,2768	&	0,3007	&	\textbf{0,2283}	&	0,1960	&	0,2222	&	\textbf{0,2082}	&	0,2882	&	0,3190	&	\textbf{0,2169}	&	0,6627	&	0,6234	&	\textbf{0,5651}	&	0,3453	&	0,3953	&	0,2924	\\
26	&	0,2823	&	0,3049	&	\textbf{0,2276}	&	0,2020	&	0,2270	&	\textbf{0,2081}	&	0,2961	&	0,3239	&	\textbf{0,2181}	&	0,6689	&	0,6294	&	\textbf{0,5671}	&	0,3559	&	0,4014	&	0,2946	\\
27	&	0,2871	&	0,3087	&	\textbf{0,2267}	&	0,2074	&	0,2313	&	\textbf{0,2078}	&	0,3031	&	0,3283	&	\textbf{0,2191}	&	0,6743	&	0,6346	&	\textbf{0,5687}	&	0,3655	&	0,4070	&	\textbf{0,2965}	\\
28	&	0,2913	&	0,3121	&	\textbf{0,2258}	&	0,2123	&	0,2353	&	\textbf{0,2074}	&	0,3094	&	0,3323	&	\textbf{0,2200}	&	0,6792	&	0,6393	&	\textbf{0,5699}	&	0,3742	&	0,4121	&	\textbf{0,2982}	\\
29	&	0,2951	&	0,3151	&	\textbf{0,2249}	&	0,2167	&	0,2389	&	\textbf{0,2068}	&	0,3151	&	0,3360	&	\textbf{0,2207}	&	0,6836	&	0,6436	&	\textbf{0,5709}	&	0,3819	&	0,4167	&	\textbf{0,2997}	\\
30	&	0,2985	&	0,3179	&	\textbf{0,2239}	&	0,2208	&	0,2422	&	\textbf{0,2062}	&	0,3202	&	0,3393	&	\textbf{0,2212}	&	0,6876	&	0,6474	&	\textbf{0,5715}	&	0,3889	&	0,4210	&	\textbf{0,3010}	\\
31	&	0,3015	&	0,3204	&	\textbf{0,2228}	&	0,2245	&	0,2453	&	\textbf{0,2056}	&	0,3249	&	0,3423	&	\textbf{0,2217}	&	0,6912	&	0,6509	&	\textbf{0,5720}	&	0,3953	&	0,4250	&	\textbf{0,3021}	\\
32	&	0,3043	&	0,3228	&	\textbf{0,2218}	&	0,2279	&	0,2481	&	\textbf{0,2048}	&	0,3292	&	0,3451	&	\textbf{0,2220}	&	0,6945	&	0,6541	&	\textbf{0,5723}	&	0,4012	&	0,4286	&	\textbf{0,3031}	\\
33	&	0,3068	&	0,3249	&	\textbf{0,2208}	&	0,2310	&	0,2507	&	\textbf{0,2041}	&	0,3332	&	0,3477	&	\textbf{0,2223}	&	0,6976	&	0,6571	&	\textbf{0,5724}	&	0,4065	&	0,4320	&	\textbf{0,3039}	\\
34	&	0,3091	&	0,3269	&	\textbf{0,2198}	&	0,2339	&	0,2532	&	\textbf{0,2033}	&	0,3368	&	0,3501	&	\textbf{0,2225}	&	0,7004	&	0,6598	&	\textbf{0,5724}	&	0,4114	&	0,4351	&	\textbf{0,3047}	\\
35	&	0,3112	&	0,3287	&	\textbf{0,2187}	&	0,2366	&	0,2555	&	\textbf{0,2026}	&	0,3401	&	0,3523	&	\textbf{0,2226}	&	0,7031	&	0,6623	&	\textbf{0,5723}	&	0,4159	&	0,4381	&	\textbf{0,3053}	\\
36	&	0,3131	&	0,3304	&	0,2177	&	0,2391	&	0,2576	&	\textbf{0,2018}	&	0,3432	&	0,3544	&	\textbf{0,2227}	&	0,7055	&	0,6646	&	\textbf{0,5721}	&	0,4202	&	0,4409	&	\textbf{0,3059}	\\
	\bottomrule
\end{tabular}
 \end{adjustwidth}
\caption[Wyniki badań miar dwuelementowych dla korpusu \emph{KIPI} poddanego podpróbkowaniu klasy negatywnej do 95\%, część 1]{Wyniki badań miar dwuelementowych dla korpusu \emph{KIPI} poddanego podpróbkowaniu klasy negatywnej do 95\%, część 1.}
\label{KIPI_subsampled_5_part_1}
\end{table}

\begin{table}[htp!]
\centering
\footnotesize\setlength{\tabcolsep}{2.5pt}
 \begin{adjustwidth}{-2cm}{}
\begin{tabular}{ l | *{15}{| r}}
	\toprule 
	\textbf{95\%} &	\textbf{16}	&	\textbf{17}	&	\textbf{18}	&	\textbf{19}	&	\textbf{20}	&	\textbf{21}	&	\textbf{22}	&	\textbf{23}	&	\textbf{24}	&	\textbf{25}	&	\textbf{26}	&	\textbf{27}	&	\textbf{28}	&	\textbf{29}	&	\textbf{30}	\\
	\midrule
1	&	\textbf{0,8385}	&	\textbf{0,8006}	&	0,6404	&	0,3636	&	0,3498	&	0,2063	&	0,7750	&	0,7224	&	0,5873	&	0,4598	&	0,4527	&	0,3037	&	0,8458	&	0,8037	&	\textbf{0,7114}	\\
2	&	0,6381	&	0,5671	&	0,3218	&	0,3201	&	0,2765	&	0,0740	&	0,6306	&	0,5486	&	0,3162	&	0,4627	&	0,4248	&	0,1453	&	0,7453	&	0,6867	&	0,4573	\\
3	&	0,1350	&	0,1116	&	0,2656	&	0,0170	&	0,0207	&	0,0549	&	0,1066	&	0,0848	&	0,2082	&	0,0256	&	0,0331	&	0,0881	&	0,1780	&	0,1458	&	0,3417	\\
4	&	0,4865	&	0,4370	&	0,5526	&	0,0609	&	0,0892	&	0,1293	&	0,3531	&	0,3068	&	0,4378	&	0,0918	&	0,1376	&	0,1973	&	0,5199	&	0,4602	&	0,6122	\\
5	&	0,4866	&	0,4370	&	0,5526	&	0,0609	&	0,0892	&	0,1293	&	0,3532	&	0,3069	&	0,4378	&	0,0918	&	0,1376	&	0,1972	&	0,5199	&	0,4602	&	0,6121	\\
6	&	0,4690	&	0,4241	&	0,5483	&	0,0576	&	0,0940	&	0,1369	&	0,3529	&	0,3098	&	0,4522	&	0,0807	&	0,1370	&	0,1967	&	0,5049	&	0,4478	&	0,6088	\\
7	&	0,2355	&	0,2014	&	0,3743	&	0,0323	&	0,0443	&	0,0957	&	0,1991	&	0,1645	&	0,3337	&	0,0414	&	0,0601	&	0,1259	&	0,2871	&	0,2427	&	0,4507	\\
8	&	0,3436	&	0,3007	&	0,4339	&	0,0501	&	0,0673	&	0,1065	&	0,2812	&	0,2381	&	0,3708	&	0,0616	&	0,0897	&	0,1468	&	0,3892	&	0,3351	&	0,5076	\\
9	&	0,1637	&	0,1342	&	0,2918	&	0,0219	&	0,0240	&	0,0584	&	0,1279	&	0,1008	&	0,2263	&	0,0332	&	0,0382	&	0,0935	&	0,2112	&	0,1727	&	0,3697	\\
10	&	0,1637	&	0,1342	&	0,2918	&	0,0219	&	0,0240	&	0,0584	&	0,1279	&	0,1008	&	0,2263	&	0,0332	&	0,0382	&	0,0935	&	0,2112	&	0,1727	&	0,3697	\\
11	&	0,7539	&	0,7150	&	\textbf{0,6717}	&	0,2200	&	0,2422	&	0,2085	&	0,6209	&	0,5687	&	0,5829	&	0,2786	&	0,3182	&	0,2943	&	0,7283	&	0,6785	&	\textbf{0,7165}	\\
12	&	0,2185	&	0,1862	&	0,3754	&	0,0261	&	0,0351	&	0,0811	&	0,1628	&	0,1349	&	0,2949	&	0,0393	&	0,0562	&	0,1264	&	0,2732	&	0,2304	&	0,4545	\\
13	&	0,7769	&	0,7356	&	\textbf{0,6807}	&	0,2825	&	0,2822	&	\textbf{0,2270}	&	0,6663	&	0,6115	&	\textbf{0,6098}	&	0,3332	&	0,3408	&	0,3128	&	0,7263	&	0,6736	&	\textbf{0,7249}	\\
14	&	0,4689	&	0,4241	&	0,5483	&	0,0576	&	0,0940	&	0,1369	&	0,3530	&	0,3099	&	0,4522	&	0,0807	&	0,1370	&	0,1967	&	0,5049	&	0,4478	&	0,6088	\\
15	&	0,7089	&	0,6699	&	\textbf{0,6659}	&	0,1603	&	0,2120	&	0,2037	&	0,5720	&	0,5227	&	0,5810	&	0,1911	&	0,2721	&	0,2751	&	0,6825	&	0,6312	&	\textbf{0,7088}	\\
16	&	0,7912	&	0,7480	&	0,6527	&	0,3070	&	0,2976	&	0,2115	&	0,6965	&	0,6389	&	0,5870	&	0,3595	&	0,3565	&	0,3011	&	0,7442	&	0,6901	&	0,7053	\\
17	&	0,4648	&	0,4194	&	0,5461	&	0,0570	&	0,0928	&	0,1357	&	0,3497	&	0,3066	&	0,4500	&	0,0795	&	0,1342	&	0,1948	&	0,4973	&	0,4396	&	0,6050	\\
18	&	0,4684	&	0,4239	&	0,5477	&	0,0577	&	0,0944	&	0,1372	&	0,3533	&	0,3106	&	0,4523	&	0,0815	&	0,1396	&	0,1984	&	0,5108	&	0,4553	&	0,6119	\\
19	&	0,4646	&	0,4192	&	0,5459	&	0,0570	&	0,0927	&	0,1357	&	0,3496	&	0,3064	&	0,4499	&	0,0794	&	0,1341	&	0,1947	&	0,4970	&	0,4393	&	0,6049	\\
20	&	0,7960	&	0,7596	&	\textbf{0,6898}	&	0,3017	&	0,3083	&	\textbf{0,2377}	&	0,6985	&	0,6495	&	\textbf{0,6250}	&	0,3851	&	0,4067	&	\textbf{0,3334}	&	0,7909	&	0,7494	&	\textbf{0,7437}	\\
21	&	0,7190	&	0,6805	&	\textbf{0,6707}	&	0,1720	&	0,2202	&	0,2076	&	0,5830	&	0,5337	&	0,5869	&	0,2037	&	0,2815	&	0,2803	&	0,6910	&	0,6403	&	\textbf{0,7133}	\\
22	&	0,7278	&	0,6896	&	\textbf{0,6747}	&	0,1829	&	0,2275	&	0,2111	&	0,5929	&	0,5435	&	0,5920	&	0,2159	&	0,2901	&	0,2849	&	0,6987	&	0,6484	&	\textbf{0,7172}	\\
23	&	0,7353	&	0,6975	&	\textbf{0,6780}	&	0,1928	&	0,2341	&	0,2141	&	0,6016	&	0,5523	&	\textbf{0,5962}	&	0,2278	&	0,2979	&	0,2890	&	0,7055	&	0,6557	&	\textbf{0,7206}	\\
24	&	0,7419	&	0,7044	&	\textbf{0,6806}	&	0,2019	&	0,2400	&	0,2166	&	0,6094	&	0,5601	&	\textbf{0,5999}	&	0,2390	&	0,3049	&	0,2927	&	0,7118	&	0,6622	&	\textbf{0,7234}	\\
25	&	0,7477	&	0,7103	&	\textbf{0,6827}	&	0,2101	&	0,2453	&	0,2189	&	0,6164	&	0,5671	&	\textbf{0,6030}	&	0,2498	&	0,3114	&	0,2960	&	0,7174	&	0,6682	&	\textbf{0,7258}	\\
26	&	0,7529	&	0,7157	&	\textbf{0,6845}	&	0,2176	&	0,2502	&	0,2208	&	0,6228	&	0,5735	&	\textbf{0,6056}	&	0,2597	&	0,3173	&	0,2989	&	0,7226	&	0,6737	&	\textbf{0,7279}	\\
27	&	0,7575	&	0,7204	&	\textbf{0,6859}	&	0,2243	&	0,2546	&	0,2224	&	0,6287	&	0,5793	&	\textbf{0,6077}	&	0,2687	&	0,3227	&	0,3015	&	0,7273	&	0,6787	&	\textbf{0,7296}	\\
28	&	0,7617	&	0,7247	&	\textbf{0,6870}	&	0,2304	&	0,2586	&	0,2237	&	0,6341	&	0,5847	&	\textbf{0,6096}	&	0,2769	&	0,3277	&	0,3037	&	0,7317	&	0,6833	&	\textbf{0,7311}	\\
29	&	0,7654	&	0,7285	&	\textbf{0,6877}	&	0,2360	&	0,2623	&	0,2249	&	0,6391	&	0,5897	&	\textbf{0,6111}	&	0,2844	&	0,3323	&	0,3058	&	0,7358	&	0,6876	&	\textbf{0,7323}	\\
30	&	0,7688	&	0,7320	&	\textbf{0,6884}	&	0,2411	&	0,2658	&	\textbf{0,2259}	&	0,6438	&	0,5943	&	\textbf{0,6124}	&	0,2913	&	0,3366	&	0,3076	&	0,7396	&	0,6916	&	\textbf{0,7333}	\\
31	&	0,7720	&	0,7352	&	\textbf{0,6888}	&	0,2458	&	0,2689	&	\textbf{0,2267}	&	0,6481	&	0,5986	&	\textbf{0,6134}	&	0,2976	&	0,3406	&	0,3091	&	0,7432	&	0,6954	&	\textbf{0,7342}	\\
32	&	0,7748	&	0,7381	&	\textbf{0,6890}	&	0,2501	&	0,2719	&	\textbf{0,2274}	&	0,6522	&	0,6027	&	\textbf{0,6143}	&	0,3034	&	0,3443	&	0,3105	&	0,7465	&	0,6989	&	\textbf{0,7349}	\\
33	&	0,7775	&	0,7409	&	\textbf{0,6892}	&	0,2542	&	0,2747	&	\textbf{0,2280}	&	0,6560	&	0,6065	&	\textbf{0,6150}	&	0,3088	&	0,3478	&	0,3117	&	0,7497	&	0,7022	&	\textbf{0,7355}	\\
34	&	0,7799	&	0,7434	&	\textbf{0,6892}	&	0,2579	&	0,2772	&	\textbf{0,2284}	&	0,6596	&	0,6101	&	\textbf{0,6156}	&	0,3138	&	0,3511	&	0,3128	&	0,7527	&	0,7054	&	\textbf{0,7360}	\\
35	&	0,7822	&	0,7457	&	\textbf{0,6892}	&	0,2614	&	0,2797	&	\textbf{0,2288}	&	0,6630	&	0,6135	&	\textbf{0,6160}	&	0,3184	&	0,3542	&	0,3138	&	0,7555	&	0,7083	&	\textbf{0,7364}	\\
36	&	0,7843	&	0,7479	&	\textbf{0,6891}	&	0,2647	&	0,2819	&	\textbf{0,2291}	&	0,6663	&	0,6167	&	\textbf{0,6164}	&	0,3228	&	0,3571	&	0,3146	&	0,7582	&	0,7111	&	\textbf{0,7368}	\\
	\bottomrule
\end{tabular}
 \end{adjustwidth}
\caption[Wyniki badań miar dwuelementowych dla korpusu \emph{KIPI} poddanego podpróbkowaniu klasy negatywnej do 95\%, część 2]{Wyniki badań miar dwuelementowych dla korpusu \emph{KIPI} poddanego podpróbkowaniu klasy negatywnej do 95\%, część 2.}
\label{KIPI_subsampled_5_part_2}
\end{table}

\begin{table}[htp!]
\centering
\footnotesize\setlength{\tabcolsep}{2.5pt}
 \begin{adjustwidth}{-2cm}{}
\begin{tabular}{ l | *{15}{| r}}
	\toprule 
	\textbf{95\%} &	\textbf{1}	&	\textbf{2}	&	\textbf{3}	&	\textbf{4}	&	\textbf{5}	&	\textbf{6}	&	\textbf{7}	&	\textbf{8}	&	\textbf{9}	&	\textbf{10}	&	\textbf{11}	&	\textbf{12}	&	\textbf{13}	&	\textbf{14}	&	\textbf{15}	\\
	\midrule
37	&	0,3148	&	0,3320	&	0,2168	&	0,2414	&	0,2596	&	\textbf{0,2010}	&	0,3460	&	0,3563	&	\textbf{0,2227}	&	0,7078	&	0,6668	&	\textbf{0,5719}	&	0,4241	&	0,4434	&	\textbf{0,3064}	\\
38	&	0,3165	&	0,3335	&	0,2158	&	0,2436	&	0,2615	&	\textbf{0,2002}	&	0,3487	&	0,3581	&	\textbf{0,2227}	&	0,7099	&	0,6688	&	\textbf{0,5716}	&	0,4277	&	0,4459	&	\textbf{0,3068}	\\
39	&	0,3180	&	0,3349	&	0,2149	&	0,2456	&	0,2633	&	\textbf{0,1994}	&	0,3512	&	0,3598	&	\textbf{0,2227}	&	0,7119	&	0,6707	&	\textbf{0,5712}	&	0,4311	&	0,4481	&	\textbf{0,3071}	\\
40	&	0,3194	&	0,3362	&	0,2140	&	0,2476	&	0,2649	&	\textbf{0,1987}	&	0,3535	&	0,3614	&	\textbf{0,2226}	&	0,7138	&	0,6725	&	\textbf{0,5708}	&	0,4343	&	0,4503	&	\textbf{0,3074}	\\
41	&	0,3207	&	0,3375	&	0,2131	&	0,2494	&	0,2665	&	\textbf{0,1979}	&	0,3557	&	0,3629	&	\textbf{0,2225}	&	0,7156	&	0,6742	&	\textbf{0,5703}	&	0,4373	&	0,4523	&	\textbf{0,3077}	\\
42	&	0,3220	&	0,3386	&	0,2122	&	0,2511	&	0,2680	&	0,1972	&	0,3578	&	0,3643	&	\textbf{0,2224}	&	0,7173	&	0,6758	&	\textbf{0,5699}	&	0,4401	&	0,4543	&	\textbf{0,3079}	\\
43	&	0,3231	&	0,3397	&	0,2114	&	0,2527	&	0,2694	&	0,1965	&	0,3598	&	0,3656	&	\textbf{0,2223}	&	0,7189	&	0,6772	&	\textbf{0,5694}	&	0,4428	&	0,4561	&	\textbf{0,3081}	\\
44	&	0,3242	&	0,3407	&	0,2106	&	0,2542	&	0,2707	&	0,1958	&	0,3616	&	0,3669	&	\textbf{0,2221}	&	0,7204	&	0,6787	&	\textbf{0,5689}	&	0,4453	&	0,4578	&	\textbf{0,3082}	\\
45	&	0,3253	&	0,3417	&	0,2098	&	0,2556	&	0,2719	&	0,1951	&	0,3634	&	0,3681	&	\textbf{0,2220}	&	0,7218	&	0,6800	&	\textbf{0,5683}	&	0,4477	&	0,4595	&	\textbf{0,3083}	\\
46	&	0,3262	&	0,3426	&	0,2090	&	0,2570	&	0,2731	&	0,1944	&	0,3651	&	0,3692	&	\textbf{0,2218}	&	0,7232	&	\textbf{0,6813}	&	\textbf{0,5678}	&	0,4499	&	0,4610	&	\textbf{0,3084}	\\
47	&	0,3271	&	0,3435	&	0,2083	&	0,2583	&	0,2743	&	0,1938	&	0,3666	&	0,3703	&	\textbf{0,2217}	&	0,7244	&	\textbf{0,6824}	&	\textbf{0,5673}	&	0,4521	&	0,4625	&	\textbf{0,3085}	\\
48	&	0,3280	&	0,3443	&	0,2076	&	0,2595	&	0,2753	&	0,1932	&	0,3682	&	0,3713	&	\textbf{0,2215}	&	0,7257	&	\textbf{0,6836}	&	\textbf{0,5667}	&	0,4541	&	0,4639	&	\textbf{0,3085}	\\
49	&	0,3288	&	0,3451	&	0,2069	&	0,2607	&	0,2764	&	0,1926	&	0,3696	&	0,3723	&	\textbf{0,2213}	&	\textbf{0,7268}	&	\textbf{0,6846}	&	\textbf{0,5662}	&	0,4561	&	0,4653	&	\textbf{0,3086}	\\
50	&	0,3296	&	0,3459	&	0,2062	&	0,2618	&	0,2773	&	0,1920	&	0,3709	&	0,3732	&	\textbf{0,2211}	&	\textbf{0,7279}	&	\textbf{0,6857}	&	\textbf{0,5657}	&	0,4579	&	0,4666	&	\textbf{0,3086}	\\
51	&	0,2950	&	0,3040	&	0,2108	&	0,2121	&	0,2176	&	0,1945	&	0,3574	&	0,3543	&	\textbf{0,2208}	&	0,7044	&	0,6592	&	\textbf{0,5669}	&	0,4292	&	0,4284	&	\textbf{0,3047}	\\
52	&	0,2787	&	0,2856	&	0,2138	&	0,1946	&	0,1988	&	0,1967	&	0,3468	&	0,3448	&	\textbf{0,2209}	&	0,6911	&	0,6461	&	\textbf{0,5681}	&	0,4110	&	0,4107	&	\textbf{0,3019}	\\
53	&	0,2572	&	0,2612	&	0,2151	&	0,1752	&	0,1777	&	0,1967	&	0,3340	&	0,3331	&	\textbf{0,2204}	&	0,6742	&	0,6291	&	\textbf{0,5679}	&	0,3884	&	0,3883	&	\textbf{0,2968}	\\
54	&	0,2364	&	0,2401	&	0,2130	&	0,1584	&	0,1609	&	0,1921	&	0,3191	&	0,3192	&	\textbf{0,2186}	&	0,6540	&	0,6082	&	\textbf{0,5652}	&	0,3643	&	0,3664	&	0,2888	\\
55	&	0,2258	&	0,2293	&	0,2071	&	0,1483	&	0,1507	&	0,1838	&	0,3007	&	0,3019	&	\textbf{0,2152}	&	0,6275	&	0,5818	&	\textbf{0,5585}	&	0,3391	&	0,3420	&	0,2781	\\
56	&	0,2180	&	0,2205	&	0,2044	&	0,1390	&	0,1408	&	0,1782	&	0,2782	&	0,2802	&	0,2092	&	0,5926	&	0,5470	&	0,5461	&	0,3094	&	0,3124	&	0,2643	\\
57	&	0,2075	&	0,2088	&	0,2011	&	0,1271	&	0,1281	&	0,1709	&	0,2519	&	0,2540	&	0,1996	&	0,5492	&	0,5030	&	0,5248	&	0,2745	&	0,2780	&	0,2465	\\
58	&	0,1918	&	0,1922	&	0,1930	&	0,1113	&	0,1116	&	0,1594	&	0,2246	&	0,2270	&	0,1863	&	0,5036	&	0,4570	&	0,4933	&	0,2349	&	0,2391	&	0,2237	\\
59	&	0,1705	&	0,1702	&	0,1793	&	0,0927	&	0,0918	&	0,1437	&	0,1953	&	0,1976	&	0,1696	&	0,4521	&	0,4060	&	0,4532	&	0,1933	&	0,1982	&	0,1973	\\
60	&	0,1446	&	0,1442	&	0,1609	&	0,0725	&	0,0717	&	0,1247	&	0,1648	&	0,1671	&	0,1516	&	0,3969	&	0,3499	&	0,4093	&	0,1509	&	0,1560	&	0,1689	\\
61	&	0,1164	&	0,1152	&	0,1390	&	0,0542	&	0,0533	&	0,1043	&	0,1339	&	0,1358	&	0,1317	&	0,3366	&	0,2914	&	0,3577	&	0,1124	&	0,1169	&	0,1405	\\
62	&	0,0872	&	0,0860	&	0,1153	&	0,0421	&	0,0418	&	0,0852	&	0,1029	&	0,1047	&	0,1109	&	0,2741	&	0,2308	&	0,3040	&	0,0807	&	0,0840	&	0,1150	\\
63	&	0,0612	&	0,0584	&	0,0925	&	0,0346	&	0,0343	&	0,0694	&	0,0746	&	0,0757	&	0,0908	&	0,2115	&	0,1723	&	0,2515	&	0,0606	&	0,0625	&	0,0945	\\
64	&	0,0441	&	0,0422	&	0,0738	&	0,0290	&	0,0290	&	0,0579	&	0,0532	&	0,0534	&	0,0742	&	0,1626	&	0,1308	&	0,2079	&	0,0483	&	0,0501	&	0,0791	\\
65	&	0,0350	&	0,0340	&	0,0600	&	0,0248	&	0,0248	&	0,0501	&	0,0402	&	0,0404	&	0,0613	&	0,1328	&	0,1057	&	0,1732	&	0,0394	&	0,0414	&	0,0667	\\
66	&	0,0289	&	0,0281	&	0,0506	&	0,0216	&	0,0217	&	0,0445	&	0,0321	&	0,0327	&	0,0513	&	0,1110	&	0,0881	&	0,1470	&	0,0329	&	0,0348	&	0,0579	\\
67	&	0,0244	&	0,0237	&	0,0438	&	0,0190	&	0,0191	&	0,0397	&	0,0266	&	0,0272	&	0,0435	&	0,0939	&	0,0744	&	0,1265	&	0,0282	&	0,0299	&	0,0517	\\
68	&	0,0210	&	0,0202	&	0,0385	&	0,0170	&	0,0170	&	0,0363	&	0,0223	&	0,0231	&	0,0381	&	0,0808	&	0,0641	&	0,1114	&	0,0245	&	0,0264	&	0,0473	\\
69	&	0,0183	&	0,0174	&	0,0347	&	0,0154	&	0,0155	&	0,0335	&	0,0191	&	0,0198	&	0,0338	&	0,0713	&	0,0563	&	0,0997	&	0,0230	&	0,0249	&	0,0444	\\
70	&	0,0155	&	0,0149	&	0,0318	&	0,0139	&	0,0141	&	0,0314	&	0,0157	&	0,0166	&	0,0307	&	0,0629	&	0,0499	&	0,0912	&	0,0220	&	0,0239	&	0,0425	\\
71	&	0,1766	&	0,1841	&	0,0659	&	0,2489	&	0,2583	&	0,0793	&	0,2672	&	0,2528	&	0,0999	&	0,6136	&	0,5520	&	0,3038	&	0,3646	&	0,3483	&	0,1489	\\
72	&	0,2291	&	0,2402	&	0,0883	&	0,2321	&	0,2408	&	0,0825	&	0,3127	&	0,2952	&	0,1234	&	0,6794	&	0,6198	&	0,3665	&	0,4199	&	0,4003	&	0,1829	\\
	\bottomrule
\end{tabular}
 \end{adjustwidth}
\caption[Wyniki badań miar dwuelementowych dla korpusu \emph{KIPI} poddanego podpróbkowaniu klasy negatywnej do 95\%, część 3]{Wyniki badań miar dwuelementowych dla korpusu \emph{KIPI} poddanego podpróbkowaniu klasy negatywnej do 95\%, część 3.}
\label{KIPI_subsampled_5_part_3}
\end{table}

\begin{table}[htp!]
\centering
\footnotesize\setlength{\tabcolsep}{2.5pt}
 \begin{adjustwidth}{-2cm}{}
\begin{tabular}{ l | *{15}{| r}}
	\toprule 
	\textbf{95\%} &	\textbf{16}	&	\textbf{17}	&	\textbf{18}	&	\textbf{19}	&	\textbf{20}	&	\textbf{21}	&	\textbf{22}	&	\textbf{23}	&	\textbf{24}	&	\textbf{25}	&	\textbf{26}	&	\textbf{27}	&	\textbf{28}	&	\textbf{29}	&	\textbf{30}	\\
	\midrule
37	&	0,7863	&	0,7499	&	\textbf{0,6889}	&	0,2677	&	0,2841	&	\textbf{0,2293}	&	0,6693	&	0,6197	&	\textbf{0,6167}	&	0,3269	&	0,3599	&	0,3154	&	0,7607	&	0,7138	&	\textbf{0,7370}	\\
38	&	0,7882	&	0,7518	&	\textbf{0,6886}	&	0,2706	&	0,2861	&	\textbf{0,2295}	&	0,6723	&	0,6226	&	\textbf{0,6169}	&	0,3307	&	0,3625	&	0,3160	&	0,7631	&	0,7163	&	\textbf{0,7372}	\\
39	&	0,7900	&	0,7535	&	\textbf{0,6883}	&	0,2733	&	0,2880	&	\textbf{0,2296}	&	0,6750	&	0,6254	&	\textbf{0,6170}	&	0,3344	&	0,3650	&	0,3166	&	0,7654	&	0,7187	&	\textbf{0,7373}	\\
40	&	0,7916	&	0,7552	&	\textbf{0,6880}	&	0,2758	&	0,2898	&	\textbf{0,2297}	&	0,6777	&	0,6280	&	\textbf{0,6171}	&	0,3378	&	0,3673	&	\textbf{0,3171}	&	0,7676	&	0,7210	&	\textbf{0,7374}	\\
41	&	0,7932	&	0,7568	&	\textbf{0,6876}	&	0,2782	&	0,2915	&	\textbf{0,2298}	&	0,6802	&	0,6305	&	\textbf{0,6171}	&	0,3410	&	0,3695	&	\textbf{0,3175}	&	0,7697	&	0,7232	&	\textbf{0,7374}	\\
42	&	0,7946	&	0,7583	&	\textbf{0,6873}	&	0,2805	&	0,2931	&	\textbf{0,2298}	&	0,6827	&	0,6329	&	\textbf{0,6171}	&	0,3441	&	0,3717	&	\textbf{0,3179}	&	0,7716	&	0,7253	&	\textbf{0,7375}	\\
43	&	0,7960	&	0,7597	&	\textbf{0,6869}	&	0,2827	&	0,2947	&	\textbf{0,2297}	&	0,6850	&	0,6352	&	\textbf{0,6171}	&	0,3469	&	0,3737	&	\textbf{0,3183}	&	0,7735	&	0,7273	&	\textbf{0,7374}	\\
44	&	\textbf{0,7974}	&	\textbf{0,7610}	&	\textbf{0,6864}	&	0,2848	&	0,2962	&	\textbf{0,2297}	&	0,6872	&	0,6374	&	\textbf{0,6170}	&	0,3497	&	0,3756	&	\textbf{0,3186}	&	0,7754	&	0,7292	&	\textbf{0,7374}	\\
45	&	\textbf{0,7986}	&	\textbf{0,7623}	&	\textbf{0,6860}	&	0,2868	&	0,2976	&	\textbf{0,2296}	&	0,6893	&	0,6394	&	\textbf{0,6170}	&	0,3524	&	0,3775	&	\textbf{0,3188}	&	0,7771	&	0,7310	&	\textbf{0,7373}	\\
46	&	\textbf{0,7998}	&	\textbf{0,7635}	&	\textbf{0,6856}	&	0,2887	&	0,2989	&	\textbf{0,2295}	&	0,6913	&	0,6414	&	\textbf{0,6168}	&	0,3549	&	0,3793	&	\textbf{0,3190}	&	0,7788	&	0,7328	&	\textbf{0,7372}	\\
47	&	\textbf{0,8009}	&	\textbf{0,7646}	&	\textbf{0,6851}	&	0,2904	&	0,3002	&	\textbf{0,2294}	&	0,6933	&	0,6434	&	\textbf{0,6167}	&	0,3573	&	0,3810	&	\textbf{0,3192}	&	0,7804	&	0,7345	&	\textbf{0,7371}	\\
48	&	\textbf{0,8020}	&	\textbf{0,7657}	&	\textbf{0,6847}	&	0,2922	&	0,3014	&	\textbf{0,2293}	&	0,6951	&	0,6452	&	\textbf{0,6166}	&	0,3596	&	0,3826	&	\textbf{0,3193}	&	0,7819	&	0,7361	&	\textbf{0,7370}	\\
49	&	\textbf{0,8031}	&	\textbf{0,7667}	&	\textbf{0,6842}	&	0,2938	&	0,3025	&	\textbf{0,2292}	&	0,6969	&	0,6470	&	\textbf{0,6164}	&	0,3618	&	0,3841	&	\textbf{0,3195}	&	0,7833	&	0,7376	&	\textbf{0,7369}	\\
50	&	\textbf{0,8041}	&	\textbf{0,7677}	&	\textbf{0,6838}	&	0,2954	&	0,3036	&	\textbf{0,2291}	&	0,6986	&	0,6487	&	\textbf{0,6162}	&	0,3640	&	0,3856	&	\textbf{0,3196}	&	0,7848	&	0,7391	&	\textbf{0,7367}	\\
51	&	0,7624	&	0,7202	&	\textbf{0,6812}	&	0,2721	&	0,2726	&	\textbf{0,2270}	&	0,6497	&	0,5948	&	\textbf{0,6090}	&	0,3168	&	0,3254	&	0,3101	&	0,7053	&	0,6522	&	\textbf{0,7222}	\\
52	&	0,7433	&	0,6994	&	\textbf{0,6795}	&	0,2601	&	0,2614	&	\textbf{0,2262}	&	0,6305	&	0,5751	&	\textbf{0,6069}	&	0,2989	&	0,3081	&	0,3061	&	0,6824	&	0,6284	&	\textbf{0,7179}	\\
53	&	0,7166	&	0,6711	&	\textbf{0,6745}	&	0,2464	&	0,2483	&	0,2245	&	0,6075	&	0,5516	&	\textbf{0,6032}	&	0,2779	&	0,2879	&	0,3001	&	0,6541	&	0,5996	&	\textbf{0,7115}	\\
54	&	0,6891	&	0,6437	&	\textbf{0,6638}	&	0,2312	&	0,2337	&	0,2215	&	0,5817	&	0,5254	&	\textbf{0,5971}	&	0,2577	&	0,2686	&	0,2900	&	0,6278	&	0,5728	&	0,6991	\\
55	&	0,6602	&	0,6132	&	0,6481	&	0,2140	&	0,2173	&	0,2165	&	0,5525	&	0,4969	&	0,5878	&	0,2359	&	0,2473	&	0,2767	&	0,5997	&	0,5435	&	0,6819	\\
56	&	0,6236	&	0,5744	&	0,6250	&	0,1945	&	0,1980	&	0,2086	&	0,5173	&	0,4619	&	0,5728	&	0,2101	&	0,2220	&	0,2603	&	0,5648	&	0,5074	&	0,6600	\\
57	&	0,5778	&	0,5262	&	0,5951	&	0,1733	&	0,1766	&	0,1967	&	0,4784	&	0,4231	&	0,5496	&	0,1813	&	0,1927	&	0,2388	&	0,5231	&	0,4638	&	0,6290	\\
58	&	0,5208	&	0,4684	&	0,5545	&	0,1513	&	0,1547	&	0,1805	&	0,4397	&	0,3833	&	0,5171	&	0,1493	&	0,1601	&	0,2134	&	0,4712	&	0,4123	&	0,5892	\\
59	&	0,4572	&	0,4047	&	0,5037	&	0,1281	&	0,1308	&	0,1620	&	0,3959	&	0,3390	&	0,4787	&	0,1171	&	0,1278	&	0,1853	&	0,4150	&	0,3576	&	0,5414	\\
60	&	0,3859	&	0,3348	&	0,4447	&	0,1039	&	0,1061	&	0,1424	&	0,3486	&	0,2907	&	0,4355	&	0,0873	&	0,0966	&	0,1559	&	0,3554	&	0,2988	&	0,4846	\\
61	&	0,3124	&	0,2643	&	0,3813	&	0,0798	&	0,0816	&	0,1214	&	0,2971	&	0,2415	&	0,3866	&	0,0652	&	0,0716	&	0,1272	&	0,2993	&	0,2465	&	0,4249	\\
62	&	0,2449	&	0,2016	&	0,3198	&	0,0583	&	0,0598	&	0,1003	&	0,2455	&	0,1931	&	0,3353	&	0,0514	&	0,0567	&	0,1039	&	0,2543	&	0,2084	&	0,3699	\\
63	&	0,1984	&	0,1617	&	0,2677	&	0,0436	&	0,0447	&	0,0810	&	0,2042	&	0,1590	&	0,2850	&	0,0425	&	0,0472	&	0,0866	&	0,2210	&	0,1799	&	0,3226	\\
64	&	0,1667	&	0,1354	&	0,2282	&	0,0350	&	0,0362	&	0,0662	&	0,1740	&	0,1343	&	0,2427	&	0,0358	&	0,0402	&	0,0740	&	0,1956	&	0,1578	&	0,2859	\\
65	&	0,1412	&	0,1139	&	0,1959	&	0,0292	&	0,0305	&	0,0554	&	0,1507	&	0,1158	&	0,2097	&	0,0304	&	0,0346	&	0,0642	&	0,1742	&	0,1401	&	0,2545	\\
66	&	0,1222	&	0,0985	&	0,1715	&	0,0248	&	0,0260	&	0,0474	&	0,1314	&	0,1008	&	0,1846	&	0,0263	&	0,0301	&	0,0568	&	0,1584	&	0,1271	&	0,2312	\\
67	&	0,1093	&	0,0876	&	0,1546	&	0,0213	&	0,0226	&	0,0411	&	0,1168	&	0,0892	&	0,1633	&	0,0233	&	0,0271	&	0,0515	&	0,1478	&	0,1182	&	0,2139	\\
68	&	0,1005	&	0,0805	&	0,1426	&	0,0183	&	0,0196	&	0,0362	&	0,1046	&	0,0798	&	0,1462	&	0,0210	&	0,0248	&	0,0477	&	0,1410	&	0,1126	&	0,2014	\\
69	&	0,0966	&	0,0775	&	0,1349	&	0,0160	&	0,0175	&	0,0327	&	0,0960	&	0,0733	&	0,1333	&	0,0203	&	0,0239	&	0,0452	&	0,1381	&	0,1103	&	0,1931	\\
70	&	0,0946	&	0,0755	&	0,1296	&	0,0139	&	0,0155	&	0,0299	&	0,0889	&	0,0681	&	0,1240	&	0,0198	&	0,0234	&	0,0436	&	0,1364	&	0,1088	&	0,1876	\\
71	&	0,7156	&	0,6623	&	0,4065	&	\textbf{0,4408}	&	\textbf{0,4221}	&	0,1694	&	\textbf{0,8427}	&	\textbf{0,7991}	&	0,5197	&	\textbf{0,5471}	&	\textbf{0,5358}	&	0,2432	&	\textbf{0,8995}	&	\textbf{0,8653}	&	0,6434	\\
72	&	0,7730	&	0,7249	&	0,4789	&	0,3927	&	0,3739	&	0,1679	&	\textbf{0,8163}	&	\textbf{0,7679}	&	0,5195	&	0,4991	&	0,4867	&	0,2414	&	\textbf{0,8805}	&	\textbf{0,8413}	&	0,6430	\\
	\bottomrule
\end{tabular}
 \end{adjustwidth}
\caption[Wyniki badań miar dwuelementowych dla korpusu \emph{KIPI} poddanego podpróbkowaniu klasy negatywnej do 95\%, część 4]{Wyniki badań miar dwuelementowych dla korpusu \emph{KIPI} poddanego podpróbkowaniu klasy negatywnej do 95\%, część 4.}
\label{KIPI_subsampled_5_part_4}
\end{table}

Zgodnie z oczekiwaniami jakość wyników spadła.
Było to do przewidzenia, jeśli wziąć pod uwagę czterokrotne zmniejszenie liczby instancji klasy pozytywnej -- wyrażeń wielowyrazowych, i zwiększenie liczby instancji z klasy negatywnej w tym problemie binarnym. 
Zestaw funkcji osiągających najlepsze wyniki nie uległ zmianie w stosunku do badania poprzedniego -- podprókowania klasy negatywnej na poziomie 80\%.

\par
Najlepszy z osiągniętych wyników w tym badaniu uplasował się na poziomie około 93\% najlepszego z rezultatów osiągniętych w badaniu poprzednim.
Jednak zaznaczyć trzeba, że w ogólności jakość wyników uległa dużym zmianom, w niektórych przypadkach nawet około dwuipółkrotnie obniżając wynik; przykładowo z 0,8174 do 0,3148 czyli o około 61\%.

\par
Wnioskiem z tego i poprzedniego badania może być to, że nawet proste funkcje, takie jak zwykła częstość czy wartość oczekiwana częstości mogą osiągać bardzo dobre wyniki, jeśli zbiór danych jest klasowo zbalansowany.
Dodatkowo, biorąc pod uwagę wyniki niniejszego badania, a także badania poprzedniego i pierwszego można stwierdzić, że w tym problemie zbalansowanie klas wydaje się mieć znaczący wpływ na jakość generowanych rozwiązań .

\newpage

\subsubsection{Wyniki badań jakości rozwiązań generowanych przez sieci neuronowe}
Tabele \ref{KIPI_nn_part_1} oraz \ref{KIPI_nn_part_2} prezentują jakość wyników osiągniętych przez 48 sieci neuronowych w 30 różnych badaniach (30 zestawów danych pozyskanych z korpusu \emph{KIPI}).
Indeksy typów badań pozostały takie same, jak w poprzednich badaniach, a indeksy zbadanych sieci neuronowych są identyczne z przedstawionymi w tabeli we wcześniejszej części tego rozdziału pracy.

\begin{table}[htp!]
\centering
\footnotesize\setlength{\tabcolsep}{2.5pt}
 \begin{adjustwidth}{-2cm}{}
\begin{tabular}{ l | *{15}{| r}}
	\toprule 
	\textbf{99\%} &	\textbf{1}	&	\textbf{2}	&	\textbf{3}	&	\textbf{4}	&	\textbf{5}	&	\textbf{6}	&	\textbf{7}	&	\textbf{8}	&	\textbf{9}	&	\textbf{10}	&	\textbf{11}	&	\textbf{12}	&	\textbf{13}	&	\textbf{14}	&	\textbf{15}	\\
	\midrule
1	&	0,0408	&	0,0329	&	0,0804	&	0,0186	&	0,0162	&	0,0453	&	0,0623	&	0,0653	&	0,1243	&	0,2594	&	0,2180	&	0,3823	&	0,0633	&	0,0500	&	0,1170	\\
2	&	0,0413	&	0,0344	&	0,0795	&	0,0186	&	0,0186	&	0,0452	&	0,0616	&	0,0693	&	0,1282	&	0,2616	&	0,2200	&	0,3863	&	0,0623	&	0,0571	&	0,1195	\\
3	&	0,0400	&	0,0349	&	0,0753	&	0,0182	&	0,0169	&	0,0454	&	0,0629	&	0,0637	&	0,1297	&	0,2596	&	0,2263	&	0,3923	&	0,0643	&	0,0557	&	0,1259	\\
4	&	0,0388	&	0,0323	&	0,0761	&	0,0180	&	0,0202	&	0,0462	&	0,0628	&	0,0690	&	0,1285	&	0,2638	&	0,2146	&	0,3975	&	0,0570	&	0,0628	&	0,1225	\\
5	&	0,0431	&	0,0342	&	0,0740	&	0,0183	&	0,0170	&	0,0456	&	0,0629	&	0,0707	&	0,1317	&	0,2429	&	0,2285	&	0,3941	&	0,0623	&	0,0624	&	0,1281	\\
6	&	0,0402	&	0,0330	&	0,0781	&	0,0182	&	0,0213	&	0,0449	&	0,0625	&	0,0713	&	0,1312	&	0,2569	&	\textbf{0,2436}	&	0,3945	&	0,0650	&	0,0663	&	0,1327	\\
7	&	0,0415	&	0,0356	&	0,0777	&	0,0179	&	0,0201	&	0,0450	&	0,0623	&	0,0742	&	0,1326	&	0,2570	&	\textbf{0,2434}	&	\textbf{0,4011}	&	0,0614	&	0,0652	&	0,1317	\\
8	&	0,0423	&	0,0352	&	0,0839	&	0,0179	&	0,0216	&	0,0450	&	0,0633	&	\textbf{0,0762}	&	0,1322	&	0,2555	&	0,2287	&	0,3941	&	0,0661	&	0,0665	&	0,1340	\\
9	&	0,0436	&	0,0354	&	0,0843	&	0,0179	&	0,0180	&	0,0449	&	0,0615	&	0,0707	&	0,1330	&	0,2588	&	0,2280	&	0,3927	&	0,0675	&	0,0641	&	\textbf{0,1358}	\\
10	&	0,0445	&	0,0349	&	0,0894	&	0,0178	&	0,0222	&	0,0449	&	0,0606	&	0,0740	&	0,1335	&	0,2588	&	0,2359	&	0,3953	&	0,0671	&	0,0646	&	0,1284	\\
11	&	0,0448	&	0,0336	&	0,0854	&	0,0180	&	0,0188	&	0,0462	&	0,0627	&	0,0713	&	0,1370	&	0,2495	&	0,2283	&	0,3970	&	0,0688	&	0,0687	&	0,1346	\\
12	&	0,0425	&	0,0338	&	0,0881	&	0,0177	&	0,0212	&	0,0455	&	0,0613	&	0,0711	&	0,1363	&	0,2420	&	0,2281	&	0,3957	&	\textbf{0,0719}	&	0,0675	&	0,1338	\\
13	&	0,0394	&	0,0283	&	0,0776	&	\textbf{0,0187}	&	0,0125	&	0,0429	&	0,0635	&	0,0615	&	0,1268	&	0,2652	&	0,2119	&	\textbf{0,4043}	&	0,0635	&	0,0523	&	0,1228	\\
14	&	0,0416	&	0,0310	&	0,0791	&	0,0186	&	0,0178	&	0,0454	&	0,0636	&	0,0688	&	0,1275	&	0,2645	&	0,2189	&	0,3906	&	0,0649	&	0,0569	&	0,1230	\\
15	&	0,0399	&	0,0276	&	0,0817	&	0,0186	&	0,0176	&	0,0459	&	\textbf{0,0642}	&	0,0657	&	0,1318	&	0,2651	&	0,2164	&	0,3971	&	0,0656	&	0,0573	&	0,1167	\\
16	&	0,0415	&	0,0282	&	0,0776	&	\textbf{0,0188}	&	0,0118	&	0,0457	&	\textbf{0,0643}	&	0,0695	&	0,1323	&	0,2635	&	0,2056	&	0,3934	&	0,0622	&	0,0576	&	0,1140	\\
17	&	0,0422	&	0,0334	&	0,0799	&	\textbf{0,0188}	&	0,0190	&	0,0445	&	0,0629	&	0,0635	&	0,1303	&	0,2652	&	0,2252	&	0,3974	&	0,0664	&	0,0575	&	0,1239	\\
18	&	0,0420	&	0,0334	&	0,0784	&	0,0182	&	0,0187	&	0,0466	&	0,0633	&	0,0658	&	0,1328	&	0,2588	&	0,2316	&	0,3893	&	0,0657	&	0,0572	&	0,1294	\\
19	&	0,0428	&	0,0352	&	0,0790	&	0,0178	&	0,0170	&	0,0433	&	0,0639	&	0,0696	&	\textbf{0,1386}	&	0,2633	&	0,2413	&	0,3949	&	0,0658	&	0,0615	&	0,1264	\\
20	&	0,0426	&	0,0310	&	0,0816	&	0,0183	&	0,0105	&	0,0457	&	0,0635	&	0,0728	&	0,1338	&	0,2579	&	0,2305	&	0,4007	&	0,0662	&	0,0649	&	0,1346	\\
21	&	0,0436	&	0,0302	&	0,0791	&	0,0182	&	0,0181	&	0,0457	&	\textbf{0,0648}	&	0,0702	&	\textbf{0,1378}	&	0,2655	&	0,2398	&	0,3981	&	0,0651	&	0,0575	&	0,1335	\\
22	&	0,0429	&	0,0306	&	0,0852	&	0,0179	&	0,0216	&	0,0446	&	0,0639	&	0,0664	&	0,1364	&	0,2627	&	0,2105	&	0,3946	&	0,0692	&	0,0580	&	0,1319	\\
23	&	0,0435	&	\textbf{0,0386}	&	0,0794	&	0,0182	&	0,0180	&	0,0450	&	0,0624	&	0,0748	&	0,1336	&	0,2652	&	0,2192	&	0,3981	&	0,0625	&	0,0655	&	0,1339	\\
24	&	0,0423	&	0,0355	&	0,0859	&	0,0180	&	0,0172	&	0,0463	&	0,0637	&	0,0723	&	0,1350	&	0,2651	&	0,2309	&	0,3952	&	0,0704	&	0,0608	&	\textbf{0,1354}	\\
25	&	0,0397	&	0,0310	&	0,0724	&	\textbf{0,0188}	&	0,0162	&	0,0453	&	0,0616	&	0,0636	&	0,1288	&	0,2572	&	0,2180	&	0,3922	&	0,0638	&	0,0538	&	0,1256	\\
26	&	0,0409	&	0,0312	&	0,0751	&	\textbf{0,0188}	&	0,0161	&	0,0453	&	0,0618	&	0,0673	&	0,1242	&	0,2600	&	0,2221	&	0,3902	&	0,0627	&	0,0569	&	0,1221	\\
27	&	0,0422	&	0,0340	&	0,0772	&	0,0186	&	0,0190	&	0,0445	&	0,0616	&	0,0676	&	0,1279	&	0,2621	&	0,2197	&	0,3962	&	0,0522	&	0,0589	&	0,1226	\\
28	&	0,0415	&	0,0361	&	0,0754	&	0,0180	&	0,0201	&	0,0456	&	0,0600	&	0,0725	&	0,1306	&	0,2618	&	0,2273	&	0,3997	&	0,0630	&	0,0639	&	0,1327	\\
29	&	0,0424	&	0,0352	&	0,0758	&	0,0180	&	0,0207	&	0,0457	&	0,0612	&	0,0749	&	0,1317	&	0,2613	&	0,2353	&	0,4003	&	0,0597	&	0,0655	&	0,1331	\\
30	&	0,0415	&	0,0359	&	0,0798	&	0,0179	&	0,0196	&	0,0444	&	0,0626	&	\textbf{0,0756}	&	0,1315	&	0,2510	&	0,2316	&	0,4005	&	0,0619	&	0,0672	&	0,1311	\\
31	&	0,0444	&	0,0354	&	0,0822	&	0,0181	&	0,0201	&	\textbf{0,0483}	&	0,0622	&	0,0729	&	0,1337	&	0,2501	&	0,2360	&	0,3943	&	0,0668	&	0,0685	&	0,1279	\\
32	&	0,0411	&	0,0358	&	0,0832	&	0,0183	&	\textbf{0,0228}	&	0,0447	&	0,0607	&	0,0712	&	0,1326	&	0,2499	&	\textbf{0,2444}	&	0,3990	&	0,0651	&	0,0650	&	0,1331	\\
33	&	\textbf{0,0449}	&	0,0359	&	0,0816	&	0,0176	&	0,0211	&	0,0449	&	0,0623	&	0,0687	&	0,1319	&	0,2510	&	0,2246	&	0,3990	&	0,0685	&	0,0692	&	0,1330	\\
34	&	0,0426	&	0,0354	&	0,0860	&	0,0178	&	0,0209	&	0,0463	&	0,0619	&	0,0669	&	0,1348	&	0,2550	&	0,2328	&	\textbf{0,4029}	&	0,0682	&	\textbf{0,0712}	&	0,1345	\\
35	&	\textbf{0,0450}	&	0,0365	&	0,0828	&	0,0180	&	\textbf{0,0227}	&	0,0469	&	0,0568	&	0,0682	&	0,1343	&	0,2497	&	0,2372	&	\textbf{0,4049}	&	0,0675	&	0,0654	&	0,1345	\\
36	&	0,0445	&	0,0352	&	\textbf{0,0937}	&	0,0181	&	0,0224	&	\textbf{0,0480}	&	0,0595	&	0,0727	&	0,1343	&	0,2542	&	0,2300	&	\textbf{0,4018}	&	0,0685	&	0,0650	&	0,1344	\\
37	&	0,0412	&	0,0300	&	0,0782	&	\textbf{0,0188}	&	0,0168	&	0,0453	&	0,0630	&	0,0648	&	0,1321	&	0,2622	&	0,2176	&	0,3954	&	0,0641	&	0,0551	&	0,1193	\\
38	&	0,0403	&	0,0331	&	0,0806	&	\textbf{0,0188}	&	0,0172	&	0,0450	&	0,0608	&	0,0657	&	0,1301	&	0,2604	&	0,2281	&	0,3901	&	0,0650	&	0,0573	&	0,1232	\\
39	&	0,0408	&	0,0279	&	0,0800	&	0,0184	&	0,0166	&	0,0458	&	0,0627	&	0,0707	&	0,1310	&	0,2661	&	0,1982	&	0,3938	&	0,0646	&	0,0595	&	0,1215	\\
40	&	0,0411	&	0,0347	&	0,0771	&	0,0186	&	0,0180	&	0,0458	&	0,0615	&	0,0664	&	0,1324	&	0,2610	&	0,2251	&	0,3961	&	0,0667	&	0,0589	&	0,1232	\\
41	&	0,0427	&	0,0321	&	0,0759	&	0,0183	&	0,0182	&	0,0439	&	\textbf{0,0643}	&	0,0732	&	0,1329	&	0,2658	&	0,2191	&	0,4005	&	0,0646	&	0,0601	&	0,1293	\\
42	&	0,0421	&	0,0304	&	0,0820	&	0,0185	&	0,0192	&	0,0442	&	0,0620	&	0,0735	&	0,1316	&	0,2597	&	0,2358	&	0,3959	&	0,0667	&	0,0613	&	0,1194	\\
43	&	0,0441	&	0,0337	&	0,0811	&	0,0179	&	0,0209	&	0,0467	&	\textbf{0,0643}	&	0,0746	&	0,1370	&	0,2533	&	0,2395	&	0,4002	&	0,0629	&	0,0616	&	0,1323	\\
44	&	0,0432	&	0,0313	&	0,0874	&	0,0181	&	0,0170	&	0,0472	&	0,0639	&	0,0743	&	0,1347	&	0,2642	&	0,2346	&	0,3990	&	0,0673	&	0,0666	&	0,1351	\\
45	&	\textbf{0,0453}	&	0,0341	&	0,0800	&	0,0183	&	0,0177	&	0,0457	&	0,0599	&	0,0739	&	0,1369	&	\textbf{0,2677}	&	0,2246	&	0,3983	&	0,0667	&	0,0582	&	\textbf{0,1367}	\\
46	&	0,0442	&	0,0317	&	0,0815	&	0,0182	&	0,0185	&	0,0441	&	\textbf{0,0646}	&	0,0711	&	0,1336	&	\textbf{0,2700}	&	0,2353	&	0,3921	&	0,0686	&	0,0647	&	0,1353	\\
47	&	\textbf{0,0450}	&	0,0325	&	0,0926	&	0,0181	&	0,0220	&	0,0456	&	0,0607	&	0,0733	&	\textbf{0,1387}	&	0,2618	&	\textbf{0,2451}	&	\textbf{0,4010}	&	0,0698	&	0,0646	&	0,1352	\\
48	&	0,0438	&	0,0362	&	0,0854	&	0,0183	&	0,0199	&	0,0466	&	0,0628	&	0,0737	&	0,1363	&	0,2618	&	0,2334	&	0,3994	&	0,0677	&	0,0654	&	\textbf{0,1355}	\\
	\bottomrule
\end{tabular}
 \end{adjustwidth}
\caption[Wyniki badań sieci neuronowych rankingujących kolokacje dwuelementowe z korpusu \emph{KIPI}, część 1]{Wyniki badań sieci neuronowych rankingujących kolokacje dwuelementowe z korpusu \emph{KIPI}, część 1.}
\label{KIPI_nn_part_1}
\end{table}

\begin{table}[htp!]
\centering
\footnotesize\setlength{\tabcolsep}{2.5pt}
 \begin{adjustwidth}{-2cm}{}
\begin{tabular}{ l | *{15}{| r}}
	\toprule 
	\textbf{99\%} &	\textbf{16}	&	\textbf{17}	&	\textbf{18}	&	\textbf{19}	&	\textbf{20}	&	\textbf{21}	&	\textbf{22}	&	\textbf{23}	&	\textbf{24}	&	\textbf{25}	&	\textbf{26}	&	\textbf{27}	&	\textbf{28}	&	\textbf{29}	&	\textbf{30}	\\
	\midrule
1	&	0,2657	&	0,1751	&	0,3701	&	0,0345	&	0,0345	&	0,0843	&	0,1699	&	0,1526	&	0,3379	&	0,0322	&	0,0322	&	0,0803	&	0,1810	&	0,1379	&	0,3380	\\
2	&	0,2481	&	0,1958	&	0,3657	&	0,0350	&	0,0387	&	0,0749	&	0,1769	&	0,1450	&	0,3343	&	0,0326	&	0,0344	&	0,0797	&	0,1794	&	0,1464	&	0,3301	\\
3	&	0,2497	&	0,1964	&	0,3585	&	0,0354	&	0,0382	&	0,0829	&	0,1816	&	0,1566	&	0,3260	&	0,0327	&	0,0326	&	0,0796	&	0,1832	&	0,1588	&	0,3394	\\
4	&	0,2512	&	0,2152	&	0,3871	&	0,0348	&	0,0400	&	0,0852	&	0,1715	&	0,1667	&	0,3277	&	0,0327	&	0,0385	&	0,0800	&	0,1736	&	0,1597	&	0,3364	\\
5	&	0,2384	&	0,2163	&	0,3835	&	0,0350	&	0,0413	&	0,0858	&	0,1801	&	0,1605	&	0,3238	&	0,0330	&	0,0355	&	0,0787	&	0,1819	&	0,1497	&	0,3324	\\
6	&	0,2447	&	0,2120	&	0,4082	&	0,0358	&	0,0412	&	0,0886	&	0,1778	&	0,1614	&	0,3227	&	0,0298	&	0,0388	&	0,0807	&	0,1811	&	0,1550	&	0,3156	\\
7	&	0,2539	&	0,2113	&	0,4079	&	0,0348	&	0,0420	&	0,0834	&	0,1767	&	0,1663	&	0,3377	&	0,0325	&	0,0363	&	0,0778	&	0,1779	&	0,1540	&	0,3516	\\
8	&	0,2617	&	\textbf{0,2321}	&	0,4059	&	0,0349	&	0,0419	&	0,0849	&	0,1795	&	0,1648	&	0,3300	&	0,0324	&	0,0399	&	0,0823	&	0,1771	&	0,1663	&	0,3379	\\
9	&	0,2579	&	\textbf{0,2326}	&	0,4138	&	0,0356	&	0,0413	&	0,0847	&	0,1768	&	0,1683	&	0,3372	&	0,0298	&	0,0394	&	0,0797	&	0,1829	&	0,1734	&	\textbf{0,3530}	\\
10	&	0,2629	&	0,2130	&	0,4126	&	0,0353	&	0,0433	&	0,0854	&	0,1735	&	\textbf{0,1703}	&	0,3317	&	0,0319	&	0,0409	&	0,0848	&	0,1734	&	0,1764	&	0,3494	\\
11	&	0,2652	&	0,2030	&	0,4105	&	0,0351	&	0,0418	&	0,0874	&	0,1729	&	0,1671	&	0,3289	&	0,0322	&	0,0406	&	0,0862	&	0,1771	&	0,1704	&	0,3403	\\
12	&	0,2708	&	0,1814	&	0,4126	&	\textbf{0,0361}	&	0,0428	&	0,0859	&	0,1742	&	0,1688	&	0,3304	&	0,0309	&	0,0428	&	\textbf{0,0884}	&	0,1649	&	0,1805	&	0,3506	\\
13	&	0,2693	&	0,1852	&	0,3832	&	0,0357	&	0,0360	&	0,0861	&	\textbf{0,1850}	&	0,1346	&	0,3312	&	0,0327	&	0,0274	&	0,0709	&	0,1842	&	0,1058	&	0,3286	\\
14	&	0,2691	&	0,1805	&	0,3914	&	0,0356	&	0,0361	&	0,0815	&	0,1819	&	0,1436	&	\textbf{0,3439}	&	0,0328	&	0,0334	&	0,0805	&	0,1825	&	0,1232	&	0,3350	\\
15	&	0,2712	&	0,2034	&	0,3848	&	0,0348	&	0,0392	&	0,0883	&	\textbf{0,1861}	&	0,1319	&	0,3345	&	0,0333	&	0,0306	&	0,0758	&	\textbf{0,1863}	&	0,1174	&	0,3331	\\
16	&	\textbf{0,2760}	&	0,1943	&	0,3964	&	0,0346	&	0,0365	&	0,0820	&	0,1803	&	0,1267	&	0,3402	&	0,0333	&	0,0326	&	0,0846	&	\textbf{0,1876}	&	0,1572	&	0,3416	\\
17	&	0,2666	&	0,1964	&	0,3701	&	\textbf{0,0361}	&	0,0362	&	0,0872	&	0,1822	&	0,1255	&	0,3404	&	\textbf{0,0336}	&	0,0342	&	0,0791	&	\textbf{0,1869}	&	0,1340	&	0,3495	\\
18	&	0,2649	&	0,1907	&	0,3862	&	0,0354	&	0,0360	&	0,0886	&	0,1690	&	0,1417	&	0,3336	&	\textbf{0,0336}	&	0,0368	&	0,0835	&	\textbf{0,1866}	&	0,1441	&	0,3487	\\
19	&	0,2618	&	0,2031	&	0,3977	&	0,0313	&	0,0376	&	0,0903	&	0,1829	&	0,1490	&	0,3340	&	0,0332	&	0,0304	&	0,0844	&	0,1795	&	0,1296	&	0,3443	\\
20	&	0,2631	&	0,1801	&	0,4048	&	0,0351	&	0,0397	&	0,0900	&	0,1790	&	0,1422	&	0,3373	&	0,0312	&	0,0350	&	0,0810	&	0,1855	&	0,1487	&	0,3449	\\
21	&	0,2603	&	0,2075	&	0,4065	&	0,0348	&	0,0408	&	0,0905	&	0,1807	&	0,1387	&	\textbf{0,3444}	&	0,0327	&	0,0355	&	0,0840	&	0,1814	&	0,1375	&	0,3386	\\
22	&	\textbf{0,2737}	&	0,1912	&	0,4112	&	0,0356	&	0,0368	&	0,0903	&	0,1760	&	0,1605	&	0,3405	&	\textbf{0,0335}	&	0,0341	&	0,0832	&	0,1772	&	0,1406	&	0,3429	\\
23	&	0,2693	&	0,2012	&	\textbf{0,4175}	&	0,0341	&	0,0402	&	\textbf{0,0915}	&	0,1671	&	0,1611	&	\textbf{0,3454}	&	0,0321	&	0,0303	&	0,0821	&	0,1819	&	0,1504	&	0,3417	\\
24	&	\textbf{0,2736}	&	0,2109	&	\textbf{0,4164}	&	0,0323	&	0,0416	&	0,0877	&	0,1784	&	0,1552	&	0,3374	&	0,0320	&	0,0353	&	0,0866	&	0,1794	&	0,1271	&	0,3473	\\
25	&	0,2645	&	0,1858	&	0,3812	&	0,0341	&	0,0368	&	0,0894	&	0,1770	&	0,1437	&	0,3377	&	0,0324	&	0,0337	&	0,0791	&	0,1811	&	0,1361	&	0,3427	\\
26	&	0,2565	&	0,2004	&	0,3620	&	0,0348	&	0,0381	&	0,0826	&	0,1816	&	0,1495	&	0,3386	&	0,0321	&	0,0349	&	0,0822	&	0,1831	&	0,1513	&	0,3498	\\
27	&	0,2292	&	0,1960	&	0,3597	&	0,0346	&	0,0390	&	0,0798	&	0,1782	&	0,1528	&	0,3345	&	0,0325	&	0,0375	&	0,0819	&	0,1834	&	0,1508	&	0,3311	\\
28	&	0,2384	&	0,2158	&	0,3966	&	0,0336	&	0,0393	&	0,0825	&	0,1771	&	0,1567	&	0,3328	&	0,0329	&	0,0363	&	0,0817	&	0,1777	&	0,1572	&	0,3433	\\
29	&	0,2444	&	0,2029	&	0,3968	&	0,0346	&	0,0410	&	0,0840	&	0,1783	&	0,1595	&	0,3260	&	0,0330	&	0,0363	&	0,0816	&	0,1835	&	0,1551	&	0,3470	\\
30	&	0,2600	&	0,2150	&	0,4088	&	0,0343	&	0,0401	&	0,0825	&	0,1790	&	0,1657	&	0,3348	&	0,0316	&	0,0389	&	0,0827	&	0,1840	&	0,1593	&	0,3492	\\
31	&	0,2623	&	0,2245	&	0,4083	&	0,0343	&	0,0428	&	0,0838	&	0,1803	&	\textbf{0,1711}	&	0,3341	&	0,0321	&	0,0411	&	0,0817	&	0,1820	&	0,1774	&	0,3506	\\
32	&	0,2583	&	0,2223	&	\textbf{0,4163}	&	0,0349	&	0,0425	&	0,0850	&	0,1791	&	\textbf{0,1711}	&	0,3317	&	0,0315	&	0,0418	&	0,0844	&	0,1677	&	0,1782	&	0,3519	\\
33	&	0,2658	&	0,2285	&	0,4094	&	0,0356	&	0,0429	&	0,0842	&	0,1773	&	0,1685	&	0,3297	&	0,0302	&	0,0416	&	0,0810	&	0,1711	&	\textbf{0,1827}	&	0,3521	\\
34	&	0,2697	&	0,2244	&	0,4085	&	0,0351	&	0,0423	&	0,0853	&	0,1738	&	\textbf{0,1708}	&	0,3375	&	0,0324	&	0,0419	&	0,0870	&	0,1730	&	\textbf{0,1821}	&	\textbf{0,3528}	\\
35	&	0,2698	&	0,2156	&	0,4113	&	0,0338	&	0,0424	&	0,0864	&	0,1679	&	0,1674	&	0,3387	&	0,0330	&	0,0426	&	\textbf{0,0887}	&	0,1771	&	0,1795	&	\textbf{0,3549}	\\
36	&	0,2700	&	0,2141	&	0,4087	&	0,0323	&	0,0414	&	0,0853	&	0,1752	&	0,1681	&	0,3328	&	0,0318	&	\textbf{0,0432}	&	\textbf{0,0887}	&	0,1756	&	0,1712	&	\textbf{0,3562}	\\
37	&	0,2687	&	0,1861	&	0,4054	&	0,0330	&	0,0380	&	0,0903	&	0,1757	&	0,1396	&	0,3416	&	0,0321	&	0,0241	&	0,0814	&	0,1835	&	0,1380	&	0,3444	\\
38	&	0,2709	&	0,1942	&	0,3877	&	0,0350	&	0,0335	&	0,0863	&	\textbf{0,1847}	&	0,1485	&	\textbf{0,3437}	&	0,0330	&	0,0281	&	0,0815	&	0,1855	&	0,1320	&	0,3438	\\
39	&	\textbf{0,2739}	&	0,2021	&	0,3959	&	0,0347	&	0,0331	&	\textbf{0,0915}	&	0,1803	&	0,1363	&	\textbf{0,3428}	&	0,0333	&	0,0297	&	0,0824	&	\textbf{0,1866}	&	0,1262	&	0,3436	\\
40	&	\textbf{0,2750}	&	0,1853	&	0,4063	&	0,0348	&	0,0386	&	0,0887	&	0,1810	&	0,1613	&	0,3363	&	0,0333	&	0,0295	&	0,0819	&	0,1855	&	0,1398	&	0,3405	\\
41	&	0,2713	&	0,1902	&	0,4116	&	0,0337	&	0,0389	&	0,0888	&	0,1782	&	0,1441	&	0,3351	&	0,0333	&	0,0358	&	0,0821	&	0,1762	&	0,1484	&	0,3453	\\
42	&	0,2669	&	0,1942	&	0,3738	&	0,0337	&	0,0348	&	\textbf{0,0913}	&	0,1806	&	0,1492	&	\textbf{0,3432}	&	0,0334	&	0,0361	&	0,0836	&	\textbf{0,1876}	&	0,1388	&	0,3464	\\
43	&	0,2710	&	0,2036	&	0,4155	&	0,0347	&	0,0410	&	0,0876	&	0,1780	&	0,1600	&	0,3375	&	0,0329	&	0,0292	&	0,0860	&	0,1820	&	0,1510	&	0,3488	\\
44	&	0,2661	&	0,1780	&	0,3972	&	\textbf{0,0362}	&	0,0384	&	0,0906	&	0,1833	&	0,1502	&	0,3390	&	\textbf{0,0338}	&	0,0336	&	0,0849	&	0,1751	&	0,1492	&	0,3368	\\
45	&	0,2671	&	0,1824	&	0,4107	&	\textbf{0,0363}	&	0,0384	&	0,0900	&	0,1730	&	0,1614	&	0,3416	&	0,0330	&	0,0370	&	0,0844	&	0,1820	&	0,1747	&	0,3448	\\
46	&	0,2684	&	0,2067	&	0,4115	&	0,0348	&	0,0406	&	0,0905	&	0,1694	&	0,1666	&	0,3398	&	0,0320	&	0,0371	&	0,0842	&	0,1818	&	0,1645	&	0,3475	\\
47	&	0,2622	&	0,2019	&	\textbf{0,4176}	&	0,0342	&	0,0410	&	\textbf{0,0913}	&	0,1695	&	0,1571	&	0,3366	&	0,0312	&	0,0362	&	0,0864	&	0,1707	&	0,1432	&	\textbf{0,3529}	\\
48	&	0,2702	&	0,1890	&	\textbf{0,4198}	&	0,0327	&	\textbf{0,0440}	&	\textbf{0,0920}	&	0,1735	&	0,1564	&	0,3382	&	0,0331	&	0,0378	&	0,0859	&	0,1841	&	0,1798	&	0,3495	\\
	\bottomrule
\end{tabular}
 \end{adjustwidth}
\caption[Wyniki badań sieci neuronowych rankingujących kolokacje dwuelementowe z korpusu \emph{KIPI}, część 2]{Wyniki badań sieci neuronowych rankingujących kolokacje dwuelementowe z korpusu \emph{KIPI}, część 2.}
\label{KIPI_nn_part_2}
\end{table}

Tabela \ref{KIPI_measures_vs_nn} prezentuje stosunek wyniku najlepszej z sieci neuronowych do wyniku najlepszej z funkcji asocjacyjnych.
Oznaczenie \emph{m} określa najlepszą z miar dla danego badania, \emph{s} najlepszą z sieci, a \emph{s/m} stosunek wyniku najlepszej z sieci do najlepszej z miar asocjacyjnych.
\begin{table}[htp!]
\centering
\footnotesize\setlength{\tabcolsep}{2.5pt}
 \begin{adjustwidth}{-2cm}{}
\begin{tabular}{ l | *{15}{| r}}
	\toprule 
	&	1	&	2	&	3	&	4	&	5	&	6	&	7	&	8	&	9	&	10	&	11	&	12	&	13	&	14	&	15	\\
	\midrule
m	&	0,0369	&	0,0470	&	0,0815	&	0,0280	&	0,0364	&	0,0491	&	0,0628	&	0,0788	&	0,1209	&	0,2545	&	0,2578	&	0,3924	&	0,0610	&	0,0792	&	0,1242	\\
s	&	0,0453	&	0,0386	&	0,0937	&	0,0188	&	0,0228	&	0,0483	&	0,0648	&	0,0762	&	0,1387	&	0,2700	&	0,2451	&	0,4049	&	0,0719	&	0,0712	&	0,1367	\\
s/m	&	\textbf{1,2281}	&	0,8216	&	\textbf{1,1500}	&	0,6732	&	0,6264	&	0,9836	&	\textbf{1,0322}	&	0,9674	&	\textbf{1,1469}	&	\textbf{1,0608}	&	0,9507	&	\textbf{1,0319}	&	\textbf{1,1793}	&	0,8993	&	\textbf{1,1007}	\\
	\bottomrule
	\toprule
	&	16	&	17	&	18	&	19	&	20	&	21	&	22	&	23	&	24	&	25	&	26	&	27	&	28	&	29	&	30	\\
	\midrule																													
m	&	0,2607	&	0,2627	&	0,4168	&	0,0643	&	0,0792	&	0,0987	&	0,2668	&	0,2680	&	0,3422	&	0,0643	&	0,0786	&	0,0993	&	0,2688	&	0,2686	&	0,3593	\\
s	&	0,2760	&	0,2326	&	0,4176	&	0,0363	&	0,0433	&	0,0915	&	0,1861	&	0,1711	&	0,3454	&	0,0338	&	0,0432	&	0,0887	&	0,1876	&	0,1827	&	0,3562	\\
s/m	&	\textbf{1,0587}	&	0,8855	&	\textbf{1,0018}	&	0,5649	&	0,5469	&	0,9270	&	0,6976	&	0,6384	&	\textbf{1,0093}	&	0,5253	&	0,5501	&	0,8932	&	0,6980	&	0,6802	&	0,9913	\\
	\bottomrule
\end{tabular}
 \end{adjustwidth}
\caption[Porównanie wyników miar asocjacyjnych i sieci neuronowych dla korpusu \emph{KIPI}]{Porównanie wyników miar asocjacyjnych i sieci neuronowych dla korpusu \emph{KIPI}.}
\label{KIPI_measures_vs_nn}
\end{table}

\par 
Porównując wyniki sieci z funkcjami asocjacyjnymi zauważyć można, że sieci okazały się dawać lepsze jakościowo wyniki jedynie w części przypadków -- badań, a szczególnie w badaniu numer jeden, gdzie przewaga najlepszej z sieci nad najlepszą miarą sięgnęła prawie 23\%.
Sumarycznie jednak perceptrony wielowarstwowe osiągają raczej gorsze wyniki niż funkcje, a powodem takiego stanu rzeczy może być duże niedoreprezentowanie klasy pozytywnej. 
Ważne jednak jest, że najlepszy z wyników osiągniętych przez sieci jest lepszy niż najlepszy wynik najlepszej z miar asocjacyjnych -- obrazuje to rezultat badania osiemnastego.
Ta różnica jest co prawda minimalna i sięga niecałych 0,2\%, niemniej sieci okazały się być najskuteczniejsze.


\subsubsection{Wyniki badań dla sieci neuronowych po podpróbkowaniu klasy negatywnej do 95\%}
Tabele \ref{KIPI_sub5_nn_part_1} oraz \ref{KIPI_sub5_nn_part_2} prezentują jakość wyników osiągniętych przez 48 sieci neuronowych w 30 różnych badaniach (30 zestawów danych pozyskanych z korpusu \emph{KIPI}).
Różnica pomiędzy tym, a poprzednim badaniem jest taka, że zbiory danych zostały poddane podpróbkowaniu klasy negatywnej do około 95\% całego zbioru.
Indeksy typów badań pozostały takie same, jak w poprzednich badaniach, a indeksy zbadanych sieci neuronowych są identyczne z przedstawionymi w tabeli we wcześniejszej części tego rozdziału pracy.

\begin{table}[htp!]
\centering
\footnotesize\setlength{\tabcolsep}{2.5pt}
 \begin{adjustwidth}{-2cm}{}
\begin{tabular}{ l | *{15}{| r}}
	\toprule 
	\textbf{99\%} &	\textbf{1}	&	\textbf{2}	&	\textbf{3}	&	\textbf{4}	&	\textbf{5}	&	\textbf{6}	&	\textbf{7}	&	\textbf{8}	&	\textbf{9}	&	\textbf{10}	&	\textbf{11}	&	\textbf{12}	&	\textbf{13}	&	\textbf{14}	&	\textbf{15}	\\
	\midrule
1	&	0,5875	&	0,7068	&	0,8654	&	0,3774	&	0,4268	&	0,5884	&	0,4044	&	0,4547	&	0,7952	&	0,7628	&	0,7661	&	0,9242	&	0,5776	&	0,5897	&	0,8289	\\
2	&	0,5695	&	0,7136	&	0,7812	&	0,3825	&	0,4297	&	\textbf{0,5909}	&	0,4100	&	0,4580	&	0,7548	&	0,7640	&	0,7700	&	\textbf{0,9332}	&	0,5953	&	0,6394	&	0,7470	\\
3	&	0,5822	&	0,7109	&	0,8122	&	0,3818	&	0,4462	&	0,5859	&	0,4077	&	0,4604	&	0,7568	&	0,7635	&	0,7737	&	0,9165	&	0,5935	&	0,6443	&	0,7552	\\
4	&	0,5877	&	\textbf{0,7316}	&	0,7649	&	0,3884	&	0,4653	&	\textbf{0,5911}	&	0,4108	&	0,4619	&	0,7753	&	0,7652	&	0,7779	&	0,9252	&	0,5923	&	0,6578	&	0,7777	\\
5	&	\textbf{0,6090}	&	0,7193	&	0,8683	&	0,3797	&	0,4639	&	0,5882	&	0,4123	&	0,4589	&	0,7657	&	0,7652	&	\textbf{0,7795}	&	\textbf{0,9346}	&	0,5968	&	0,6797	&	0,7760	\\
6	&	0,5888	&	0,7241	&	0,8044	&	0,3854	&	0,4605	&	0,5901	&	0,4118	&	0,4637	&	0,7497	&	0,7665	&	0,7772	&	\textbf{0,9333}	&	0,5976	&	0,6626	&	0,7462	\\
7	&	0,5792	&	\textbf{0,7320}	&	0,7083	&	0,3755	&	0,4527	&	\textbf{0,5907}	&	0,4139	&	0,4626	&	0,7258	&	0,7676	&	0,7741	&	\textbf{0,9306}	&	0,5908	&	0,6617	&	0,7252	\\
8	&	0,5781	&	0,7261	&	0,8091	&	0,3858	&	0,4614	&	0,5860	&	0,4135	&	0,4654	&	0,7461	&	0,7649	&	\textbf{0,7784}	&	\textbf{0,9322}	&	0,5980	&	0,6624	&	0,7390	\\
9	&	0,5555	&	\textbf{0,7361}	&	0,7075	&	0,3806	&	0,4506	&	0,5802	&	0,4160	&	0,4648	&	0,7021	&	0,7650	&	0,7775	&	\textbf{0,9365}	&	0,5885	&	0,6802	&	0,7303	\\
10	&	0,5814	&	0,7287	&	0,7622	&	0,3800	&	0,4494	&	0,5812	&	0,4122	&	0,4662	&	0,7402	&	0,7673	&	0,7779	&	\textbf{0,9397}	&	0,5926	&	0,6621	&	0,7115	\\
11	&	0,5732	&	0,7267	&	0,7621	&	0,3823	&	0,4603	&	0,5789	&	0,4162	&	0,4639	&	0,7485	&	\textbf{0,7681}	&	\textbf{0,7798}	&	\textbf{0,9359}	&	\textbf{0,6090}	&	0,6666	&	0,7540	\\
12	&	0,5679	&	\textbf{0,7303}	&	0,7256	&	0,3892	&	0,4495	&	0,5807	&	0,4136	&	0,4640	&	0,7137	&	0,7660	&	\textbf{0,7795}	&	\textbf{0,9396}	&	0,5873	&	0,6853	&	0,6779	\\
13	&	0,5970	&	0,7176	&	0,8682	&	0,3902	&	0,4706	&	\textbf{0,5921}	&	0,4105	&	0,4590	&	0,7868	&	\textbf{0,7711}	&	0,7724	&	\textbf{0,9353}	&	0,5994	&	0,6743	&	0,8542	\\
14	&	0,5839	&	0,7137	&	\textbf{0,8735}	&	0,3893	&	0,4718	&	\textbf{0,5932}	&	0,4159	&	0,4620	&	0,7945	&	\textbf{0,7683}	&	0,7766	&	\textbf{0,9384}	&	0,5961	&	0,6738	&	\textbf{0,8875}	\\
15	&	0,5860	&	0,7239	&	\textbf{0,8741}	&	0,3882	&	\textbf{0,4725}	&	0,5888	&	0,4153	&	0,4656	&	0,7966	&	\textbf{0,7725}	&	0,7771	&	0,9299	&	0,5867	&	0,6687	&	\textbf{0,8902}	\\
16	&	0,5924	&	0,7140	&	\textbf{0,8727}	&	0,3922	&	\textbf{0,4738}	&	\textbf{0,5906}	&	0,4176	&	0,4674	&	\textbf{0,8001}	&	\textbf{0,7717}	&	0,7760	&	\textbf{0,9337}	&	0,5803	&	0,6941	&	\textbf{0,8874}	\\
17	&	0,5963	&	0,7110	&	\textbf{0,8778}	&	0,3858	&	0,4719	&	0,5863	&	0,4177	&	0,4696	&	0,7988	&	\textbf{0,7705}	&	\textbf{0,7786}	&	\textbf{0,9341}	&	0,5744	&	\textbf{0,6982}	&	0,8747	\\
18	&	0,5719	&	0,7179	&	0,8678	&	0,3876	&	\textbf{0,4741}	&	\textbf{0,5949}	&	0,4189	&	0,4706	&	\textbf{0,8002}	&	\textbf{0,7712}	&	\textbf{0,7821}	&	0,9298	&	0,5835	&	\textbf{0,6976}	&	0,8707	\\
19	&	0,5980	&	0,7221	&	\textbf{0,8751}	&	0,3870	&	0,4628	&	0,5892	&	\textbf{0,4214}	&	0,4691	&	\textbf{0,8034}	&	\textbf{0,7687}	&	\textbf{0,7806}	&	\textbf{0,9356}	&	0,5824	&	0,6953	&	\textbf{0,8957}	\\
20	&	0,5721	&	\textbf{0,7307}	&	\textbf{0,8744}	&	0,3837	&	0,4664	&	\textbf{0,5949}	&	0,4190	&	0,4671	&	0,7926	&	\textbf{0,7716}	&	\textbf{0,7830}	&	\textbf{0,9381}	&	0,5886	&	\textbf{0,6962}	&	0,8783	\\
21	&	0,5851	&	0,7007	&	0,8649	&	0,3905	&	0,4716	&	0,5884	&	0,4160	&	0,4693	&	\textbf{0,8032}	&	\textbf{0,7718}	&	\textbf{0,7819}	&	\textbf{0,9396}	&	0,5769	&	\textbf{0,6998}	&	0,8631	\\
22	&	0,5717	&	0,7262	&	0,8644	&	0,3865	&	0,4700	&	\textbf{0,5932}	&	\textbf{0,4201}	&	0,4692	&	\textbf{0,8001}	&	\textbf{0,7727}	&	\textbf{0,7842}	&	\textbf{0,9330}	&	0,5909	&	\textbf{0,6966}	&	0,8703	\\
23	&	0,5776	&	0,7176	&	0,8716	&	0,3870	&	0,4662	&	\textbf{0,5963}	&	0,4171	&	\textbf{0,4747}	&	\textbf{0,8029}	&	\textbf{0,7735}	&	\textbf{0,7844}	&	\textbf{0,9369}	&	0,5905	&	\textbf{0,6967}	&	0,8347	\\
24	&	0,5796	&	0,7181	&	0,8705	&	0,3903	&	0,4662	&	\textbf{0,5952}	&	0,4178	&	\textbf{0,4760}	&	\textbf{0,8029}	&	\textbf{0,7734}	&	\textbf{0,7813}	&	\textbf{0,9386}	&	0,5810	&	\textbf{0,6984}	&	0,8371	\\
25	&	0,5785	&	0,7102	&	0,8662	&	0,3895	&	0,4339	&	0,5828	&	0,4039	&	0,4549	&	0,7944	&	0,7621	&	0,7720	&	0,9298	&	0,5947	&	0,5867	&	0,8488	\\
26	&	\textbf{0,6120}	&	0,7192	&	0,8592	&	0,3748	&	0,4260	&	0,5831	&	0,4086	&	0,4603	&	0,7721	&	0,7635	&	0,7739	&	\textbf{0,9346}	&	0,5907	&	0,6161	&	0,8047	\\
27	&	\textbf{0,6084}	&	0,7188	&	0,8720	&	0,3884	&	0,4181	&	0,5856	&	0,4095	&	0,4614	&	0,7681	&	0,7626	&	0,7752	&	\textbf{0,9305}	&	0,5893	&	0,6763	&	0,7735	\\
28	&	0,5835	&	0,7163	&	0,7853	&	0,3921	&	0,4502	&	0,5862	&	0,4123	&	0,4618	&	\textbf{0,8016}	&	0,7654	&	0,7775	&	\textbf{0,9336}	&	\textbf{0,6058}	&	0,6638	&	0,8266	\\
29	&	0,5997	&	0,7229	&	0,8391	&	0,3853	&	0,4614	&	0,5883	&	0,4145	&	0,4623	&	0,7666	&	0,7654	&	0,7771	&	\textbf{0,9339}	&	\textbf{0,6029}	&	0,6693	&	0,8097	\\
30	&	0,5975	&	\textbf{0,7297}	&	0,8518	&	0,3813	&	0,4679	&	0,5897	&	0,4143	&	0,4620	&	0,7908	&	0,7650	&	0,7775	&	\textbf{0,9392}	&	0,5972	&	0,6728	&	0,8506	\\
31	&	0,5864	&	0,7093	&	0,8533	&	0,3845	&	0,4603	&	0,5845	&	0,4134	&	0,4637	&	0,7894	&	0,7665	&	\textbf{0,7795}	&	\textbf{0,9376}	&	0,5906	&	0,6903	&	0,8239	\\
32	&	0,5727	&	0,7235	&	0,8161	&	0,3807	&	0,4677	&	0,5815	&	0,4106	&	0,4622	&	0,7786	&	0,7668	&	\textbf{0,7787}	&	\textbf{0,9361}	&	0,5973	&	0,6703	&	0,7862	\\
33	&	0,5756	&	\textbf{0,7289}	&	0,8609	&	0,3860	&	0,4501	&	\textbf{0,5904}	&	0,4138	&	0,4607	&	0,7952	&	0,7642	&	0,7782	&	\textbf{0,9382}	&	0,5912	&	0,6878	&	0,7632	\\
34	&	0,5754	&	0,7257	&	0,8655	&	0,3846	&	0,4442	&	0,5821	&	0,4146	&	0,4652	&	0,7620	&	0,7676	&	\textbf{0,7789}	&	\textbf{0,9367}	&	0,5913	&	0,6848	&	0,8211	\\
35	&	0,5712	&	0,7218	&	0,8353	&	0,3878	&	0,4447	&	0,5809	&	0,4118	&	0,4658	&	0,7867	&	0,7648	&	\textbf{0,7842}	&	\textbf{0,9340}	&	0,5884	&	0,6700	&	0,8292	\\
36	&	0,5743	&	0,7221	&	0,8363	&	0,3847	&	0,4494	&	0,5786	&	0,4129	&	0,4634	&	0,7940	&	\textbf{0,7678}	&	\textbf{0,7803}	&	\textbf{0,9362}	&	0,5844	&	0,6812	&	0,8152	\\
37	&	0,6031	&	0,7153	&	0,8715	&	\textbf{0,3964}	&	0,4690	&	\textbf{0,5929}	&	0,4095	&	0,4631	&	0,7884	&	0,7662	&	0,7742	&	\textbf{0,9363}	&	0,5930	&	0,6509	&	0,8834	\\
38	&	0,5588	&	\textbf{0,7294}	&	\textbf{0,8753}	&	0,3916	&	0,4668	&	\textbf{0,5922}	&	0,4176	&	0,4656	&	0,7992	&	0,7670	&	0,7768	&	\textbf{0,9313}	&	0,5784	&	0,6825	&	0,8872	\\
39	&	0,6059	&	0,7167	&	0,8692	&	0,3870	&	\textbf{0,4731}	&	\textbf{0,5953}	&	0,4150	&	0,4656	&	0,7965	&	\textbf{0,7698}	&	0,7782	&	\textbf{0,9337}	&	0,5890	&	0,6930	&	0,8816	\\
40	&	0,6000	&	0,7128	&	\textbf{0,8750}	&	0,3841	&	\textbf{0,4751}	&	\textbf{0,5939}	&	0,4176	&	0,4674	&	0,7985	&	\textbf{0,7688}	&	\textbf{0,7807}	&	\textbf{0,9344}	&	0,5889	&	0,6917	&	0,8815	\\
41	&	0,5852	&	\textbf{0,7327}	&	0,8700	&	0,3901	&	\textbf{0,4741}	&	\textbf{0,5913}	&	0,4181	&	0,4664	&	\textbf{0,8019}	&	\textbf{0,7700}	&	\textbf{0,7788}	&	\textbf{0,9366}	&	0,5997	&	\textbf{0,6972}	&	0,8852	\\
42	&	0,5900	&	0,7210	&	0,8706	&	0,3854	&	\textbf{0,4769}	&	\textbf{0,5906}	&	\textbf{0,4201}	&	0,4691	&	0,7944	&	\textbf{0,7724}	&	\textbf{0,7817}	&	\textbf{0,9340}	&	0,5744	&	\textbf{0,6979}	&	\textbf{0,8934}	\\
43	&	0,5660	&	\textbf{0,7310}	&	0,8720	&	0,3925	&	0,4704	&	\textbf{0,5940}	&	0,4185	&	0,4689	&	\textbf{0,8024}	&	\textbf{0,7730}	&	\textbf{0,7816}	&	\textbf{0,9360}	&	0,5851	&	\textbf{0,6977}	&	0,8492	\\
44	&	0,5703	&	0,7210	&	\textbf{0,8813}	&	0,3889	&	0,4719	&	\textbf{0,5926}	&	\textbf{0,4221}	&	0,4683	&	\textbf{0,8011}	&	\textbf{0,7743}	&	\textbf{0,7838}	&	\textbf{0,9376}	&	0,5764	&	0,6949	&	\textbf{0,8930}	\\
45	&	0,5694	&	0,7252	&	\textbf{0,8744}	&	\textbf{0,3931}	&	0,4670	&	\textbf{0,5909}	&	\textbf{0,4239}	&	0,4707	&	0,7994	&	\textbf{0,7730}	&	\textbf{0,7837}	&	\textbf{0,9396}	&	0,5625	&	0,6919	&	\textbf{0,8962}	\\
46	&	0,5675	&	0,7212	&	\textbf{0,8752}	&	0,3892	&	0,4664	&	\textbf{0,5930}	&	0,4182	&	\textbf{0,4723}	&	\textbf{0,8018}	&	\textbf{0,7741}	&	\textbf{0,7819}	&	\textbf{0,9376}	&	0,5812	&	\textbf{0,7024}	&	0,8858	\\
47	&	0,5619	&	0,7252	&	\textbf{0,8784}	&	0,3865	&	0,4666	&	0,5902	&	\textbf{0,4243}	&	\textbf{0,4721}	&	0,7995	&	\textbf{0,7754}	&	\textbf{0,7862}	&	\textbf{0,9335}	&	0,5909	&	\textbf{0,6999}	&	\textbf{0,8915}	\\
48	&	0,5650	&	0,7238	&	0,8663	&	0,3902	&	0,4648	&	\textbf{0,5918}	&	\textbf{0,4204}	&	\textbf{0,4728}	&	\textbf{0,8076}	&	\textbf{0,7698}	&	\textbf{0,7821}	&	\textbf{0,9380}	&	0,5935	&	0,6928	&	0,8853	\\
	\bottomrule
\end{tabular}
 \end{adjustwidth}
\caption[Wyniki badań sieci neuronowych rankingujących kolokacje dwuelementowe z korpusu \emph{KIPI} po podpróbkowaniu klasy negatywnej do 95\%, część 1]{Wyniki badań sieci neuronowych rankingujących kolokacje dwuelementowe z korpusu \emph{KIPI} po podpróbkowaniu klasy negatywnej do 95\%, część 1.}
\label{KIPI_sub5_nn_part_1}
\end{table}

\begin{table}[htp!]
\centering
\footnotesize\setlength{\tabcolsep}{2.5pt}
 \begin{adjustwidth}{-2cm}{}
\begin{tabular}{ l | *{15}{| r}}
	\toprule
	\textbf{99\%} &	\textbf{16}	&	\textbf{17}	&	\textbf{18}	&	\textbf{19}	&	\textbf{20}	&	\textbf{21}	&	\textbf{22}	&	\textbf{23}	&	\textbf{24}	&	\textbf{25}	&	\textbf{26}	&	\textbf{27}	&	\textbf{28}	&	\textbf{29}	&	\textbf{30}	\\
	\midrule
1	&	0,7678	&	\textbf{0,8372}	&	0,7597	&	0,3857	&	0,4083	&	0,6070	&	0,7703	&	0,7595	&	\textbf{0,8903}	&	0,4435	&	0,5400	&	\textbf{0,7269}	&	0,7947	&	0,8121	&	\textbf{0,9130}	\\
2	&	0,7544	&	0,8300	&	0,8258	&	0,3838	&	0,4281	&	0,6069	&	0,7707	&	0,7644	&	\textbf{0,8904}	&	0,4865	&	0,5373	&	0,7217	&	0,7768	&	0,8123	&	\textbf{0,9137}	\\
3	&	0,7377	&	0,8245	&	0,8407	&	0,3876	&	0,4424	&	0,6067	&	0,7650	&	0,7435	&	\textbf{0,8911}	&	0,4842	&	0,5206	&	\textbf{0,7264}	&	0,7863	&	0,8146	&	\textbf{0,9152}	\\
4	&	0,7402	&	0,8112	&	0,8295	&	0,3994	&	0,4520	&	0,6086	&	0,7666	&	0,7718	&	\textbf{0,8909}	&	0,4745	&	0,5400	&	\textbf{0,7283}	&	0,7870	&	0,8165	&	\textbf{0,9135}	\\
5	&	0,7455	&	0,7947	&	0,8176	&	0,3974	&	0,4593	&	0,6033	&	0,7689	&	0,7647	&	\textbf{0,8894}	&	0,5031	&	0,5403	&	0,7226	&	0,8073	&	0,8075	&	\textbf{0,9141}	\\
6	&	0,7544	&	0,8002	&	0,8279	&	0,3932	&	0,4564	&	0,6041	&	0,7653	&	0,7633	&	\textbf{0,8892}	&	0,5051	&	0,5607	&	0,7230	&	0,7921	&	0,8202	&	\textbf{0,9136}	\\
7	&	0,7538	&	0,7859	&	0,8310	&	0,3930	&	0,4607	&	0,6031	&	0,7699	&	0,7694	&	\textbf{0,8916}	&	0,4993	&	0,5539	&	0,7238	&	0,8018	&	0,8162	&	\textbf{0,9136}	\\
8	&	0,7363	&	0,7835	&	0,8227	&	0,3965	&	0,4637	&	0,6043	&	0,7695	&	0,7681	&	\textbf{0,8887}	&	0,5085	&	0,5608	&	0,7214	&	0,8015	&	0,8212	&	\textbf{0,9113}	\\
9	&	0,7485	&	0,7936	&	0,7983	&	0,3944	&	0,4604	&	0,6036	&	0,7705	&	0,7749	&	\textbf{0,8891}	&	0,5130	&	0,5610	&	0,7260	&	0,8073	&	0,8223	&	\textbf{0,9137}	\\
10	&	0,7538	&	0,7835	&	0,8093	&	0,4148	&	0,4628	&	0,6054	&	0,7689	&	0,7704	&	\textbf{0,8902}	&	0,5101	&	0,5620	&	\textbf{0,7281}	&	0,8211	&	0,8194	&	\textbf{0,9149}	\\
11	&	0,7496	&	0,7832	&	0,8358	&	0,4077	&	0,4656	&	0,6034	&	0,7698	&	0,7731	&	\textbf{0,8912}	&	0,5209	&	0,5526	&	0,7196	&	0,8154	&	0,8177	&	\textbf{0,9132}	\\
12	&	0,7558	&	0,7729	&	0,8488	&	0,4085	&	\textbf{0,4677}	&	0,6023	&	0,7714	&	0,7725	&	\textbf{0,8892}	&	0,5030	&	0,5648	&	0,7261	&	0,8249	&	0,8205	&	\textbf{0,9121}	\\
13	&	0,7584	&	0,8005	&	0,9533	&	0,3970	&	0,4536	&	\textbf{0,6157}	&	0,7755	&	0,7739	&	\textbf{0,8875}	&	0,5101	&	0,5153	&	0,7245	&	0,8128	&	0,8221	&	\textbf{0,9148}	\\
14	&	0,7346	&	0,8055	&	0,8589	&	0,4074	&	0,4618	&	\textbf{0,6116}	&	0,7760	&	0,7816	&	\textbf{0,8910}	&	0,5015	&	0,5537	&	\textbf{0,7281}	&	0,8063	&	0,8224	&	\textbf{0,9140}	\\
15	&	0,7487	&	0,7826	&	\textbf{0,9616}	&	0,3881	&	\textbf{0,4670}	&	\textbf{0,6105}	&	0,7755	&	0,7806	&	\textbf{0,8903}	&	0,5140	&	0,5618	&	\textbf{0,7335}	&	0,8067	&	0,8160	&	\textbf{0,9136}	\\
16	&	0,7347	&	0,7888	&	0,8527	&	0,4058	&	0,4659	&	\textbf{0,6112}	&	0,7735	&	0,7814	&	\textbf{0,8902}	&	0,5182	&	0,5497	&	0,7191	&	0,8085	&	0,8150	&	\textbf{0,9139}	\\
17	&	0,7418	&	0,7822	&	\textbf{0,9618}	&	0,4010	&	\textbf{0,4686}	&	0,6080	&	0,7740	&	0,7805	&	\textbf{0,8880}	&	0,5199	&	0,5716	&	\textbf{0,7279}	&	0,8181	&	0,8219	&	\textbf{0,9128}	\\
18	&	0,7561	&	0,7888	&	\textbf{0,9660}	&	0,4175	&	\textbf{0,4702}	&	0,6088	&	0,7734	&	0,7808	&	\textbf{0,8900}	&	0,5325	&	0,5750	&	0,7180	&	0,8167	&	0,8200	&	\textbf{0,9099}	\\
19	&	0,7518	&	0,7907	&	\textbf{0,9614}	&	0,4131	&	\textbf{0,4670}	&	\textbf{0,6103}	&	0,7734	&	0,7806	&	\textbf{0,8868}	&	0,5292	&	0,5746	&	0,7201	&	0,8227	&	0,8283	&	\textbf{0,9124}	\\
20	&	0,7429	&	0,7868	&	\textbf{0,9570}	&	0,4199	&	\textbf{0,4686}	&	0,6081	&	0,7730	&	0,7800	&	\textbf{0,8878}	&	0,5254	&	0,5764	&	0,7206	&	0,8190	&	0,8196	&	\textbf{0,9125}	\\
21	&	0,7497	&	0,7843	&	0,8760	&	0,4201	&	\textbf{0,4703}	&	0,6082	&	0,7735	&	0,7805	&	\textbf{0,8896}	&	0,5213	&	0,5813	&	0,7232	&	0,8274	&	0,8275	&	\textbf{0,9143}	\\
22	&	0,7497	&	0,7808	&	\textbf{0,9664}	&	0,4157	&	\textbf{0,4676}	&	0,6087	&	0,7733	&	0,7802	&	\textbf{0,8863}	&	0,5334	&	0,5826	&	\textbf{0,7267}	&	0,8276	&	0,8199	&	\textbf{0,9152}	\\
23	&	0,7465	&	0,7824	&	0,8910	&	0,4201	&	\textbf{0,4699}	&	0,6094	&	0,7734	&	0,7797	&	\textbf{0,8862}	&	0,5242	&	\textbf{0,5896}	&	0,7204	&	0,8301	&	0,8269	&	\textbf{0,9139}	\\
24	&	0,7566	&	0,7834	&	0,9097	&	0,4200	&	\textbf{0,4706}	&	0,6070	&	0,7728	&	0,7802	&	0,8817	&	0,5222	&	0,5812	&	0,7230	&	0,8343	&	0,8271	&	\textbf{0,9122}	\\
25	&	0,7730	&	\textbf{0,8394}	&	0,8598	&	0,3897	&	0,4080	&	\textbf{0,6105}	&	0,7606	&	0,7635	&	\textbf{0,8901}	&	0,4810	&	0,5415	&	0,7204	&	0,7695	&	0,8105	&	\textbf{0,9139}	\\
26	&	0,7679	&	\textbf{0,8402}	&	0,8692	&	0,3950	&	0,4211	&	\textbf{0,6097}	&	0,7672	&	0,7663	&	\textbf{0,8911}	&	0,4926	&	0,5356	&	0,7256	&	0,7826	&	0,8050	&	\textbf{0,9142}	\\
27	&	0,7591	&	0,8239	&	0,8731	&	0,3932	&	0,4415	&	0,6076	&	0,7697	&	0,7716	&	\textbf{0,8891}	&	0,4843	&	0,5191	&	0,7197	&	0,8001	&	0,8150	&	\textbf{0,9140}	\\
28	&	0,7371	&	0,8126	&	0,7116	&	0,3917	&	0,4405	&	0,6059	&	0,7686	&	0,7677	&	\textbf{0,8903}	&	0,4904	&	0,5332	&	0,7260	&	0,7744	&	0,8080	&	\textbf{0,9148}	\\
29	&	0,7597	&	0,8122	&	0,7163	&	0,3928	&	0,4450	&	0,6063	&	0,7693	&	0,7665	&	\textbf{0,8909}	&	0,4951	&	0,5450	&	0,7225	&	0,7869	&	0,8086	&	\textbf{0,9147}	\\
30	&	0,7554	&	0,8128	&	0,7813	&	0,3900	&	0,4564	&	0,6077	&	0,7685	&	0,7639	&	\textbf{0,8908}	&	0,5042	&	0,5354	&	0,7232	&	0,7819	&	0,8147	&	\textbf{0,9133}	\\
31	&	0,7606	&	0,7864	&	0,8159	&	0,3973	&	0,4593	&	0,6025	&	0,7641	&	0,7683	&	\textbf{0,8897}	&	0,5058	&	0,5570	&	0,7239	&	0,7966	&	0,8088	&	\textbf{0,9139}	\\
32	&	0,7562	&	0,7909	&	0,7548	&	0,3940	&	0,4619	&	0,6055	&	0,7649	&	0,7682	&	\textbf{0,8896}	&	0,5163	&	0,5355	&	0,7228	&	0,7955	&	0,8193	&	\textbf{0,9124}	\\
33	&	0,7481	&	0,7806	&	0,7798	&	0,4035	&	0,4645	&	0,6056	&	0,7694	&	0,7687	&	\textbf{0,8909}	&	0,5089	&	0,5601	&	0,7219	&	0,8195	&	0,8247	&	\textbf{0,9141}	\\
34	&	0,7497	&	0,7752	&	0,7480	&	0,4104	&	0,4640	&	0,6010	&	0,7697	&	0,7703	&	\textbf{0,8904}	&	0,5111	&	0,5746	&	0,7239	&	0,8291	&	0,8206	&	\textbf{0,9126}	\\
35	&	0,7556	&	0,7769	&	0,8018	&	0,4150	&	0,4633	&	0,6048	&	0,7695	&	0,7668	&	\textbf{0,8898}	&	0,5008	&	0,5615	&	0,7256	&	0,8243	&	0,8238	&	\textbf{0,9135}	\\
36	&	0,7417	&	0,7838	&	0,8025	&	0,4091	&	\textbf{0,4669}	&	0,6046	&	0,7693	&	0,7722	&	\textbf{0,8904}	&	0,5037	&	0,5718	&	0,7230	&	0,8196	&	0,8205	&	\textbf{0,9136}	\\
37	&	0,7433	&	0,7987	&	0,9546	&	0,3928	&	0,4534	&	\textbf{0,6103}	&	0,7747	&	0,7810	&	\textbf{0,8897}	&	0,5180	&	0,5622	&	0,7209	&	0,8189	&	0,8095	&	\textbf{0,9122}	\\
38	&	0,7417	&	0,7928	&	0,9473	&	0,3989	&	0,4634	&	\textbf{0,6145}	&	0,7761	&	0,7772	&	\textbf{0,8881}	&	0,5199	&	0,5305	&	0,7260	&	0,7994	&	0,8129	&	\textbf{0,9113}	\\
39	&	0,7437	&	0,7952	&	\textbf{0,9595}	&	0,4057	&	0,4659	&	\textbf{0,6133}	&	0,7742	&	0,7816	&	\textbf{0,8880}	&	0,5236	&	0,5482	&	0,7228	&	0,8057	&	0,8099	&	\textbf{0,9128}	\\
40	&	0,7437	&	0,7953	&	0,9537	&	0,4006	&	\textbf{0,4683}	&	\textbf{0,6130}	&	0,7746	&	0,7805	&	\textbf{0,8890}	&	0,5242	&	0,5518	&	0,7238	&	0,8056	&	0,8178	&	\textbf{0,9127}	\\
41	&	0,7495	&	0,7832	&	0,9091	&	0,3991	&	\textbf{0,4691}	&	0,6074	&	0,7728	&	0,7784	&	\textbf{0,8886}	&	0,5232	&	0,5777	&	0,7243	&	0,8150	&	0,8242	&	\textbf{0,9133}	\\
42	&	0,7483	&	0,7844	&	\textbf{0,9588}	&	0,4194	&	\textbf{0,4693}	&	\textbf{0,6100}	&	0,7727	&	0,7789	&	0,8815	&	0,5199	&	0,5785	&	0,7178	&	0,8200	&	0,8281	&	\textbf{0,9109}	\\
43	&	0,7511	&	0,7847	&	\textbf{0,9589}	&	0,4175	&	\textbf{0,4700}	&	0,6086	&	0,7734	&	0,7798	&	\textbf{0,8857}	&	0,5219	&	0,5756	&	0,7213	&	0,8190	&	0,8269	&	\textbf{0,9128}	\\
44	&	0,7432	&	0,7848	&	0,9304	&	0,4068	&	\textbf{0,4691}	&	\textbf{0,6106}	&	0,7730	&	0,7799	&	\textbf{0,8904}	&	0,5337	&	0,5832	&	0,7207	&	0,8098	&	0,8250	&	\textbf{0,9124}	\\
45	&	0,7478	&	0,7843	&	\textbf{0,9568}	&	0,4218	&	\textbf{0,4693}	&	0,6051	&	0,7728	&	0,7780	&	\textbf{0,8881}	&	0,5321	&	0,5765	&	\textbf{0,7305}	&	0,8217	&	0,8257	&	\textbf{0,9104}	\\
46	&	0,7517	&	0,7839	&	\textbf{0,9592}	&	0,4210	&	\textbf{0,4701}	&	0,6076	&	0,7732	&	0,7797	&	\textbf{0,8875}	&	0,5348	&	0,5782	&	0,7176	&	0,8188	&	0,8290	&	\textbf{0,9127}	\\
47	&	0,7402	&	0,7820	&	0,9263	&	0,4203	&	\textbf{0,4707}	&	\textbf{0,6097}	&	0,7735	&	0,7808	&	\textbf{0,8841}	&	0,5216	&	\textbf{0,5904}	&	0,7235	&	0,8291	&	0,8267	&	\textbf{0,9123}	\\
48	&	0,7565	&	0,7785	&	\textbf{0,9598}	&	0,4210	&	\textbf{0,4700}	&	0,6056	&	0,7730	&	0,7803	&	\textbf{0,8870}	&	0,5232	&	\textbf{0,5887}	&	\textbf{0,7286}	&	0,8346	&	0,8250	&	\textbf{0,9147}	\\
	\bottomrule
\end{tabular}
 \end{adjustwidth}
\caption[Wyniki badań sieci neuronowych rankingujących kolokacje dwuelementowe z korpusu \emph{KIPI} po podpróbkowaniu klasy negatywnej do 95\%, część 2]{Wyniki badań sieci neuronowych rankingujących kolokacje dwuelementowe z korpusu \emph{KIPI} po podpróbkowaniu klasy negatywnej do 95\%, część 2.}
\label{KIPI_sub5_nn_part_2}
\end{table}

Poniższa tabela \ref{KIPI_sub5_measures_vs_nn} prezentuje stosunek wyniku najlepszej z sieci neuronowych do wyniku najlepszej z funkcji asocjacyjnych.
Oznaczenie \emph{m} określa najlepszą z miar dla danego badania, \emph{s} najlepszą z sieci, a \emph{s/m} stosunek wyniku najlepszej z sieci do najlepszej z miar asocjacyjnych.
Pogrubioną czcionką zostały oznaczone te wyniki badań, w których najlepsza sieć neuronowa osiągnęła lepszą jakość rozwiązania niż najlepsza z miar asocjacyjnych.
\begin{table}[htp!]
\centering
\footnotesize\setlength{\tabcolsep}{2.5pt}
 \begin{adjustwidth}{-2cm}{}
\begin{tabular}{ l | *{15}{| r}}
	\toprule 
	&	1	&	2	&	3	&	4	&	5	&	6	&	7	&	8	&	9	&	10	&	11	&	12	&	13	&	14	&	15	\\
	\midrule
m	&	0,3540	&	0,3708	&	0,2293	&	0,3698	&	0,3672	&	0,2082	&	0,4210	&	0,4046	&	0,2243	&	0,7649	&	0,7165	&	0,5749	&	0,5299	&	0,5152	&	0,3101	\\
s	&	0,6120	&	0,7361	&	0,8813	&	0,3964	&	0,4769	&	0,5963	&	0,4243	&	0,4760	&	0,8076	&	0,7754	&	0,7862	&	0,9397	&	0,6090	&	0,7024	&	0,8962	\\
s/m	&	\textbf{1,7288}	&	\textbf{1,9854}	&	\textbf{3,8433}	&	\textbf{1,0722}	&	\textbf{1,2988}	&	\textbf{2,8640}	&	\textbf{1,0078}	&	\textbf{1,1767}	&	\textbf{3,6009}	&	\textbf{1,0137}	&	\textbf{1,0972}	&	\textbf{1,6343}	&	\textbf{1,1494}	&	\textbf{1,3635}	&	\textbf{2,8899}	\\
	\bottomrule
	\toprule
	&	16	&	17	&	18	&	19	&	20	&	21	&	22	&	23	&	24	&	25	&	26	&	27	&	28	&	29	&	30	\\
	\midrule
m	&	0,8385	&	0,8006	&	0,6898	&	0,4408	&	0,4221	&	0,2377	&	0,8427	&	0,7991	&	0,6250	&	0,5471	&	0,5358	&	0,3334	&	0,8995	&	0,8653	&	0,7437	\\
s	&	0,7730	&	0,8402	&	0,9664	&	0,4218	&	0,4707	&	0,6157	&	0,7761	&	0,7816	&	0,8916	&	0,5348	&	0,5904	&	0,7335	&	0,8346	&	0,8290	&	0,9152	\\
s/m	&	0,9219	&	\textbf{1,0495}	&	\textbf{1,4009}	&	0,9571	&	\textbf{1,1150}	&	\textbf{2,5907}	&	0,9210	&	0,9781	&	\textbf{1,4265}	&	0,9775	&	\textbf{1,1019}	&	\textbf{2,2001}	&	0,9279	&	0,9581	&	\textbf{1,2306}	\\
	\bottomrule
\end{tabular}
 \end{adjustwidth}
\caption[Porównanie wyników miar asocjacyjnych i sieci neuronowych dla korpusu \emph{KIPI}]{Porównanie wyników miar asocjacyjnych i sieci neuronowych dla korpusu \emph{KIPI}.}
\label{KIPI_sub5_measures_vs_nn}
\end{table}

Zbalansowanie klas spowodowało znaczny wzrost jakości wyników generowanych zarówno przez sieci neuronowe, jak i przez miary asocjacyjne.
Analizując wyniki zauważyć jednak można, że wzrost jakości wyników sieci jest znacznie większy niż wzrost wyników miar, a dodatkowo w większości badań perceptrony wielowarstwowe okazały się nawet prawie czterokrotnie lepsze (3,8433) niż miary asocjacyjne.
Ponadto wyniki najlepszej z sieci uplasowały się na poziomie większym o ponad 7\% niż wynik najlepszej z miar asocjacyjnych. 
Perceptron wielowarstwowy zdołał osiągnąć wynik na poziomie ponad \emph{0,966} i mimo, że okazał się być lepszy tylko o 7\%, trzeba mieć na uwadze poziom tego wyniku jak i jakość najlepszej z miar, która osiągnęła prawie \emph{0,899}.
Innymi słowy w takiej skali polepszenie wyniku o ponad 7\% może być znaczące\footnote{Zwiększenie jakości wyniku z prawie 0,9 do niemal 0,97 jest duże, ponieważ maksymalna możliwa do osiągnięcia jakość rozwiązania jest równa 1.} dla pewnych systemów.
Na uwagę zasługuje też wynik sieci w badaniu numer trzy, gdzie najlepsza z nich osiągnęła wynik prawie czterokrotnie lepszy niż miary.
Badanie to było przeprowadzone na zbiorze danych bez użycia filtru relacji opartego o części mowy, a mimo to perceptron osiągnął wynik na poziomie \emph{0,8813}, co stanowi prawie \emph{0,98}\% wyniku osiągniętego przez najlepszą z miar asocjacyjnych w dowolnym z badań, a zaznaczyć trzeba, że maksymalna możliwa do osiągnięcia kompletność na zbiorze numer trzy jest większa niż na jakimkolwiek innym zbiorze filtrowanym za pomocą operatorów \emph{WCCL}.
Dodatkowo brak zastosowania jakichkolwiek filtrów pozwolił zachować maksymalną możliwą do osiągnięcia kompletność przy jednoczesnym uplasowaniu się wyniku najlepszej z sieci neuronowych na poziomie \emph{0,61}.
Ciekawym może być fakt, że w drugiej piętnastce części badań sieci osiągały zawsze lepsze wyniki na zbiorach poddanych maksymalnej filtracji (częstość, Morfeusz, relacje); w zbiorach poddanych częściowej filtracji (Morfeusz, relacje) sieci okazały się lepsze już tylko w trzech z pięciu przypadków, a w przypadku wykorzystania już tylko filtrów \emph{WCCL} miary przodowały w każdym z pięciu badań.
Wnioskiem z tej obserwacji może być to, że miary są bardziej odporne na gorszą jakość danych.

\par
Zauważyć można, że sieci neuronowe o mniejszej wartości parameteru odpowiedzialnego za szybkość uczenia osiągały lepsze wyniki -- więcej rezultatów w okolicy najlepszych z wyników i o lepszej jakości.
Jeśli chodzi o liczbę neuronów w warstwach ukrytych, różnice w osiąganych wynikach nie zmieniają się zbytnio przy zwiększaniu rozmiarów warstw, a znacznie bardziej istotne jest przygotowanie danych.
Najlepsze wyniki sieci osiągały w badaniach numer 12 i 30, czyli w przypadkach ekstrakcji tylko relacji częstych i po wykonaniu filtracji z wykorzystaniem Morfeusza i częstości.
Dodatkowo zauważyć można, że w tych przypadkach, a także w badaniu numer 24, sieci zwracały wyniki o podobnej jakości mimo różnic w parametrach tych perceptronów wielowarstwowych.

\par
Podsumowując analizę niniejszego badania zauważyć należy, że zbalansowanie klas ma ogromny wpływ na jakość wyników generowanych przez perceptrony wielowarstwowe.
Dodatkowo potrafią one radzić sobie w tego typu zadaniach kilkukrotnie lepiej niż pojedyncze miary asocjacyjne.
Zaznaczyć trzeba jednak, że niniejszy zbiór danych został utworzony sztucznie, a tym samym nie oddaje dokładnie realiów problematyki omawianej w niniejszej pracy, jednak mimo to klasyfikatory mogą okazać się bardzo dobrym narzędziem w procesie ekstrakcji wyrażeń wielowyrazowych.


\subsubsection{Rezultaty badań dla sieci neuronowych po podpróbkowaniu klasy negatywnej do 95\% i dodaniu częstości jako cechy}
Tabele \ref{KIPI_sub5_nn_add_freq_part_1} oraz \ref{KIPI_sub5_nn_add_freq_part_2} prezentują jakość wyników osiągniętych przez 48 sieci neuronowych w 30 różnych badaniach (30 zestawów danych pozyskanych z korpusu \emph{KIPI}).
Różnica pomiędzy tym, a poprzednim badaniem jest taka, że dodana do zbioru cech została częstość kandydata w korpusie.
Indeksy typów badań pozostały takie same jak w poprzednich badaniach, a indeksy zbadanych sieci neuronowych są identyczne z przedstawionymi w tabeli we wcześniejszej części tego rozdziału pracy.

\begin{table}[htp!]
\centering
\footnotesize\setlength{\tabcolsep}{2.5pt}
 \begin{adjustwidth}{-2cm}{}
\begin{tabular}{ l | *{15}{| r}}
	\toprule 
	\textbf{99\%} &	\textbf{1}	&	\textbf{2}	&	\textbf{3}	&	\textbf{4}	&	\textbf{5}	&	\textbf{6}	&	\textbf{7}	&	\textbf{8}	&	\textbf{9}	&	\textbf{10}	&	\textbf{11}	&	\textbf{12}	&	\textbf{13}	&	\textbf{14}	&	\textbf{15}	\\
	\midrule
1	&	0,5738	&	0,6681	&	0,6352	&	0,3677	&	\textbf{0,4708}	&	0,5677	&	0,4034	&	0,4506	&	0,7703	&	0,7575	&	0,7665	&	0,9314	&	0,6056	&	0,4971	&	0,7102	\\
2	&	0,5817	&	0,7266	&	0,6811	&	0,3612	&	0,4673	&	0,5712	&	0,4085	&	0,4515	&	0,7616	&	0,7581	&	0,7663	&	0,9299	&	0,5852	&	0,5172	&	0,6052	\\
3	&	0,6164	&	\textbf{0,7321}	&	0,6442	&	0,3717	&	0,4613	&	0,5710	&	0,4112	&	0,4624	&	0,7821	&	0,7558	&	0,7680	&	0,9051	&	0,6082	&	0,5622	&	0,6046	\\
4	&	0,6027	&	0,7236	&	0,5242	&	0,3692	&	0,4551	&	0,5684	&	0,4134	&	0,4536	&	0,7020	&	0,7565	&	0,7670	&	0,8946	&	0,5965	&	0,5989	&	0,6032	\\
5	&	0,6136	&	0,6815	&	0,5730	&	0,3706	&	0,4678	&	0,5705	&	0,4151	&	0,4570	&	0,6918	&	0,7582	&	0,7704	&	0,9010	&	0,5961	&	0,6075	&	0,5761	\\
6	&	0,5854	&	0,7267	&	0,6103	&	0,3646	&	0,4616	&	0,5684	&	0,4141	&	0,4603	&	0,7531	&	0,7563	&	0,7703	&	0,9126	&	0,6069	&	0,6272	&	0,6073	\\
7	&	0,6303	&	0,6769	&	0,4440	&	0,3616	&	0,4457	&	0,5694	&	0,4172	&	0,4635	&	0,6407	&	\textbf{0,7584}	&	0,7706	&	0,8627	&	0,5957	&	0,6349	&	0,5951	\\
8	&	0,6149	&	0,7169	&	0,4089	&	0,3769	&	0,4511	&	0,5729	&	0,4176	&	0,4642	&	0,7005	&	\textbf{0,7596}	&	\textbf{0,7720}	&	0,8926	&	0,5827	&	0,6033	&	0,5548	\\
9	&	0,6263	&	0,7190	&	0,5102	&	0,3637	&	0,4580	&	0,5730	&	0,4171	&	0,4598	&	0,7068	&	0,7568	&	\textbf{0,7720}	&	0,8589	&	0,5679	&	0,6468	&	0,6123	\\
10	&	0,5938	&	0,7212	&	0,4474	&	0,3617	&	0,4481	&	0,5759	&	0,4206	&	0,4631	&	0,6523	&	0,7552	&	\textbf{0,7730}	&	0,8926	&	0,5980	&	0,6243	&	0,5509	\\
11	&	\textbf{0,6606}	&	0,7246	&	0,4329	&	0,3589	&	0,4458	&	0,5666	&	0,4186	&	0,4629	&	0,6501	&	0,7540	&	\textbf{0,7723}	&	0,9226	&	0,5931	&	0,6458	&	0,6446	\\
12	&	0,6214	&	0,6697	&	0,4088	&	0,3653	&	0,4472	&	0,5679	&	0,4200	&	0,4716	&	0,6965	&	0,7568	&	\textbf{0,7731}	&	0,9248	&	0,5687	&	0,6173	&	0,5985	\\
13	&	\textbf{0,6650}	&	0,6932	&	0,8623	&	0,3694	&	0,4677	&	0,5866	&	0,4113	&	0,4595	&	0,8038	&	\textbf{0,7623}	&	0,7708	&	\textbf{0,9468}	&	0,6161	&	0,5405	&	0,8115	\\
14	&	0,6107	&	\textbf{0,7354}	&	0,8628	&	0,3809	&	0,4562	&	0,5812	&	0,4143	&	0,4669	&	0,7955	&	\textbf{0,7614}	&	\textbf{0,7728}	&	\textbf{0,9480}	&	0,6259	&	0,6487	&	0,7317	\\
15	&	0,6410	&	\textbf{0,7344}	&	0,8628	&	0,3710	&	0,4564	&	0,5819	&	0,4166	&	0,4670	&	0,8086	&	\textbf{0,7622}	&	0,7710	&	\textbf{0,9468}	&	0,6061	&	0,6532	&	0,7710	\\
16	&	0,6516	&	\textbf{0,7326}	&	0,8698	&	0,3668	&	0,4501	&	0,5847	&	0,4189	&	0,4674	&	0,8013	&	\textbf{0,7620}	&	\textbf{0,7750}	&	\textbf{0,9493}	&	0,6014	&	0,6364	&	0,8608	\\
17	&	0,6330	&	0,7206	&	\textbf{0,8760}	&	0,3742	&	0,4642	&	0,5810	&	0,4199	&	0,4682	&	\textbf{0,8127}	&	\textbf{0,7611}	&	\textbf{0,7759}	&	\textbf{0,9437}	&	0,5928	&	0,6536	&	0,7786	\\
18	&	0,6291	&	0,7245	&	0,8647	&	0,3782	&	0,4586	&	0,5816	&	0,4222	&	0,4664	&	0,7976	&	\textbf{0,7619}	&	\textbf{0,7768}	&	\textbf{0,9464}	&	0,5938	&	0,6506	&	0,7772	\\
19	&	0,6439	&	0,7221	&	0,8428	&	0,3765	&	0,4654	&	0,5806	&	0,4238	&	0,4683	&	\textbf{0,8120}	&	\textbf{0,7631}	&	\textbf{0,7749}	&	\textbf{0,9444}	&	0,6086	&	0,6630	&	0,7313	\\
20	&	0,6436	&	0,7140	&	0,8645	&	\textbf{0,3831}	&	0,4604	&	0,5730	&	0,4245	&	0,4706	&	0,7860	&	\textbf{0,7659}	&	\textbf{0,7748}	&	\textbf{0,9446}	&	0,5844	&	0,6449	&	0,7645	\\
21	&	0,6525	&	0,7216	&	0,8676	&	0,3811	&	0,4681	&	0,5755	&	\textbf{0,4256}	&	0,4717	&	\textbf{0,8143}	&	\textbf{0,7637}	&	\textbf{0,7743}	&	\textbf{0,9491}	&	0,6065	&	\textbf{0,6734}	&	0,7692	\\
22	&	0,6545	&	0,7071	&	\textbf{0,8775}	&	\textbf{0,3841}	&	0,4525	&	0,5809	&	\textbf{0,4281}	&	\textbf{0,4799}	&	\textbf{0,8088}	&	\textbf{0,7630}	&	\textbf{0,7755}	&	0,9387	&	0,6184	&	0,6423	&	0,7903	\\
23	&	0,6460	&	0,7258	&	\textbf{0,8800}	&	0,3787	&	0,4613	&	0,5806	&	\textbf{0,4262}	&	\textbf{0,4772}	&	\textbf{0,8152}	&	\textbf{0,7652}	&	\textbf{0,7793}	&	\textbf{0,9451}	&	0,5833	&	0,6603	&	0,5916	\\
24	&	0,6406	&	0,7183	&	0,8106	&	0,3808	&	0,4661	&	0,5794	&	\textbf{0,4261}	&	\textbf{0,4764}	&	0,8001	&	\textbf{0,7600}	&	\textbf{0,7770}	&	\textbf{0,9467}	&	0,6056	&	\textbf{0,6781}	&	0,7599	\\
25	&	0,6325	&	0,7135	&	0,8249	&	0,3705	&	\textbf{0,4715}	&	0,5613	&	0,3925	&	0,4584	&	0,8012	&	0,7570	&	0,7676	&	0,9382	&	\textbf{0,6359}	&	0,5194	&	0,7795	\\
26	&	0,5857	&	0,7211	&	0,7407	&	0,3688	&	0,4678	&	0,5750	&	0,4073	&	0,4603	&	0,7851	&	0,7578	&	0,7672	&	0,9092	&	0,5614	&	0,5424	&	0,6842	\\
27	&	\textbf{0,6654}	&	\textbf{0,7319}	&	0,8000	&	0,3681	&	0,4546	&	0,5706	&	0,4097	&	0,4607	&	0,7594	&	0,7552	&	0,7678	&	0,9378	&	0,5727	&	0,5892	&	0,7593	\\
28	&	0,6178	&	0,7135	&	0,7471	&	0,3561	&	\textbf{0,4736}	&	0,5647	&	0,4133	&	0,4598	&	0,7593	&	0,7569	&	0,7671	&	0,9348	&	0,5796	&	0,6005	&	0,5788	\\
29	&	0,6387	&	0,7206	&	0,6801	&	0,3727	&	0,4554	&	0,5770	&	0,4125	&	0,4633	&	0,7611	&	0,7575	&	0,7687	&	0,9352	&	0,5842	&	0,5884	&	0,6017	\\
30	&	\textbf{0,6598}	&	0,7064	&	0,7401	&	0,3619	&	0,4530	&	0,5732	&	0,4156	&	0,4642	&	0,6994	&	0,7573	&	0,7675	&	0,9355	&	0,5798	&	0,6229	&	0,6455	\\
31	&	0,6378	&	0,7108	&	0,6340	&	0,3579	&	0,4581	&	0,5722	&	0,4174	&	0,4602	&	0,7895	&	0,7554	&	0,7708	&	0,8995	&	0,5632	&	\textbf{0,6715}	&	0,6170	\\
32	&	0,6550	&	0,6588	&	0,5740	&	0,3547	&	0,4512	&	0,5729	&	0,4194	&	0,4672	&	0,7495	&	\textbf{0,7591}	&	0,7691	&	0,9065	&	0,5460	&	0,6555	&	0,6212	\\
33	&	0,6465	&	0,6931	&	0,4666	&	0,3680	&	0,4586	&	0,5661	&	0,4182	&	0,4694	&	0,7630	&	\textbf{0,7586}	&	0,7701	&	0,9021	&	0,5953	&	0,6597	&	0,5463	\\
34	&	0,6277	&	0,7254	&	0,5801	&	0,3744	&	0,4514	&	0,5715	&	0,4202	&	0,4649	&	0,7351	&	\textbf{0,7592}	&	\textbf{0,7733}	&	0,8981	&	0,5531	&	0,6480	&	0,5874	\\
35	&	0,6457	&	0,7217	&	0,5799	&	0,3685	&	0,4512	&	0,5671	&	0,4206	&	0,4661	&	0,7326	&	\textbf{0,7602}	&	\textbf{0,7737}	&	0,9310	&	0,5899	&	0,6666	&	0,6447	\\
36	&	0,6330	&	0,7187	&	0,5033	&	0,3647	&	0,4498	&	0,5679	&	0,4210	&	0,4649	&	0,7312	&	\textbf{0,7600}	&	\textbf{0,7721}	&	0,9283	&	0,6101	&	0,6471	&	0,6472	\\
37	&	0,6426	&	\textbf{0,7337}	&	0,8678	&	0,3651	&	0,4682	&	\textbf{0,5910}	&	0,4109	&	0,4625	&	0,8078	&	\textbf{0,7629}	&	\textbf{0,7721}	&	\textbf{0,9485}	&	0,6005	&	0,6001	&	\textbf{0,8719}	\\
38	&	0,6414	&	0,7257	&	0,8704	&	0,3691	&	0,4645	&	0,5811	&	0,4123	&	0,4652	&	\textbf{0,8099}	&	\textbf{0,7617}	&	\textbf{0,7727}	&	\textbf{0,9460}	&	0,5849	&	0,6209	&	0,8464	\\
39	&	0,6462	&	\textbf{0,7381}	&	\textbf{0,8731}	&	0,3735	&	0,4515	&	\textbf{0,5926}	&	0,4175	&	0,4677	&	\textbf{0,8116}	&	\textbf{0,7621}	&	\textbf{0,7736}	&	\textbf{0,9442}	&	0,5830	&	0,6360	&	\textbf{0,8716}	\\
40	&	0,6472	&	0,7263	&	\textbf{0,8811}	&	0,3768	&	0,4617	&	\textbf{0,5919}	&	0,4180	&	0,4718	&	\textbf{0,8125}	&	\textbf{0,7618}	&	\textbf{0,7727}	&	\textbf{0,9507}	&	0,5901	&	0,6629	&	0,8237	\\
41	&	0,6538	&	0,7226	&	0,8715	&	\textbf{0,3851}	&	0,4660	&	0,5856	&	0,4209	&	0,4712	&	\textbf{0,8168}	&	\textbf{0,7620}	&	\textbf{0,7735}	&	\textbf{0,9491}	&	0,5975	&	0,6448	&	0,8330	\\
42	&	0,6396	&	\textbf{0,7321}	&	0,8704	&	0,3767	&	0,4590	&	0,5824	&	0,4203	&	0,4724	&	\textbf{0,8138}	&	\textbf{0,7618}	&	\textbf{0,7739}	&	\textbf{0,9489}	&	0,5871	&	0,6669	&	0,7884	\\
43	&	0,6382	&	0,7277	&	0,8671	&	0,3725	&	0,4529	&	0,5811	&	0,4250	&	0,4736	&	0,8007	&	\textbf{0,7600}	&	\textbf{0,7749}	&	\textbf{0,9480}	&	0,5740	&	\textbf{0,6772}	&	0,8450	\\
44	&	0,6400	&	0,7304	&	\textbf{0,8792}	&	0,3754	&	0,4628	&	0,5736	&	0,4244	&	0,4704	&	\textbf{0,8149}	&	\textbf{0,7610}	&	\textbf{0,7750}	&	\textbf{0,9460}	&	0,5924	&	0,6678	&	0,8366	\\
45	&	0,6366	&	0,7136	&	\textbf{0,8724}	&	\textbf{0,3820}	&	0,4551	&	0,5732	&	\textbf{0,4282}	&	0,4737	&	\textbf{0,8098}	&	\textbf{0,7624}	&	\textbf{0,7760}	&	\textbf{0,9460}	&	0,5963	&	\textbf{0,6776}	&	0,7812	\\
46	&	0,6414	&	0,7232	&	0,8655	&	0,3734	&	0,4575	&	0,5859	&	\textbf{0,4290}	&	0,4701	&	\textbf{0,8108}	&	\textbf{0,7646}	&	\textbf{0,7769}	&	\textbf{0,9450}	&	0,5852	&	\textbf{0,6743}	&	0,6737	\\
47	&	0,6496	&	0,7064	&	0,8711	&	0,3787	&	0,4541	&	0,5746	&	\textbf{0,4294}	&	\textbf{0,4763}	&	\textbf{0,8110}	&	\textbf{0,7655}	&	\textbf{0,7789}	&	\textbf{0,9473}	&	0,5900	&	0,6622	&	0,7838	\\
48	&	0,6484	&	0,7187	&	0,8690	&	0,3800	&	0,4607	&	0,5793	&	\textbf{0,4298}	&	\textbf{0,4770}	&	0,7942	&	\textbf{0,7626}	&	\textbf{0,7759}	&	\textbf{0,9431}	&	0,5829	&	0,6665	&	0,6937	\\
	\bottomrule
\end{tabular}
 \end{adjustwidth}
\caption[Wyniki badań sieci neuronowych rankingujących kolokacje dwuelementowe z korpusu \emph{KIPI} po podpróbkowaniu klasy negatywnej do 95\% i dodaniu częstości jako cechy, część 1]{Wyniki badań sieci neuronowych rankingujących kolokacje dwuelementowe z korpusu \emph{KIPI} po podpróbkowaniu klasy negatywnej do 95\% i dodaniu częstości jako cechy, część 1.}
\label{KIPI_sub5_nn_add_freq_part_1}
\end{table}

\begin{table}[htp!]
\centering
\footnotesize\setlength{\tabcolsep}{2.5pt}
 \begin{adjustwidth}{-2cm}{}
\begin{tabular}{ l | *{15}{| r}}
	\toprule
	\textbf{99\%} &	\textbf{16}	&	\textbf{17}	&	\textbf{18}	&	\textbf{19}	&	\textbf{20}	&	\textbf{21}	&	\textbf{22}	&	\textbf{23}	&	\textbf{24}	&	\textbf{25}	&	\textbf{26}	&	\textbf{27}	&	\textbf{28}	&	\textbf{29}	&	\textbf{30}	\\
	\midrule
1	&	0,7611	&	\textbf{0,8476}	&	0,9012	&	0,3985	&	0,4454	&	0,6143	&	0,7700	&	\textbf{0,7833}	&	\textbf{0,8928}	&	0,5047	&	0,5349	&	0,7207	&	0,8282	&	0,8218	&	\textbf{0,9136}	\\
2	&	0,7281	&	\textbf{0,8416}	&	0,9103	&	0,4026	&	0,4472	&	0,6141	&	0,7638	&	\textbf{0,7836}	&	\textbf{0,8934}	&	0,5022	&	0,5617	&	0,7269	&	0,8259	&	0,8246	&	\textbf{0,9135}	\\
3	&	0,7629	&	\textbf{0,8421}	&	0,8764	&	0,4092	&	0,4588	&	0,6144	&	0,7708	&	\textbf{0,7835}	&	\textbf{0,8939}	&	0,5090	&	0,5629	&	0,7207	&	0,8257	&	0,8177	&	\textbf{0,9127}	\\
4	&	0,7716	&	0,8294	&	0,9516	&	0,4101	&	0,4654	&	0,6157	&	0,7685	&	\textbf{0,7836}	&	\textbf{0,8946}	&	0,4959	&	0,5562	&	0,7228	&	0,8306	&	0,8136	&	\textbf{0,9114}	\\
5	&	0,7748	&	0,8076	&	0,9365	&	0,4150	&	0,4667	&	0,6119	&	0,7679	&	\textbf{0,7841}	&	\textbf{0,8945}	&	0,5131	&	0,5693	&	0,7208	&	\textbf{0,8388}	&	0,8185	&	\textbf{0,9116}	\\
6	&	0,7733	&	0,7877	&	0,9339	&	0,4155	&	0,4679	&	0,6152	&	0,7665	&	\textbf{0,7841}	&	\textbf{0,8949}	&	0,5127	&	0,5765	&	0,7220	&	0,8343	&	0,8163	&	\textbf{0,9116}	\\
7	&	0,7669	&	0,8032	&	0,9186	&	0,4178	&	0,4690	&	0,6118	&	0,7694	&	\textbf{0,7825}	&	\textbf{0,8928}	&	0,5009	&	0,5617	&	0,7264	&	\textbf{0,8397}	&	0,8206	&	\textbf{0,9086}	\\
8	&	\textbf{0,7971}	&	0,7948	&	0,8978	&	0,4181	&	0,4685	&	0,6135	&	\textbf{0,7718}	&	\textbf{0,7829}	&	\textbf{0,8942}	&	0,4975	&	0,5831	&	0,7255	&	\textbf{0,8388}	&	0,8229	&	\textbf{0,9077}	\\
9	&	0,7866	&	0,7996	&	0,9290	&	0,4191	&	0,4693	&	0,6139	&	0,7704	&	\textbf{0,7840}	&	\textbf{0,8950}	&	0,5020	&	0,5842	&	0,7236	&	0,8361	&	0,8212	&	\textbf{0,9078}	\\
10	&	0,7802	&	0,7954	&	0,8879	&	0,4205	&	\textbf{0,4713}	&	0,6155	&	0,7688	&	\textbf{0,7833}	&	\textbf{0,8956}	&	0,5011	&	0,5755	&	0,7256	&	\textbf{0,8431}	&	0,8186	&	\textbf{0,9121}	\\
11	&	0,7602	&	0,7860	&	0,8977	&	0,4201	&	\textbf{0,4715}	&	0,6128	&	\textbf{0,7739}	&	0,7787	&	\textbf{0,8941}	&	0,5057	&	0,5838	&	0,7243	&	\textbf{0,8437}	&	0,8181	&	\textbf{0,9074}	\\
12	&	0,7819	&	0,8183	&	0,9222	&	\textbf{0,4213}	&	\textbf{0,4716}	&	0,6111	&	\textbf{0,7753}	&	\textbf{0,7836}	&	\textbf{0,8937}	&	0,5032	&	0,5878	&	0,7222	&	\textbf{0,8398}	&	0,8168	&	0,9055	\\
13	&	\textbf{0,7902}	&	\textbf{0,8407}	&	0,9506	&	0,4079	&	0,4634	&	0,6191	&	\textbf{0,7733}	&	\textbf{0,7867}	&	\textbf{0,8931}	&	0,5195	&	0,5720	&	\textbf{0,7333}	&	0,8275	&	\textbf{0,8295}	&	\textbf{0,9128}	\\
14	&	0,7817	&	0,8349	&	\textbf{0,9659}	&	0,4146	&	0,4678	&	0,6189	&	\textbf{0,7769}	&	\textbf{0,7857}	&	\textbf{0,8928}	&	0,5206	&	0,5632	&	\textbf{0,7375}	&	0,8321	&	0,8208	&	\textbf{0,9129}	\\
15	&	0,7845	&	0,7989	&	\textbf{0,9646}	&	0,4168	&	0,4703	&	\textbf{0,6213}	&	\textbf{0,7788}	&	\textbf{0,7847}	&	\textbf{0,8931}	&	0,5235	&	0,5698	&	\textbf{0,7371}	&	0,8335	&	0,8102	&	\textbf{0,9151}	\\
16	&	0,7839	&	0,7950	&	\textbf{0,9620}	&	0,4184	&	\textbf{0,4709}	&	\textbf{0,6200}	&	\textbf{0,7772}	&	\textbf{0,7845}	&	\textbf{0,8886}	&	0,5177	&	0,5777	&	0,7264	&	\textbf{0,8392}	&	0,8097	&	\textbf{0,9132}	\\
17	&	0,7744	&	0,7954	&	\textbf{0,9617}	&	0,4202	&	\textbf{0,4722}	&	\textbf{0,6211}	&	\textbf{0,7769}	&	\textbf{0,7845}	&	\textbf{0,8937}	&	0,5247	&	0,5799	&	\textbf{0,7332}	&	0,8356	&	0,8252	&	\textbf{0,9100}	\\
18	&	0,7900	&	0,8019	&	\textbf{0,9593}	&	\textbf{0,4231}	&	\textbf{0,4720}	&	0,6185	&	\textbf{0,7753}	&	\textbf{0,7848}	&	\textbf{0,8945}	&	0,5187	&	0,5880	&	\textbf{0,7326}	&	0,8362	&	0,8234	&	\textbf{0,9105}	\\
19	&	0,7648	&	0,7677	&	\textbf{0,9636}	&	\textbf{0,4227}	&	\textbf{0,4732}	&	\textbf{0,6254}	&	\textbf{0,7764}	&	\textbf{0,7838}	&	\textbf{0,8931}	&	0,5166	&	0,5800	&	\textbf{0,7327}	&	0,8365	&	0,8162	&	\textbf{0,9104}	\\
20	&	0,7683	&	0,7894	&	\textbf{0,9605}	&	\textbf{0,4235}	&	\textbf{0,4742}	&	\textbf{0,6250}	&	\textbf{0,7758}	&	\textbf{0,7836}	&	\textbf{0,8918}	&	0,5233	&	0,5784	&	\textbf{0,7304}	&	\textbf{0,8431}	&	\textbf{0,8310}	&	\textbf{0,9108}	\\
21	&	0,7897	&	0,7946	&	\textbf{0,9657}	&	\textbf{0,4244}	&	\textbf{0,4744}	&	0,6170	&	\textbf{0,7755}	&	\textbf{0,7829}	&	\textbf{0,8924}	&	0,5180	&	\textbf{0,5924}	&	0,7286	&	\textbf{0,8396}	&	\textbf{0,8318}	&	\textbf{0,9128}	\\
22	&	0,7835	&	0,7859	&	\textbf{0,9585}	&	\textbf{0,4246}	&	\textbf{0,4743}	&	0,6191	&	\textbf{0,7765}	&	\textbf{0,7844}	&	\textbf{0,8905}	&	0,5186	&	\textbf{0,5946}	&	0,7237	&	0,8327	&	\textbf{0,8261}	&	\textbf{0,9119}	\\
23	&	0,7718	&	0,7878	&	\textbf{0,9609}	&	\textbf{0,4226}	&	\textbf{0,4741}	&	\textbf{0,6205}	&	\textbf{0,7764}	&	\textbf{0,7839}	&	\textbf{0,8920}	&	0,5153	&	\textbf{0,5912}	&	0,7297	&	\textbf{0,8439}	&	\textbf{0,8282}	&	\textbf{0,9125}	\\
24	&	0,7708	&	0,7736	&	\textbf{0,9629}	&	\textbf{0,4246}	&	\textbf{0,4747}	&	0,6186	&	\textbf{0,7773}	&	\textbf{0,7840}	&	\textbf{0,8902}	&	0,5157	&	0,5848	&	\textbf{0,7325}	&	\textbf{0,8428}	&	0,8233	&	\textbf{0,9123}	\\
25	&	0,7629	&	0,8367	&	0,9297	&	0,3957	&	0,4517	&	0,6142	&	0,7660	&	\textbf{0,7819}	&	\textbf{0,8931}	&	0,4787	&	0,5454	&	0,7238	&	0,8252	&	0,8212	&	\textbf{0,9132}	\\
26	&	0,7765	&	0,8259	&	0,9556	&	0,4069	&	0,4514	&	0,6144	&	\textbf{0,7724}	&	\textbf{0,7833}	&	\textbf{0,8929}	&	0,5002	&	0,5462	&	0,7232	&	0,8258	&	0,8200	&	\textbf{0,9117}	\\
27	&	0,7384	&	0,7784	&	0,9467	&	0,4085	&	0,4549	&	0,6154	&	0,7686	&	\textbf{0,7826}	&	\textbf{0,8933}	&	0,5071	&	0,5599	&	0,7238	&	0,8308	&	0,8234	&	\textbf{0,9111}	\\
28	&	0,7455	&	0,7843	&	0,9179	&	0,4112	&	0,4657	&	0,6132	&	\textbf{0,7712}	&	\textbf{0,7838}	&	\textbf{0,8940}	&	0,5032	&	0,5502	&	0,7248	&	0,8287	&	0,8185	&	\textbf{0,9100}	\\
29	&	0,7623	&	0,7703	&	0,9441	&	0,4099	&	0,4647	&	0,6155	&	\textbf{0,7730}	&	\textbf{0,7826}	&	\textbf{0,8938}	&	0,4985	&	0,5475	&	0,7296	&	0,8188	&	0,8151	&	\textbf{0,9132}	\\
30	&	0,7872	&	0,7468	&	0,9202	&	0,4162	&	0,4658	&	0,6159	&	\textbf{0,7739}	&	\textbf{0,7829}	&	\textbf{0,8946}	&	0,5142	&	0,5616	&	0,7257	&	0,8333	&	0,8184	&	\textbf{0,9114}	\\
31	&	0,7673	&	0,7609	&	0,9059	&	0,4182	&	0,4689	&	0,6133	&	\textbf{0,7718}	&	\textbf{0,7832}	&	\textbf{0,8949}	&	0,4939	&	0,5832	&	0,7248	&	0,8334	&	0,8198	&	\textbf{0,9117}	\\
32	&	0,7587	&	0,7457	&	0,9476	&	0,4193	&	0,4700	&	0,6126	&	\textbf{0,7734}	&	\textbf{0,7834}	&	\textbf{0,8931}	&	0,5101	&	0,5639	&	0,7298	&	\textbf{0,8449}	&	0,8200	&	\textbf{0,9111}	\\
33	&	\textbf{0,7913}	&	0,7662	&	0,9081	&	0,4201	&	0,4681	&	0,6141	&	\textbf{0,7736}	&	\textbf{0,7839}	&	\textbf{0,8929}	&	0,5004	&	0,5794	&	0,7225	&	0,8329	&	0,8213	&	\textbf{0,9102}	\\
34	&	0,7501	&	0,7488	&	0,9042	&	0,4201	&	\textbf{0,4713}	&	0,6146	&	\textbf{0,7715}	&	\textbf{0,7834}	&	\textbf{0,8940}	&	0,4915	&	0,5788	&	0,7251	&	\textbf{0,8372}	&	\textbf{0,8269}	&	\textbf{0,9093}	\\
35	&	0,7311	&	0,7755	&	0,9020	&	\textbf{0,4215}	&	0,4687	&	0,6132	&	0,7708	&	\textbf{0,7827}	&	\textbf{0,8924}	&	0,4951	&	0,5825	&	\textbf{0,7322}	&	\textbf{0,8419}	&	\textbf{0,8276}	&	\textbf{0,9092}	\\
36	&	0,7825	&	0,7792	&	0,9007	&	\textbf{0,4212}	&	0,4681	&	0,6110	&	\textbf{0,7736}	&	\textbf{0,7833}	&	\textbf{0,8921}	&	0,5132	&	0,5871	&	0,7220	&	\textbf{0,8449}	&	\textbf{0,8289}	&	\textbf{0,9095}	\\
37	&	0,7745	&	0,8258	&	0,9404	&	0,4110	&	0,4644	&	0,6190	&	\textbf{0,7767}	&	\textbf{0,7856}	&	\textbf{0,8930}	&	0,5141	&	0,5714	&	\textbf{0,7349}	&	0,8300	&	\textbf{0,8261}	&	\textbf{0,9116}	\\
38	&	\textbf{0,7906}	&	0,7613	&	\textbf{0,9641}	&	0,4151	&	0,4680	&	0,6160	&	\textbf{0,7775}	&	\textbf{0,7848}	&	\textbf{0,8928}	&	0,5094	&	0,5800	&	\textbf{0,7339}	&	0,8237	&	\textbf{0,8309}	&	\textbf{0,9126}	\\
39	&	0,7575	&	0,7899	&	\textbf{0,9637}	&	0,4169	&	\textbf{0,4714}	&	0,6158	&	\textbf{0,7770}	&	\textbf{0,7834}	&	\textbf{0,8939}	&	\textbf{0,5315}	&	0,5752	&	\textbf{0,7339}	&	0,8292	&	0,8043	&	\textbf{0,9105}	\\
40	&	0,7781	&	0,7825	&	\textbf{0,9655}	&	0,4190	&	\textbf{0,4709}	&	\textbf{0,6222}	&	\textbf{0,7768}	&	\textbf{0,7834}	&	\textbf{0,8918}	&	\textbf{0,5319}	&	0,5857	&	0,7278	&	0,8334	&	0,8226	&	\textbf{0,9118}	\\
41	&	\textbf{0,7982}	&	0,7928	&	\textbf{0,9592}	&	0,4202	&	\textbf{0,4719}	&	0,6172	&	\textbf{0,7770}	&	\textbf{0,7848}	&	\textbf{0,8908}	&	0,5246	&	\textbf{0,5898}	&	\textbf{0,7370}	&	0,8350	&	0,8167	&	\textbf{0,9111}	\\
42	&	0,7675	&	0,7884	&	\textbf{0,9635}	&	\textbf{0,4218}	&	\textbf{0,4721}	&	\textbf{0,6196}	&	\textbf{0,7759}	&	\textbf{0,7837}	&	\textbf{0,8942}	&	0,5219	&	0,5778	&	0,7226	&	0,8364	&	0,8236	&	\textbf{0,9125}	\\
43	&	0,7886	&	0,7738	&	\textbf{0,9590}	&	\textbf{0,4232}	&	\textbf{0,4740}	&	\textbf{0,6232}	&	\textbf{0,7764}	&	\textbf{0,7837}	&	\textbf{0,8917}	&	0,5147	&	0,5861	&	0,7300	&	0,8342	&	0,8232	&	\textbf{0,9130}	\\
44	&	0,7761	&	0,7761	&	0,9312	&	\textbf{0,4232}	&	\textbf{0,4732}	&	0,6181	&	\textbf{0,7767}	&	\textbf{0,7845}	&	\textbf{0,8869}	&	0,5244	&	0,5856	&	\textbf{0,7310}	&	0,8332	&	\textbf{0,8285}	&	\textbf{0,9094}	\\
45	&	\textbf{0,7946}	&	0,7744	&	\textbf{0,9640}	&	\textbf{0,4242}	&	\textbf{0,4734}	&	0,6184	&	\textbf{0,7766}	&	\textbf{0,7815}	&	\textbf{0,8944}	&	0,5204	&	\textbf{0,5912}	&	0,7289	&	\textbf{0,8436}	&	0,8191	&	\textbf{0,9113}	\\
46	&	0,7705	&	0,7739	&	0,9245	&	\textbf{0,4239}	&	\textbf{0,4746}	&	\textbf{0,6208}	&	\textbf{0,7764}	&	\textbf{0,7844}	&	\textbf{0,8924}	&	0,5131	&	\textbf{0,5929}	&	0,7251	&	\textbf{0,8456}	&	\textbf{0,8340}	&	\textbf{0,9090}	\\
47	&	0,7682	&	0,7805	&	0,9553	&	\textbf{0,4240}	&	\textbf{0,4748}	&	\textbf{0,6208}	&	\textbf{0,7769}	&	\textbf{0,7842}	&	\textbf{0,8914}	&	0,5123	&	\textbf{0,5891}	&	0,7278	&	\textbf{0,8428}	&	\textbf{0,8319}	&	\textbf{0,9104}	\\
48	&	\textbf{0,7902}	&	0,7708	&	\textbf{0,9636}	&	\textbf{0,4249}	&	\textbf{0,4751}	&	\textbf{0,6202}	&	\textbf{0,7760}	&	\textbf{0,7829}	&	\textbf{0,8907}	&	0,5126	&	0,5873	&	\textbf{0,7325}	&	\textbf{0,8447}	&	\textbf{0,8264}	&	\textbf{0,9098}	\\
	\bottomrule
\end{tabular}
 \end{adjustwidth}
\caption[Wyniki badań sieci neuronowych rankingujących kolokacje dwuelementowe z korpusu \emph{KIPI} po podpróbkowaniu klasy negatywnej do 95\% i dodaniu częstości jako cechy, część 2]{Wyniki badań sieci neuronowych rankingujących kolokacje dwuelementowe z korpusu \emph{KIPI} po podpróbkowaniu klasy negatywnej do 95\% i dodaniu częstości jako cechy, część 2.}
\label{KIPI_sub5_nn_add_freq_part_2}
\end{table}

Tabela \ref{KIPI_sub5_add_freq_measures_vs_nn} prezentuje stosunki wyników najlepszej z sieci w tym badaniu z wynikami najlepszej sieci z poprzedniego badania, dla każdego z badań z osobna.
Wytłuszczoną czcionką zostały zapisane te stosunki, które są większe od 1, czyli oznaczone zostały takie wyniki, w których sieć, której zestaw cech został wzbogacony o częstość okazała się lepsza od sieci korzystającej z zestawu cech bez częstości krotki.
Litera \emph{f} oznacza zestaw cech łącznie z częstością, \emph{!f} zestaw cech bez częstości, a \emph{s} stosunek $ f / !f $.
\clearpage	% too manz floats
\begin{table}[htp!]
\centering
\footnotesize\setlength{\tabcolsep}{2.5pt}
 \begin{adjustwidth}{-2cm}{}
\begin{tabular}{ l | *{15}{| r}}
	\toprule 
	&	1	&	2	&	3	&	4	&	5	&	6	&	7	&	8	&	9	&	10	&	11	&	12	&	13	&	14	&	15	\\
	\midrule
f	&	0,6654	&	0,7381	&	0,8811	&	0,3851	&	0,4736	&	0,5926	&	0,4298	&	0,4799	&	0,8168	&	0,7659	&	0,7793	&	0,9507	&	0,6359	&	0,6781	&	0,8719	\\
!f	&	0,6120	&	0,7361	&	0,8813	&	0,3964	&	0,4769	&	0,5963	&	0,4243	&	0,4760	&	0,8076	&	0,7754	&	0,7862	&	0,9397	&	0,6090	&	0,7024	&	0,8962	\\
\hline																															
s	&	\textbf{1,0872}	&	\textbf{1,0026}	&	0,9997	&	0,9715	&	0,9931	&	0,9939	&	\textbf{1,0128}	&	\textbf{1,0081}	&	\textbf{1,0114}	&	0,9878	&	0,9912	&	\textbf{1,0118}	&	\textbf{1,0442}	&	0,9653	&	0,9729	\\
	\bottomrule
	\toprule
	&	16	&	17	&	18	&	19	&	20	&	21	&	22	&	23	&	24	&	25	&	26	&	27	&	28	&	29	&	30	\\
	\midrule
f	&	0,7982	&	0,8476	&	0,9659	&	0,4249	&	0,4751	&	0,6254	&	0,7788	&	0,7867	&	0,8956	&	0,5319	&	0,5946	&	0,7375	&	0,8456	&	0,8340	&	0,9151	\\
!f	&	0,7730	&	0,8402	&	0,9664	&	0,4218	&	0,4707	&	0,6157	&	0,7761	&	0,7816	&	0,8916	&	0,5348	&	0,5904	&	0,7335	&	0,8346	&	0,8290	&	0,9152	\\
\hline																															
s	&	\textbf{1,0325}	&	\textbf{1,0088}	&	0,9995	&	\textbf{1,0072}	&	\textbf{1,0093}	&	\textbf{1,0158}	&	\textbf{1,0035}	&	\textbf{1,0065}	&	\textbf{1,0045}	&	0,9946	&	\textbf{1,0072}	&	\textbf{1,0054}	&	\textbf{1,0132}	&	\textbf{1,0060}	&	1,0000	\\
	\bottomrule
\end{tabular}
 \end{adjustwidth}
\caption[Porównanie wyników miar asocjacyjnych i sieci neuronowych dla korpusu \emph{KIPI}]{Porównanie wyników miar asocjacyjnych i sieci neuronowych dla korpusu \emph{KIPI}.}
\label{KIPI_sub5_add_freq_measures_vs_nn}
\end{table}

Zauważyć można, że dodanie do zestawu cech częstości kandydata wpłynęło na wyniki pozytywnie tylko w części przypadków, ale w ich większej liczbie -- w 19 z 30 badań.
Dodatkowo w dużej liczbie badań różnice w wyniku są na poziomie mniejszym od 0,01, co równie dobrze można uznać za przypadek ze względu na mniej lub bardziej korzystne losowanie danych w procesie walidacji krzyżowej.
Jednak w niektórych badaniach różnica ta sięga nawet około 0,035, co wydaje się być bardziej znaczącym wynikiem.
Po dodaniu częstości do zestawu cech najlepszy z wyników został jednak nieznacznie pogorszony -- może być to kwestia przypadku.
Podsumowanie tego badania może być takie, że zmiana zestawu cech w niewielki sposób zmieniła wyniki działania sieci, zaś wpłynęła na nie korzystnie tylko w części badań.


\section{Miary wieloelementowe, korpus KIPI}
Podobnie jak w przypadku miar dwuelementowych, także i tutaj należało przygotować zbiory danych do badań.
Zostały one przygotowane i zbadane w taki sam sposób, jak w przypadku miar dwuelementowych, ale z użyciem innych relacji -- 3-elementowych.
Wyjątkiem jest także liczba podkorpusów danych, które zostały utworzone, zaledwie cztery.
Powodem takiego działania była niewystarczająca ilość RAM\footnote{Przechowywanie i przetwarzanie wszystkich możliwych kandydatów trójelementowych, ciągłych i nieciągłych, oknowych oraz relacyjnych, a także metadanych o nich wymaga posiadania dużej ilości pamięci operacyjnej.}.
Zmianie uległ także sposób oceny relacji częstych -- w tym przypadku próg nie był ustalony równo na 1\% (z pewnymi wyjątkami, jak w przypadku badań miar dwuelementowych).
Zamiast tego relacje częste były wyznaczane poprzez porównanie procentowej zawartości jednostek wielowyrazowych wśród wszystkich kandydatów z danej relacji z procentem kolokacji wśród wszystkich kandydatów ze wszystkich relacji.
Dodatkowo z grona częstych odrzucone zostały relacje oknowe.


\subsection{Przygotowanie i zbadanie podkorpusów KIPI} 
Niniejszy rozdział prezentuje wyniki sprawdzenia utworzonych podkorpusów pod kątem udziału procentowego jednostek wielowyrazowych i wyznaczenia zbiorów relacji częstych.
Badania zostały wykonane tak samo, jak w przypadku podkorpusów utworzonych do badań metod ekstrakcji kolokacji dwuelementowych.


\subsubsection{Podzbiór \protect\textit{3W}}
Tabela \ref{KIPI_3W_stats} prezentuje wyniki przebadania miar na podkorpusie zawierającym wszystkich kandydatów utworzonych z wykorzystaniem relacji oknowych.

\begin{table}[h!]
\centering
\begin{tabular}{ l | r | r | r | l }
	\toprule
	relacje 	& liczba krotek & liczba JW & procent JW & częsta? 	\\
	\midrule
	Window3P0	&	78426220	&	2044	&	0,0026	&	nie	\\
	Window3P1	&	78426220	&	80	&	0,0001 &	nie	\\
	\midrule									
	Suma		&	156852440	&	2124	&	0,00135	&		\\
	\bottomrule
\end{tabular}
\caption[Statystyki podzbioru danych \emph{KIPI} 3W]{Statystyki dotyczące podzbioru danych 3W pozyskanego z korpusu \emph{KIPI}.}
\label{KIPI_3W_stats}
\end{table}

Obserwacją z niniejszego badania jest to, że wyrażeń wielowyrazowych 3-elementowych jest około 250-krotnie mniej niż w analogicznym zbiorze w badaniach nad kolokacjami dwuelementowymi, co stanowi o poziomie trudności niniejszego problemu.


\subsubsection{Podzbiór \protect\textit{3R}}
\begin{table}[h!]
\centering
\begin{tabular}{ l | r | r | r | l }
	\toprule
	\textbf{relacje} 	& \textbf{liczba krotek} & \textbf{liczba JW} & \textbf{procent JW} & \textbf{częsta?} 	\\
	\midrule
	AdjAdjSubst	&	541277	&	9	&	0,0017	&	nie	\\
	AdjPrepSubst	&	716945	&	120	&	0,0167	&	$ TAK $	\\
	AdjSubstAdj	&	967750	&	83	&	0,0086	&	$ TAK $	\\
	AdjSubstSubst	&	1260577	&	67	&	0,0053	&	nie	\\
	SubstAdjAdj	&	312186	&	24	&	0,0077	&	nie	\\
	SubstAdjSubst	&	1526866	&	83	&	0,0054	&	nie	\\
	SubstAdvAdj	&	113720	&	4	&	0,0035	&	nie	\\
	SubstConjSubst	&	601678	&	22	&	0,0037	&	nie	\\
	SubstPrepSubst	&	1604498	&	204	&	0,0127	&	$ TAK $	\\
	SubstSubstAdj	&	941736	&	84	&	0,0089	&	$ TAK $	\\
	SubstSubstSubst	&	1192601	&	100	&	0,0084	&	$ TAK $	\\
	\midrule									
	Suma wszystkich	&	9779834	&	800	&	0,0082	&		\\
	Suma częstych	&	5423530	&	591	&	0,0109	&		\\
	\bottomrule
\end{tabular}
\caption[Statystyki podzbioru danych \emph{KIPI} 3R]{Statystyki dotyczące podzbioru danych 3R pozyskanego z korpusu \emph{KIPI}.}
\label{KIPI_3R_stats}
\end{table}


\subsubsection{Podzbiór \protect\textit{3RW}}
Podzbiór \emph{3RW} powstał poprzez połączenie \emph{3R} oraz \emph{3W}, czyli analogicznie do sposobu powstania podzbioru \emph{2RW}.


\subsubsection{Podzbiór \protect\textit{3R1H}}
\begin{table}[h!]
\centering
\begin{tabular}{ l | r | r | r | l }
	\toprule
	\textbf{relacje} 	& \textbf{liczba krotek} & \textbf{liczba JW} & \textbf{procent JW} & \textbf{częsta?} 	\\
	\midrule
	AdjAdjSubst	&	541277	&	9	&	0,0017	&	nie	\\
	AdjAdjSubstH1P0	&	245555	&	2	&	0,0008	&	nie	\\
	AdjAdjSubstH1P1	&	1151945	&	5	&	0,0004	&	nie	\\
	AdjPrepSubst	&	716945	&	120	&	0,0167	&	$ TAK $	\\
	AdjPrepSubstH1P0	&	647731	&	32	&	0,0049	&	$ TAK $	\\
	AdjPrepSubstH1P1	&	827820	&	46	&	0,0056	&	$ TAK $	\\
	AdjSubstAdj	&	967750	&	83	&	0,0086	&	$ TAK $	\\
	AdjSubstAdjH1P0	&	1127086	&	3	&	0,0003	&	nie	\\
	AdjSubstAdjH1P1	&	859167	&	18	&	0,0021	&	nie	\\
	AdjSubstSubst	&	1260577	&	67	&	0,0053	&	$ TAK $	\\
	AdjSubstSubstH1P0	&	2113142	&	8	&	0,0004	&	nie	\\
	AdjSubstSubstH1P1	&	1062241	&	14	&	0,0013	&	nie	\\
	SubstAdjAdj	&	312186	&	24	&	0,0077	&	$ TAK $	\\
	SubstAdjAdjH1P0	&	984252	&	10	&	0,001	&	nie	\\
	SubstAdjAdjH1P1	&	256799	&	1	&	0,0004	&	nie	\\
	SubstAdjSubst	&	1526866	&	83	&	0,0054	&	$ TAK $	\\
	SubstAdjSubstH1P0	&	1684299	&	10	&	0,0006	&	nie	\\
	SubstAdjSubstH1P1	&	2143032	&	30	&	0,0014	&	nie	\\
	SubstAdvAdj	&	113720	&	4	&	0,0035	&	nie	\\
	SubstAdvAdjH1P0	&	70397	&	0	&	0	&	nie	\\
	SubstAdvAdjH1P1	&	231161	&	2	&	0,0009	&	nie	\\
	SubstConjSubst	&	601678	&	22	&	0,0037	&	$ TAK $	\\
	SubstConjSubstH1P0	&	528323	&	11	&	0,0021	&	nie	\\
	SubstConjSubstH1P1	&	575422	&	8	&	0,0014	&	nie	\\
	SubstPrepSubst	&	1604498	&	204	&	0,0127	&	$ TAK $	\\
	SubstPrepSubstH1P0	&	1274522	&	58	&	0,0046	&	$ TAK $	\\
	SubstPrepSubstH1P1	&	1679465	&	74	&	0,0044	&	$ TAK $	\\
	SubstSubstAdj	&	941736	&	84	&	0,0089	&	$ TAK $	\\
	SubstSubstAdjH1P0	&	977246	&	8	&	0,0008	&	nie	\\
	SubstSubstAdjH1P1	&	1631955	&	39	&	0,0024	&	nie	\\
	SubstSubstSubst	&	1192601	&	100	&	0,0084	&	$ TAK $	\\
	SubstSubstSubstH1P0	&	2227830	&	10	&	0,0004	&	nie	\\
	SubstSubstSubstH1P1	&	2267238	&	17	&	0,0007	&	nie	\\
	\midrule									
	Suma ciągłych	&	9779834	&	800	&	0,0082	&		\\
	Suma nieciągłych	&	24566628	&	406	&	0,0017	&		\\
	Suma wszystkich	&	34346462	&	1206	&	0,0035	&		\\
	Suma ciągłych częstych	&	5423530	&	591	&	0,0109	&		\\
	Suma nieciągłych częstych	&	8130845	&	406	&	0,0050	&		\\
	Suma wszystkich częstych	&	13554375	&	997	&	0,0074	&		\\
\bottomrule									
\end{tabular}
\caption[Statystyki podzbioru danych \emph{KIPI} 3R1H]{Statystyki dotyczące podzbioru danych 3R1H pozyskanego z korpusu \emph{KIPI}.}
\label{KIPI_3R1H_stats}
\end{table}

Podobnie jak w przypadku kolokacji dwuelementowych, także i tutaj dodanie relacji nieciągłych zmniejsza znacznie stężenie wyrażeń wielowyrazowych wśród wszystkich kandydatów.
Można spróbować wysunąć wniosek, że kolokacje 3-elementowe występują przeważnie w sposób ciągły, jednak liczba kolokacji nieciągłych nie może zostać pominięta, ponieważ stanowi znaczny procent wszystkich wyrażeń wielowyrazowych.


\subsection{Zbiór testowy}
Wykorzystany w tych badaniach zbiór testowy został pozyskany ze \emph{Słowosieci}, jednak okazał się być zbiorem bardzo małym i niepochodzącym ze zbioru, na którym badania były prowadzone.
Fakt ten rodził dwa następujące problemy: bardzo niski stosunek klasy pozytywnej do negatywnej oraz brak poprawnego oznakowania instancji na szeroką skalę\footnote{Zbiór danych nie był w pełni oznaczony jak na przykład w artykule Pavla Peciny i Pavla Schlesingera.}.
Pierwszy problem wynika z faktu, że liczba kandydatów na jednostki trójelementowe jest znacznie większa od liczby poprawnych jednostek wielowyrazowych wskazanych przez zbiór testowy.
Drugi problem polegał na tym, że wszyscy kandydaci, którzy nie znaleźli się w zbiorze testowym byli uznawani za negatywnych, a tak być nie powinno -- to także jest problem małego zbioru testowego.

\par
Miara oceny i sposób porównywania kandydatów ze zbiorem testowym pozostały takie same jak w przypadku badań ekstrakcji kolokacji dwuelementowych.


\subsection{Szczegółowy opis przebiegu tej części badań}
Tabela \ref{KIPI_3_research_types} przedstawia zestaw różnych wariantów badań przeprowadzonych dla funkcji 3-elementowych na korpusie \emph{KIPI}.
Zestaw ten jest mniejszy ze względu na złożoność pamięciową problemu.

\begin{table}[h!]
\centering
\footnotesize\setlength{\tabcolsep}{2.5pt}
\begin{tabular}{ l || l | l | l }
	\toprule
	\textbf{nr} 	& \textbf{źródło danych statystycznych}	& \textbf{źródło kandydatów}			& \textbf{filtry}					\\
	\midrule
	1	& okno ciągłe 			& okno ciągłe				& morfeusz, częstość $>$ 5	\\
	2	& relacje ciągłe		& relacje ciągłe			&				\\
	3	& relacje ciągłe		& relacje ciągłe			& morfeusz					\\
	4	& relacje ciągłe		& relacje ciągłe			& morfeusz, częstość $>$ 5	\\
	5	& relacje ciągłe		& częste relacje ciągłe 		&							\\
	6	& relacje ciągłe		& częste relacje ciągłe 		& morfeusz					\\
	7	& relacje ciągłe		& częste relacje ciągłe 		& morfeusz, częstość $>$ 5	\\
	8	& relacje ciągłe i nieciągłe	& relacje ciągłe i nieciągłe 		& 							\\
	9	& relacje ciągłe i nieciągłe	& relacje ciągłe i nieciągłe		& morfeusz					\\
	10	& relacje ciągłe i nieciągłe	& relacje ciągłe i nieciągłe		& morfeusz, częstość $>$ 5	\\
	11	& relacje ciągłe i nieciągłe	& częste relacje ciągłe i nieciągłe	& 							\\
	12	& relacje ciągłe i nieciągłe	& częste relacje ciągłe i nieciągłe	& morfeusz					\\
	13	& relacje ciągłe i nieciągłe	& częste relacje ciągłe i nieciągłe	& morfeusz, częstość $>$ 5	\\
	\bottomrule
\end{tabular}
\caption[Zestaw przeprowadzonych badań dla funkcji 3-elementowych na korpusie \emph{KIPI}]{Zestaw przeprowadzonych badań dla funkcji 3-elementowych na korpusie \emph{KIPI}.}
\label{KIPI_3_research_types}
\end{table}

Zestaw zbadanych funkcji asocjacyjnych jest taki sam, jak dla badań nad miarami ekstrakcji wyrażeń dwuelementowych, a tym samym tabela \ref{KIPI_2_function_set} przestawia jednocześnie zestaw zbadanych funkcji 3-elementowych na korpusie \emph{KIPI}.
Dodatkowo zbadane zostały także funkcje zamieszczone w tabeli \ref{KIPI_3_add_function_set} -- nie były one zbadane dla kolokacji dwuelementowych, ponieważ wyniki byłyby takie same jak dla ich funkcji wewnętrznych.
Sama funkcja wewnętrzna została ustalona na etapie badań miar ekstrahujących wyrażenia dwuelementowe i została nią miara $ Specific \: Exponential \: Correlation $ dla wartości parametru wykładnika równej 4,2.
\begin{table}[h!]
\centering
\footnotesize\setlength{\tabcolsep}{2.5pt}
\begin{tabular}{ l | l }
	\toprule
	\textbf{nr}  	& \textbf{nazwa}  \\
	\midrule
	73	& Fair Dispersion Point Normalization \\
	74	& Average Bigram \\
	75	& Smoothed Bigram \\
	76	& Minimal Bigram \\
	\bottomrule
\end{tabular}
\caption[Zestaw zbadanych funkcji dwuelementowych na korpusie \emph{KIPI}]{Zestaw zbadanych funkcji dwuelementowych na korpusie \emph{KIPI}.}
\label{KIPI_3_add_function_set}
\end{table}

Zestaw zbadanych klasyfikatorów etykietujących krotki 3-elementowe na korpusie \emph{KIPI} oraz wykorzystany zestaw cech pozostał taki sam, jak w przypadku badań nad wyrażeniami wieloelementowymi.
Motywacją do pozostawienia takich cech były głównie informacje o korelacji miar pozyskane z artykułu Mariusza Paradowskiego \cite{paradowski_beta}, ponieważ w przypadku doboru cechy ze względu na wynik pojawił się problem dominacji miary \emph{Specific Exponential Correlation} nad innymi.


\subsection{Wyniki}
Podczas przedstawiania jakości osiąganych wyników przez miary trójelementowe dla korpusu \emph{KIPI} używane będą odpowiednie numery badań zamiast ich pełnych nazw w celu identyfikacji rodzaju badania -- analogicznie jak miało to miejsce w przypadku kolokacji dwuelementowych.
Analogicznie stosowane będą identyfikatory miar zamiast ich pełnych nazw.
Wyniki badań zostały przygotowane i przedstawione w analogiczny sposób do badań nad kolokacjami dwuelementowymi.
Wyniki oznaczone czcionką pogrubioną także symbolizują to co w badaniu poprzednim -- dana funkcja jest najlepszą lub zbliżoną do niej jakościowo.

\par
Zakres uśrednianego rankingu pozostał taki sam, jak w badaniach nad ekstrakcją wyrażeń dwuelementowych -- od 10\% do 20\%.


\subsubsection{Wyniki badań miar trójelementowych}
Dwie tabele \ref{KIPI_3_part_1} oraz \ref{KIPI_3_part_2} prezentują jakość wyników osiągniętych przez 72 funkcje w 13 różnych badaniach.
Etykiety kolumn odpowiadają numerowi badania, a wiersze numerom funkcji -- tak samo jak w przypadku badań nad wyrażeniami dwuelementowymi.
Informacje o typach badań (sposobie przygotowania korpusu \emph{KIPI}) zostały zamieszczone we wcześniejszej części tego rozdziału w tabeli \ref{KIPI_3_research_types}.

\begin{table}[htp!]
\centering
\footnotesize\setlength{\tabcolsep}{2.5pt}
 \begin{adjustwidth}{-1cm}{}
\begin{tabular}{ l | *{15}{| r}}
	\toprule
	\textbf{99\%} &	\textbf{1}	&	\textbf{2}	&	\textbf{3}	&	\textbf{4}	&	\textbf{5}	&	\textbf{6}	&	\textbf{7}	&	\textbf{8}	&	\textbf{9}	&	\textbf{10}	&	\textbf{11}	&	\textbf{12}	&	\textbf{13}	\\
	\midrule
1	&	0,0019	&	0,0080	&	0,0155	&	0,0078	&	0,0076	&	0,0144	&	0,0089	&	0,0024	&	0,0070	&	0,0027	&	0,0047	&	0,0113	&	0,0049	\\
2	&	0,0001	&	0,0001	&	0,0004	&	0,0001	&	0,0001	&	0,0004	&	0,0001	&	0,0000	&	0,0002	&	0,0000	&	0,0000	&	0,0002	&	0,0000	\\
3	&	0,0023	&	0,0003	&	0,0065	&	0,0003	&	0,0004	&	0,0059	&	0,0004	&	0,0001	&	0,0027	&	0,0001	&	0,0003	&	0,0077	&	0,0003	\\
4	&	0,0062	&	0,0060	&	0,0234	&	0,0063	&	0,0060	&	0,0184	&	0,0080	&	0,0011	&	0,0089	&	0,0013	&	0,0051	&	0,0233	&	0,0058	\\
5	&	0,0062	&	0,0060	&	0,0234	&	0,0062	&	0,0060	&	0,0183	&	0,0080	&	0,0011	&	0,0089	&	0,0013	&	0,0051	&	0,0238	&	0,0058	\\
6	&	0,0026	&	0,0003	&	0,0075	&	0,0003	&	0,0004	&	0,0070	&	0,0004	&	0,0001	&	0,0032	&	0,0001	&	0,0004	&	0,0092	&	0,0004	\\
7	&	0,0033	&	0,0005	&	0,0129	&	0,0005	&	0,0006	&	0,0122	&	0,0007	&	0,0001	&	0,0049	&	0,0001	&	0,0003	&	0,0136	&	0,0003	\\
8	&	0,0035	&	0,0006	&	0,0135	&	0,0006	&	0,0008	&	0,0125	&	0,0009	&	0,0001	&	0,0050	&	0,0001	&	0,0005	&	0,0141	&	0,0005	\\
9	&	0,0014	&	0,0003	&	0,0069	&	0,0003	&	0,0004	&	0,0053	&	0,0004	&	0,0001	&	0,0023	&	0,0001	&	0,0003	&	0,0068	&	0,0003	\\
10	&	0,0014	&	0,0003	&	0,0070	&	0,0003	&	0,0004	&	0,0054	&	0,0004	&	0,0001	&	0,0023	&	0,0001	&	0,0003	&	0,0068	&	0,0003	\\
11	&	0,0057	&	0,0145	&	0,0306	&	0,0139	&	0,0126	&	0,0256	&	0,0148	&	0,0036	&	0,0139	&	0,0042	&	0,0095	&	0,0275	&	0,0101	\\
12	&	0,0034	&	0,0006	&	0,0094	&	0,0006	&	0,0007	&	0,0083	&	0,0008	&	0,0002	&	0,0039	&	0,0002	&	0,0006	&	0,0111	&	0,0006	\\
13	&	0,0038	&	0,0120	&	0,0233	&	0,0118	&	0,0115	&	0,0214	&	0,0131	&	0,0038	&	0,0116	&	0,0042	&	0,0083	&	0,0202	&	0,0087	\\
14	&	0,0057	&	0,0016	&	0,0151	&	0,0016	&	0,0019	&	0,0131	&	0,0022	&	0,0004	&	0,0062	&	0,0004	&	0,0018	&	0,0174	&	0,0019	\\
15	&	0,0093	&	0,0071	&	0,0273	&	0,0075	&	0,0073	&	0,0229	&	0,0096	&	0,0014	&	0,0114	&	0,0016	&	0,0069	&	0,0292	&	0,0079	\\
16	&	0,0021	&	0,0082	&	0,0158	&	0,0080	&	0,0079	&	0,0147	&	0,0092	&	0,0026	&	0,0074	&	0,0028	&	0,0052	&	0,0122	&	0,0054	\\
17	&	0,0057	&	0,0016	&	0,0151	&	0,0016	&	0,0019	&	0,0131	&	0,0022	&	0,0004	&	0,0062	&	0,0004	&	0,0018	&	0,0174	&	0,0019	\\
18	&	0,0055	&	0,0016	&	0,0151	&	0,0016	&	0,0018	&	0,0131	&	0,0022	&	0,0004	&	0,0062	&	0,0004	&	0,0017	&	0,0173	&	0,0018	\\
19	&	0,0055	&	0,0016	&	0,0151	&	0,0016	&	0,0018	&	0,0131	&	0,0022	&	0,0004	&	0,0062	&	0,0004	&	0,0017	&	0,0174	&	0,0018	\\
20	&	0,0002	&	0,0002	&	0,0013	&	0,0002	&	0,0002	&	0,0013	&	0,0002	&	0,0001	&	0,0006	&	0,0001	&	0,0001	&	0,0007	&	0,0001	\\
21	&	0,0096	&	0,0084	&	0,0292	&	0,0088	&	0,0085	&	0,0247	&	0,0109	&	0,0016	&	0,0121	&	0,0019	&	0,0078	&	0,0309	&	0,0089	\\
22	&	0,0099	&	0,0098	&	0,0310	&	0,0103	&	0,0097	&	0,0259	&	0,0124	&	0,0019	&	0,0129	&	0,0022	&	0,0088	&	0,0323	&	0,0099	\\
23	&	0,0101	&	0,0113	&	0,0329	&	0,0117	&	0,0109	&	0,0277	&	0,0137	&	0,0021	&	0,0137	&	0,0025	&	0,0096	&	0,0337	&	0,0110	\\
24	&	0,0103	&	0,0126	&	0,0341	&	0,0131	&	0,0120	&	0,0287	&	0,0149	&	0,0024	&	0,0145	&	0,0029	&	0,0105	&	0,0350	&	0,0119	\\
25	&	\textbf{0,0104}	&	0,0140	&	0,0351	&	0,0144	&	0,0130	&	0,0294	&	0,0160	&	0,0027	&	0,0152	&	0,0033	&	0,0113	&	0,0359	&	0,0127	\\
26	&	\textbf{0,0105}	&	0,0152	&	0,0359	&	0,0155	&	0,0139	&	0,0301	&	0,0169	&	0,0030	&	0,0159	&	0,0036	&	0,0119	&	0,0368	&	0,0133	\\
27	&	\textbf{0,0105}	&	0,0163	&	0,0366	&	0,0164	&	0,0147	&	0,0305	&	0,0177	&	0,0033	&	0,0164	&	0,0040	&	0,0125	&	0,0376	&	0,0139	\\
28	&	\textbf{0,0105}	&	0,0171	&	0,0372	&	0,0171	&	0,0153	&	0,0309	&	0,0182	&	0,0036	&	0,0169	&	0,0043	&	0,0129	&	\textbf{0,0380}	&	0,0142	\\
29	&	\textbf{0,0104}	&	0,0177	&	0,0377	&	0,0177	&	0,0158	&	0,0313	&	0,0187	&	0,0039	&	0,0173	&	0,0046	&	0,0133	&	\textbf{0,0380}	&	0,0145	\\
30	&	0,0103	&	0,0183	&	0,0381	&	0,0181	&	0,0162	&	0,0317	&	0,0191	&	0,0041	&	0,0176	&	0,0049	&	0,0135	&	\textbf{0,0382}	&	0,0147	\\
31	&	0,0102	&	0,0187	&	\textbf{0,0384}	&	0,0184	&	0,0165	&	\textbf{0,0320}	&	0,0194	&	0,0043	&	0,0178	&	0,0051	&	0,0137	&	\textbf{0,0383}	&	\textbf{0,0148}	\\
32	&	0,0100	&	0,0190	&	\textbf{0,0386}	&	\textbf{0,0186}	&	0,0168	&	\textbf{0,0322}	&	\textbf{0,0196}	&	0,0045	&	0,0181	&	0,0053	&	\textbf{0,0138}	&	\textbf{0,0381}	&	\textbf{0,0149}	\\
33	&	0,0099	&	\textbf{0,0192}	&	\textbf{0,0385}	&	\textbf{0,0188}	&	\textbf{0,0170}	&	\textbf{0,0322}	&	\textbf{0,0197}	&	0,0047	&	0,0183	&	0,0054	&	\textbf{0,0138}	&	\textbf{0,0379}	&	\textbf{0,0149}	\\
34	&	0,0097	&	\textbf{0,0193}	&	\textbf{0,0384}	&	\textbf{0,0188}	&	\textbf{0,0170}	&	\textbf{0,0322}	&	\textbf{0,0197}	&	0,0048	&	\textbf{0,0184}	&	0,0056	&	\textbf{0,0138}	&	0,0378	&	\textbf{0,0148}	\\
35	&	0,0095	&	\textbf{0,0193}	&	\textbf{0,0382}	&	\textbf{0,0188}	&	\textbf{0,0171}	&	\textbf{0,0321}	&	\textbf{0,0197}	&	0,0049	&	\textbf{0,0185}	&	0,0057	&	\textbf{0,0138}	&	0,0375	&	\textbf{0,0148}	\\
36	&	0,0093	&	\textbf{0,0193}	&	0,0380	&	\textbf{0,0188}	&	\textbf{0,0170}	&	\textbf{0,0321}	&	\textbf{0,0196}	&	0,0050	&	\textbf{0,0185}	&	0,0057	&	\textbf{0,0137}	&	0,0373	&	0,0147	\\
	\bottomrule
\end{tabular}
 \end{adjustwidth}
\caption[Wyniki badań miar trójelementowych dla korpusu \emph{KIPI}, część 1]{Wyniki badań miar trójelementowych dla korpusu \emph{KIPI}, część 1.}
\label{KIPI_3_part_1}
\end{table}

\begin{table}[htp!]
\centering
\footnotesize\setlength{\tabcolsep}{2.5pt}
 \begin{adjustwidth}{-1cm}{}
\begin{tabular}{ l | *{15}{| r}}
	\toprule
	\textbf{99\%} &	\textbf{1}	&	\textbf{2}	&	\textbf{3}	&	\textbf{4}	&	\textbf{5}	&	\textbf{6}	&	\textbf{7}	&	\textbf{8}	&	\textbf{9}	&	\textbf{10}	&	\textbf{11}	&	\textbf{12}	&	\textbf{13}	\\
	\midrule 
37	&	0,0092	&	\textbf{0,0193}	&	0,0378	&	\textbf{0,0187}	&	\textbf{0,0170}	&	0,0319	&	\textbf{0,0195}	&	0,0050	&	\textbf{0,0185}	&	\textbf{0,0058}	&	0,0137	&	0,0369	&	0,0146	\\
38	&	0,0090	&	\textbf{0,0192}	&	0,0375	&	0,0186	&	\textbf{0,0170}	&	0,0317	&	0,0195	&	0,0051	&	\textbf{0,0185}	&	\textbf{0,0058}	&	0,0135	&	0,0365	&	0,0144	\\
39	&	0,0088	&	0,0191	&	0,0371	&	0,0185	&	\textbf{0,0169}	&	0,0314	&	0,0194	&	\textbf{0,0051}	&	\textbf{0,0184}	&	\textbf{0,0058}	&	0,0134	&	0,0360	&	0,0143	\\
40	&	0,0086	&	0,0190	&	0,0368	&	0,0183	&	0,0168	&	0,0313	&	0,0192	&	\textbf{0,0051}	&	0,0183	&	\textbf{0,0058}	&	0,0133	&	0,0355	&	0,0141	\\
41	&	0,0084	&	0,0188	&	0,0364	&	0,0182	&	0,0167	&	0,0311	&	0,0191	&	\textbf{0,0051}	&	0,0182	&	\textbf{0,0058}	&	0,0132	&	0,0351	&	0,0140	\\
42	&	0,0082	&	0,0186	&	0,0360	&	0,0181	&	0,0166	&	0,0309	&	0,0189	&	\textbf{0,0051}	&	0,0181	&	\textbf{0,0058}	&	0,0130	&	0,0347	&	0,0138	\\
43	&	0,0080	&	0,0185	&	0,0356	&	0,0179	&	0,0164	&	0,0307	&	0,0188	&	\textbf{0,0051}	&	0,0179	&	\textbf{0,0058}	&	0,0129	&	0,0342	&	0,0137	\\
44	&	0,0079	&	0,0183	&	0,0353	&	0,0177	&	0,0163	&	0,0304	&	0,0186	&	\textbf{0,0051}	&	0,0178	&	\textbf{0,0058}	&	0,0128	&	0,0338	&	0,0135	\\
45	&	0,0077	&	0,0182	&	0,0349	&	0,0176	&	0,0162	&	0,0301	&	0,0185	&	\textbf{0,0051}	&	0,0177	&	0,0058	&	0,0126	&	0,0333	&	0,0133	\\
46	&	0,0075	&	0,0180	&	0,0345	&	0,0174	&	0,0161	&	0,0299	&	0,0183	&	0,0051	&	0,0175	&	0,0057	&	0,0125	&	0,0329	&	0,0132	\\
47	&	0,0074	&	0,0178	&	0,0341	&	0,0172	&	0,0159	&	0,0297	&	0,0182	&	0,0050	&	0,0174	&	0,0057	&	0,0124	&	0,0324	&	0,0130	\\
48	&	0,0072	&	0,0176	&	0,0338	&	0,0171	&	0,0158	&	0,0294	&	0,0180	&	0,0050	&	0,0172	&	0,0057	&	0,0122	&	0,0320	&	0,0129	\\
49	&	0,0071	&	0,0175	&	0,0334	&	0,0169	&	0,0157	&	0,0292	&	0,0179	&	0,0050	&	0,0170	&	0,0056	&	0,0121	&	0,0316	&	0,0127	\\
50	&	0,0069	&	0,0173	&	0,0331	&	0,0168	&	0,0156	&	0,0289	&	0,0177	&	0,0050	&	0,0169	&	0,0056	&	0,0119	&	0,0312	&	0,0126	\\
51	&	0,0040	&	0,0123	&	0,0237	&	0,0120	&	0,0117	&	0,0217	&	0,0133	&	0,0039	&	0,0119	&	0,0043	&	0,0085	&	0,0207	&	0,0090	\\
52	&	0,0042	&	0,0125	&	0,0241	&	0,0122	&	0,0119	&	0,0221	&	0,0136	&	0,0040	&	0,0122	&	0,0044	&	0,0088	&	0,0213	&	0,0092	\\
53	&	0,0044	&	0,0128	&	0,0246	&	0,0125	&	0,0122	&	0,0225	&	0,0139	&	0,0040	&	0,0125	&	0,0045	&	0,0091	&	0,0219	&	0,0095	\\
54	&	0,0046	&	0,0131	&	0,0251	&	0,0128	&	0,0124	&	0,0230	&	0,0142	&	0,0041	&	0,0128	&	0,0046	&	0,0094	&	0,0226	&	0,0098	\\
55	&	0,0049	&	0,0134	&	0,0256	&	0,0132	&	0,0127	&	0,0235	&	0,0145	&	0,0042	&	0,0131	&	0,0047	&	0,0097	&	0,0233	&	0,0101	\\
56	&	0,0051	&	0,0137	&	0,0262	&	0,0135	&	0,0130	&	0,0239	&	0,0148	&	0,0043	&	0,0135	&	0,0048	&	0,0100	&	0,0241	&	0,0105	\\
57	&	0,0054	&	0,0141	&	0,0268	&	0,0138	&	0,0133	&	0,0245	&	0,0152	&	0,0043	&	0,0139	&	0,0049	&	0,0103	&	0,0249	&	0,0108	\\
58	&	0,0057	&	0,0145	&	0,0274	&	0,0142	&	0,0136	&	0,0251	&	0,0155	&	0,0044	&	0,0143	&	0,0049	&	0,0106	&	0,0257	&	0,0111	\\
59	&	0,0060	&	0,0149	&	0,0282	&	0,0146	&	0,0139	&	0,0257	&	0,0159	&	0,0045	&	0,0147	&	0,0050	&	0,0108	&	0,0267	&	0,0114	\\
60	&	0,0064	&	0,0152	&	0,0289	&	0,0149	&	0,0142	&	0,0263	&	0,0162	&	0,0045	&	0,0152	&	0,0051	&	0,0110	&	0,0276	&	0,0117	\\
61	&	0,0067	&	0,0155	&	0,0297	&	0,0152	&	0,0143	&	0,0269	&	0,0164	&	0,0045	&	0,0156	&	0,0051	&	0,0112	&	0,0286	&	0,0119	\\
62	&	0,0070	&	0,0159	&	0,0304	&	0,0155	&	0,0145	&	0,0278	&	0,0166	&	0,0045	&	0,0160	&	0,0051	&	0,0115	&	0,0296	&	0,0121	\\
63	&	0,0073	&	0,0162	&	0,0313	&	0,0158	&	0,0147	&	0,0285	&	0,0168	&	0,0045	&	0,0163	&	0,0051	&	0,0115	&	0,0305	&	0,0121	\\
64	&	0,0075	&	0,0165	&	0,0322	&	0,0161	&	0,0148	&	0,0291	&	0,0170	&	0,0043	&	0,0165	&	0,0050	&	0,0114	&	0,0311	&	0,0121	\\
65	&	0,0077	&	0,0167	&	0,0331	&	0,0163	&	0,0148	&	0,0297	&	0,0169	&	0,0042	&	0,0166	&	0,0048	&	0,0112	&	0,0318	&	0,0120	\\
66	&	0,0077	&	0,0166	&	0,0336	&	0,0162	&	0,0144	&	0,0299	&	0,0166	&	0,0038	&	0,0164	&	0,0044	&	0,0105	&	0,0320	&	0,0112	\\
67	&	0,0076	&	0,0164	&	0,0340	&	0,0159	&	0,0137	&	0,0297	&	0,0159	&	0,0034	&	0,0160	&	0,0039	&	0,0097	&	0,0315	&	0,0104	\\
68	&	0,0074	&	0,0160	&	0,0340	&	0,0155	&	0,0129	&	0,0291	&	0,0148	&	0,0030	&	0,0155	&	0,0035	&	0,0089	&	0,0309	&	0,0096	\\
69	&	0,0069	&	0,0147	&	0,0334	&	0,0142	&	0,0114	&	0,0280	&	0,0131	&	0,0024	&	0,0148	&	0,0029	&	0,0077	&	0,0298	&	0,0084	\\
70	&	0,0065	&	0,0135	&	0,0324	&	0,0132	&	0,0100	&	0,0267	&	0,0118	&	0,0020	&	0,0137	&	0,0024	&	0,0070	&	0,0283	&	0,0076	\\
71	&	0,0020	&	0,0044	&	0,0075	&	0,0044	&	0,0040	&	0,0063	&	0,0044	&	0,0021	&	0,0027	&	0,0022	&	0,0052	&	0,0107	&	0,0053	\\
72	&	0,0020	&	0,0045	&	0,0076	&	0,0045	&	0,0040	&	0,0065	&	0,0045	&	0,0021	&	0,0028	&	0,0023	&	0,0053	&	0,0110	&	0,0054	\\
	\bottomrule
\end{tabular}
 \end{adjustwidth}
\caption[Wyniki badań miar trójelementowych dla korpusu \emph{KIPI}, część 2]{Wyniki badań miar trójelementowych dla korpusu \emph{KIPI}, część 2.}
\label{KIPI_3_part_2}
\end{table}

Tabela \ref{KIPI_3_part_3} zawiera natomiast wyniki dla czterech funkcji, których badanie ma sens jedynie dla wyrażeń przynajmniej trójelementowych.

\begin{table}[htp!]
\centering
\footnotesize\setlength{\tabcolsep}{2.5pt}
 \begin{adjustwidth}{-1cm}{}
\begin{tabular}{ l | *{15}{| r}}
	\toprule
	\textbf{95\%} &	\textbf{1}	&	\textbf{2}	&	\textbf{3}	&	\textbf{4}	&	\textbf{5}	&	\textbf{6}	&	\textbf{7}	&	\textbf{8}	&	\textbf{9}	&	\textbf{10}	&	\textbf{11}	&	\textbf{12}	&	\textbf{13}	\\
	\midrule 
73	&	\textbf{0,0071}	&	\textbf{0,0160}	&	\textbf{0,0295}	&	\textbf{0,0158}	&	\textbf{0,0201}	&	\textbf{0,0317}	&	\textbf{0,0179}	&	\textbf{0,0056}	&	\textbf{0,0175}	&	\textbf{0,0067}	&	\textbf{0,0114}	&	\textbf{0,0290}	&	\textbf{0,0138} \\
74	&	0,0004	&	0,0006	&	0,0032	&	0,0006	&	0,0007	&	0,0031	&	0,0006	&	0,0002	&	0,0012	&	0,0003	&	0,0002	&	0,0012	&	0,0002 \\
75	&	0,0032	&	0,0122	&	0,0227	&	0,0119	&	0,0159	&	0,0246	&	0,0139	&	0,0038	&	0,0120	&	0,0045	&	0,0090	&	0,0227	&	0,0110 \\
76	&	0,0005	&	0,0009	&	0,0056	&	0,0009	&	0,0011	&	0,0052	&	0,0009	&	0,0003	&	0,0018	&	0,0003	&	0,0003	&	0,0018	&	0,0003 \\
	\bottomrule
\end{tabular}
 \end{adjustwidth}
\caption[Wyniki badań miar trójelementowych dla korpusu \emph{KIPI}, część 3]{Wyniki badań miar trójelementowych dla korpusu \emph{KIPI}, część 3.}
\label{KIPI_3_part_3}
\end{table}

Wyniki jasno obrazują, że najlepszą miarą (przynajmniej dla przeprowadzonych i opisanych tutaj badań) spośród grona wszystkich kandydatów jest \emph{Specific Exponential Correlation} z dokładnością do jej parametru -- wykładnika.
Najlepszy globalnie wynik został osiągnięty dla badania numer trzy i wartości wykładnika równej 4,2, natomiast dla parametru w okolicach wartości 4,6 funkcja wydaje się być najlepsza w ogólności -- dostarcza dobrych jakościowo wyników dla wszystkich rodzajów przeprowadzonych badań.
Trzeba jednak mieć na uwadze, że część wyników osiągniętych przez miarę \emph{Fair Dispersion Point Normalization} okazała się lepsza niż najlepsze z wyników poprzednich 72 miar, ale jednak najlepszy wynik globalnie został osiągnięty przez \emph{SEC}.
Ciekawą obserwacją jest, że \emph{FDPN} okazała się być najlepszą z czterech miar heurystycznych tutaj zastosowanych -- w każdym z badań.
Interesujące może być to, że najlepszy z wyników został osiągnięty w badaniu, które nie stosuje najmocniejszych filtrów, a jego zestaw relacji nie został ograniczony tylko do relacji częstych.
Taka anomalia może jednak być spowodowana specyfiką niniejszego zbioru danych, a raczej małym zbiorem testowym wyrażeń wielowyrazowych długości trzech elementów.
Innym wnioskiem może być gorszy sposób doboru relacji częstych niż w przypadku kolokacji dwuelementowych ze względu na niskie jakościowo wyniki podzbiorów wykorzystujących ten zestaw relacji.


\subsubsection{Wyniki badań jakości rozwiązań generowanych przez sieci neuronowe}
Tabela \ref{KIPI_3_nn} prezentuje jakość wyników osiągniętych przez perceptron wielowarstwowy w 48 wariantach w 13 różnych badań.
Etykiety kolumn odpowiadają numerom badania a wiersze numerom wariantów sieci neuronowych.
Typy badań i ich indeksy są takie same, jak w badaniu poprzednim.

\begin{table}[htp!]
\centering
\footnotesize\setlength{\tabcolsep}{2.5pt}
 \begin{adjustwidth}{-1cm}{}
\begin{tabular}{ l | *{15}{| r}}
	\toprule
	\textbf{95\%} &	\textbf{1}	&	\textbf{2}	&	\textbf{3}	&	\textbf{4}	&	\textbf{5}	&	\textbf{6}	&	\textbf{7}	&	\textbf{8}	&	\textbf{9}	&	\textbf{10}	&	\textbf{11}	&	\textbf{12}	&	\textbf{13}	\\
	\midrule
1	&	0,0054	&	0,0073	&	0,0124	&	0,0067	&	0,0058	&	0,0295	&	0,0064	&	\textbf{0,0032}	&	0,0069	&	0,0004	&	0,0005	&	0,0203	&	\textbf{0,0077}	\\
2	&	0,0029	&	0,0063	&	0,0245	&	0,0019	&	0,0009	&	0,0253	&	0,0111	&	0,0003	&	0,0083	&	0,0005	&	0,0005	&	0,0274	&	0,0007	\\
3	&	0,0033	&	0,0022	&	0,0196	&	0,0071	&	0,0073	&	0,0094	&	0,0034	&	0,0005	&	0,0102	&	0,0031	&	0,0002	&	0,0267	&	0,0028	\\
4	&	0,0035	&	0,0060	&	0,0212	&	0,0045	&	0,0003	&	0,0151	&	0,0118	&	0,0011	&	0,0122	&	0,0029	&	0,0037	&	0,0232	&	0,0018	\\
5	&	0,0037	&	0,0032	&	0,0184	&	0,0034	&	0,0096	&	0,0181	&	0,0029	&	0,0029	&	0,0070	&	0,0003	&	0,0045	&	0,0147	&	0,0035	\\
6	&	0,0028	&	0,0093	&	0,0197	&	0,0018	&	0,0076	&	0,0225	&	0,0064	&	0,0008	&	0,0091	&	0,0031	&	0,0033	&	0,0224	&	0,0058	\\
7	&	0,0037	&	0,0028	&	0,0077	&	0,0095	&	0,0084	&	0,0279	&	0,0029	&	0,0027	&	0,0103	&	0,0019	&	0,0036	&	0,0260	&	0,0065	\\
8	&	\textbf{0,0062}	&	0,0086	&	0,0236	&	0,0097	&	0,0056	&	0,0195	&	0,0089	&	0,0013	&	0,0068	&	0,0002	&	0,0055	&	0,0170	&	0,0041	\\
9	&	0,0024	&	0,0032	&	0,0248	&	0,0062	&	0,0073	&	0,0164	&	0,0111	&	0,0022	&	0,0060	&	0,0025	&	0,0052	&	0,0208	&	0,0050	\\
10	&	\textbf{0,0059}	&	0,0028	&	0,0246	&	0,0075	&	0,0081	&	0,0175	&	0,0089	&	0,0021	&	0,0098	&	0,0031	&	0,0062	&	0,0139	&	0,0037	\\
11	&	0,0040	&	0,0088	&	0,0227	&	0,0082	&	0,0078	&	0,0182	&	0,0065	&	0,0013	&	0,0117	&	0,0024	&	0,0058	&	0,0206	&	0,0036	\\
12	&	0,0038	&	0,0063	&	\textbf{0,0286}	&	0,0054	&	0,0075	&	0,0228	&	0,0130	&	0,0012	&	0,0078	&	0,0024	&	\textbf{0,0065}	&	0,0154	&	0,0063	\\
13	&	0,0038	&	0,0010	&	0,0240	&	0,0035	&	0,0080	&	0,0233	&	0,0078	&	0,0004	&	0,0040	&	0,0005	&	0,0045	&	0,0225	&	0,0041	\\
14	&	0,0040	&	0,0047	&	0,0093	&	0,0076	&	0,0027	&	0,0052	&	0,0026	&	0,0003	&	0,0078	&	0,0016	&	\textbf{0,0062}	&	0,0071	&	0,0024	\\
15	&	0,0051	&	0,0040	&	0,0262	&	0,0084	&	0,0020	&	0,0123	&	0,0094	&	0,0011	&	0,0059	&	0,0015	&	0,0030	&	0,0192	&	0,0034	\\
16	&	0,0011	&	0,0014	&	0,0143	&	0,0034	&	0,0004	&	0,0184	&	0,0053	&	0,0014	&	0,0050	&	0,0030	&	0,0055	&	0,0178	&	0,0002	\\
17	&	0,0022	&	0,0047	&	0,0173	&	\textbf{0,0124}	&	0,0041	&	0,0259	&	0,0018	&	0,0005	&	0,0060	&	0,0004	&	0,0059	&	0,0223	&	0,0030	\\
18	&	0,0028	&	0,0057	&	0,0143	&	0,0080	&	0,0087	&	0,0195	&	0,0090	&	0,0003	&	0,0057	&	0,0032	&	0,0044	&	0,0163	&	0,0029	\\
19	&	0,0032	&	0,0017	&	0,0185	&	0,0043	&	0,0107	&	0,0298	&	0,0007	&	0,0004	&	0,0084	&	0,0011	&	0,0036	&	0,0161	&	0,0012	\\
20	&	0,0014	&	0,0076	&	0,0216	&	0,0032	&	0,0025	&	0,0266	&	0,0090	&	0,0016	&	0,0104	&	0,0007	&	0,0040	&	0,0081	&	0,0009	\\
21	&	0,0021	&	0,0018	&	0,0238	&	0,0115	&	0,0079	&	0,0060	&	0,0047	&	0,0002	&	0,0029	&	0,0013	&	0,0046	&	0,0110	&	\textbf{0,0075}	\\
22	&	0,0033	&	0,0095	&	0,0252	&	0,0065	&	0,0075	&	0,0148	&	0,0052	&	0,0006	&	0,0102	&	0,0024	&	0,0055	&	0,0144	&	0,0048	\\
23	&	0,0034	&	\textbf{0,0114}	&	0,0213	&	0,0060	&	0,0022	&	0,0115	&	0,0030	&	0,0003	&	0,0087	&	0,0009	&	0,0052	&	0,0107	&	0,0031	\\
24	&	0,0033	&	0,0081	&	0,0167	&	0,0064	&	0,0083	&	0,0165	&	0,0052	&	0,0012	&	0,0076	&	0,0016	&	0,0030	&	0,0119	&	0,0036	\\
25	&	0,0026	&	0,0105	&	0,0216	&	0,0075	&	0,0062	&	0,0243	&	0,0020	&	0,0023	&	0,0088	&	0,0035	&	0,0050	&	0,0228	&	0,0053	\\
26	&	0,0041	&	0,0092	&	0,0182	&	0,0025	&	0,0062	&	0,0166	&	0,0090	&	0,0018	&	0,0078	&	0,0020	&	0,0030	&	0,0271	&	0,0053	\\
27	&	0,0030	&	0,0103	&	0,0228	&	0,0048	&	0,0100	&	\textbf{0,0325}	&	0,0012	&	0,0011	&	0,0046	&	0,0021	&	0,0060	&	0,0036	&	0,0063	\\
28	&	0,0037	&	0,0064	&	0,0199	&	0,0072	&	0,0097	&	0,0274	&	0,0062	&	0,0012	&	0,0116	&	0,0034	&	\textbf{0,0063}	&	0,0212	&	0,0059	\\
29	&	0,0035	&	0,0032	&	0,0217	&	0,0048	&	0,0098	&	0,0211	&	0,0057	&	0,0019	&	0,0082	&	0,0029	&	0,0046	&	0,0267	&	0,0021	\\
30	&	0,0037	&	0,0098	&	0,0167	&	0,0080	&	0,0089	&	0,0217	&	0,0126	&	0,0008	&	0,0108	&	0,0034	&	\textbf{0,0063}	&	0,0167	&	0,0046	\\
31	&	0,0028	&	0,0108	&	0,0173	&	0,0076	&	0,0089	&	\textbf{0,0337}	&	0,0061	&	0,0024	&	0,0119	&	0,0023	&	0,0019	&	0,0132	&	0,0050	\\
32	&	0,0046	&	0,0104	&	0,0139	&	0,0066	&	0,0096	&	0,0309	&	0,0072	&	0,0015	&	0,0084	&	0,0022	&	0,0058	&	0,0234	&	0,0041	\\
33	&	0,0030	&	0,0090	&	0,0132	&	0,0064	&	0,0080	&	0,0236	&	0,0109	&	0,0011	&	0,0084	&	0,0028	&	0,0052	&	0,0205	&	0,0050	\\
34	&	0,0051	&	0,0070	&	0,0249	&	0,0089	&	0,0084	&	0,0218	&	0,0083	&	0,0012	&	0,0077	&	\textbf{0,0035}	&	0,0049	&	0,0214	&	0,0046	\\
35	&	0,0053	&	0,0097	&	0,0269	&	0,0075	&	\textbf{0,0117}	&	0,0191	&	0,0095	&	0,0012	&	0,0090	&	0,0023	&	0,0038	&	0,0161	&	0,0038	\\
36	&	0,0021	&	0,0077	&	0,0250	&	0,0029	&	\textbf{0,0114}	&	0,0146	&	0,0093	&	0,0014	&	0,0101	&	0,0018	&	0,0025	&	0,0246	&	0,0021	\\
37	&	0,0028	&	0,0028	&	0,0182	&	0,0032	&	0,0089	&	0,0218	&	0,0006	&	0,0012	&	0,0104	&	\textbf{0,0037}	&	0,0060	&	0,0095	&	0,0048	\\
38	&	0,0023	&	0,0007	&	0,0145	&	0,0092	&	0,0017	&	0,0171	&	0,0024	&	0,0005	&	0,0090	&	0,0030	&	0,0045	&	\textbf{0,0309}	&	0,0046	\\
39	&	0,0048	&	0,0054	&	0,0228	&	0,0047	&	0,0102	&	0,0219	&	0,0102	&	0,0001	&	0,0070	&	0,0034	&	0,0052	&	0,0179	&	0,0038	\\
40	&	0,0051	&	0,0060	&	0,0218	&	0,0065	&	0,0060	&	0,0089	&	0,0068	&	\textbf{0,0031}	&	0,0097	&	0,0025	&	0,0023	&	0,0137	&	0,0047	\\
41	&	0,0039	&	0,0094	&	0,0206	&	0,0034	&	0,0080	&	0,0225	&	0,0027	&	0,0020	&	0,0057	&	0,0034	&	0,0039	&	0,0103	&	0,0029	\\
42	&	0,0023	&	0,0079	&	0,0244	&	0,0083	&	0,0105	&	0,0317	&	0,0088	&	0,0008	&	0,0061	&	0,0025	&	0,0061	&	0,0144	&	0,0038	\\
43	&	0,0026	&	0,0067	&	0,0232	&	0,0077	&	0,0080	&	0,0228	&	\textbf{0,0142}	&	0,0006	&	0,0092	&	0,0021	&	0,0049	&	0,0044	&	0,0033	\\
44	&	0,0033	&	0,0071	&	0,0116	&	0,0071	&	0,0049	&	0,0108	&	0,0109	&	0,0009	&	0,0105	&	0,0028	&	0,0033	&	0,0189	&	0,0021	\\
45	&	0,0039	&	0,0060	&	0,0244	&	0,0024	&	\textbf{0,0120}	&	0,0131	&	0,0073	&	0,0009	&	0,0112	&	\textbf{0,0035}	&	0,0030	&	0,0163	&	0,0030	\\
46	&	0,0021	&	0,0049	&	0,0245	&	0,0013	&	0,0077	&	0,0123	&	0,0056	&	0,0016	&	0,0071	&	0,0033	&	0,0019	&	0,0175	&	0,0032	\\
47	&	0,0039	&	0,0080	&	0,0159	&	0,0046	&	0,0089	&	0,0247	&	0,0095	&	0,0029	&	\textbf{0,0139}	&	0,0020	&	0,0043	&	0,0167	&	0,0021	\\
48	&	0,0028	&	0,0092	&	0,0124	&	0,0062	&	0,0082	&	0,0158	&	0,0062	&	0,0026	&	0,0045	&	0,0032	&	0,0059	&	0,0145	&	0,0043	\\
	\bottomrule
\end{tabular}
 \end{adjustwidth}
\caption[Wyniki badań ekstrakcji kolokacji trójelementowych za pomocą sieci neuronowych dla korpusu \emph{KIPI}]{Wyniki badań ekstrakcji kolokacji trójelementowych za pomocą sieci neuronowych dla korpusu \emph{KIPI}.}
\label{KIPI_3_nn}
\end{table} 

Tabela \ref{KIPI_3_nn_vs_measures} prezentuje wyniki porównawcze badań miar asocjacyjnych i sieci neuronowych.
Pierwszy wiesz to wyniki najlepszej z miar dla każdego z badań, drugi najlepszej z sieci, a trzeci to stosunek wyniku najlepszej z sieci do wyniku najlepszej z miar.

\begin{table}[htp!]
\centering
\footnotesize\setlength{\tabcolsep}{2.5pt}
 \begin{adjustwidth}{-1cm}{}
\begin{tabular}{ l | *{15}{| r}}
	\toprule
	\textbf{95\%} &	\textbf{1}	&	\textbf{2}	&	\textbf{3}	&	\textbf{4}	&	\textbf{5}	&	\textbf{6}	&	\textbf{7}	&	\textbf{8}	&	\textbf{9}	&	\textbf{10}	&	\textbf{11}	&	\textbf{12}	&	\textbf{13}	\\
	\midrule
m	&	0,0105	&	0,0193	&	0,0386	&	0,0188	&	0,0171	&	0,0322	&	0,0197	&	0,0051	&	0,0185	&	0,0058	&	0,0138	&	0,0383	&	0,0149	\\
s	&	0,0062	&	0,0114	&	0,0286	&	0,0124	&	0,0120	&	0,0337	&	0,0142	&	0,0032	&	0,0139	&	0,0037	&	0,0065	&	0,0309	&	0,0077	\\
\hline																											
s/m	&	0,5905	&	0,5907	&	0,7409	&	0,6596	&	0,7018	&	\textbf{1,0466}	&	0,7208	&	0,6275	&	0,7514	&	0,6379	&	0,4710	&	0,8068	&	0,5168	\\
	\bottomrule
\end{tabular}
 \end{adjustwidth}
\caption[Wyniki badań ekstrakcji kolokacji trójelementowych za pomocą sieci neuronowych dla korpusu \emph{KIPI}]{Wyniki badań ekstrakcji kolokacji trójelementowych za pomocą sieci neuronowych dla korpusu \emph{KIPI}.}
\label{KIPI_3_nn_vs_measures}
\end{table} 

Wyniki pokazują, że jedynie w jednym z badań sieć osiągnęła lepszy wynik niż miary.
Powodem tego może być bardzo duże niezrównoważenie klas oraz niedopracowany i mały zbiór testowy, a także zwiększony poziom trudności problemu.


\section{Podejście miar mieszanych}
Podejście miar mieszanych polegało na wykorzystaniu kombinacji liniowej rankingów wygenerowanych przez pewien zestaw poszczególnych funkcji asocjacyjnych.

\subsection{Optymalizacja wag dla rankingów w modelu kombinacji liniowej}
Narzędziem wykorzystanym w optymalizacji był opisany we wcześniejszej części pracy algorytm ewolucyjny \footnote{por. rozdz. 4.6.10.}.
Jako zbiór uczący wybrano korpus \emph{KIPI}, a testowym był cały zbiór \emph{KGR7}.
Oba wykorzystane zbiory danych były przygotowane do tego zadania w inny sposób niż opisano w części pracy traktującej o wykorzystanych zbiorach danych -- nie były one tagowane od nowa, a jedynie zastosowana była konwersja tagsetów z \emph{KIPI} do \emph{NKJP}.
Zauważyć jednak należy, że zarówno zbiór testowy jak i uczący były przygotowane w taki sam sposób.

\par
Proces optymalizacji wag był wykonany dwuetapowo.
Pierwszym krokiem było dostrojenie parametrów algorytmu genetycznego, a drugim optymalizacja wag dla rankingów.
Parametry, które autor pracy optymalizował, to: szansa krzyżowania, rozmiar turnieju -- dotyczy operatora selekcji, szansa na mutację genotypu osobnika.
Zbadane zostały wartości parametrów przedstawione w tabeli \ref{optimizer_parameters_optimization}.
\begin{table}[h!]
\centering
\begin{tabular}{ l | l }
	\toprule
	\textbf{parametr} & \textbf{wartości} \\
	\midrule
	rozmiar turnieju & [2:5], skok co 1\\
	szansa krzyżowania & [0.4:0.8], skok co 0.05\\
	szansa mutacji & [0.03:0.11], skok co 0.01\\ 
	\bottomrule
\end{tabular}
\caption[Sprawdzone wartości parametrów algorytmu genetycznego]{Sprawdzone wartości parametrów algorytmu genetycznego.}
\label{optimizer_parameters_optimization}
\end{table}

Optymalizacja parametrów algorytmu została wykonana na korpusie uczącym przy progu odcięcia na poziomie dziesięciotysięcznej pozycji rankingu.
Każda wartość parametru była dostrajana przez 50 iteracji dla rozmiaru populacji równego 25.
Konkretny parametr był dostrajany przy utrzymaniu stałych wartości innych parametrów przez cały proces jego optymalizacji.
Ręczna ocena wyników polegała na sprawdzeniu, dla których parametrów wzrost jakości rozwiązani jest najszybszy, ale przy jednoczesnym stosunkowo powolnym zbieganiu się najlepszego, średniego i najgorszego z rozwiązań.
Ostatecznie wybrany został zestaw parametrów zaprezentowany poniżej:

$$ rozmiar \: turnieju = 5, szansa \: krzyzowania = 0.75, szansa \: mutacji = 0.05 $$

\par
Po ustaleniu parametrów dla algorytmu ewolucyjnego wykonany został drugi krok -- optymalizacja wag dla rankingów.
Polegała ona na wybraniu zestawu miar, które dawały dobre wyniki w ekstrakcji miar dwuelementowych, a następnie wykorzystaniu ich w procesie tworzenia rankingów do kombinacji liniowej.
Dobrane zostały następujące miar:
\begin{enumerate}
	\item \emph{Loglikelihood},
	\item \emph{Mutual Expectation},
	\item \emph{Specific Frequency Biased Mutual Dependency},
	\item \emph{Jaccard},
	\item \emph{W Specific Correlation},
	\item \emph{W Specific Exponential Correlation} z parameterami 1.35, 1.375 oraz 1.4 stosowanymi zamiennie.
\end{enumerate}

\par
Algorytm genetyczny dostrajał wagi przeszukując przestrzeń rozwiązań, wybierając jakiś zestaw wag i sprawdzając rozwiązanie.
Zastosowany agregator to \emph{maksymalna suma}, a funkcje normalizujące to \emph{Borda score} oraz \emph{Zipf's Borda score}.
Użyta miara oceny to \emph{Average precision on hit} przy długości rankingu równej 10000.
Liczba osobników w populacji to 25, a liczba iteracji była równa przynajmniej 100.

\par
Podjętych zostało wiele prób optymalizacji wag dla dwóch zakresów możliwych wag dla każdego z rankingu -- od 0 do 1 oraz od -1 do 1, dwóch różnych sposobów punktowania rankingów oraz dla zmiennej wartości parametru funkcji \emph{W Specific Exponential Correlation}.


\subsection{Wyniki i obserwacje po zastosowaniu optymalizacji}
Badania wykazały bardzo słaby wynik dla zbioru testowego, w uczącym wynik był poprawiony o około 8\%, ale w przypadku testowego uplasował się na pozycji około 20-25\% gorszym niż najlepsze funkcje.
Powodem takie wyniku jest najprawdopodobniej zły jakościowo oraz nie do końca przemyślany sposób przeprowadzenia procesu optymalizacji parameterów i testów po wykonaniu optymalizacji.
Istotnym mógł być też źle dobrany sposób sprawdzania jakości wyniku po krokach optymalizacji algorytmem genetycznym -- funkcja oceny.