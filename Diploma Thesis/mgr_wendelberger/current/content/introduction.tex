
\chapter{Wstęp}

\section{Cel pracy}
Celem niniejszej pracy jest przeprowadzenie badań metod stosowanych do wydobywania wyrażeń wielowyrazowych i wyznaczenie metody lub zestawu metod, które osiągają najlepsze wyniki w procesie wydobywania kolokacji.
Informacje o metodach mają być pozyskane głównie z literatury.
Celem pracy ma być także zaproponowanie przez autora nowego rozwiązania oraz sprawdzenie wyników przez nie generowanych.
Ważnym aspektem zadania jest przeprowadzenie badań na dużych zbiorach tekstowych w języku polskim.


\section{Struktura pracy}
Praca składa się z sześciu rozdziałów.

\par
Pierwszy rozdział jest wprowadzeniem do niniejszej pracy -- wstępem.

\par
Drugi porusza problematykę związaną z definicją kolokacji, przedstawione zostały ich cechy oraz przykładowe definicje kolokacji odnalezione w literaturze.
Na końcu rozdziału podana została definicja wyrażenia wielowyrazowego stosowana na potrzeby pozyskania zbioru testowego wykorzystywanego w badaniach opisanych w pracy.

\par
Trzeci rozdział zawiera opisy metod ekstrakcji wyrażeń wielowyrazowych pozyskane z literatury.
Opisany tutaj został także problem wydobywania kolokacji dłuższych niż dwuelementowe, poruszany w literaturze.

\par
Rozdział czwarty powstał z zamiarem opisania wykorzystywanych na potrzeby realizacji badań narzędzi i bibliotek programistycznych.
Opisane tutaj narzędzia i zasoby są w większości wykorzystywane pośrednio lub bezpośrednio do współpracy (poza narzędziem \emph{SuperMatrix}) z pakietem \emph{MWeXtractor}.
Znaczna część tego fragmentu pracy jest opisem wspomnianego pakietu programistycznego i narzędzi stworzonych przez autora pracy w celu ekstrakcji wyrażeń wielowyrazowych z dużych korpusów tekstowych.

\par
Piąty rozdział to prezentacja wykorzystanych zbiorów danych, ich statystyk, rezultatów ich weryfikacji, badań.
Zawarte i omówione tutaj są także sposoby badania metod ekstrakcji wyrażeń wielowyrazowych i wyniki osiągnięte przez te metody.

\par
Ostatnio z rozdziałów -- szósty -- to podsumowanie pracy, zsumowany opis wkładu własnego autora pracy, a także wnioski wyciągnięte na podstawie przeprowadzonych badań i zdobytej dzięki zajmowaniu się tematyką związaną z poruszanym tutaj problemem wiedzy.


\section{Wyjaśnienie ważniejszych terminów}
Zamieszczona poniżej lista przedstawia wykaz ważniejszych terminów wraz z ich definicjami ustalonymi na potrzeby niniejszej pracy:

\begin{itemize}
	\setlength{\itemsep}{1em}
	\item \textbf{wyraz} - na potrzeby niniejszej pracy termin ten określa dowolne słowo występujące w języku; w szczególności za wyraz uznawany będzie także pojedynczy znak; nie może on jednak zawierać żadnych znaków białych; termin ten jest stosowany zamiennie z określeniem \emph{słowo};
	\item \textbf{zdanie} - sekwencja wyrazów w określonej kolejności, zakończona kropką;
	\item \textbf{segment} - część tekstu ciągłego z zachowaniem kolejności jego składowych (przykładowo: zdanie, zbiór zdań lub paragraf);
	\item \textbf{token} - pojedyncze wystąpienie jakiegoś bytu \cite[str. 22]{mit}; jednostka będąca obiektem rozważań lingwistycznych; element składowy danych tekstowych wykorzystywany w ich przetwarzaniu; często pojedynczy wyraz lub zdanie, rzadziej całe segmenty; token zawierać może też znaki specjalne, które nie wchodzą w skład wyrazów;
	\item \textbf{częstość} - liczba, która określa ile razy w rozważanych danych tekstowych wystąpił dany byt, najczęściej słowo lub konkretny zbiór wyrazów;
	\item \textbf{tagset} - zestaw metadanych wykorzystywany do opisu składowych tekstów języka;
	\item \textbf{tager} - narzędzie wykonujące znakowanie morfosyntaktyczne elementów tekstu z wykorzystaniem określonego tagsetu; ewentualnie spełniać może także funkcję ujednoznaczniania znaczenia słów w przypadku ich polisemiczności;
	\item \textbf{korpus} - zebrany na potrzeby analiz lingwistycznych zbiór tekstów; dane zawarte w korpusie mogą zawierać różne metadane opisujące zebrane w nim informacje; przykładem popularnych metadanych jest oznakowanie morfosyntaktyczne słów zawartych w tym zbiorze; pożądaną cechą korpusu jest zrównoważenie polegające na zapewnieniu w nim odpowiednich stosunków ilości danych ze wszystkich rejestrów stylistycznych języka\footnote{Styl języka rozumiany jako jeden z obszarów jego użycia, na przykład mowa potoczna, słownictwo naukowe, teksty pseudonaukowe, wiadomości, terminologia związana ze sztuką czy innymi dziedzinami wiedzy lub działalności ludzkiej itp.}, co ma zapewnić jego dużą reprezentatywność; część korpusów jest także określona mianem równoległych, co oznacza, że każdemu tekstowi zawartemu w korpusie przyporządkowany jest tekst z co najmniej jednego innego języka, a zestawione mogą być na różnym poziomie - zdania czy akapitu;
	\item \textbf{słowo przetwarzane} - termin wprowadzony na potrzeby niniejszej pracy przez jej autora; określenie wyrazu, który podczas przeglądania korpusu jest aktualnie czytany;
\end{itemize}  
