\begin{abstract}
\par
Niniejsza praca prezentuje wyniki badań metod wydobywania wyrażeń wielowyrazowych z tekstów języka polskiego.
Zbadane zostały zarówno funkcje asocjacyjne, perceptron wielowarstwowy czy kombinacja liniowa z optymalizacją parametrów za pomocą algorytmu ewolucyjnego.
Metody pochodzące z literatury, jak i kilka funkcji asocjacyjnych zaproponowanych przez autora pracy.
Poruszona została także problematyka ekstrakcji wyrażeń dwu- oraz trójelementowych, a zestaw tekstów, na którym badania były prowadzone zawiera teksty składające się z ponad 250 milionów słów.
Dodatkowo dane z korpusu zostały pozyskane na dziesiątki sposobów, a każdy z nich został zbadany osobno z wykorzystaniem ponad dwudziestu funkcji, w tym dwóch parametrycznych.
Dla ostatnich sprawdzone zostały wyniki osiągane dla różnej wartości ich parametru.
Zbadano także jakość rozwiązań z wykorzystaniem perceptronu wielowarstwowego z 48 różnymi zestawami parametrów.
Sprawdzono także wpływ funkcji dyspersji i różnych metod filtracji na jakość generowanych rozwiązań.

\par
\begin{center}
\textbf{Abstract}
\end{center}
\par

This paper presents the results of research on methods of multi-word entities extraction from Polish language corpora.
Association functions, multilayer perceptron and linear combination of measures with evolutionary algorithm parameters optimization have been examined.
Methods from literature, as well as some associations functions proposed by the author of this paper.
The case of two- and three-element length collocation extraction was examined and the corpora used to conduct the research contained more than 250 million words.
Moreover, the data from corpora was extracted in dozens of ways and every one was examined separately with use of more than twenty association functions including two parametric methods.
The results generated by the last two functions for different parameters' values were examined.
Multilayer perceptron results for 48 network variants were also included.
The influence of the dispersion function and filters on results was also discussed.
\end{abstract}
