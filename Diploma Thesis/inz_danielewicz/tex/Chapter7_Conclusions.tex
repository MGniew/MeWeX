\chapter{Conclusions}

All described in this thesis steps contibuted the improvement of the MeWeX system. Unit testing led to detection of 
misimplementations of several fuctions and some other minor errors in code, what in result allowed to fix all of them. 
New association measure C-value adopted from TermoPL \cite{termopl} - tool for extraction terminology, appeared to be effective also in extraction of collocations. 
Generated list of MWE contained many proper candidates, what states about its quality, but also verification with use of Cross-validation 
method confirmed its effectivness. Aggregator of measures during training with Particle Swarm Optimization were achieving promising results, 
but cross-validation revealed much worse score. This could be caused by improper tuning methods, but several clues indicate other reason. 
At first, for creating folds during verification, the same dataset was used as for training, so overfitting would not diminish quality of 
the result to this level. Same problem met author of the \cite{mgr} using other methods for tuning weigths, so that can exclude 
problem with the algorithm itself. Possible justification may be error in implementation of aggregator of vector association measure.

\section{Further development}
MeWeX is a system which still leaves great field for improvements. At first set of relations should be carefully examined. 
Selecting candidates with WCCL operators is a first step of extraction and significantly affects following stages. 
Having availble such large corpus like plWordNet Corpus 10.0 with inapropriate set of relations it can generate huge number of improper candidates, 
what makes further calculations longer and worsens the results. That is why it is important to create set of more specific relatons, 
which will produce more valuable candidates, reducing computation time and increasing efficacy of training and improving quality of results.

Next possible enchancement is to implement additional stage of collocation extraction procces, similar to the one described in \cite{termopl}.
This checks the pointwise mutual information for all bigrams included in tuples with more elements and basing on obtained values it can 
filter out subtuples with low value what in fact suggest that these are improper callocations. That additional step will reduce the number of 
not correct MWE what will lead to improving the quality of the score.

Finally, code of aggregator of vector association measures should be investigated to find out what caused such a disproportion between results
obtained during training and performing Cross-validation. Some misimplementation is likely to be an explanation, taking into consideration, 
that performed unit tests showed that some errors were present in the code. As it was described in chapter \ref{vam_descr} aggregator of measures 
can greatly increase quality of the result, so this can be the most profitable improvement with the least cost.




% czw:
%     skończyć analizę funckyjną i niefunkcyjną
%     poukładać to, żeby się nie rozjeżdżało
%     wrzucić na gita!!!