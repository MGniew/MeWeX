\chapter{Training classifier}

\section{Dataset}
During tuning weigths vector 
All association measures base on statistical data, so in order to obtain reliable results large dataset should be provided.
Machine learning algorithms also need vast and differentiated resources to produce satisfactory result.
Below there are described datasets and its sources which was used for purpose of this thesis.

\subsection{Manually annotated subcorpus of NKJP}
Manually annotated subcorpus of NKJP is a part of the National Corpus of Polish, which is a initiative of four institutions: 
Institute of Computer Science at the Polish Academy of Sciences, Institute of Polish Language at the Polish Academy of Sciences, 
Polish Scientific Publishers PWN, and the Department of Computational and Corpus Linguistics at the University of Łódź.
It contains one milion tokens and was manually morfosyntactically "and semantically" annotated by lexicographers. 
Similarly to the main corpus of NKJP it contains text from diverse sources like classic literature, daily newspapers, 
specialist periodicals and journals, transcripts of conversations, and a variety of short-lived and internet texts. 
Manual annotation guarantee correctness of data and it is relatively large what makes it valuable choice for training data.

\subsection{kgr10 Corpus}
"It contains 40 000 000 000 tokens". Enormous size of this corpus caused computation time to process such amount of information 
long enough to make it impossible to perform tuning weigths and verification of the results on whole corpus. 
That was motivation to utilize only part of that resource. To preserve variety of text, new subcorpus was prepared by randomly picking 
10\% of sentences and dropping rest. Then it was automatically annotated by "Morphodita (add cite)".

\subsection{PLWordNet}
Wordnet is a "lexically-semantic" web, which nodes are lexical units and edges are semantic relations connecting those units.
PLWordNet is the biggest polish wordnet, it was founded by G4.19 Research Group in 2006 and still it is continously developed, 
for now it contains 170000 synsets, 244000 lexical units oraz 670000 semantic relations. Author of this thesis decided to use that resource 
as a source of proper collocations. They were obtained as a set of wccl operators, while MeWeX accepts only MWE listed in file one per line, 
so they had to be preprocessed.

\subsection{Set of relations}
First stage of collocation extraction, selecting candidates, requires set of relations, to reduce number of candidates only to syntactically reasonable. 
These relations has to be in form of WCCL operators. They can specify size of the tuple, part of speech and other properties of member "words", 
order ot tuple or even ... Set used for purpose of this thesis was taken from re

\section{Methods used for training}

\section{Results}