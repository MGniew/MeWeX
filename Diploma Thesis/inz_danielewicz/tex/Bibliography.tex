\begin{thebibliography}{9}
    % \nocite{*}

\bibitem{mgr} 
Wendelberger M., 
\textit{Automatyczne wydobywanie i klasyfikowanie kolokacji z korpusów języka polskiego.}, 
"magisterka",Wrocław, 2015.

\bibitem{evert}
Evert S.,
\textit{The Statistics of Word Cooccurences Word Pairs and Collocation Extraction}, 
PhD dissertation, University of Stuttgart, 2004.

\bibitem{ramisch} 
Ramisch C., 
\textit{Multiword Espressions Acquisition A Generic and Open Framework.} 
Springer, 2015.

\bibitem{kgr10} 
Piasecki M., Czachor G., Janz A., Kaszewski D., Kędzia P.,
\textit{Wordnet-based Evaluation of Large Distributional Models for Polish.} 
G4.19 Research Group, Computational Intelligence Department, Wrocław University of Science and Technology, Wrocław, 2017.

\bibitem{termopl} 
Marciniak M., Mykowiecka A., Rychlik P.,
\textit{TermoPL — a Flexible Tool for Terminology Extraction.} 
Institute of Computer Science PAS, Warszawa 2016.

\bibitem{nkjp} 
\textit{Narodowy Korpus Języka Polskiego}, 
URL: http://nkjp.pl/ (available 24.01.2018)

\bibitem{boost} 
Rozental G., \textit{Boost Unit Test Framework}, \\
URL: http://www.boost.org/doc/libs/1\_45\_0/libs/test/doc/html/utf.html (available 24.01.2018)

\bibitem{pso_wiki} 
Wikipedia, \textit{Particle Swarm Optimization}, \\
URL: https://en.wikipedia.org/wiki/Particle\_swarm\_optimization (available 24.01.2018)

\bibitem{wordnet} 
\textit{plWordNet}, 
URL: http://plwordnet.pwr.wroc.pl/wordnet/ (available 24.01.2018)

\bibitem{klyk}	
Kłyk Ł., \emph{Metody sztucznej inteligencji w zwiększaniu skuteczności klasyfikatora}, \\
"praca magisterska napisana na Wydziale Informatyki i Zarządzania Politechniki Wrocławskiej", Wrocław 2013.

\bibitem{wccl}
Radziszewski A., Wardyński A., Śniatowski T., \emph{WCCL: A Morpho-syntactic Feature Toolkit}, [in:] \emph{Proceedings of the Balto-Slavonic Natural Language Processing Workshop}, \\
Springer, 2011.


\end{thebibliography}