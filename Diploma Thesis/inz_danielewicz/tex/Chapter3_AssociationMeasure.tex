\chapter{New association measures}


\section{Termopl}
Natural Language Processing is developing very quickly nowadays, but extraction of collocations is still quite low advanced field. 
In recent years "many" papers or books have been published in this topic, but most of them describes method for extraction of MWE in english 
and since this task is strongly language dependent, polish literature about this subject is much more valuable for this thesis.
In 2016 Małgorzata Marciniak, Agnieszka Mykowiecka and Piotr Rychlik from Institute of Computer Science PAS published paper about their application 
for polish terminology extraction. Method that they use for that task is very similar to described in section \ref{extraction_method}. 
Whole extraction is a 3-stage process. At first candidates for terms are selected from text basing on grammar rules. 
Second step is evaluation of candidates using term quality measure and making ranking accordingly to the obtained score. 
Final step is filtering out general terms used in many domains by comparing phrases containing that terms to those obtained from non-domain corpora.
"Principle of working" is similar to the one used in MeWeX, also definidion of terminology used in that paper 
partialy cover definition of collocation used in this thesis. The difference is, that in opposition to MWE, term can be a single word, 
also terminology is domain-specific only while collocation can be also non-compositional general expression. 
These discrepancies may be easily "walkaround" by adding some simple constraints and then the same method can be used for extracting MWE. 
First stage of processing is very similar to already implemented, so this can be skipped. Last one filters out general terms, 
which is not desired step in collocation extraction. Middle stage uses quality measure to evaluate candidates and sort them 
accordingly to the obtained score, but TermoPL uses for that purpose C-value measure, which is not implemented in MeWeX. 
Extending availble list of association measures by this new function can improve efficiency of the system.

\section{C-Value}

\subsection{Motivation}
MeWeX contains many association measure which base on frequency, expected frequency, order, number of relation etc. 
and using vector association measure it can 

\subsection{Measure description}
\[ 
    C{\text -}value(p) = \begin{cases}
        l(p) * (freq(p) - \frac{1}{r(LP)}\sum_{lp \in LP}{freq(lp)} & \text{if } r(LP) > 0\\
        l(p) * freq(p)            & \text{if } r(LP) = 0\\
    \end{cases}
\]
Where: \\
\(p\)  - phrase \\
\(l(p) = ln(length(p))\) \\
\(LP\)  - set of phrases containing \(p\) \\
\(r(LP)\) - number of elements of \(LP\) \\

\subsection{Measure implementation}
\begin{lstlisting}
double Cvalue::rankUsingTable(
                    TupleId pTupleId, 
                    ContingencyTable const& pContingencyTable) const
{
    double LP = 0.0, LPsum = 0.0;
    for(int i = 1; i < pContingencyTable.size() - 1;i++)
    {
        LPsum += pContingencyTable[i].observed;
        if(pContingencyTable[i].observed > 0)
            LP += 1.0;
    }
    double l = log(pContingencyTable.tupleSize());

    if(LP == 0)
    {
        return (l * pContingencyTable[0].observed);
    }
    else
    {
        return (l * (pContingencyTable[0].observed - (LPsum/LP)));
    }
}
\end{lstlisting}

\subsection{Quality evaluation}