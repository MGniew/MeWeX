\section*{Abstract}
Thesis describe process of improving web based system for extraction of
collocations called MeWeX. At first it characterizes the problem of automatic extraction of collocations 
and introduces the system itself, its structure and functioning. Next chapters present all steps 
which were performed in order to improve quality of the MeWeX, that includes: performing unit tests,
implementing new association measure, implementing new algorithm for tuning
weights and training on new data. Finally it verifies the results achived by the new, 
updated, version of the system comparing it to old results.

\vspace{5em}

\section*{Streszczenie}
Praca opisuje proces ulepszania systemu webowego do ekstrakcji kolokacji, z korpusu tekstów języka polskiego, zwanego MeWeX.
Przedstawia ona problem automatycznej ekstrakcji kolokacji i prezentuje wspomniany system, opisując jego strukturę i 
sposób działania. Kolejne rozdziały przybliżają wszystkie kroki jakie zostały podjęte do polepszenia jakości wyników 
otrzymywanych przez MeWeXa, w tym: wykonanie testów jednostkowych, implementacja nowej miary asocjacyjnej,
implementacja nowego algorytmu do trenowania wektora wag, dostrojenie agregatora miar z użyciem nowego algorytmu 
na nowych danych treningowych. Ostatnim krokiem jest weryfikacja wyników otrzymanych z ulepszonej wersji systemu 
i porównanie ich do poprzednich rezultatów.