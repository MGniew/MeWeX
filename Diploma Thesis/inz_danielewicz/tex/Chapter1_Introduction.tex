
\chapter{Introduction}

\section{Definition of collocation}
In relevant literature we can find many different deffinitions of collocation, all of them have one common characteristic, collocation is a multiword expression that 

\section{Methods for extraction of collocations}
Automatic extraction of collocations may be performed in many different ways. Method of extraction is also dependent on choosen definition of MWE and language. Usually extraction is done in three steps. At first candidates for collocation are 

\section{MeWex}

\subsection{General description}
MeWeX is a system designed for extraction of collocations. It is composed from many

\subsection{Workflow of the program}

\subsection{Structure of the program}

\section{Scope of the thesis}

\subsection{General}
Purpose of this thesis is to improve efficiency of the MeWeX. To achieve that many different steps have to be performed.

\subsection{Test association measure functions}
First step is to perform unit test of association measures implementation. Author of the MeWeX did not prepared unit test for this part, so there is possibility, that those functions are incorrectly implemented.

\subsection{Implement new association measures}

\subsection{Imlement new mechanism for training feature set for classifiers}

\subsection{Train classifier on new data}

\subsection{Compare new results with previous one}
