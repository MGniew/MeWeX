
\chapter{Introduction}

\section{Definition of collocation}
There is not one fixed definition of collocation. Lexicographers argued over the years on correct defining what collocation is.
In relevant literature we can find many different deffinitions and most of them have one common characteristic, 
collocation is a multiword expression that are syntactically and/or semantically idiomatic. Because this thesis will base on MeWeX system 
it assumes the same definition as in Michał Wendelberger thesis, for which purpose MeWeX was created.
% wstawić definicję kolokacji w cytat czy cuś
\\ \textit Collocation is a multiword specialistic term or noncompositional general term. It may be both continous or not and both in fixed or flexible order.
% Za wyrażenie wielowyrazowe uznawane są wieloelementowe terminy specjalistyczne oraz
% niekompozycyjne terminy ogólne. Mogą być one zarówno ciągłe, w szyku przemiennym, jak i
% ustalonym.

\section{Methods for extraction of collocations}
Automatic extraction of collocations is a very hard task. First of all precise specification of collocation is required.
Method of extraction is dependent on choosen definition of MWE but also on other factors like language. 
Even having precisly specified all conditions it may be performed in many different ways. 
Usually finding MWE is done in three steps. At first candidates for collocation are extracted from the text using some rules and filters, 
which most often base on grammar of given language.

\section{MeWex}

\subsection{General description}
MeWeX is a system designed for extraction of collocations. It is composed from many

\subsection{Workflow of the program}

\subsection{Structure of the program}

\section{Scope of the thesis}

\subsection{General}
Purpose of this thesis is to improve efficiency of the MeWeX. To achieve that many different steps have to be performed.

\subsection{Test association measure functions}
First step is to perform unit test of association measures implementation. Author of the MeWeX did not prepared unit test for this part, 
so there is possibility, that those functions are incorrectly implemented.

\subsection{Implement new association measures}

\subsection{Imlement new mechanism for training feature set for classifiers}

\subsection{Train classifier on new data}

\subsection{Compare new results with previous one}
