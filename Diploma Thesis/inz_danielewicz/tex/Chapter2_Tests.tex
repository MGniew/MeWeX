\chapter{Tests}

\section{Scope of unit tests}
To ensure correctness fo the results unit tests should be performed. They cover implementation of all association measures, 
as they were not tested before, so there exists a chance that they are incorrectly written. This should be carefuly examined, 
because errors in that stage will propagate further making improvement of next stages pointless.



\section{Used technology}

\subsection{Availble test frameworks}
There is a lot of unit test frameworks for C++. Thay vary in availble functionalties, in design, differ in their complexity and purpose, 
some of them are heavy-duty, some are ligthweigth and simple. Few exemplary test frameworks are listed below.
\begin{itemize}
    \item CppUnit
    \item Boost.Test
    \item CppUnitLite
    \item NanoCppUnit
    \item Unit++
    \item CxxTest
\end{itemize}
From those frameworks Boost.Test has been chosen. There are few reasons why this one was the choice. 
At first Boost library is already used by MeWeX, so using this UTF instead of other does not increase number of dependencies of the project. 
Second reason was that Boost library is the only one known to the author of this thesis, so it was also an advantage.

\subsection{Boost UTF}
The Boost Test Library Unit Test Framework gives both a simply to use and flexible way of implementation and organization C++ unit test. 
Writing a unit test module is simple and intuitive for begginers, but framework allows more advanced users to perform complex tests. 
Test module gives possibility to have many small test cases and organise them into test suites. It also provide a feedback for a long test by 
showing the test progress during its work. It does not require any additional library anf for long term usage users of a unit test framework 
it is able to build it as a standalone library. The Boost.Test keeps track of all passed/failed testing tools assertions, 
provides an ability to check the test progress and generates a result report in several different formats.

\section{Method of testing}
TODO

\section{Result}
Results of unit test showed that several association measures were incorrectly implemented. In some cases error lies in lacking minus like in 
LogLikelihood, so it only reverse the score, but in other cases wrong order of operations like in WOrder makes the result completly miscalculated.
Table \ref{tbl_test} shows the results of performed unit tests, red names indicates incorrect implementation, orange represents function, 
which was properly written, but it uses outcome of faulty measure. In Remaining functions no errors were found.
\begin{table}[t]
    \centering
    % \begin{tabular}{|l|l|}
    \begin{tabular*}{0.9\textwidth}{l @{\extracolsep{\fill}} l}
        \hline \\
        Dice & \textcolor{red}{LogLikelihood} \\
        \textcolor{red}{WOrder} & SmoothedBigram \\
        TScore & UnigramSubtuples \\
        ZScore & PearsonsChiSquare \\
        Jaccard & ExpectedFrequency \\
        \textcolor{orange}{WTFOrder} & MutualExpectation \\
        Frequency & SpecificCorrelation \\
        OddsRatio & WSpecificCorrelation \\
        Sorgenfrei & InversedExpectedFrequency \\
        \textcolor{red}{ConsonniT1} & SpecificExponentialCorrelation \\
        ConsonniT2 & WSpecificExponentialCorrelation \\
        WChiSquare & FairDispersionPointNormalization \\
        AverageBigram & SpecificFrequencyBiasedMutualDependency \\
        MinimalBigram \\
        \\\hline
    \end{tabular*} 
    \caption{Association functions results}
    \label{tbl_test}
\end{table}