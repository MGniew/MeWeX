%%&pdflatex
\documentclass[12pt]{report}


\usepackage[utf8]{inputenc}

%docstyle
\usepackage{sectsty}
\sectionfont{\fontsize{13}{15}\selectfont}
\subsectionfont{\fontsize{13}{15}\selectfont}

\usepackage{etoolbox}
\makeatletter
\patchcmd{\@makechapterhead}{\huge}{\fontsize{14}{1}\selectfont}{}{}% for \chapter
\patchcmd{\@makechapterhead}{\Huge}{\fontsize{14}{1}\selectfont}{}{}% for \chapter
\patchcmd{\@makeschapterhead}{\Huge}{\huge}{\fontsize{14}{1}\selectfont}{}{}% for \chapter*
\makeatother

\usepackage{csquotes}
\AtBeginEnvironment{quote}{\itshape}

\usepackage{indentfirst} %indent paragraphs
\usepackage[margin=2.5cm]{geometry}
\usepackage{pdfpages}
\usepackage[T1]{fontenc}
\usepackage{mathptmx} %Times New Roman
% \usepackage{fullpage}
% \usepackage{placeins} %for floats position barrier
% \usepackage{float}

%tables
% \usepackage{caption} 
% \usepackage{multirow, supertabular}
% \usepackage{array}
\usepackage{tabularx}
% \usepackage{makecell}
% \usepackage{hhline}
% \usepackage{changepage}

% \captionsetup[table]{skip=14pt}

%bibliography
% \usepackage{url}
% \bibliographystyle{abbrv}

%graphics
% \usepackage{graphicx}
% \usepackage{subcaption}

%math
\usepackage{amsmath}

\title{Web-Based System for the Extraction of Collocations from Corpora of Polish Texts Equipped with Mechanism for Tunning on Training Data.}
\author{Igor Danielewicz\\ \and Supervisor: dr inż. Michał Przewoźniczek}


%DOCUMENT
\begin{document}
\begin{titlepage}
	\includepdf[pages={1}]{titlepage.pdf}
\end{titlepage}


\section*{Abstract}
bla bla bla
\section*{Streszczenie}
to samo tylko po polsku

\tableofcontents


\chapter{Introduction}

\section{Definition of collocation}
There is not one fixed definition of collocation. Lexicographers argued over the years on correct defining what collocation is.
In relevant literature we can find many different deffinitions and most of them have one common characteristic, 
collocation is a multiword expression that are syntactically and/or semantically idiomatic. Because this thesis will base on MeWeX system 
it assumes the same definition as in Michał Wendelberger thesis, for which purpose MeWeX was created.
% wstawić definicję kolokacji w cytat czy cuś
\\ \textit Collocation is a multiword specialistic term or noncompositional general term. It may be both continous or not and both in fixed or flexible order.
% Za wyrażenie wielowyrazowe uznawane są wieloelementowe terminy specjalistyczne oraz
% niekompozycyjne terminy ogólne. Mogą być one zarówno ciągłe, w szyku przemiennym, jak i
% ustalonym.

\section{Methods for extraction of collocations}
Automatic extraction of collocations is a very hard task. First of all precise specification of collocation is required.
Method of extraction is dependent on choosen definition of MWE but also on other factors like language. 
Even having precisly specified all conditions it may be performed in many different ways. 
Usually finding MWE is done in three steps. At first candidates for collocation are extracted from the text using some rules and filters, 
which most often base on grammar of given language.

\section{MeWex}

\subsection{General description}
MeWeX is a system designed for extraction of collocations. It is composed from many

\subsection{Workflow of the program}

\subsection{Structure of the program}

\section{Scope of the thesis}

\subsection{General}
Purpose of this thesis is to improve efficiency of the MeWeX. To achieve that many different steps have to be performed.

\subsection{Test association measure functions}
First step is to perform unit test of association measures implementation. Author of the MeWeX did not prepared unit test for this part, 
so there is possibility, that those functions are incorrectly implemented.

\subsection{Implement new association measures}

\subsection{Imlement new mechanism for training feature set for classifiers}

\subsection{Train classifier on new data}

\subsection{Compare new results with previous one}

\chapter{Tests}
To ensure correctness fo the results unit testsing should be performed. They cover implementation of all association measures, 
as they were not tested before, so there exists a chance that they are incorrectly written. This should be carefuly examined, 
because errors in that stage will propagate further making improvement of next stages pointless.

\section{Availble test frameworks}
There is a lot of unit test frameworks for C++. Thay vary in availble functionalties, in design, differ in their complexity and purpose, 
some of them are heavy-duty, some are ligthweigth and simple. Few exemplary test frameworks are listed below.
TODO
\begin{itemize}
    \item \textbf{CppUnit} - test framework that allows testing of C sources as well as C++ with minimal source code modification.
            It was started as a port of JUnit for Windows and ported to Unix. The library is released under the GNU LGP License.
            The framework runs tests in suites. Test result output is sent to a filter, the most basic being a simple pass or fail count printed out, 
            or more advanced filters allowing XML output compatible with continuous integration reporting systems.\footnote{https://en.wikipedia.org/wiki/CppUnit}
    \item \textbf{Boost.Test} - this UTF provides simple writing test cases by using various testing tools. 
            It allows to organize test cases into a test tree. Ease error detection by reporting duties and framework runtime parameters processing.
            More precise description can be found in section \ref{test_boost}.
    \item \textbf{CppUnitLite} - this is a simple C++ testing framework developed by author of the original CppUnit. 
            Unlike some other frameworks, this one is a barebones framework intended to be extended by its users to support their particular needs.
            This approach makes writing individual test easy, one TEST macro which registers the test automatically.\footnote{http://wiki.c2.com/?CppUnitLite}
    \item \textbf{Aeryn} - its design is clean, simple has no dependencies upon other libraries. Although it is primarily intended for unit testing, 
            it is adaptable enough to handle integration testing and can be adapted for most other forms of C++ testing.
            Aeryn requires a modern C++ compiler.\footnote{https://accu.org/index.php/journals/1326}
    \item \textbf{CxxTest} - is a unit testing framework for C++ that is similar in spirit to JUnit, CppUnit, and xUnit. 
            CxxTest is easy to use because it does not require precompiling a CxxTest testing library, 
            it employs no advanced features of C++ (e.g. RTTI) and it supports a very flexible form of test discovery.\footnote{http://cxxtest.com/}
\end{itemize}
From those frameworks Boost.Test has been chosen. There are few reasons why this one was the choice. 
At first Boost library is already used by MeWeX, so using this UTF instead of other does not increase number of dependencies of the project. 
Second reason was that Boost library is the only one known to the author of this thesis, so it was also an advantage. 
Finally its capabilities are sufficient for purpose of this project while use of this framework is simple and easy.

\subsection{Used technology}\label{test_boost}
The Boost Test Library Unit Test Framework gives both a simply to use and flexible way of implementation and organization C++ unit test. 
Writing a unit test module is simple and intuitive for begginers, but framework allows more advanced users to perform complex tests \cite{boost}. 
Test module gives possibility to have many small test cases and organise them into test suites. It also provide a feedback for a long test by 
showing the test progress during its work. It does not require any additional library anf for long term usage users of a unit test framework 
it is able to build it as a standalone library. The Boost.Test keeps track of all passed/failed testing tools assertions, 
provides an ability to check the test progress and generates a result report in several different formats.

\section{Method of testing}
To test association measures at first it was neccesary to prepare contingency tables and additionally for several fuctions additionaly RankerData. 
For purpose of generating that objects set of short CCL was used with content prepared to cover all edge cases in evaluation of tuples scores. 
Text reading, tuples generation and creating contingency table source together with RankerData was performed directly with use of availble functions 
and class instead of using separate program. It was performed only once and stored as a Fixture. Every function was covered by one test case which 
compared result of function with correct result for every tuple in all prepared texts. All test cases were groupped to one test suite.

\section{Result}
Results of unit test showed that several association measures were incorrectly implemented. In some cases error lies in lacking minus like in 
LogLikelihood, so it only reverse the score, but in other cases wrong order of operations like in WOrder makes the result completly miscalculated.
Table \ref{tbl_test} shows the results of performed unit tests, red names indicates incorrect implementation, orange represents function, 
which was properly written, but it uses outcome of faulty measure. In remaining functions no errors were found. 
Last step of this part was to fix those implementations which consisted errors.
\begin{table}[t]
    \centering
    % \begin{tabular}{|l|l|}
    \begin{tabular*}{0.9\textwidth}{l @{\extracolsep{\fill}} l}
        \hline \\
        Dice & \textcolor{red}{LogLikelihood} \\
        \textcolor{red}{WOrder} & SmoothedBigram \\
        TScore & UnigramSubtuples \\
        ZScore & PearsonsChiSquare \\
        Jaccard & ExpectedFrequency \\
        \textcolor{orange}{WTFOrder} & MutualExpectation \\
        Frequency & SpecificCorrelation \\
        OddsRatio & WSpecificCorrelation \\
        Sorgenfrei & InversedExpectedFrequency \\
        \textcolor{red}{ConsonniT1} & SpecificExponentialCorrelation \\
        ConsonniT2 & WSpecificExponentialCorrelation \\
        WChiSquare & FairDispersionPointNormalization \\
        AverageBigram & SpecificFrequencyBiasedMutualDependency \\
        MinimalBigram \\
        \\\hline
    \end{tabular*} 
    \caption{Association functions results}
    \label{tbl_test}
\end{table}
\chapter{New association measures}

\section{Termopl}

\subsection{Measure description}

\subsection{Measure implementation}
\chapter{Imlement algorithm for training weigths for vector association measure}\label{vam_descr}

Different association measures evaluate their scores basing on different statistics like frequency of the tuple, 
expected frequency, frequency of subtuples, size of tuple etc, so to improve quality of the results MeWeX uses 
aggregators of the association measures. This feature allows to join properties of many measures in order to achieve 
better score. Influence on final result of each measure can be controlled by vector of weigths. Proper adjustment 
of weigths values is crucial. It can be done with use of machine learning algorithms, which can tune vector of weigths 
in order to maximize correctness of the result. MeWeX contains several heuristic and metaheuristic algorithms, 
which were implemented by Łukasz Kłyk in his \(Optimizer\) and adopted by the autor of the MeWeX to work with the rest of the system.
\begin{itemize}
    \setlength\itemsep{0em}
    \item Evolutionary Algorithm 
    \item Hill Climbing 
    \item Tabu Search 
    \item Random Search 
    \item Simulated Annealing
\end{itemize}

\section{Particle Swarm Optimization}\label{pso_def}

Particle Swarm Optimization is an algorithm for optimization of a problem. It tries to improve a candidate solution 
accordingly to a given measure of quality. PSO have population of particles called swarm and solves problem by iteratively moving those particles 
through the search space using formula that involves particle velocity, local best solution and global best solution. 

Algorithm starts with placing all particles in search space choosing random uniformly distributed positions for them and initializing its velocity to 0. 
Then for first particle solution is evaluated, next it assigns its starting position as a local and global best position. 
This step is repeated for all particles in the swarm, but global best position is changed only when new one achieves higher score. 
From that point all steps are performed in a loop till stop condition will not be satisfied. For every paricle, at first 
velocity is updated, then its position is changed by the velocity vector and quality of new position is evaluated. 
If new score is is higher than local or global best, then they are updated accordingly. When algorithm reaches stop conition 
then position of global best solution is returned as a result.
In the figure \ref{pso_flowchart} there is presented flow chart of Particle Swarm Optimization.
\begin{figure}[ht]
    \centering
    \includegraphics[scale=0.4]{img/pso_flowchart.png}
    \caption{Flow chart of Particle Swarm Optimization}
    \label{pso_flowchart}
\end{figure}

Particles move toward local and global maxima making swarm more dense in surroundings of best known positions. 
That property entail swarm for searching better solutions.
Below there is formula for updating velocity of the particles.
\[ 
    V \gets \omega V + \omega _{l} r _{l} (p - x) + \omega _{g} r _{g} (g - x)
\]
Where: \\
\(V\)  - velocity of given particle, it is vector of size equal to number of dimensions in search space \\
\(r _{l}, r _{g}\) - random numbers from interval \((0,1)\) \\
\(\omega, \omega _{l}, \omega _{g}\) - parameters, chosen by the user, which controls behaviour of the PSO \\
\(x\)  - position of given particle \\
\(p\)  - position of best solution achieved by given particle \\
\(g\)  - position of global best solution achieved by the swarm \\

One of the factor for calculating new velocity is its previous value. Preserving partially old velocity makes particle behavior 
like it would have momentum, smoothing its movement. Particle aims toward global and local best, but force of those both factors is randomized.
New velocity is composed from three vectors and strength of influence for each of them can be adjusted by changing value 
of corresponding \(\omega\) parameter. By modifing those parameters it is possible to control behavior of the swarm, for example 
by reducing momentum or increasing importance of local maxima.

Particle Swarm Optimization is quite simple algorithm, but it can achieve good result. Its simplicity allows easy modification and adjusting 
algorithm for needs of the user. It has three parameters that can influence behavior of the swarm, but their tuning is intuitive. 
Disadvantage of this algorithm is that it has tendency to fall in the local maxima. Next problem is that for high dimensional search spaces 
swarm has to be more populated what greatly increases computational time. 

\subsection{Motivation}
One of the method to increase efficiency of the MeWeX was to improve selection of weigths for aggregators. 
Implementing new algorithm Particle Swarm Optimization was reasonable solution taking into account its simplicity 
connected with good efficacy. Susceptibility on modifications and easy adjustment of parameters make it also a good choice.

\subsection{Implementation}
Expanded structure and modular architecture of MeWeX made simple further extensions of code. Class diagram presented on figure \ref{img_pso_class}
presents the structure of implemented algorithm.
\begin{figure}[ht]
\centering
    \includegraphics[scale=0.2]{img/pso_class.png}
    \caption{Class diagram of Particle Swarm Optimization}
    \label{img_pso_class}
\end{figure}


To make possible using new algorithm in the same way as already implemented few steps was neccesary to perform.
At first, class \(ParticleSwarmOptimization\) had to be created base on implementation of other machine learning algorithms. 
It does not need to inherit any other class, but it has to be template class, with the same set of template parameters like in other classes. 
Next requirement is that it must define constructor \textit{ParticleSwarmOptimization(const Point\& rStartPoint, Evaluator* pEvaluator, const ArgumentsType\& rArgs)} 
and fuction \(Point\ start()\) which executes whole algorithm and returns best found solution.
Next step was to implement two classes, one which inherits by \(BaseCallPoliciesArguments\) and defines parameters that can be passed by user, 
and second which inherits by \(BaseReport\) and specifies format of creating report of training, which is written to file during algorithm work.
Then, those classes had to be added to \(EvaluatorWrapper\) which is responsible for running chosen algorithm, 
passing arguments from user and creating report.
    % \begin{lstlisting}
    % class EvaluatorWrapper
    % {
    % public:

    %         ...

    %     typedef particle_swarm_optimization::CallPoliciesArgumentsPSO	PSOCallPolicy;

    %     enum MethodType
    %     {
    %         ...

    %         PSO,  // Particle Swarm Optimization
    %         EMPTY
    %     };
    %         ...

    %     EvaluatorWrapper(
    %         EvaluatorPtrR const&	pEvaluator,
    %         MethodType              pMethodType,
    %         Point const&	        pStartPoint,
    %         PSOCallPolicy const& 	pPolicy);

    %         ...
        
    %     void setParticleSwarmOptimizationPolicy(PSOCallPolicy const& pPolicy);

    % private:
    %         ...
        
    %     PSOCallPolicy mPSOCPA;
    % };
    % \end{lstlisting}
        
    % \begin{lstlisting}
    %         ...

    % EvaluatorWrapper::EvaluatorWrapper(
    %     EvaluatorPtrR const&	pEvaluator,
    %     MethodType 			pMethodType,
    %     Point const&		pStartPoint,
    %     PSOCallPolicy const& 	pPolicy)
    % :
    %     mMethodType(pMethodType),
    %     mStartPoint(pStartPoint),
    %     mEvaluator(pEvaluator),
    %     mPSOCPA(pPolicy)
    % {}

    % auto EvaluatorWrapper::parseMethodType(std::string const& pMethod) -> MethodType
    % {
    %     if(boost::iequals(pMethod, "RS"))
    %     {
    %         return RS;
    %     }    
        
    %         ...
            
    %     else if(boost::iequals(pMethod, "PSO"))
    %     {
    %         return PSO;
    %     }
    %         ...

    % void EvaluatorWrapper::setParticleSwarmOptimizationPolicy(PSOCallPolicy const& pPolicy)
    % {
    %     mPSOCPA = pPolicy;
    % }

    % auto EvaluatorWrapper::start() -> Point
    % {
    %         ...

    % typedef particle_swarm_optimization::ParticleSwarmOptimization<
    %             particle_swarm_optimization::CallPoliciesArgumentsPSO,
    %             unsigned int,
    %             Step,
    %             time_t,
    %             Timer,
    %             particle_swarm_optimization::ReportPSO> PSOAlgorithm;

    %         ...

    % switch(mMethodType)
    % {
    %         ...
            
    %     case PSO:
    %     {
    %         return PSOAlgorithm(mStartPoint, mEvaluator, mPSOCPA).start();
    %     }
    %     break;

    %         ...
    % \end{lstlisting}
Last step is to implement algorithm itself, but it needs a class that represents particle and MeWeX machine learning module implements only class \(Point\).
Class \(Particle\) was created by author of this thesis by inheriting class \(Point\), adding velocity vector and best local position and defining 
function \(void\ move(const\ Point\&\ rBest)\).
    % \begin{lstlisting}
    % void Particle::move(const Point& rBest)
    % {
    %     double c1 = 1.0, c2 = 0.2, c3 = 0.8;
    %     double r1 = Random::random(), r2 = Random::random(), r3 = Random::random();
    %     for(int i = 0; i < mVelocity.size(); i++)
    %     {
    %         double v,mX,rX,cX;
    %         auto data = mVelocity[i]->getValueAt(0);
    %         mVelocity[i]->getValueAt(0).get(v);
    %         mBest->getParameterAt(i).getValueAt(0).get(mX);
    %         mParameters[i]->getValueAt(0).get(cX);
    %         rBest.getParameterAt(i).getValueAt(0).get(rX);
    %         v = (c1 * r1 * v) + (c2 * r2 * (mX - cX)) + (c2 * r2 * (rX - cX));
    %         data.set(v);
    %         mVelocity[i]->setValueAt(0, data);
    %         data.set(cX + v);
    %         mParameters[i]->setValueAt(0, data);
    %     }
    % }
    % \end{lstlisting}
Having implemented class \(Particle\) it was possible to write fuction \(Point\ start()\) which executes algorithm. 



\subsection{Improvements}\label{pso_improv}
Initial investigation of algorithm work showed that algorithm improves the score at early stage of work, but then it finds local maximum 
and stops seeking for better solutions. It was caused by the situation where all particles reached the position between 
local and global best solution, so their velocity lowered to values close to zero. Figure \ref{img_pso_base} shows the quality of solution 
during algorithm work, where it is visible how swarm has fallen in local maximum and make no progress.
\begin{figure}[ht]
\centering
	\includegraphics[scale=0.4]{img/pso_base.png}
	\caption{Plot of PSO algorithm}
	\label{img_pso_base}
\end{figure}

To prevent that issue author of this thesis proposed a function which reallocates all particles when algorithm encounter that problem. 
To detect when swarm stopped search, it calculates sum of velocity vector length of all particles. 
If that sum drops below some specified value, function, which places all particle in random position is called. 
Figure \ref{img_pso_improv} shows how the improved algorithm performs. It is visible from the plot that this improvement 
increased efficacy of the algorithm by realocating particles when they stop in place and do not search for new solutions.
\begin{figure}[ht]
	\centering
	\includegraphics[scale=0.37]{img/pso_improv.png}
	\caption{Plot of PSO algorithm}
	\label{img_pso_improv}
\end{figure}

Second problem with PSO was that search space had a lot of dimensions, and by reason of long computation time to evaluate single particle solution, 
size of the swarm was limited. This caused that only small part of search space was explored. Solution of that problem, 
proposed by the author of this thesis was to change the distribution of starting positions. Instead of uniform distribution, positions was randomized 
with more problability beeing close to edge. This distibution is presented on figure \ref{img_pso_imp_dist}.

\begin{figure}[ht]
	\centering
	\includegraphics[scale=0.4]{img/pso_dist.png}
	\caption{Probability distribution for selecting particle position}
	\label{img_pso_imp_dist}
\end{figure}

For creation of that probability distribution there was used class, provided by C++11 Standard Library, 
\textit{template< class RealType = double > class normal\_distribution;} which produces random numbers with normal probability
with specified mean and standard deviation. Method, which converts normal distribution to the shown on the plot in the figure \ref{img_pso_imp_dist} 
is shown below.

\begin{lstlisting}
std::normal_distribution<double> Random::normal(0.5, 0.1);
std::mt19937_64 Random::generator(time(nullptr));

double Random::random_inv_normal(void)
{
    double val = normal(generator);
    if(val < 0.5)
    {
        val += 0.5;
        if(val < 0)
            val = 0;
    }
    else
    {
        val -= 0.5;
        if(val > 1)
            val = 1;
    }
    return val;
}
\end{lstlisting}
\chapter{Training classifier}

\section{Methods used for training}

\section{Dataset}

\subsection{kgr10 Corpus}

\subsection{PLWordNet}

\subsection{Set of relations}

\section{Results}
\chapter{Verification of the result}

\section{Methods of results verification}

\section{Setup for verification}

\section{Results}
\chapter{Conclusions}

All described in this thesis steps contibuted the improvement of the MeWeX system. Unit testing led to detection of 
misimplementations of several fuctions and some other minor errors in code, what in result allowed to fix all of them. 
New association measure C-value adopted from TermoPL \cite{termopl} - tool for extraction terminology, appeared to be effective also in extraction of collocations. 
Generated list of MWE contained many proper candidates, what states about its quality, but also verification with use of Cross-validation 
method confirmed its effectivness.

\section{Further development}
MeWeX is a system which still leaves great field for improvements. At first set of relations should be carefully examined. 
Selecting candidates with WCCL operators is a first step of extraction and significantly affects following stages. 
Having availble such large corpus like plWordNet Corpus 10.0 with inapropriate set of relations it can generate huge number of improper candidates, 
what makes further calculations longer and worsens the results. That is why it is important to create set of more specific relatons, 
which will produce more valuable candidates, reducing computation time and increasing efficacy of training and improving quality of results.

Next possible enchancement is to implement additional stage of collocation extraction procces, similar to the one described in \cite{termopl}.
This checks the pointwise mutual information for all bigrams included in tuples with more elements and basing on obtained values it can 
filter out subtuples with low value what in fact suggest that these are improper callocations. That additional step will reduce the number of 
not correct MWE what will lead to improving the quality of the score.

Finally aggregators





% czw:
%     chap 6
%     5.2 i 5.3
%     1 obrazek (arkitekczer)
%     zacząć analizę funckyjną i niefunkcyjną
%     wrzucić na gita!!!
% pt:
%     chap 7
%     dokończyć analizę
%     słownik pojęć uzupełnić
%     dokończyć bibliografie
%     literówki
%     wysłać do obu


% Shared library available in C++ and package for Python offer same method with similar syntax - method Lemmatize with parameters :\begin{itemize}
% \item inflected phrase
% \item base form of words space-separated
% \end{itemize}
        
% \begin{figure}[ht]
% 	\centering
% 	\includegraphics[scale=0.7]{img/karina.jpg}
% 	\caption{Lemmatization Module Input}
% 	%\label{polem_input}
% \end{figure}
        
% At the end of the file the summary of the lemmatization is given concerning the method of lemmatization and concerning the category of phrases. E
% \begin{figure}[h]
% 	\centering
% 	%\includegraphics[scale=0.7]{karina.jpg}
% 	\caption{Lemmatization Module Output Table}
% 	%\label{polem_output_table}
% \end{figure}

		
%BIBLIOGRAPHY
\addcontentsline{toc}{chapter}{List of tables}
\listoftables
\addcontentsline{toc}{chapter}{List of figures}
\listoffigures
\addcontentsline{toc}{chapter}{Bibliography}
% 			\bibliography{bibliography.bbl}
% 			\bibliography{mybib}{}
% \nocite{*}
%         \bibliographystyle{plain}
% \bibliography{bib}
\begin{thebibliography}{9}
    % \nocite{*}

\bibitem{evert}
Evert S.,
\textit{The Statistics of Word Cooccurences Word Pairs and Collocation Extraction}, 
PhD dissertation, University of Stuttgart, 2004.

\bibitem{klyk}	
Kłyk Ł., \emph{Metody sztucznej inteligencji w zwiększaniu skuteczności klasyfikatora}, \\
Wroclaw University of Science and Technology, Wrocław 2013.

\bibitem{termopl} 
Marciniak M., Mykowiecka A., Rychlik P.,
\textit{TermoPL — a Flexible Tool for Terminology Extraction.} 
Institute of Computer Science PAS, Warszawa 2016.

\bibitem{kgr10} 
Piasecki M., Czachor G., Janz A., Kaszewski D., Kędzia P.,
\textit{Wordnet-based Evaluation of Large Distributional Models for Polish.} 
G4.19 Research Group, Computational Intelligence Department, Wrocław University of Science and Technology, Wrocław, 2017.

\bibitem{wccl}
Radziszewski A., Wardyński A., Śniatowski T., \emph{WCCL: A Morpho-syntactic Feature Toolkit}, [in:] \emph{Proceedings of the Balto-Slavonic Natural Language Processing Workshop}, \\
Springer, 2011.

\bibitem{ramisch} 
Ramisch C., 
\textit{Multiword Espressions Acquisition A Generic and Open Framework.} 
Springer, 2015.

\bibitem{boost} 
Rozental G., \textit{Boost Unit Test Framework}, \\
URL: http://www.boost.org/doc/libs/1\_45\_0/libs/test/doc/html/utf.html (available 24.01.2018)

\bibitem{mgr} 
Wendelberger M., 
\textit{Automatyczne wydobywanie i klasyfikowanie kolokacji z korpusów języka polskiego.}, 
Wroclaw University of Science and Technology,Wrocław, 2015.

\bibitem{nkjp} 
\textit{Narodowy Korpus Języka Polskiego}, 
URL: http://nkjp.pl/ (available 24.01.2018)

\bibitem{wordnet} 
\textit{plWordNet}, 
URL: http://plwordnet.pwr.wroc.pl/wordnet/ (available 24.01.2018)

\bibitem{pso_wiki} 
Wikipedia, \textit{Particle Swarm Optimization}, \\
URL: https://en.wikipedia.org/wiki/Particle\_swarm\_optimization (available 24.01.2018)


\end{thebibliography}

\end{document}
