%%&pdflatex
\documentclass[12pt]{report}


\usepackage[utf8]{inputenc}

%\usepackage{fullpage}
\usepackage{indentfirst} %indent paragraphs
\usepackage[margin=2.5cm]{geometry}
\usepackage{placeins} %for floats position barrier
\usepackage{float}
\usepackage{pdfpages}

%tables
\usepackage{caption} 
\usepackage{multirow, supertabular}
\usepackage{array}
\usepackage{tabularx}
\usepackage{makecell}
\usepackage{hhline}
\usepackage{changepage}

% \captionsetup[table]{skip=14pt}

%bibliography
\usepackage{url}
\bibliographystyle{abbrv}

%graphics
\usepackage{graphicx}
\usepackage{subcaption}

\title{Web-Based System for the Extraction of Collocations from Corpora of Polish Texts Equipped with Mechanism for Tunning on Training Data.}
\author{Igor Danielewicz\\ \and Supervisor: dr inż. Michał Przewoźniczek}


%DOCUMENT
\begin{document}
\begin{titlepage}
	\includepdf[pages={1}]{titlepage.pdf}
\end{titlepage}
%\maketitle


\section*{Abstract}
\section*{Streszczenie}

\tableofcontents

\chapter{Introduction}
\section{Goal of the project}
\section{Motivation}

\chapter{Introduction}

\section{Definition of collocation}
There is not one fixed definition of collocation. Lexicographers argued over the years on correct defining what collocation is.
In relevant literature we can find many different deffinitions and most of them have one common characteristic, 
collocation is a multiword expression that are syntactically and/or semantically idiomatic. Because this thesis will base on MeWeX system 
it assumes the same definition as in Michał Wendelberger thesis, for which purpose MeWeX was created.
% wstawić definicję kolokacji w cytat czy cuś
\\ \textit Collocation is a multiword specialistic term or noncompositional general term. It may be both continous or not and both in fixed or flexible order.
% Za wyrażenie wielowyrazowe uznawane są wieloelementowe terminy specjalistyczne oraz
% niekompozycyjne terminy ogólne. Mogą być one zarówno ciągłe, w szyku przemiennym, jak i
% ustalonym.

\section{Methods for extraction of collocations}
Automatic extraction of collocations is a very hard task. First of all precise specification of collocation is required.
Method of extraction is dependent on choosen definition of MWE but also on other factors like language. 
Even having precisly specified all conditions it may be performed in many different ways. 
Usually finding MWE is done in three steps. At first candidates for collocation are extracted from the text using some rules and filters, 
which most often base on grammar of given language.

\section{MeWex}

\subsection{General description}
MeWeX is a system designed for extraction of collocations. It is composed from many

\subsection{Workflow of the program}

\subsection{Structure of the program}

\section{Scope of the thesis}

\subsection{General}
Purpose of this thesis is to improve efficiency of the MeWeX. To achieve that many different steps have to be performed.

\subsection{Test association measure functions}
First step is to perform unit test of association measures implementation. Author of the MeWeX did not prepared unit test for this part, 
so there is possibility, that those functions are incorrectly implemented.

\subsection{Implement new association measures}

\subsection{Imlement new mechanism for training feature set for classifiers}

\subsection{Train classifier on new data}

\subsection{Compare new results with previous one}

\chapter{State of the art}
\section{Lemmatization}
\section{Keyword extraction}
\chapter{Lemmatization module}
\section{General}
Lemmatization module used in the system is single-handedly implemented tool named Polem which is described in "Lemmatization of Multi-word Common Noun Phrases and Named Entities in Polish" paper by Ph.D. Marcinczuk. It is a command-line tool, implemented in C++, that calculates the effectiveness of lemmatization and is also available as shared library in C++ and package in Python. Command-line tool can be run with different parameters like pathname to input file which is obligatory, pathname to look for dictionary and other files necessary to conduct lemmatization when module is not installed and if lemmatization should be evaluated case and space sensitive.
Shared library available in C++ and package for Python offer same method with similar syntax - method Lemmatize with parameters :\begin{itemize}
\item inflected phrase
\item base form of words space-separated
\item tags of words space-separated
\item boolean values if the words of phrase should be space-separated
\item phrase category
\end{itemize}
Command-line tool takes tsv file as an input. Each line consist of proper lemma of a phrase, an inflected phrase, base forms of words from the phrase, tags of each word, boolean values if there should be space after each word and a category of a phrase e.g.
        
\begin{figure}[ht]
	\centering
	\includegraphics[scale=0.7]{img/karina.jpg}
	\caption{Lemmatization Module Input}
	%\label{polem_input}
\end{figure}
        
Evaluation is based on comparing the proper lemma that was given at input to the result of the lemmatization. Output of the module is also a tsv file with effectiveness of the lemmatization calculated. Each line consists of a boolean value if the lemmatization of the phrase was conducted successfully, inflected phrase, proper lemma of the phrase, category, method that was used to lemmatize given phrase, base form of words in phrase and tags e.g.
        
\begin{figure}[ht]
	%\includegraphics[scale=0.53]{karina.jpg}
	\caption{Lemmatization Module Output}
	%\label{polem_Output}
\end{figure}
At the end of the file the summary of the lemmatization is given concerning the method of lemmatization and concerning the category of phrases. Each method or category is listed with an amount of phrases successfully and unsuccessfully lemmatized, effectiveness and coverage of all phrases by given method or category e.g.
\begin{figure}[h]
	\centering
	%\includegraphics[scale=0.7]{karina.jpg}
	\caption{Lemmatization Module Output Table}
	%\label{polem_output_table}
\end{figure}
\section{Process Description}
Lemmatization is conducted with use of different, partial lemmatizers like dictionary lemmatizers, inflection rule-based lemmatizers for specific categories etc. Dictionary lemmatizer takes as an input list of inflected phrases with assigned lemmas. Rule based lemmatizers identifies the phrase by tags and follow the rules to transform inflected phrase into its lemma. In this process there are several partial lemmatizers. They was named with such working titles and used in such order : 
\begin{itemize}
	\item Nelexicon Lemmatizer
	\item MorfGeo Lemmatizer 
	\item NamLivPerson Lemmatizer
	\item NamLoc Lemmatizer
	\item Rule Lemmatizer
	\item Orth Lemmatizer
\end{itemize}
Nelexicon Lemmatizer is dictionary lemmatizer. File to build it was taken from gazetter Nelexicon2. File consist of proper names and the group of phrase for each. To make it useful, lemmas of each name were added. Part of aforementioned file : 
\begin{figure}[h]
	\centering
	%\includegraphics[scale=0.7]{karina.jpg}
	\caption{Nelexicon Dictionary file}
	%\label{polem_output_table}
\end{figure}
        
MorfGeo Lemmatizer is another dictionary lemmatizer however file that it uses as dictionary is a list of inflected geological names and locations with their lemmas. NamLivPerson lemmatizer is partial dictionary and partial rule lemmatizer as it uses both dictionary and rules to generate or find lemmas. Dictionary that it also taken from the Nelexicon tool however it consists only of phrases related to and of human full names. Rules for inflection for NamLivPerson are in a form of list with tag of an inflected phrase, ending of an inflected phrase, ending of lemma of the phrase and numerical value that represents frequency of given ending occurring in texts. NamLoc lemmatizer uses only inflection rules to generate lemmas. Rules are in identical form as in NamLivPerson lemmatizer, the only difference between NamLoc and NamLivPerson rule-based part is the category of phrases and dictionary file. Rule Lemmatizer uses rules developed by Ph.D. Marczinczuk and described in "Lemmatization of Multi-word Common Noun Phrases and Named Entities in Polish" paper. //TODO CHANGE AFOREMENTIONED SENTENCE// Rules are in a form of XML DOM file and consists of constraints and transformations. Constraints are encoded using WCCL formalism. Sample rule : 
        
After entering values, process goes into Nelexicon Lemmatizer part. Dictionary file is searched for lemma for given phrase. If lemma is found the process returns result otherwise process proceeds to MorfGeo Lemmatizer and searches the dictionary file for lemma. When lemma is not found and category of the phrase matches person's name category process starts NamLivPerson Lemmatizer and searches its dictionary file and in case lemma is still not found inflection rules are applied in such manner, that if tag and ending of a word fits the rule, the ending of a lemma from the rule with highest occurrence value is taken and lemma is generated by replacing word's ending with ending from the rule. Phrases are dismembered into words before searching for fitting rule. Lemma is found when there is a rule for every word in a phrase. When taking into consideration the occurrence values, only rules with values higher then certain value are used. If lemma is not found and category of the phrase matches location, NamLoc Lemmatizer is started. For every word in phrase rules of lemmatizer are searched for in same manner that in NamLivPerson rule-based part.  //TODO 
\section{Evaluation()MORELIKE IMPLEMENTATION }
//write bout files and result 
\chapter{Keyword extraction module}
\chapter{Description of the system as a whole}
System consists of keyword extractor and lemmatization module. Lemmatization module was added to default process of the extractor.
\chapter{Implementation}
\chapter{Summary}
		
%BIBLIOGRAPHY
\addcontentsline{toc}{chapter}{List of tables}
\listoftables
\addcontentsline{toc}{chapter}{List of figures}
\listoffigures
\addcontentsline{toc}{chapter}{Bibliography}
% 			\bibliography{bibliography.bbl}
% 			\bibliography{mybib}{}
\nocite{*}
%             \bibliographystyle{plain}
\bibliography{main.bib}
\end{document}
